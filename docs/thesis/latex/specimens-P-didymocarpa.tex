\section*{\textit{Physaria didymocarpa subsp. didymocarpa}}

%%% UNKOWN LOCALITIES
  \textbf{Unkown State:}
  \textbf{Unknown County:}
Rocky Mountain Flora. \textit{Hall and Harbour 47} (NY).
Palliser's Brit. N. Am. Expl. Expedition, Rocky Mountains. 1858;
\textit{Bourgeau s.n.} (NY).
Franklin's Journey. \textit{Hooker s.n.} (NY).
Rocky Mtns. \textit{Durand s.n.} (NY).
Platte Rocks. \textit{Durand s.n.} (NY).
%%% CANADIAN PROVINCIAL SPECIMENS
  \textbf{Canada:}
  \textbf{Alberta:}
Bow River Valley, Banff. 4,500' elevation; 9-18 Jun 1906;
\textit{Brown 123} (MO).
Bow River. Jul 1897; \textit{Van Brunt and Van Brunt 70} (NY).
Gravelly and rocky slopes, vicinity of Banff. 4,700-7,000' elevation;
5 Jul 1899; \textit{McCalla 2270} (NY).
Devil's Lake, Banff; Alberta, Canada. 5,800' elevation; 5 Jul 1907;
\textit{Butters and Holway 41} (RM, NY).
Morley, foothills of Rocky Mtns. 24 Jun 1885; \textit{Macoun s.n.} (NY);
18 Jul 1885; \textit{Macoun s.n.} (NY).
Campsite Island Lake. 4,500' elevation; 114º40'00"W, 49º38'00"N;
22 Jul 1969; \textit{Warrington and Nagy 551} (MONT).
5 mi. N of Pincher Creek, dry south facing slope above Oldman River.
30 May 1963; \textit{Mulligan and Mosquin 2668} (NY).
Dry, open, south-facing slope, Rowe Creek; Canadian zone. 5,800' elevation;
21 Jul 1953; \textit{Breitung 16347} (NY).
East end Lakeview Ridge. 5,400' elevation; 113º53'00"W, 49º09'00"N; 14 May 1969;
\textit{Nagy and Blais 170} (MONT).
Flora of Waterton Lakes National Park; gravelly river bar, 1 mi. E of Waterton
Lake; transition zone. 4,100' elevation; 10 Jul 1953;
\textit{Breitung 15714} (NY).
Open south-facing slope, trail to Crypt Lake; Canadian Zone. 5,000' elevation;
31 Jul 1954; \textit{Breitung 17009} (F).
Pipestone Creek. 7 Jul 1904; \textit{Macoun 64432} (NY).
  \textbf{Saskatchewan:}
Saskatchewan, plants of Paliser's Brit. N. Amer. Expl. Exped. 1858;
\textit{Bourgeau s.n.} (NY).
Junction of North Fork of North Branch Saskatchewan. 15 Jun 1908;
\textit{Brown 917} (MO).
Kootany Plains North Branch Saskatchewan River. 17 Jun 1908.
\textit{Brown 970} (NY).
%%% MONTANA STATE SPECIMENS
  \textbf{Montana:}
  \textbf{Unknown County:}
Beer Creek. 12 Jun 1884; \textit{Anderson s.n.} (UC-10767).
Midway Station, M. Y. Line. 20 Jun 1899; \textit{Nelson and Nelson 5457} (RM).
West Fork of Sun River, Mont. 5,400' elevation; 2 Sep 1923;
\textit{Fickwood 2358} (RM).
Montana. 12 Mar 1909; \textit{Rydberg s.n.} (NY).
Montana. \textit{Williams 10460} (NY).
% Glacier
  \textbf{Glacier County:}
Many Glaciers RD, Glacier National Park. 27 Jun 1964; 27 Jun 1964;
\textit{Harvey 7067} (MONTU).
100 yd. W St. Mary bridge. 4,530' elevation; T35N R14W S33 NE1/4;
17 Jun 1947; \textit{McMullen 2821} (RM), \textit{McMullen 2836} (RM).
East side of Divide Mountain in talus fields. Off BIA near cell tower.
7,193' elevation; 15 Aug 2014; 48º44.10'N, 113º23.64'W;
\textit{Ratcliff and Horsch 56} (ISTC).
Divide Mtn., Glacier National Park; collected on east slope of peak in talus.
2,200m elevation; 9 Aug 1964; \textit{Harvey and Pemble 7213} (MONTU).
Divide Mtn. 16 Jul 1897; \textit{Williams s.n.} (MONT-9298).
Above timber line St. Mary Lake area on Goat Mountain above Baring Creek on
steep rock west facing slides of talus about even with Sexton Glacier,
Glacier National Park, Montana. 3 Jul 1953; \textit{McMullen 1133} MONTU.
2 mi. below summit of Siyeh Pass - Baring Basin. 15 Jul 1934;
\textit{McLaughlin 3277} (F).
Two Medicine Ridge at U.S. 89. 1,820m elevation; 13 Jul 1954;
\textit{Harvey 5836} (MONT).
Mountain sides, Mt. Henry; Midvale, Montana. 9 Jul 1903;
\textit{Umbach 305} (RM-86904, RM-168738, MONT, F, NY).
Glacier National Park, Two Medicine Trail to Cutbank Pass, just below pass, in
rocks. 7,500' elevation; 14 Aug 1939; \textit{Bailey and Bailey 470} (UC).
Rare in shale-derived soil on a steep, west-facing roadcut above Summit Creek
ca. 7 mi. SW of East Glacier. 5,100' elevation; T31N R13W S15; 24 Jul 1986;
\textit{Lesica 3986} (MONTU).
Rising Wolf Ranch at U.S. 2. 1,220m elevation; 3 Jul 1950;
\textit{Harvey 4174} (MONT).
Gravel soil in alpine tundra on Mt. Baldy, 10 mi. S of East Glacier.
7,500' elevation; 27 Jun 1981; \textit{Shaw 3433} (MONTU).
% Flathead
  \textbf{Flathead County:}
Gravelly stream bank at the foot of Summit Mtn. just north of Maria's Pass.
22 Jun 1978; \textit{Lesica 511} (MONTU).
Common in barren shaly soil on the crest of a ridge on the north end of the
Blacktail Hills. 5,700' elevation; 31 May 1993; \textit{Lesica 5935} (MONTU).
River bar Shafer Ranger Station. 4,400' elevation; 17 Jul 1937;
\textit{Root s.n.} (MONTU-103050).
0.5 mi. N of Pagoda Mtn. approx. 32 mi. SE of Spotted Bear Ranger Station.
8,047' elevation; T22N R13W S3; 10 Jul 1996; \textit{Wirt 89} (ISTC).
Common in stony limestone soil of an exposed ridge south of Sphinx Peak.
8,300' elevation; T22N R12W S28; 30 Jul 2004;
\textit{Lesica and Hanna 8910} (NY, MONTU).
% Pondera
  \textbf{Pondera County:}
Common in stony limestone-derived soil on a ridge crest 1/2 mi. S of Scarface
Mountain. 7,600' elevation; T28N R11W S15; 15 Jul 2005;
\textit{Lesica 9438} (MONTU).
1.0 mi. S of Family Pk., 9.75 mi. E Schafer Meadows Ranger Station - air strip
area, ca. 2.5 mi. of Beaver Lake. 7,000' elevation; T28N R11W S33; 11 Aug 1995;
\textit{Wirt 79} (ISTC).
% Teton
  \textbf{Teton County:}
East Front Mountains. 7,800' elevation; T26N R10W S14 N1/2; 28 Jul 1983;
\textit{Lackschewitz 10565} (MONTU).
Common on loose limestone-rubble slides on S- to S- slopes of the mountain.
7,800' elevation; T26N R10W S19 NE1/4; 27 Jun 1983;
\textit{Lackschewitz 10532} (MONTU).
Front Range, MT. Wright, west of Chouteau; SSE facing dry wash. Ca. 6,600'
elevation; T26N R10W S36 NW SE1/4; 23 Jul 1982; \textit{Ramsden 1110} (MONTU).
East Front Mountains; Chouteau Mtn. 8,000' elevation; 7 Jul 1977;
\textit{Lackschewitz 4432} (MONTU).
Two and 1/2 mi. E below confluence of Teton River forks. 4,800' elevation;
T25N R8W S25S S36N; 8 Jul 1973; \textit{Lackschewitz 4494} (MONTU).
Along road Chouteau-N-Fk. Teton River. T25N R8W S25 SW1/4; 30 Jun 1988;
\textit{Lackschewitz 11452} (GH).
Roadside Teton Canyon RD, 27 mi. NW of Choteau. 4,150' elevation;
4 Jul 1964; \textit{Finley s.n.} (MONTU).
Approx. 1 mi. E of bridge across the Teton River County RD from Ear Mtn.
Ranger Station to Choteau (20 mi.). 4,800' elevation; 9 Jun 1972;
\textit{Lackschewitz 3577} (MONTU).
Quarter mi. W of Pine Butte Swamp. 4,800' elevation; T24N R8W S11; 26 May 1995;
\textit{Lesica 6580} (MONTU).
Rocky Mountain Front Range, above and to the west of Our Lake at low pass.
7,760' elevation; T24N R9W S12 NW1/4; 9 Aug 1989;
\textit{Schassberger and Evenden 345} (MONTU).
Bob Marshall Wilderness; Trail across Headquarters Pass. 7,650' elevation;
T24N R9W S19 E1/2; 29 Jul 1978; \textit{Lackschewitz 8476} (MONTU).
Common in shaley, barren soil on a steep west-facing slope at the head of
Rierdon Gulch. 7,500' elevation; T24N R8W S23; 20 Aug 1995;
\textit{Lesica 7104} (MONTU).
Foot of Rocky Mtns., along HWY US-287 0.2 mi. N of Sevenmile Hill; 6.8 mi. SSW
of Chouteau. 4,300' elevation; T23N R5W S27 SE1/4; 13 Jul 1975;
\textit{Stickney 2261} (RM, MONT-66338, MONT-66339).
North Fork Teton River, 15 mi. W Chouteau. 4,000' elevation; 2 May 1981;
\textit{Shaw 3402} (MONTU).
Chouteau, about 12 mi. S. 11 May 1956; \textit{Booth 56261} (RM).
% Hill
  \textbf{Hill County:}
Bear's Paw Mtns., steep south-facing roadcut at the head of Big Sandy Creek,
ca. 20 mi. S of Havre. 5,400' elevation; T28N R16E S20; 28 Jun 1988;
\textit{Lesica 4609} (GH, MONTU).
% Chouteau
  \textbf{Chouteau County:}
Eagle Creek Landing, floating the Missouri White Cliffs area. 2,500' elevation;
T25N R13E S21; 27 Jun 1990; \textit{Lackschewitz 11656} (MONT).
% Missoula
  \textbf{Missoula County:}
Missoula County. \textit{Kennedy s.n.} (MONT-9356).
Miller Creek Canyon. 1,090m elevation; 9 May 1950; \textit{Harvey 4103} (MONT).
Mouth of Lolo Canyon. 10 Apr 1938; \textit{Keilman 9} (MONT).
Growing on talous slope E of Missoula, Montana. 9 May 1960;
\textit{Ruff s.n.} (UTC-00112173).
% Ravalli
  \textbf{Ravalli County:}
Bitterroot vicinity - Bitterroot Valley, tributary of Three Mile Creek; 10 mi.
NE of Stevensville. 4,000' elevation; T10N R19W S24 SE1/4; 25 May 1965;
\textit{Stickney 1195} (RM).
E St. Joseph Peak. 9,000' elevation; 4 Jul 1970;
\textit{Lackschewitz 2126} (NY).
% Granite
  \textbf{Granite County:}
Along N-side of HWY 38 (Skalkaho RD) between E-Fk. Rock Creed RD and the Middle
Fk. RD 8 May 1977; \textit{Lackschewitz 7187} (RM, MONTU).
Shale banks ca. 32 mi. E of Missoula. 20 Jun 1944;
\textit{Hitchcock and Muhlick 9119}
(RM, NY, MONT, UC, CAS-303267, CAS-303275, IDS, UTC).
Roadcut, slide above old HWY 10, Bearmouth Area. 3 Jun 1983;
\textit{Lackschewitz 10433} (MONTU).
On roadcut above the old HWY 90 in the Bearmouth Area. 13 May 1977;
\textit{Lackschewitz 7176} (RM, MONTU).
Highway 10, 35 mi. E of Missoula. 3,220' elevation; 19 May 1938;
\textit{Rose 107} (MONTU, MONT).
On sandy bank near Hiway, ca. 40 mi. E of Missoula. 5 May 1934;
\textit{Hitchcock 2293} (RM, MONT, RSA-POM, MO).
Ca. 5.9 mi. W of Drummond. 3,862' elevation; 14 Jun 2014; 46º42.7216'N,
113º14.7275'W; \textit{Ratcliff and O'Kane, Jr. 46} (ISTC).
Ca. 5.5 mi. W of Drummond between Rattlesnake Gulch and Mulkey Gulch. 1,250m
elevation; 46º42'40"N, 113º14'43"W; 10 Jun 1996; \textit{O'Kane Jr. 3794} (MO).
46 mi. E of Missoula. \textit{Gillett and Moulds 12437} (NY).
Common in stony, calcareous soil of exposed ridge 1/2 mi. W of the Powell Mine.
8,400' elevation; T7N R12W S22; 2 Aug 2000; \textit{Lesica 8142} (MONTU).
% Powell
  \textbf{Powell County:}
Flathead National Forest; on top of Shale Mtn., 5 mi. NE of Big Prairie Ranger
Station. 22 Jul 1948; \textit{Hitchcock 18620} (RM, UC).
Flathead National Forest; Top of Gordon Mtn., 6 mi. S of Big Prairie Ranger
Station. 8,300' elevation; 22 Jul 1948; \textit{Hitchcock 18861} (RM, UC).
% Lewis and Clark
  \textbf{Lewis and Clark County:}
Bob Marshall Wilderness, mountain S of Sock Lake, Upper E-Slope.
8,100' elevation; T24N R11W S31 SE1/4; 26 Jul 1979;
\textit{Lackschewitz 9106} (RM).
Sawtooth Ranch. 4,200' elevation; T21N R7W S19 NE1/4; 24 May 1988;
\textit{Lackschewitz 11372} (GH).
Haystack Butte. 5,000' elevation; 27 May 1973;
\textit{Lackschewitz 4263} (MONTU).
Common in limestone talus on an E-facing slope of Crown Mtn. near the pass.
7,200' elevation; T19N R9W S28; 6 Jun 1986; \textit{Lesica 4043} (MONTU).
Flathead Range; uncommon in a limestone fellfield on a gentle southeast-facing
slope of Flint Mtn. 8,900' elevation; T18N R10W S8; 25 Jul 2009;
\textit{Lesica and Hanna 10206} (MONTU).
Scapegoat Mountain. 7,600-8,400' elevation; 11 Aug 1975;
\textit{Craighead 62} (MONTU).
Scapegoat plateau. 8,200' elevation; 30 Jul 1972;
\textit{Craighead 122} (MONTU).
Lewis and Clark National Forest. 5,500' elevation; T18N R8W S9; 25 Jul 1928;
\textit{Harris 9} (RM).
Low ridges along Mission RD 1 mi. W of the Dearborn River. 4,000' elevation;
T17N R4W S27; 10 Jun 2012; \textit{Lesica 10808} (MONTU).
7.5 mi. NE of Wolf Creek. 3,800' elevation; 27 Jun 1963;
\textit{Mulligan and Mosquin 2820} (CAS).
Prickly Pear Cañon. 5 Aug 1887; \textit{Williams 515} (MONT).
York bridge on Missouri River NE of Helena. T11N R2W S13 NW1/4; 6 May 1980;
\textit{Ramsden 524} (MONTU).
About 2 mi. NE of Lakeside at mouth of canyon. 3,700' elevation; 4 Jul 1967;
\textit{McKinney s.n.} (MONT-63192).
Helena, Montana. \textit{Brandegee 368} (UC-117221, UC-117506).
% Cascade
  \textbf{Cascade County:}
Valley of the Sun River in the Rocky Mtns. 17 May 1884;
\textit{Doty 22} (MO-3930875).
Gravelly hills of Sun River near the Rocky Mtns. 17 May 1884;
\textit{Doty 27} (MO-3833626).
North Fork of Sun River. 5,400' elevation; 2 Sep 1925;
\textit{Kirkwood 2358} (RM, UC).
Belt River. 2 Jun 1888; \textit{Williams s.n.} (MONT-9297).
Belt River (Milk Ranch) Montana. 12 Jun 1884; \textit{Anderson 41} (MONT).
% Judith Basin
  \textbf{Judith Basin County:}
Arrow creek above mouth of Coffee Creek. 3 Jul 1901; \textit{Spragg 185} (MONT).
% Fergus
  \textbf{Fergus County:}
Judith Mtns.; common in limestone scree on a steep west-facing slope above
Maiden Canyon ca. 4 mi. NE of Giltedge. 4,600' elevation; T16N R20E S9;
20 Jun 1987; \textit{Lesica 4330} (GH, MONTU).
Big Snowy Mtns.; common in a limestone fellfield on top of Greathouse Peak.
8,600' elevation; T12N R19E S29; 19 Jul 2007;
\textit{Lesica and Hanna 9833} (ISTC, MONTU).
East-facing slope of Greathouse Peak. 8,200' elevation; T12N R19E S29;
24 Jul 1981; \textit{Lesica 1658} (MONTU).
Top of Grayhouse [sic] Peak, Big Snowy Mtns. 3 Jul 1947;
\textit{Hitchcock 16052} (NY, UC, RSA-POM, UTC).
On shale bank in Half Moon Canyon 5 mi. from mouth, Big Snowy Mtns. 5 Jul 1945;
\textit{Hitchcock and Muhlick 12012} (NY, CAS, UTC).
Shale bank at western base of Little Snowy Mtns. 4 Jul 1945;
\textit{Hitchcock and Muhlick 11931} (RM, NY, MONT, MO, CAS, UTC, UC).
% Golden Valley
  \textbf{Golden Valley County:}
Ryegate, 8 mi. NE.  17 Jun 1957; Booth 57127 (MONT).
% Wheatland
  \textbf{Wheatland County:}
Harlowton, west of town in bunchgrass high cinder content soil. 22 May 1955;
\textit{Booth 5514} (MONT).
Harlowton, 10-12 mi. S. 17 Jun 1957; \textit{Booth 5750} (MONT).
% Meagher
  \textbf{Meagher County:}
King's Hill. 4 Jul 1948; \textit{Rose 4015} (MONTU).
Little Belt Mtns., near the Pass. 7,000' elevation; 10 Aug 1896;
\textit{Flodman 496} (NY, MO), 596 (NY).
11.4 mi. N of White Sulphur Springs. 4 Jul 1963;
\textit{Mosquin and Gillett 5222} (RM, UC, UTC).
Red shale outcrop near Four Mile Ranger Station, northeastern base of Castle
Mtns. 8 Jul 1945; \textit{Hitchcock and Muhlick 12075}
(RM, NY, MO, UTC, UC, MONT, RSA-POM).
South-facing slope above Richardson Creek. 6,200' elevation;
46º32.146'N, 110º42.330'W; 17 Jul 2012; \textit{Lesica 10820} (MONTU).
On moist ridge 1 mi. N of Baldy Mtn., Big Belt Mtns. 8,500' elevation;
16 Jul 1945; \textit{Hitchcock and Muhlick 12377} (CAS).
Belt Mountains. 8 Jul 1886; \textit{Anderson 411} (NY).
Belt Mountains. 8 Jul 1886; \textit{Anderson 42} (F, MONTU).
Belt Mtns. Montana. 12 Jul 1886; \textit{Anderson s.n.} (MONT-9302).
Red sandstone outcrop ca. 6 mi. W of Lennep. 2 Jul 1951;
\textit{Hitchcock 15983}
(RM, NY, MONT, RSA-POM, UC, UTC, IDS-030425, IDS-030426).
Cottonwood Creek. 5,000' elevation; 30 Jul 1896; \textit{Flodman 495} (NY).
% Broadwater
  \textbf{Broadwater County:}
Big Belt Mountains, East Fork Cabin Gulch, ca. 1 mi. N of U.S. HWY 12, ca. 15
mi. E of Townsend. 5000' elevation; T7N R4E S22 N1/2; 12 Jun 1986;
\textit{Shelly 1100} (MONTU).
% Silver Bow
  \textbf{Silver Bow County:}
Near Butte Montana. 1893; \textit{Moore s.n.} (MO-3833622).
% Madison
  \textbf{Madison County:}
Mountains near Indian Creek, Montana. 8,000' elevation; 22 Jul 1897;
\textit{Rydberg and Bessey 4166} (NY, F).
Ruby Peak, Ruby Mountains, East of Dillon, Montana. 9,300' elevation;
T6S R5W S16; 22 Aug 1982; \textit{Rosentreter 2836} (MONTU).
Gravelly Range; common in stony limestone-derived soil of the exposed summit of
Baldy Mtn. 9,500' elevation; T7S R3W S27; 17 Jun 2006;
\textit{Lesica and Kittelson 9636} (MONTU).
Above the Ruby River 1 mi. N of the dam. 5,750' elevation; T7S R4W S4;
19 Jun 2003; \textit{Lesica 8648} (MONTU).
Madison Range; W Slope, 8 mi. S of Jeffers. 17 Jun 1930;
\textit{Young 9350} (MONT).
Cedar Mountain, Montana. 10,000' elevation; 16 Jul 1897;
\textit{Rydberg and Bessey 4168} (NY).
Northern Rocky Mtn., Beaverhead National Forest; Lewis Creek reseeding area.
6,300' elevation; T9S R3W S28; 20 Jun 1937;
\textit{Short and Aicher S-525} (RM, MONT).
Beaverhead National Forest; NRM Ranger Station,Schoolmarm Gulch.
6,050' elevation; T9S R3W S17 SW1/4; 18 May 1952;
\textit{Schmautz JES-28} (RM, MONT).
Uncommon on a rocky, south-facing slope of Cave Mtn. ca. 25 mi. S of
Virginia City, plot 14. 9,600' elevation; T10S R1W S32; 20 Jul 1989;
\textit{Lesica and Cooper 4914} (NY, MONTU).
Madison Range; common in gravelly soil on top of a limestone ridge in
Koch Basin, plot 59. 9,600' elevation; T9S R2E S23; 2 Aug 1991;
\textit{Lesica and Cooper 5571} (NY, MONTU).
Madison Range; common in limestone-derived soil on an exposed ridge in
Koch Basin. 9,800' elevation; T9S R2E S23; 28 Jul 1006;
\textit{Lesica and Cooper 9650} (MONTU).
On mountain 1/2 mi. N of Koch Peak; Taylor Mtns. 2 Aug 1946;
\textit{Hitchcock and Muhlick 15225} (NY, UC, MO).
Madison Range, Gallatin National Forest; Koch Basin on the east side of
Koch Peak at head of Tumbledown Creek, ca. 25 mi. SE of Ennis.
9,800' elevation; T9S R2E S14; 7 Aug 1995; \textit{Evert 30558} (RM).
% Gallatin
  \textbf{Gallatin County:}
Rocky hillside, near Maudlow. 1 Jun 1956; \textit{Denton s.n.} (MONT-52764).
Rocky Cañon. 5,000' elevation; 14 Jun 1905;
\textit{Blankinship 63} (F-190109, MO-3833621).
Manhattan, Montana; 10 mi. N in Horseshoe Hills. 15 May 1949;
\textit{s.n. s.n.} (MONT-42418).
Sedan. 30 May 1901; \textit{Jones s.n.} (MONT-9306).
Sedan, 3 mi. SW exposed dry slope. 26 Jun 1949;
\textit{Metcalf s.n.} (MONT-77931).
Gallatin N. F., Bridger Mountains; East slopes and summit of Mount Sacagawea,
including the Bridger Divide (saddle between Mt. Sacagawea and Hardscrabble
Peak); collected along trail from Fairy Lake to Sacagawea summit.
8,000' elevation; T2N R6E S22 and S27; 7 Aug 1990;
\textit{Bayer, Lebedyk, and Joncas MT-620} (RM).
Just N of the summit of Sacagawea Peak ca. 17 mi. N of Bozeman; Bridger Range.
9,000-9,500' elevation; T2N R6E S27 NW4; 10 Aug 1989; \textit{Evert 18466} (RM).
Gallatin National Forest, Sacagawea Peak. 9,800' elevation; 8 Aug 1935;
\textit{Whitham 1805} (RM, MONT).
Sacagawea Peak, Bridger Mtns. 8,800' elevation; 19 Jul 1969;
\textit{Dorn 913} (RM).
Collected from among rocks at ca. 9,000' from Sacagawea Pk. Bridger Range.
16 Jul 1972; \textit{Schaack 675} (MONTU).
Valley north of Sacagawea Peak, Bridger Range. 8,100-9,200' elevation;
31 Jul 1938; \textit{Pennell, Cotner, and Schaeffer 23832} (NY, US).
Mt. Bridger. 8,000' elevation; 5 Jul 1905; \textit{Blankinship 64}
(RM-89447, F-190110, MONTU-3784).
Mt. Bridger. 9,000' elevation; 3 Jul 1900;
\textit{Blankinship s.n.} (MONT-9309).
Mt. Bridger. 7,000-8,000' elevation; 11 Jul 1903;
\textit{Blankinship s.n.} (MONT-9351).
Mt. Bridger. 8,500' elevation; 11 Aug 1903; \textit{Blankinship s.n.}
(MONTU-095452).
Bridger Peak. Jun 1889; \textit{Kock s.n.} (MONT-9301).
Bridger Peak. 8,900' elevation; 25 Jun 1933;
\textit{Young s.n.} (RM-139984, MONT-22915, MONTU-28127, UTC-15083).
Summit of Mt. Bridger, Bozeman. 9,000' elevation; 26 Jun 1899;
\textit{Blankinship s.n.} (MONT-9300).
Mt. Baldy, Bridger Mtns. 7 Jul 1977; \textit{Forcella 69354} (MONT).
Bridger Mountains, Mont. 9,000’ elevation; 15 Jun 1897;
\textit{Young 4167} (RM, MONT, F).
Bridger Canyon. 5 Jun 1921; \textit{Hall s.n.} (MONT-26081).
Bridger Pass. 22 May 1898; \textit{Wilcox s.n.} (MONT-9303).
Bridger Pass. 22 May 1897; \textit{Blankinship s.n.} (MONT-9305).
Bridger foot-hills. 24 May 1921; \textit{Kindsley 33} (MONT).
Shale bank at the south base of Bridger Mtns., 8 mi. NE of Bozeman. 18 Jul 1945;
\textit{Hitchcock and Muhlick 12470} (RM, NY, UC, MONT, UTC).
Foothills 3 mi. NE of Bozeman. 30 Apr 1921; \textit{Savage 15} (MONT).
2.5 mi. NE Bozeman. 30 Apr 1921; Powers 14 (MONT).
0.1 mi. SW of Kelly Canyon RD on Bridger Canyon RD (HWY 86). 5,013'
elevation; 14 Jun 2014; 45º42.1746'N, 110º55.8793'W;
\textit{Ratcliff and O'Kane, Jr. 47} (ISTC).
East of Bozeman; near electric powerhouse. 23 May 1900;
\textit{Wilcox 159} (NY).
Near Bozeman. 8 Jun 1883; \textit{Scribner 8} (NY).
Bozeman. Jul-Aug 1906; \textit{Blankinship 63}
(RM-89448, UC-311010, RSA-POM-143083, UTC-00159819).
Bozeman. 10 May 1892; \textit{Blankinship s.n.} (MONT-9296).
Bozeman. 10 May 1904; \textit{Jackson s.n.} (MONT-9295).
Bozeman. 14 May 1905; \textit{Flaherty s.n.} (UC-165233).
Bozeman. 16 May 1904; \textit{Maynard s.n.} (MONT-9311).
Bozeman. 24 May 1901; \textit{Moore s.n.} (RM-73253, MO-3833623).
Bozeman. 24 May 1901; \textit{Jones s.n.} (RM-122279).
Bozeman. 10 Jun 1921; \textit{Bohars s.n.} (MONT-26011).
Bozeman. 20 Jun 1902; \textit{Jones s.n.} (UC-165173).
Outside Gallatin National Forest. 6,100' elevation; T2S R7E S13 E1/2;
20 Jun 1928; \textit{Swim 714} (RM, MONT).
Ft. Ellis to the Yellowstone. Jul 1891; \textit{Porter s.n.} (NY).
Madison Range; common in limestone talus on a west-facing slope of Cone Peak.
9,600' elevation; T10S R4E S26; 10 Jul 2007;
\textit{Lesica and Kittleson 9808} (MONTU, ISTC).
% Park
  \textbf{Park County:}
Suksdorf's Gulch, 9 mi. NW of Wilsall. 19 Jul 1921; \textit{Suksdorf 513} (NY).
Several large colonies near the crest of the ridge; mountain N of Sunlight Lake.
9,200' elevation; T4N R11E S8 NW1/4; 30 Jul 1980;
\textit{Lackschewitz 9374} (MONTU).
Ridge 1/8 - 1/4 mi. N of Cokedale RD, 6 mi. W of Livingston; Gallatin Range
foothills. 5,100-5,300' elevation; T2S R8E S24; 14 Jun 1996;
\textit{Evert 31157} (RM).
Livingston, Montana. 20 May 1901; \textit{Scheuber s.n.}
(NY-2526, UC-991364, MONT-9308).
Livingston. 1901; \textit{Scheuber 363} (NY).
1 mi. NW Livingston. 20 May 1951; \textit{Wright 25} (MONT).
About 3 mi. E of Livingston on a rocky slope, Cemetery RD. 17 May 1959;
\textit{Hoversten s.n.} (MONT-56052).
Common in open soil and in alpine turf on the ridge just east of Elephanthead
Mtn. Limestone Parent. 9,000' elevation; T3S R11E S30; 25 Jul 1985;
\textit{Antibus and Lesica 3569} (MONTU).
Absaroka Mtns., common in limestone talus near the summit of Elephanthead Mtn.
9,400' elevation; T3S R11E S30; 14 Aug 1998; \textit{Lesica 7719} (NY, MONTU).
Chico Hot Springs. 15 Jul 1921;
\textit{Suksdorf 443} (RM, NY, UTC, MO, RSA-POM).
Absaroka Forest. 5,000' elevation; T6S R8E S27; 2 Jun 1924;
\textit{Moir 73} (RM).
Electric Peak. 10,000' elevation; 26 Jul 1902;
\textit{Rev. Earnest and Smith 25} (F).
% Sweet Grass
  \textbf{Sweet Grass County:}
Beartooth Mountains; in foothills, ca. 1/4 mi. W of Forest RD 482
(Grouse Ridge / Sliderock Mountain RD); ca 5 air mi. S of I-90. 5,000-5,200'
elevation; T2S R15E S7 and S8; 4 Jun 1993; \textit{Evert 24792} (RM).
McLeod. 4 Jun 1923; \textit{Pope 145}.
Common in shallow, sparsely-vegetated metamorphic-derived soil of a ponderosa
pine woodland on a south-facing slope above Jim's Gulch. 5,200' elevation;
T2S R15E S34; 12 Jun 2005; \textit{Lesica 9143} (NY).
Along Sliderock Mtn. - Grouse Ridge RD (F.S. \#482), Gallatin National Forest;
Beartooth Mtns. 7,100' elevation; T3S R14E S14; 20 Jul 1991;
\textit{Evert 22091} (RM).
Common in stony, calcareous soil on the rim of the saddle north of Picket Pin
Mtn. 9,400' elevation; T4S R14E S19; 28 Jul 1998; \textit{Lesica 7672} (NY).
Beartooth Mtns., common in gravelly limestone fellfield on the mountain ca.
1 mi. N of Picket Pin Mtn. 9,500' elevation; T4S R14E S20; 10 Aug 1987;
\textit{Lesica 4481} (GH).
Saddle between Picket Pin Mtn. and unnamed pinnacled limestone ridge,
Custer National Forest, just N of Picket Pin RD ca. 12 mi. W of Nye; Beartooth
Mtns. 9,000' elevation; T4S R14E S29; 21 Jul 1994; \textit{Evert 28321} (RM).
% Stillwater
  \textbf{Stillwater County:}
Beartooth Mountains, ca. 1/4 mi. N of Castle Creek, in foothills ca. 7 mi. NW
of Nye. 6,000' elevation; T4S R15E S30 NE4; 9 Jul 1993;
\textit{Evert 25821} (RM).
Along the road from Mouat Mill to Horseman’s Flat. 5,300' elevation;
20 Jun 1976; \textit{Robertson 1119} (RM).
Nye, Montana. 19 Jun 1937; \textit{Peterson 736} (MONTU).
Midnight Canyon. 23 Apr 1923; \textit{s.n. s.n.} (MONT-34167).
Ca. 1/4-1/2 mi. S of Cliff Swallow fishing access site ca. 10 mi. W of
Absarokee. 4,700' elevation; T4S R17E S4; 12 Jun 1992;
\textit{Evert 22811} (RM).
Absorokee, Mont. 20 Jun 1924; \textit{Hawkins s.n.} (UC-372122).
Absorokee, Mont. 21 Jun 1914; \textit{Hawkins s.n.} (MONT-34168).
Absorokee, Mont. 26 Jun 1922; \textit{Hawkins s.n.} (MONT-34163).
Absorokee, Mont. 28 Jun 1922; \textit{Hawkins s.n.} (MONT-34166).
Jackstone Creek; Absorokee, Mont. 26 Jul 1923;
\textit{Hawkins s.n.} (MONT-34162).
Fishtail, Montana. 26 May 1922; \textit{Hawkins s.n.} (MONT-34164).
2 mi. W of Columbus; HWY 10. 25 May 1925;
\textit{J.C. Wright and A. Wright s.n.} (MONT-43983).
4 mi. N of Columbus, Montana; breaks of the Yellowstone River. 31 May 1948;
\textit{Payne s.n.} (MONT-38787).
% Carbon
  \textbf{Carbon County:}
South Fork Grove Creek near the mouth of the canyon ca. 6 mi. S of Red Lodge.
5,500' elevation; T8S R20E S35; 14 Jun 1984; \textit{Lesica 3014} (GH).
A few individuals on open bank of a roadcut near the Sage Creek Campground 10
mi. NE of Warren. 5,600' elevation; T7S R26E S20; 23 Jun 1983;
\textit{Lesica 2645} (MONTU).
%%% WYOMING STATE SPECIMENS
  \textbf{Wyoming:}
  \textbf{Park County:}
Valley of the Yellowstone River. \textit{Hines s.n.} (MO-3833624).
Yellowstone National Park. \textit{Freeman s.n.} (RM).
Yellowstone National Park, common on bare slopes near Mammoth Hot Springs Area.
12 Aug 1947; \textit{Beetle 5148} (CAS).
Mammoth Hot Springs. 6,300' elevation; \textit{Dewart s.n.} (UC, NY).
Mammoth Hot Springs. 1,900m elevation; 15 Jun 1902;
\textit{Mearns 1137} (RSA-POM).
Mammoth Hot Springs. 6,200' elevation; 13 Jul 1906; \textit{Cooper s.n.} (RM).
Mammoth Hot Springs. Jul 1904; \textit{Oleson 239} (RM).
Near Mammoth Hot Springs. 6,000' elevation;
\textit{Burglehaus s.n.} (NY, UC, MO).
Mammoth Hot Springs, Yellowstone National Park. 6,300-6,800' elevation;
25 Jun 1925; \textit{Conard NA} (RM).
Jackson Hole, on Little Green River. 4,500' elevation; 6 Jul 1860;
\textit{Hayden s.n.} (MO-3833625).
Glen Creek. 29 Jun 1899; \textit{Nelson and Nelson 5570} (NY).
7.5 mi. W of Tower Junction. 7,000' elevation; 13 Sep 1955;
\textit{Hermann 12542} (RM).
Rock Creek, Yellowstone Park. Aug 1922; \textit{Hawkins 960} (MONT-34165).
Absaroka Mountains, North Fork Shoshone River Drainage; north side of Crow Peak
ca. 1-2 mi. N of U.S. HWY 14, 16, and 20. 7,500-8,500' elevation; T52N R109W S2;
28 Jun 1987; \textit{Evert 12677} (RM).
Ca. 1/2 mi. E of Pahaska along US HWY 14, 16, and 20. 6,800' elevation;
T52N R109W S2 SW1/4; 13 Jul 1981; \textit{Evert 3123} (RM).
Northeast-southwest trending ridge ca. 1/4 mi. E of Grinnell Creek and ca.
1/8-3/4 mi. N of U.S. HWY 14, 16, and 20. 6,800-7,600' elevation;
T52N R109W S12, R108W S6; 23 Jun 1987; \textit{Evert 12460} (RM, NY).
Ridge ca. 0.5 mi. W of Mormon Creek; ca. 0.5-2 mi. N of U.S. HWY 14, 16, and 20.
7,600-8,600’ elevation; T52N R108W S8; 18 Jul 1985; \textit{Evert 8392} (RM).
Absaroka Range, South Absaroka Wilderness; Fishhawk Creek, 0.5-1 mi. SSW of
confluence with the North Fork Shoshone River. 6,500’ elevation;
T52N R108W S27 and S34; 7 Jul 1978; \textit{Hartman 8361} (RM).
Along Elk Fork Creek Trail, ca. 5-6 mi. S of U.S. HWY 14, 16 and 20.
6,500-6,700' elevation; T51N R107W S13, S7; 26 Jul 1985;
\textit{Evert 8918} (RM).
Ca. 3-4 mi. S of US HWY 14, top of Clayton Mountain. 10,100' elevation;
T51N R107W S10 NW1/4; 20 Jul 1982; \textit{Evert 4362} (RM).
Slopes on S side of summit rim of Clayton Mountain and northernmost end of
saddle connecting Clayton Mountain and Double Mountain, ca, 1.1 air mi. E of
East Fork Blackwater Creek and 3.2 mi. S of US HWY 14, 16, and 20. 9,500-10,000'
elevation; T51N R107W S10 S1/2 NW1/4; 17 Jul 1996; \textit{Fertig 16871} (RM).
Top of Clayton Mountain, ca. 3 mi. S of U.S. HWY 14, 16, and 20. 9,800'
elevation; T51N R107W S10; 24 Jul 1985; \textit{Evert 8782} (RM).
N-S trending ridge immediately E of June Creek, ca. 1/4 mi. S of Norht Fork
Shoshone River. 6,400-6,600’ elevation; T52N R107W S10 and S3; 4 Jun 1989;
\textit{Evert 16391} (RM).
East of Newton Creek, ca. 1.5 mi. S of US HWY 14. 6,800' elevation; T52N R107W
S21 SW1/4; 7 Aug 1982; \textit{Evert 4694} (RM).
Ridge/divide between Moss Creek and Clearwater Creek ca. 1.5-2.5 mi. N of HWY
14, 16, and 20. 7,600-8,300' elevation; T52N R107W S10 and S3; 6 Jul 1989;
\textit{Evert 17578} (RM).
Many pinnacled ridge E of Clearwater Creek, ca. 3-4 mi. N of U.S. HWY 14, 16,
and 20. 7,200-8,000' elevation; T52N R107W S2; 9 Jun 1986,
\textit{Evert 10137} (RM).
Ridge, 0.5 mi NW of Signal Peak, ca. 1 mi. N of U.S. HWY 14, 16, and 20. 7,000'
elevation; T52N R106W S18; 13 Jun 1986; \textit{Evert 9930} (RM).
N-S trending ridge immediately W of Clearwater Creek ca. 1/8 mi. N of HWY 14, 16,
and 20. 6,200' elevation; T52N R106W S20; 4 Jun 1989; \textit{Evert 16363} (RM).
Elk Fork Creek ca. 3/4 mi. S of US HWY 14. 6,000' elevation; T52N R106W S29
NE1/4; 15 Jul 1982; \textit{Evert 4247} (RM).
Ridge E of Sweetwater Creek, ca. 4-5 mi. N of U.S. HWY 14, 16, and 20.
7,500-8,000' elevation; T52N R106W S5 and S33; 16 Jun 1986;
\textit{Evert 10040} (RM).
Ridge between Horse and Sweetwater creeks, ca. 1.5-3 mi. N of U.S. HWY 14, 16,
and 20. 6,800-7,500' elevation; T52N R106W S4 and S9; 16 Jun 1987;
\textit{Evert 12281} (RM, NY).
North-south trending ridge between Horse Creek and Grizzly Creek ca. 2-5 mi. N
of U.S. HWY 14, 16, and 20. 7,500-8,500' elevation; T52N R106W S3, T53N R106W
S35; 25 Jun 1987; \textit{Evert 12582} (RM).
North ridge of Ptarmigan Mountain on divide between Cougar and Pagoda Creeks,
ca. 3.9 mi. E of Elk Fork Creek, ca. 5 mi. S of US HWY 14, 16 and 20. 10,200'
elevation; T51N R106W S15 SW1/4 NE1/4; 30 Jul 1996; \textit{Fertig 16970} (RM).
Divide between Cougar and Pagoda Creeks below N ridge of Ptarmigan Mountain,
ca. 3 air mi. E of Elk Fork Creek, ca. 3.5 mi. S of US HWY 14, 16 and 20.
8,600' elevation; T51N R106W S10 NE1/4 SW1/4; 30 Jul 1996;
\textit{Fertig 16965} (RM).
On ridge between Cougar and Pagoda Creeks, ca. 4 mi. S of US HWY 14, 16, and 20.
8,600' elevation; T51N R106W S10 SW4; 22 Jul 1981; \textit{Evert 3275} (RM).
Ridge system on N side of North Fork Shoshone River between Signal Peak and
Anvil Rock, ca. 1 mi. N of US HWY 14/16/20 and ca. 5 air mi. W of Wapiti.
7,200-7,400' elevation; T52N R105W S18 E1/2 SW1/4; 26 Jun 1997;
\textit{Fertig 17583} (RM).
On ridge between Fishhawk and Mesa creeks, ca. 2-3 mi. S of U.S. HWY 14, 16,
and 20. 7,000-8,000' elevation; T51N R105W S2 and S3; 8 Aug 1984;
\textit{Evert 7451} (RM).
Top of Four Bear (Black) Mountain, ca. 2.5 mi. N of U.S. HWY 14, 16, and 20.
7,200-7,600' elevation; T52N R104W S5; 14 Jun 1988; \textit{Evert 14434} (RM).
North side of Logan Mountain. 5,850' elevation; T52N R104W S11 SE1/4;
19 Jun 1982; \textit{Evert 3939} (RM).
SW flank of Rattlesnake Mtn. ca. 6-7 mi. W of Cody and ca. 1 mi. N of HWY 14,
16, and 20. 6,400-7,000' elevation; T52N R103W S2; 15 Jun 1989;
\textit{Evert 16761} (RM).
Dry gravelly soil, Middle Shoshone Canyon. 6,500' elevation; 7 Jul 1921;
\textit{Wiegand, Castle, Dann, and Douglas 1008} (F).
Eastern Absaroka Mtns., ca. 6 air mi. N of Dead Indian Peak, ca. 0.25 air mi. N
of Sunlight Creek. 7,000' elevation, T55N R106W S4 SW1/4 NE1/4; 19 Jul 1995;
\textit{Mills 39} (RM).
Windy Mountain Summit and upper NW slope, 6-7 air mi. NNW of sunlight
ranger station. 9,900' elevation; T56N R106W S26 and S35; 17 Aug 1985;
\textit{Hartman and Nelson 21835} (RM).
Bighorn Basin, Kimball Bench, ca. 10 air mi. SSW of Clark. 4,800-5,000'
elevation; T55N R103W S3 and S4; 24 May 1983;
\textit{Hartman and Hamann 14354} (NY).
SW-facing slopes below summit cone of Heart Mountain, NE of WY HWY 120, ca. 9
air mi. NNW of Cody. 7,100' elevation; T54N R102W S15 SE1/4 NW1/4; 30 Jun 1997;
\textit{Fertig and Lenard 17641} (RM).
Bighorn Basin, Spring Creek, 2 air mi. WNW of Meeteetse. 5,800' elevation;
T48N R100W S6; 21 Jun 1983; \textit{Hartman and Nelson 15525} (UTC).
South Fork Shoshone River. 7,000' elevation; T48N R106W S18 NW1/4; 24 Aug 1987;
\textit{Dorn 4799} (RM).
Absaroka Mountains, ca. 33 air mi. SW of Cody at the junction of Boulder Creek
and Little Boulder Creek. 7,300' elevation; T49N R105W S30 and S29; 1 Jul 1983;
\textit{Kirkpatrick 613} (RM, GH).
Eleanor Creek N to ridge. 10,500' elevation; T48N R105W S30, S32; 25 Aug 1984;
\textit{Hartman 19364} (RM).
Jack Creek Trail ca. 0.2 mi. N of junction with Haymaker-Timber Creek Trail,
ca. 14 air mi. W of Sunshine Reservoir Dam. 9,200' elevation; T47N R104W S10
SE1/4 NW1/4; 29 Jun 1988; \textit{Marriott 10872} (RM).
Upper Greybull River between Steer Creek and Cow Creek, 4.5-7 air mi. NW of
Kirwin. 9,300' elevation; T46N R104W S17, S18, S20, S29; 22 Aug 1983;
\textit{Hartman 17305} (RM).
Southeastern Absaroka Mtns., ca. 3 air mi. NE of Chief Mtn., ca. 0.5 air mi. N
of Jojo Crk. and Wood River junction. 8,480' elevation; T46 R103W S21 NE1/4
SE1/4; 7 Jul 1995; \textit{Mills 16} (RM).
Ca. 4 air mi. NE of Chief Mtn., ca. 0.5 air mi. N of Jojo Crk. and Wood River
junction. 8,560' elevation; T46N R103W S22 SW1/4 NW1/4; 7 Jul 1995;
\textit{Mills 17} (RM).
Ca. 5 air mi. ENE of Chief Mtn., across road from Brown Creek Camp Ground.
7,600' elevation; T46N R103W S23 SW1/4 SE1/4; 8 Jul 1995;
\textit{Mills 22} (RM).
Ca. 6 air mi. ENE of Chief Mtn., ca. 0.5 air mi. N of the Wood River. 7,800'
elevation; T46N R103W S24 S1/2; 7 Jul 1995; \textit{Mills 20} (RM).
Absaroka Mountains, ca. 22 air mi. SW of Meeteetse, in the vicinity and W of the
junction of the Middle Fork Wood River and Beaver Creek. 7,900' elevation;
T46N R103W S35 and S36; 26 Jul 1984; \textit{Kirkpatrick 4997} (RM).
Growing on shale slopes of southwest-facing slopes of mountains above the South
Fork of the Wood River, Meeteetsee. 7,588' elevation; Zone 12T WGS 84,
Easting 0652708 Northing 4864916; 25 Aug 2012;
\textit{Smith 551, 552, 553} (ISTC).
% Hot Springs
  \textbf{Hot Springs County:}
East Foothills Absaroka Mountains; Mount 7049, ca. 1.25 air mi. NW of summit of
Adam Weiss Peak, ca. 33.5 air mi. NW of Thermopolis. 6,600-6,800' elevation;
T45N R99W S6 SE1/4 NW1/4; 9 Jul 1992; \textit{Fertig 12949b} (RM).
Ridge ca. 1 mi. NNW of summit of Twin Buttes, ca. 35 air mi. W of Thermopolis.
6,600-7,100' elevation; T45N R100W S2 SW1/4 and S11 NW1/4; 10 Jul 1992;
\textit{Fertig 12967} (RM).
Ridge on north side of Grass Creek Road (BLM RD 1317), ca. 0.75 mi. NNW of the
summits of Twin Buttes, ca. 35 air mi. W of Thermopolis. 6,600' elevation;
T45N R100W S11 SE1/4 NE1/4 NW1/4; 18 Jun 1998; \textit{Welp 7840} (RM).
Ca. 20 air mi. S of Meeteetse, between Twin Buttes and Little Grass Creek.
6,800' elevation; T45N R100W S14, S11, and S13; 27 Jun 1984;
\textit{Kirkpatrick 3093} (RM).
Ridge on north side of Grass Creek 1 mi. W of confluence of Sanford Creek.
7,200' elevation; T45N R101W S13; 11 Jul 1992; \textit{Fertig 13009} (RM).
Along Negro Creek near junction with Cottonwood Creek, ca. 3.5 air mi. NE
of Squaw Teat Butte, ca. 36.5 air mi. WNW of Thermopolis. 7,000' elevation;
T44N R100W S7; 5 Jul 1984; \textit{Nelson 11272} (RM).
Ca. 4.5 air mi. N of Anchor Reservoir, ca. 33 air mi. WNW of Thermopolis.
6,780-6,900' elevation; T44N R100W S34 SW1/4 SW1/4; 4 Jul 1992;
\textit{Fertig 12871} (RM).
Ca. 4.5 air mi. N of Anchor Reservoir, ca. 33 air mi. WNW of Thermopolis.
6,800-6,900' elevation; T44N R100W S33 SE1/4; 4 Jul 1992;
\textit{Fertig 12865} (RM).
Ridge NW of summit of Hill 7493, ca. 4.5 air mi. N of Anchor Reservoir, ca.
33.5 air mi WNW of Thermopolis. 7,340-7,400' elevation; T44N R100W S33 SW1/4
SW1/4, S32 SE1/4 SE1/4; 4 Jul 1992; \textit{Fertig 12869} (RM).
Mesa northwest of Anchor Reservoir, where Eagle Nest Ranch RD (RD 20) meets
North Fork Owl Creek. 6,709' elevation; 16 Jun 2014;
43º41.9165'N, 108º52.6981'W; \textit{Ratcliff and O'Kane, Jr. 50} (ISTC).
Owl Creek Mountains; butte on north side of North Fork Owl Creek and S of
North Fork Road, ca. 33 air mi. WNW of Thermopolis. 6,900' elevation;
T43N R100W S16 NW1/4; 16 Jun 1991; \textit{Marriott 11345} (RM).
Ridge ca. 1.25 air mi. W of Anchor Reservoir, ca. 32.5 air mi. W of Thermopolis.
6,780-6,820' elevation; T43N R100W S28 NW1/4; 7 Jul 1992;
\textit{Fertig 12913} (RM).
Along Rock Creek, ca. 2.5 mi. E of Washakie Needles, ca. 49 air mi. WNW of
Thermopolis. 9,600' elevation; T44N R103W S25; 24 Jul 1983;
\textit{Nelson and Kirkpatrick 10341} (RM, NY).
% Fremont
  \textbf{Fremont County:}
Northern Wind River Range, ca. 1 air mi. SE of Whickey Mountain Peak, along
Whiskey Mountain trail. 9,640' elevation; T40N R106W S20 center of N1/2;
18 Aug 1995; \textit{Mills 229} (RM).
NE Wind River Range: S end of Torrey Rim, ca. 0.2 mi. N of Trail Lake trailhead,
ca. 1.75 mi. W of Trail Lake, ca. 7 air mi. S of Dubois. 7,700-7,850' elevation;
T40N R106W S16 SE1/4 SE1/4, S21 N1/2 E1/4; 14 Jun 1996;
\textit{Fertig 16632} (RM).
Absaroka Mountains, East fork of the Big Wind River, 7 air mi. ESE of Dubois.
7,000' elevation; T41N R105W S20; 23 Jun 1983; \textit{Hartman 15583} (RM).
Wind River Indian Reservation, along U.S. Hwy 287, ca. 14 mi. SE of Dubois.
6,600' elevation; 2 Jul 1983; \textit{Evert 5268} (RM).
Wind River Indian Reservation. 8,300' elevation; T7N R5W S13 NW1/4; 10 Jul 1985;
\textit{Lichvar 4561} (RM).
Absaroka Range, Wind River Indian Reservation, ca. 20 air mi. E of Dubois.
9,700' elevation; T7N R4W S28; 31 Jul 1981; \textit{Day and Berner 25} (RM).
Wind River Indian Reservation, gypsum formation. 6,000' elevation; T6N R3W S16;
23 Jun 1982; \textit{Lichvar 5168} (RM).
Wind River Reservation, Pasup Creek, Circle Ridge Road. 6,965' elevation;
T7N 3W S35; 1 Aug 1942; \textit{Murphey s.n.} (RM-404206).
Heavy clay soil, hillside, 15 mi. NW of Fort Washakie. 22 Jul 1983;
\textit{Rollins 83330} (NY).
Wind River Indian Reservation, ca. 25 mi. NNW of Morton. 22 Jun 1981;
\textit{Rollins and Rollins 81386} (RM, NY, GH, UC, US).
Shoshone N. F.; outcrop, ca. 1/2 mi WNW of USFS Sinks Canyon Campground, ca.
7 mi SW of Lander. 8,500' elevation; T32N R101W S13 SE1/4; 15 Jun 1978;
\textit{Johnston and Lucas 1680} (RM).
Southern Wind River Range, ca. 8 mi. SW of Lander, SE side of Fossil Hill.
7,840' elevation; T32N R100W S30 SW1/4 SW/14; 28 Jun 1995;
\textit{Mills 6} (RM).
E of Lander. 6,000' elevation; T32N R99W S11 SW1/4; 18 May 1981;
\textit{Lichvar 4215} (RM).
SE edge of Wind River Range; E of Dry Lake and NE of US 287/WY 28 junction, ca.
7 air mi. SE of Lander; Lee Ranch. 5,600' elevation; T32N R99W S23; 16 Jun 1989;
\textit{Marriott 11013} (RM).
Red Canyon Rim on E side of Red Canyon, ca. 11 air mi. SSE of Lander.
5,700-5,800' elevation; T31N R99W S10 SW1/4 NW1/4 and S9 NE1/4 NE1/4;
7 Jun 1994; \textit{Fertig 14811} (RM, NY).
E slope Wind River Range; Red Canyon Rim, on east side of county road, ca. 1.25
mi. W of WY HWY 28. 5,800-5,900' elevation; T31N R99W S10 NW1/4; 14 May 1994;
\textit{Fertig 14672} (RM).
Red Canyon, slopes W of Red Canyon Creek ca. 13 air mi. SSE of Lander.
6,000-6,200' elevation; T31N R99W S15 SW1/4; 24 May 1993;
\textit{Fertig and Studenmund 13573} (RM).
Red Canyon Rim, ca. 14 air mi. SSE of Lander. 6,100' elevation; T31N R99W S23,
S25, and S26; 20 Jun 1986; \textit{Haines 6712} (RM).
East slope Wind River Range, Red Canyon Rim above Foster Draw ca. 15 air mi. SSE
of Lander.  6,350' elevation; T31N R99W S26 NE1/4 SE1/4; 8 Jun 1994;
\textit{Fertig 14829} (RM).
17.9 mi. S of Lander on Hwy 28; E side of road. 2,059m elevation;
42º36'47"N, 108º36'20"W; 26 Jun 1996; \textit{Salywon and Dierig 3153} (ISTC).
Near State Route 28, 18.4 mi. SW of Lander. 22 Jul 1983;
\textit{Rollins and Rollins 83331} (NY, GH, US).
21 mi. S of Lander. 20 Jun 1969; \textit{Barneby 15100} (NY, GH).
17.65 mi. S of Lander on Atlantic City RD. 8,100' elevation; T30N R99W S17;
19 May 1946; \textit{Wiegand 207} (RM).
13 mi. SE of Lander. 5,800' elevation; T32N R98W S33; 1 Jul 1991;
\textit{Dorn 5242} (RM, NY).
18 mi. SSE of Lander. 6,800' elevation; T30N R98W S8 NE1/4 SW1/4; 12 Jun 1991;
\textit{Dorn 5177} (RM, NY, MO).
Great Divide Basin Area, Popo Agie River Drainage, Box Spring, ca. 20 air mi.
SE of Lander; ca. 17 air mi. W of Sweetwater Station. 6,220-6,600' elevation;
T30N R98W S12; 5 Aug 1995; \textit{Welp 7397} (RM).
Hills along Twin Creek. 6,100' elevation; T31N R98W S36 NW1/4; 30 May 1990;
\textit{Dorn 5054} (RM, NY).
Sheep Mountain, ca. 11 air mi. SE of junction U.S. HWY 287 and WY HWY 28.
6,200' elevation; T31N R98W S36, R97W S31; 31 May 1985;
\textit{Hartman 20103} (NY).
Great Divide Basin Area, Popo Agie River Drainage, above Red Bluff Canyon, ca.
6 air mi. N of Schoettlin Mountain; ca. 5.5 air mi. SE of Weiser Pass.
6,600-6,700' elevation; T30N R97W S4; 4 Aug 1995; \textit{Welp 7180} (RM).
Sweetwater River Plateau, south end of Beaver Rim, ridge on N side of Red Canyon
on east bank of Beaver Creek, 2-2.5 mi. W of U.S. HWY 287. 5,800-6,100' elevation;
T30N R97W S1 NE1/4 SW1/4 and N1/2 SE1/4; 30 Jun 1995;
\textit{Fertig and Studenmund 15808} (RM).
Beaver Rim Divide, ca. 5 mi NW from Sweetwater Junction on HWY 287, and over 1
mi. NW on abandoned highway. 6,720' elevation; T30N R96W S3 NE1/4 SE1/4;
9 Jun 2003; \textit{Heidel 2300} (RM, NY).
Beaver Rim, ca. 6 air mi. NW of Sweetwater Station. 6,680' elevation;
T30N R96W S2; 20 Jul 1986; \textit{Haines 6926} (RM).
Beaver Hill, about 35 mi. SE of Lander along Route 287. 6,400' elevation;
T30N R96W S11; 8 Jun 1960; \textit{Wetherell 256} (RM).
Top of Beaver Rim, Devil's Gap, ca. 7 air mi. NNW of Sweetwater Station.
6,800' elevation; T31N R96W S25; 14 Jun 1986;
\textit{Haines 6429 and 6387} (RM).
West Wind River Basin, Beaver Rim; ridges in vicinity of Devil's Gap,
ca. 1.5 air mi. W of Dishpan Butte. 6,900-7,000' elevation; T31N R95W S30 NW1/4;
13 Jun 1994; \textit{Fertig 14848} (RM).
Sweetwater River Plateau, Beaver Divide, SE end of Dishpan Butte, 1.25 air mi.
W of junction of Dishpan Butte Road and WY HWY 135. 6,820-6,880' elevation;
T31N R95W S29 NE1/4 of SE1/4 of NW1/4; 9 Jul 1997;
\textit{Fertig and Welp 17668} (RM).
SW edge of Cedar Rim, ca. 0.7 mi. E of Route 135, 28 air mi. SSE of Riverton.
T31N R95W S27 SW1/4 and S34 NW1/4; 11 Jun 1993; \textit{Anderson 14371} (NY).
6 mi. N of Sweetwater Station. 6,700' elevation; T31N R95W S27 SW1/4 and S34
NW1/4; 1 Jul 1991; \textit{Dorn 5246} (NY, MO).
Beaver Rim, ca. 9 mi. N of Sweetwater Station. 6,800' elevation; T31N R95W S9;
27 Jun 1981; \textit{Dueholm 11670} (RM).
Ca. 3 air mi. SSW of Big Sand Draw oil and gas field. 6,000' elevation;
T32N R95W S28; 3 Jul 1981; \textit{Hartman 13510} (RM).
Beaver Rim. 7,100' elevation; T32N R95W S26 S1/2 S1/2 NE1/4; 2 Jul 1991;
\textit{Dorn 5252} (RM).
23.75 road mi. S from HWY 136 junction with HWY 135, ca. 200 yards W of HWY 135.
6,819' elevation; 11 Jun 2014; 42º41.5022'N, 108º10.7641'W;
\textit{Ratcliff and O'Kane, Jr. 39} (ISTC).
Beaver Rim on the Road from Jeffrey City to Bairoil. Dry dismal area.
6,800' elevation; T31N R91W S3; 10 Jun 1974; \textit{Hill and Knight 1521} (RM).
Sweetwater River Plateau, Wind River Basin; toe of Beaver Rim, near head of
tributary draw of Coyote Creek, ca. 1.5 air mi. S of Wild Horse Springs and ca.
2.5 mi. ENE of Mud Springs. 6,900' elevation; T32N R90W S30 SW1/4 NE1/4;
11 Jul 1997; \textit{Fertig and Welp 17686} (RM).
Badlands on north toe of Beaver Divide, at head of Willow Springs Draw, ca. 4
mi. S of the Lucky Mac Uranium Mill. 7,000-7,100' elevation; T32N R90W S9 SW1/4
SE1/4; 12 Jul 1997; \textit{Fertig and Welp 17689} (RM).

% NORTH CENTRAL WYOMING PHYSARIA %%%
\section*{\textit{Physaria didymocarpa} subsp. \textit{lanata}}

  \textbf{Montana:}
% Big Horn
  \textbf{Big Horn County:}
Ca. 8 mi. NNW of Decker, near Route 314. 1,125-1,245m elevation;
T8S R39E S14 and S22; 1 May 1993; \textit{Prodgers 2752} (MONT).
  \textbf{Wyoming:}
% Big Horn
  \textbf{Big Horn County:}
Big Horn Mountains. Aug 1889; \textit{Unkown Collector s.n.} (MO-3930874).
Unknown location. \textit{Worthley 145} (RM).
Bighorn Range, W slope of Duncum Mountain, E of FS RD 11, ca. 3.75 mi. N of
US HWY Alt 14A and 10 mi. S of the Montana state line. 9,650' elevation;
T56N R91W S6 NE1/4 SW1/4 NE1/4; 28 Jul 2001;
\textit{Fertig 19779} (NY, UTC).
Duncum Mountain, ca. 27 air mi. E of Lovell, ca. 17 air mi. NW of Burgess
Junction. 9,600' elevation; T56N R91W S6; 11 Jul 1980; \textit{Nelson 6203} (NY).
On the slope of Duncum Mountain directly north of FS RD 11, ca. 3 air mi. N of 
Porcupine Ranger Station. 9,680' elevation; T56N R91W S6 NW1/4; 8 Aug 1999; 
\textit{O'Dea 32} (RM).
Rocky (calcareous) hillsides and summit of Medicine Mountain.
9,500-10,000' elevation; T56N R92W S22; 3 Jul 1936;
\textit{Williams and Williams 3228} (RM, NY, MO).
Medicine Mountain. 14 Aug 1948; \textit{Ownsbey 3161} (NY, CAS).
South side of ridge between Medicine Wheel and summit of Medicine Mountain.
9,450' elevation; T56N R92W S22; 8 Aug 1977; \textit{Johnston 1463} (RM).
Dry, open, south-facing limestone slopes of Medicine Mountain; off U.S. Route
14a 18 mi. W of Burgess Junction. 9,800' elevation; 13 Jul 1965;
\textit{Stolze 737} (F).
% Sheridan
  \textbf{Sheridan County:}
Whitney Coal Site. 3,540-3,790' elevation; T58N R83W S32 NW1/4 SW1/4;
2 May 1977; \textit{Mayer 937} (RM).
Eastern slope of the Big Horn Mountains, about 6 mi. W of Dayton.
6,500' elevation; 11 Jul 1957; \textit{Rollins 57180} (NY, GH, MO, US).
Ridge system on N side of Smith Creek ca. 7.3 air mi. WNW of Dayton.
5,000' elevation; T57N R87W S21 W1/8 and S20 E1/8; 3 Jun 1994;
\textit{Fertig and Britt 14773} (RM).
4 mi. W of Dayton. 4,800' elevation; T57N R87W S34; 23 May 1982;
\textit{Lichvar 4767} (RM).
Wildlife refuge, ca. 3 air mi. W of Dayton. 4,600' elevation; T57N R87W S35;
30 Jun 1979; \textit{Hartman 9838} (RM, UC).
East face of Big Horn Mtns., above Dayton. 17 Jun 1946;
\textit{Ripley and Barneby 8025} (CAS).
Hillside, HWY 14, 0.1 mi W of milepost 72 (15.2 mi. E of alt 14 junction).
2,071m elevation; 24 Jun 1996; \textit{Salywon and Dierig 3140} (ISTC).
7.5 mi. SW of Dayton. 5,750' elevation; 9 Jul 1965;
\textit{Mulligan and Crompton 3086} (NY).
East slope Bighorn Mountains; Little Rapid Creek, ca. 6.5 air mi. S of Beckton.
5,500' elevation; T54N R85W S18 SW1/4 NE1/4; 19 Jul 1994;
\textit{Fertig and Britt 15082} (RM).
% Johnson
  \textbf{Johnson County:}
Rolling plains between Sheridan and Buffalo. 6,000' elevation; 15 Jun 1900;
\textit{Tweedy 3585} (NY).
Bud Love Big Game Winter Range, ca. 8.5 (air) mi. WNW Buffalo. 5,600' elevation;
T51N R83W S4, S5, S8, and S9; 21 Jul 1979; \textit{Dueholm 8334} (RM).
Mosier Gulch on N side of US HWY 16 ca. 1.1 air mi. E of Bighorn NF boundary,
ca. 7 air mi. W of Buffalo. 6,400' elevation; T50N R83W S4 NE1/4 SE1/4;
16 Jul 1999; \textit{Fertig 18795} (RM).
Off U.S. 16 along Clear Creek, ca. 6.5 air mi. WSW of Buffalo.
5,600' elevation; T50N R83W S3; 16 Jun 1980; \textit{Nelson 5572} (RM).
Along N side of Clear Creek ca. 5.5 mi. W of Buffalo just S of HWY 16.
5,400' elevation; T50N R83W S1; 29 May 1989; \textit{Evert 16289} (RM).
Large roadcut north of Clear Creek, 6.1 mi. W of Buffalo. West of city limit,
on HWY 6. 6,343' elevation; 17 Jun 2014; 44º19.84'N, 106º50.75'W;
\textit{Ratcliff and O'Kane, Jr. 52} (ISTC).
7 mi. W of Buffalo. 6,250' elevation; 9 Jul 1965;
\textit{Mulligan and Mosquin 3080} (UC).
Hillside, 6.5 mi. W of Buffalo on HWY 16. 1,926m elevation; 44º19'44"N,
106º50'33"W; 24 Jun 1996; \textit{Salywon and Dierig 3138} (ISTC).
Hillside, 0.8 mi. S on Moiser Gultch RD, 5 mi. W of Buffalo. 1,717m elevation;
44º19'32"N, 106º48'49"W; 24 Jun 1996; \textit{Salywon and Dierig 3136} (ISTC).
On cut banks along HWY 16 about 10 mi. W of Buffalo, eastern slope of the
Big Horn Mtns. 7,000' elevation; T50N R83W S7; 29 Jun 1953;
\textit{Porter 6255} (RM, MO).
North Fork Crazy Woman Creek, ca. 13 air mi. SW of Buffalo. 6,600' elevation;
T49N R83W S22 and 27; 28 Jun 1979; \textit{Hartman 9664} (RM, GH, MONTU).
Ca. 22 (air) mi. NNE of Sussex. 4,400' elevation; T46N R77W S10; 27 Jun 1978;
\textit{Hartman and Sanguinetti 7496} (RM).
% Campbell
  \textbf{Campbell County:}
Road ca. 7 mi. S of HWY 16 and 1 mi. E of Rozet. 4,800' elevation;
T49N R69W S31; 22 May 1978; \textit{Hartman and Dueholm 5291} (RM).
Ca. 5.5 mi. N of Reno Junction and 1 mi. E of WY HWY 87. 4,500' elevation;
T45N R71W S30; 13 Jul 1973; \textit{Pierce 1971} (RM).
Ca. 3 air mi. W of Hilight. 5,010' elevation; T45N R71W S19 and S30;
13 Jul 1978; \textit{Dueholm and Sanguinetti 4098} (RM).
Pumpkin Buttes, North Butte. 6,049' elevation; T44N R76W S11, S13, S14, S15,
and S16; 28 May 1978; \textit{Hartman and Dueholm 5671} (RM).

% IDAHO PHYSARIA %%%
\section*{\textit{Physaria didymocarpa} subsp. \textit{lyrata}}

  \textbf{Idaho:}
% Lemhi
  \textbf{Lemhi County:}
4.0 road mi. W of Williams Creek RD from HWY 93, Southwest of Salmon.
4,535' elevation; 45º4.9216'N, 113º57.7358"W; 14 Jun 2014;
\textit{Ratcliff and O'Kane, Jr. 45} (ISTC).
Salmon National Forest; between Mud Spring and Atchinson Spring.
6,800' elevation; 5 Jun 1930; \textit{Phillips 50} (RM).
10 mi. W of Salmon, Idaho on road to Leesburg. 13 Jul 1945;
\textit{Christ and Ward 14727} (NY).
7 mi. SW of Salmon. 4,400' elevation; 23 Jun 1965;
\textit{Mulligan and Crompton 2962} (NY, GH).
In talus along E slope of Salmon Mtns. near mouth of North Fork Williams Creek.
1 Jul 1945; \textit{Hitchcock and Muhlick 14305} (RM, MO, CAS, UC).
Williams Creek, 10 mi. SE of Salmon, Idaho. 4,050' elevation; 10 Jul 1982;
\textit{Hawiger 4} (MONTU).
Approximately 4.5 mi. up Williams Creek road in talus slopes and road cuts on
right hand side of road. 4,900' elevation; 25 May 1988;
\textit{Rittenhouse 102} (IDS).
Williams Creek, 10 mi. SE of Salmon. 4,050' elevation; T20N R21E S10;
10 May 1982; \textit{Rosentreter 2683} (MO, NY, IDS-77676).
Williams Creek, 10 mi. SE of Salmon. 4,050' elevation; T20N R21E S10;
22 Aug 1982; \textit{Rosentreter 2881} (MONTU, NY).
Upper Basin Creek, SW of Tendoy, Idaho. 5,000' elevation; T19N R23E S31 and
NW1/4 of NE1/4 S32; 21 May 1982; \textit{Rosentreter 2696} (MONTU, NY, NY).
Pattee Creek, 6 km NE of Tendoy. 5,200' elevation; T19N R24E S11 S1/2;
7 Jul 1986; \textit{Rosentreter 3855} (NY).
Agency Creek. 5,000' elevation; T19N R25E S18 SE1/4;
20 May 1982; \textit{Rosentreter 2689} (NY, MONTU, NY).
Agency Creek. 5,000' elevation; T19N R25E S18 SE1/4; 17 Jun 1982;
\textit{Rosentreter 2727} (MONTU).
West Fork Little Eightmile Creek talus, 1/2 mi. Maryland Mine, 10 mi. N of
Leadore in Lunistene talus. 7,200' elevation; T17N R25E S23 N1/2 NE1/4;
27 Jun 1990; \textit{Atwood and Moseley 13891} (UC).
Beaverhead Mountains, talus slope on the N side of Mahogany Canyon.
7,300' elevation; 1 Jul 1976; \textit{Henderson 3196} (NY).
Targhee National Forest, Beaverhead Range; above South Fork of Worthing Canyon,
ca. 10.5 air mi. SE of Nicholia. 8,000-8,600' elevation; T11N R30E S34;
18 Jun 1992; \textit{Markow 7784} (RM).
North side of Meadow Canyon, ca. 1 mi. W of junction. 16 Jul 1975;
\textit{Henderson 2641} (NY).
Lemhi Range, Peak 10,652 just N of Meadow Canyon. 24 Jul 1975;
\textit{Henderson 2867} (NY).
% Butte
  \textbf{Butte County:}
Cirque below N face of Diamond Peak, Lemhi Range, Targhee NF; T10N R28E
unsurveyed sections, ca. 9 mi. W of Blue Dome. 9,400' elevation; 9 Jul 1989;
\textit{Moseley 1508} (NY).
Lemhi Range, Targhee National Forest, 0.5 mi. NE of Pek 10,020, on divide
between Rocky Can and Sawmill Can, ca. 12 mi. NW of Blue Dome. 9,600' elevation;
T10N R28E S11 NE1/4; 22 Jul 1984; \textit{Moseley 420} (RM, NY).
% Clark
  \textbf{Clark County:}
Targhee National Forest, along and near top of ridge. 8,100' elevation;
T13N R34E S9; 10 Jul 1926; \textit{Pickett 303(P-56)} (RM).
Targhee National Forest, Beaverhead Range; slope W of Gallagher Peak, ca. 30
air mi. W of Dubois. 9,500-9,700' elevation; T10N R31E S27; 4 Aug 1992;
\textit{Markow 10270} (RM).
  \textbf{Montana:}
% Beaverhead
  \textbf{Beaverhead County:}
Tendoy Mountains, Bell Canyon; ca. 4 air mi. SW of Red Rock. 7,000’ elevation;
T11S R11W S24; 26 Jun 1993; \textit{Vanderhorst 4983} (MONT).
Tendoy Mtns., locally common in barren gravelly soil on a steep, west-facing
slope N of Poison Lakes. 8,000’ elevation; T11S R11W S27; 5 Jul 2002;
\textit{Lesica 8475} (NY, MONTU).
Dewey, Montana. 7,000' elevation; 24 Jun 1902; Blankinship s.n. (MONT-9352).
Tendoy Mtns., common in stony limestone soils on a steep southwest-facing slope
on the ridge W of Pileup canyon. 7,600' elevation; T14S R10W S32; 10 Jul 1993;
\textit{Lesica 6073} (NY, MONTU).
Tendoy Mtns., uncommon in limestone talus at the base of a steep south-facing
slope above Big Sheep Creek ca. 12 mi. SW of Lima. 6,800' elevation;
T15S R10W S4 SE1/4; 13 Jul 1985; \textit{Lesica 3522} (MONTU).
Mt. Lima, Montana. 30 Jun 1895; \textit{Shear 3406} (NY).
Lima Peaks; locally common in gravelly, shale-derived soil on Ridge 9,204 ca.
1.5 mi. SW of Garfield Mtn., ca. 8 mi. S of Lima. 9,200' elevation;
T15S R8W S21 SW1/4; 8 Jul 1988; \textit{Lesica 4640} (MONTU).
Beaverhead Mtns., common in shallow calcareous soil on a steep, W-facing slope
of the spur ridge of the unnamed peak 2 mi. E of Red Conglomerate Peak.
8,400’ elevation; T15S R8W S36; 8 Jul 1986; \textit{Lesica 3938} (GH).
Abundant in sandy, eroding soil on an east-facing slope above the East Fork
Peet Creek. 6,850' elevation; T14S R4W S27; 7 Jul 2005;
\textit{Lesica 9393} (NY, MONTU).
Centennial Valley-Sheep Mountain, approx. 1.5 air mi. SE of Lakeview.
2,700m elevation; T14S R1W S28 E2; 20 Jun 1993;
\textit{Culver and Lavin 254} (MONT-72332, MONT-72333).

