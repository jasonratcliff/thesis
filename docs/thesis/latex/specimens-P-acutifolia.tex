\section*{\textit{Physaria acutifolia}}

%%% UNKNOWN LOCALITIES
  \textbf{Unknown State:}
  \textbf{Unknown County:}
Fremont's (2nd) Expedition to California. 1843-1844;
\textit{Unknown Collector s.n.} (NY, NY, NY).
%%% MONTANA STATE SPECIMENS
  \textbf{Montana:}
  \textbf{Prairie County:}
Head of Powder River. Sep 1859; \textit{Hayden s.n.} (MO-3093900).
% Petroleum
  \textbf{Petroleum County:}
Approx. 13 mi. N and 7 mi. NE of Winnett. 3,000' elevation; T18N R27E S33 NW1/4;
10 Jul 1999; \textit{Heidel 1896} (ISTC).
% Garfield
  \textbf{Garfield County:}
9 mi. E. Mosby; HWY 18. 17 May 1952; \textit{Wright and Wright 102} (RM).
Hell Creek State Park; One colony on dry south slope of high ridge above meadow
with camping facilities. 10 Jun 1978; \textit{Lackschewitz 8092} (MONTU, MONT).
% Musselshell
  \textbf{Musselshell County:}
Roundup, 10 mi. NW. 18 Jun 1957; \textit{Booth 57901} (RM).
% Treasure
  \textbf{Treasure County:}
Divide between Bighorn and Hysham; HWY 10. 24 May 1952;
\textit{Wright and Wright s.n.} (MONT-44019, MONT-44025).
% Rosebud
  \textbf{Rosebud County:}
Forsyth. 24 Jul 1901; \textit{Blankinship s.n.} (MONT-9307).
Fort Union Basin. T2N R41E S20; 23 May 1975; \textit{Lackschewitz 5999} (MONTU).
Custer NF, Ashland District, East Fork (off O'Dell Cr.) ca. 1.5 air mi.
E of O'Dell RD. 3,750' elevation; T5S R44E S10 NE1/4 NE1/4; 13 Jun 1995;
\textit{Marriott 11510} (MONT).
Zook Creek Wilderness Study Area. 3,375' elevation; T6S R42E S2 NW1/4;
20 Jun 2001; \textit{Taylor 8669} (MONTU).
% Custer
  \textbf{Custer County:}
Miles City. 2,500' elevation; 16 Jun 1936; \textit{Roberts 257} (MONT).
% Powder River
  \textbf{Powder River County:}
Top of the hills at the head of Wilbur Creek. 3,800' elevation; T3S R46E S1;
20 Jun 2011; \textit{Lesica 10596} (RM).
On the north side of Cow Creek 1 mi. above Otter Creek. 3,700' elevation;
T6S R45E S25; 8 May 2005; \textit{Lesica 9029} (NY, MONTU).
0.5 mi. SW of Fort Howed Work Center. 3,500' elevation; T6S R45E S25 NE1/4;
23 Jun 1999; \textit{Heidel 1845} (ISTC).
Spear Hills, over 3 mi. S of Moorhead. 3,760' elevation; T9S R48E S30 SE1/4;
30 May 1999; \textit{Heidel 1811} (MONTU).
% Yellowstone
  \textbf{Yellowstone County:}
7 mi. S of Custer Station. 6 Jul 1890; \textit{Blankinship 117} (MONT, MO).
Billings vicinity. 19 Jun 2002; \textit{Clawson s.n.} (MONT-77887).
Grassland 5-10 mi. S.E. on HWY 87 toward Hardin from Billings, Montana.
14 May 1941; \textit{White s.n.} (MONT-54041).
Growing 14 mi. N of Billings along Montana 3. 9 Jun 1966;
\textit{Davidse and Collotzi 657} (NY).
5.6 mi S. Billings on \# 87. 20 May 1955; \textit{Scharff s.n.} (MONT-49106).
Billings, 5 mi. SE on \# 87. 2 Jun 1955; \textit{Scharff s.n.} (MONT- 49613).
% Carbon
  \textbf{Carbon County:}
Beartooth Forest. 5,500' elevation; T7S R18E S1; 3 Jul 1928;
\textit{Coster 10} (RM).
Just north of N. Fk. Dry Creek on north-south trending ridge with sandstone
outcroppings ca. 8 mi. NW of Belfry. 4,060-4,280' elevation; T7S R22E S7 SW1/4;
5 Jun 1994; \textit{Evert 26694} (RM).
SE of Belfry. T9S R23E; 15 Jun 1938; \textit{Shunk s.n.} (MONT-53767).
Ridge 1 mi. W of Belfry and 1/4 mi. N of HWY 308. 4,100-4,200' elevation;
T8S R22E S16 and S9; 6 Jun 1996; \textit{Evert 30967} (RM).
Bighorn Basin; Hollenbeck Road E of MT HWY 72 and ca. 0.5-1 mi. S of Clarks
Fork Yellowstone River. 4,000-4,300' elevation; T9S R22E S9 E1/2 and S16 NE1/4;
22 May 2009; \textit{Hartman 88777} (MONTU).
Hollenbeck RD E of MT HWY 72 and S of Clarks Fork Yellowstone River.
4,000-4,300' elevation; T9S R22E S4 SE1/4 and S9 E1/2; 22 May 2009;
\textit{Hartman 88766} (NY).
East side of Long Draw. 4,100' elevation; T8S R23E S31; 20 May 1991;
\textit{Lesica 5294} (NY, MONTU).
Sage Creek RD (Pryor Mountain RD), ca. 7 air mi. SE of Bridger; South Fork
Bridger Creek, road to Depression Reservoir and to top of hogback to SW.
4,120-4,450' elevation; T7S R24E S32 N1/4; 23 May 2009;
\textit{Hartman 88842} (NY, MONTU).
Along Sage Creek RD, ca. 9.5 air mi. SE of Bridger. 4,540-4,740' elevation;
T7S R24E S21 SW1/4; 19 Jun 2008; \textit{Nelson 74468} (ISTC).
Bluewater Creek ca. 6 mi. E of Bridger. 4,200' elevation; T6S R24E S15;
14 Jun 1984; \textit{Lesica 3002} (MONTU).
South-facing slopes above Gypsum Creek ca. 1 mi. NW of Gypsum Springs.
4,700' elevation; T9S R27E S32; 21 May 1991; \textit{Lesica 5304} (NY, MONTU).
Southern end of Pryor Mountain Range, ca. 1 mi. N of Gypsum Springs RD. 12.7 air
mi. N of Lovell, WY. 4,773' elevation; 1,454.81m elevation;
45º01'00"N, 108º27'00"W; 20 Jun 2004; \textit{Grady 67} (ISTC).
Red Desert area, 1 mi. NW of Gypsum Springs, Red Pryor Mtn. 1,643m elevation;
T9S R27E S29 SW 1/4; 5 Jun 1994; \textit{McCarthy 58} (MONT).
Tan-purple ridge E of Helt and Crooked Creek RD intersection, Red Pryor Mtn.
Quad. 1,565m elevation; 12 May 1992; \textit{Jacobsen 206} (MONT).
Along Crooked Creek RD ca. 13 mi. N of Lovell, WY. 4,300' elevation;
T9S R27E S33; 16 Jun 1990; \textit{Evert 18942} (RM).
Ridge east of Helt RD and Crooked Creek RD intersection. 4,716' elevation;
45º0.5243'N, 108º25.4114W; 15 Jun 2014;
\textit{Ratcliff and O'Kane, Jr. 48} (ISTC).
Slopes ca. 1 mi. N of Gypsym Springs. 4,800' elevation; T9S R27E S28 SW1/4;
10 Jun 1991; \textit{Lesica 5366} (MONTU).
Spring, NE 4.8 km on road, then SE 0.8 km on jeep road into canyon
below, Red Pryor Mtn. 1,368m elevation; T9S R27E S27 NW 1/4; 19 Jun 1994;
\textit{McCarthy 226} (MONT).
Southern footslopes of Pryor Mountains along RD 3085 west of Penney Peak.
1,524m elevation; 45º03'22"N, 108º24'55"W; 1 Jul 1998;
\textit{O'Kane, Jr. and Prather-O'Kane 4510} (ISTC, MO).
Crooked Creek RD (BLM RD 1017) ca. 0.5 air mi. W of Demijohn Flat.
5,000-5,320' elevation; T9S R27E S9 SE1/4 and S10 SW1/4; 25 May 2009;
\textit{Hartman 88978} (MONTU, NY).
Near Gypsum Creek. 5,400' elevation; 1,645m elevation; T9S R27E; 15 may 1976;
\textit{Dorn 2547} (MONT).
Hills S of Pryor Mtns., dry plains between scattered pines. 5,100' elevation;
14 Jun 2005; \textit{Hjalmarsson 5105} (MONT).
Wassin Canyon. 4,600' elevation; T8S R28E S24; 10 May 1983;
\textit{Lichvar 5586} (RM).
Medicine Creek Campground. 3,700' elevation; T8S R29E S16; 13 Jun 1983;
\textit{Lichvar 6093} (RM).
Barry's Landing Campground. 19 Jun 1982; \textit{Thompson 2300} (MONTU).
Big Horn Canyon. 3,500' elevation; 1,070m elevation; T8S R29E S31; 15 May 1976;
\textit{Dorn 2559} (MONT).
Bighorn Canyon RD 16 mi. N of Lovell, WY. 19 Jun 1982;
\textit{Thompson 2282} (MONTU).
Common on rocky limestone ridge tops on the east side of Crooked Creek
ca. 10 mi. NW of Lovell, WY. 4,500' elevation; T9S R28E S36; 18 Jun 1983;
\textit{Lesica 2607} (MONTU).
West of Yellow Hill. 4,800' elevation; T9S R28E S27; 14 May 1983;
\textit{Lichvar 5636} (RM).
Pryor Mountain Wild Horse Range; Sykes Ridge (BLM RD 1019) along spine E of
Big Coulee. 4,650-4,950' elevation; T9S R28E S28 and S21 SE1/4; 1 Jun 2009;
\textit{Hartman 89242} (MONTU).
Pryor Mountain National Wild Horse Range; on Sykes Ridge or BLM RD 1019 at
the state line, ca. 37 air mi. SE of Bridger. 4,640-4,720' elevation;
T9S R28E S33 NE1/4; 16 Jun 2008; \textit{Nelson 74315} (ISTC).
%%% IDAHO STATE SPECIMENS
  \textbf{Idaho:}
% Bannock
  \textbf{Bannock County:}
Soda Springs. 5,700' elevation; 18 Jun 1920;
\textit{Payson and Payson 1701} (RM, NY, MO).
Soda Springs. \textit{Christ 3400} (NY).
% Caribou
  \textbf{Caribou County:}
Crow Creek RD, 11.6 km (7.2 mi.) SW of the Wyoming border, 22.5 km (14 mi.)
air distance SW of downtown Afton. 6,560' elevation; 2,000m elevation;
42º35'10"N, 111º07'56"W; 23 Jun 2003;
\textit{Holmgren and Holmgren 14877} (UTC, NY, ISTC).
% Bear Lake
  \textbf{Bear Lake County:}
Georgetown Summit. 6,232' elevation; 11 May 1978;
\textit{Shultz and Shultz 2434} (UTC).
Preuss Range, Georgetown Canyon, 3.7 mi. E of Georgetown
(junction with U.S. HWY 30N) 6,400' elevation; T11S R44E S4; 23 May 1971;
\textit{Holmgren and Holmgren 4782} (NY).
Near U.S. HWY 89, 3.8 mi. NE of Montpelier. 15 Jun 1981;
\textit{Rollins and Rollins 81317} (RM, NY, UC, F).
4 mi. E of Montpelier on Montpelier Canyon RD, just E of Home Canyon.
6,300' elevation; 14 Jul 1978; \textit{Shultz and Shultz 2765} (RM, UTC, IDS).
1/2 mi. W of Geneva Summit, off U.S. HWY 89. 24 Jun 1990;
\textit{Rollins and Rollins 8692} (RM, UTC, GH, NY).
One mi. SW of Geneva Summit. 8,000' elevation; 15 Jun 1981;
\textit{Rollins and Rollins 81319} (RM, NY, UC, US).
5.4 mi. N of Geneva. 15 Jun 1981;
\textit{Rollins and Rollins 81321} (RM, NY, US, UC).
Cache National Forest; B. M. Fish Haven Ridge. 8,000' elevation; T16S R43E S5;
26 May 1928; \textit{Averill A-8} (RM).
Top of hills east of Montpelier. 28 Jun 1930;
\textit{Davis s.n.} (IDS-30405, IDS-30406).
Between Geneva and Montpelier. 11 May 1987; \textit{Christ and Ward 7127} (NY).
8.5 mi. E of Montpelier. 6,800' elevation; 11 Jul 1965;
\textit{Mulligan and Crompton 3092} (NY).
% Oneida
  \textbf{Oneida County:}
Two-Mile Canyon. T14S R37E S32 SW1/4 SW1/4; 21 Jun 1992;
\textit{John 771} (UTC).
%%% WYOMING STATE SPECIMENS
  \textbf{Wyoming:}
  \textbf{Unknown County:}
Sagebrush slope, Flora of Northwestern Wyoming. 5 Jul 1940;
\textit{Wehmeyer, Martin Jr., and Loveland 5314} (NY, MO).
20 mi. E of Green River. \textit{McCosh s.n.} (NY).
Green River Mountains. \textit{Sheppard 26} (MO-1076929).
% Park
  \textbf{Park County:}
Bighorn Basin; on road between Silvertip and Elk Basin Oil Fields, SSE of Elk
Basin Oil Field; ca. 14.5 air mi. NW of Powell. 4,600' elevation;
T57N R100W S1; 21 Jun 1987; \textit{Nelson	13688} (NY).
Bighorn Basin; near the head of Spring Creek, ca. 6 air mi. S of Cody.
5,500' elevation; T52N R101W S31; 26 Jun 1983; \textit{Nelson 9944} (ISTC).
Bighorn Basin; north end of Oregon Basin, ca. 8 air mi. SE of Cody.
5,300-5,600' elevation; T52N R100W S8, S16, and S17; 25 May 1983;
\textit{Hartman and Hamann 14449} (ISTC).
Along W side of Oregon Basin RD, ca. 1.5 mi. S of HWY 14, 16, 20.
5,250' elevation; T52N R100W S15 and S16;	13 Jun 1990; \textit{Evert 18832} (RM).
Bighorn Basin; Burlington Meeteetse RD, 16.5 mi. NE of intersection with
WY HWY 120. T51N R98W S14, S15; 24 May 1980;
\textit{Hartman with Dueholm 11146} (ISTC, UTC).
Bighorn Basin; on the divide between Gooseberry Creek and Renner Draw, ca.
1.5 air mi. S of Gooseberry Creek; ca. 11 air mi. S of Meeteetse.
6,400' elevation; T47N R100W S33; 27 Jun 1983; \textit{Nelson 9999} (ISTC).
% Big Horn
  \textbf{Big Horn County:}
Bighorn Basin, above Dry Creek, ca. 6 air mi. NNE of Cowley, ca. 8.5 air mi. NNW
of Lovell. 4,300' elevation; T58N R96W S34; 28 Jun 1987;
\textit{Nelson 14016} (UTC).
Sykes Mtn. 4,100' elevation; T57N R95W S13; 14 May 1983;
\textit{Lichvar 5619} (RM).
Sykes Mtn. 4,000' elevation; T57N R95W S13; 10 Jun 1983;
\textit{Lichvar 6037} (RM).
Big Horn Mountains; John Blue Canyon RD, ca. 5.9 mi. NNE of U.S. 14A.
6,000' elevation; T57N R94W S12 and S13; 28 May 1980;
\textit{Hartman 11367} (RM, GH).
5.5 mi. NE of Kane. 13 Jun 1964; \textit{Wight 68} (RM).
Big Horn Mountains; Cottonwood Canyon, ca. 17 air mi. E of Lovell.
5,400' elevation; T56N R93W S4; 10 Jun 1980; \textit{Nelson 5367} (RM).
Bighorn Basin, NW end of Sheep Mountain, near Ribbon Canyon. 4,200' elevation;
T54N R94W S20; 28 May 1980; \textit{Hartman and Dueholm 11349} (UTC).
3-4 air mi. N of Greybull. T53N R93W S22; 22 May 1983;
\textit{Hartman and Hamann 14275} (NY).
Red Gluch, south of Trapper Canyon. 7 May 1926; \textit{Finley 10} (RM).
West flank of Bighorn Mountains; west side of Medicine Lodge Creek Canyon, ca.
7 air mi. NE of Hyattville. 5,200' elevation; T51N R89W S10 NE1/4 NW1/4;
22 Jun 1989; \textit{Marriott 11018} (RM).
Mouth of Dry Medicine Lodge Canyon (S9 and S16) and ridge above Medicine Lodge
Canyon (S10 and S15). 5,000' elevation; T50N R89W S9, S16, S10, and S15;
26 May 1980; \textit{Dueholm and Hartman 9487} (RM).
4 mi. SE of Hyattsville. 5,200' elevation; T49N R89W S22 S1/2; 21 May 1979;
\textit{Lichvar 1660} (RM).
% Wakashie
  \textbf{Wakashie County:}
Steep slope of red soil, near WY HWY 436, ca. 5 mi. SE of Ten Sleep.
23 Jun 1981; \textit{Rollins and Rollins 81392} (NY).
Big Cedar Ridge, W of BLM RD 1411, ca. 24 air mi. SE of Worland.
5,300' elevation; T45N R89W S28; 9 Jul 2011; \textit{Heidel 3540b} (RM).
9 mi. WSW of Neiber. 4,500' elevation; T45N R95W S3 SW1/4; 6 Jun 1995;
\textit{Dorn 5924} (RM).
% Hot Springs
  \textbf{Hot Springs County:}
Wind River Valley. 29 May 1860; \textit{Hayden s.n.} (MO-1923237).
Gravelly hills in Wind River Valley. 5,500' elevation; 18 May 1860;
\textit{Hayden s.n.} (MO-137159).
Near Long Creek Tavern, HWY 287 approaching Wind River Valley from NW.
14 Jul 1942; \textit{Cantelow s.n.} (CAS).
Bighorn Basin; west side of Hillberry Rim, ca. 0.8 mi. E of WY HWY 120.
5,600' elevation; T47N R99W S13 SW1/4 NE1/4; 1 Jul 1998; \textit{Welp 7860} (RM).
Ridge along north side of Grass Creek RD, ca. 6 mi. W of WY HWY 120.
5,800' elevation; T46N R99W S14; 27 Jun 1993; \textit{Evert 25349} (RM).
Along east side of Murphy Draw RD, ca. 3/4 mi. N of WY HWY 431.
5,200' elevation; T47N R97W S28; 11 Jun 1995; \textit{Evert 28855} (RM).
Off WY HWY 120 on Red Ridge, ca. 5.5 air mi. E of Grass Creek Post Office;
ca. 26.5 mi. NW of Thermopolis. 5,500' elevation; T46N R97W S19; 29 Jun 1983;
\textit{Nelson 10164} (NY).
Ca. 5 air mi. SE of Grass Creek; ca. 25 air mi. NW of Thermopolis.
5,500' elevation; T45N R98W S1; 3 Jun 1981; \textit{Nelson 7463} (ISTC).
North slope of Ilo Ridge, ca. 1 mi. S of WY HWY 171 and 1.2 mi.
S of Grass Creek.	5,500' elevation; T45N R98W S1 NW1/4 SW1/4 NW1/4,
T45N R98W S2 SW1/4 NE1/4; 29 Jun 1998; \textit{Welp 7851, 7855} (RM).
Ca. 11.5 air mi. SE of Grass Creek; ca. 19 air mi. NW of
Thermopolis on the Cottonwood Creek RD. 5,000' elevation; T45N R97W S14;
3 Jun 1981; \textit{Nelson 7505} (NY, UTC).
E Foothills Absaroka Mountains; Mount 7049, ca. 1.25 air mi. NW of summit of
Adam Weiss Peak, ca. 33.5 air mi. NW of Thermopolis. 6,600-6,800' elevation;
T45N R99W S6 SE1/4 NW1/4; 9 Jul 1992; \textit{Fertig 12949a} (RM).
S Bighorn Basin; ridge on western rim of Wagonhound Bench, ca. 27 air mi. NW of
Themopolis. 5,740-6,060' elevation; T44N R99W S1, S2, S3; 10 Jul 1992;
\textit{Fertig 12980} (RM).
Foothills of the Absaroka and Owl Creek Mountains; west end of Padlock Rim, ca.
3 air mi. SE of Hamilton Dome, ca. 17 air mi. WNW of Thermopolis.
5,300' elevation; T44N R97W S29, S31 NE1/4, and S32 NW1/4; 6 Jul 1992;
\textit{Fertig 12896} (RM).
S Bighorn Basin; Sand Draw Ridge, ca. 13.5 air mi. NNW of Thermopolis.
4,760-4,900' elevation; T44N R96W S3 SE1/4 and S10 NE1/4; 3 Jul 1992;
\textit{Fertig 12855} (RM).
Bighorn Basin; ca. 2 mi. W of Gebo (abandoned townsite), ca. 12 mi. N of
Thermopolis. 4,500' elevation; T44N R95W S9; 7 Jun 1995;
\textit{Evert 28732} (RM).
Ca. 1 mi. W of Gebo; ca. 10.5 air mi. NNW of Thermopolis. 4,500' elevation;
T44N R95W S10; 18 Jun 1987; \textit{Nelson 13540} (UTC).
Ca. 10 air mi. E of Thermopolis, chugwater cliffs and adjacent plains;
T42N R93W S3 and S10; 28 May 1981; \textit{Hartman and Dueholm 12821} (UTC).
Bridger Creek Road, 0.25 mi. N of Fremont County line. 5,740' elevation;
T41N R90W S34 SW, SE; 	28 Jun 1981; \textit{Martin 1575} (RM).
20 mi. SW of Grass Creek. 14 Jul 1964; \textit{Despain 28} (RM).
Red Canyon. 5,200' elevation; T42N R92W; 21 May 1979; \textit{Lichvar 1671} (NY).
Owl Creek Mountains; South Fork Owl Creek, north rim of canyon below Anchor Dam,
ca. 30 air mi. W of Thermopolis. 6,800' elevation; T43N R100W S26	SE1/4 NE1/4;
21 Jun 1991; \textit{Marriott 11356} (RM).
% Fremont
  \textbf{Fremont County:}
Wind River Indian Reservation, south flank of the Owl Creek Mountains.
6,400' elevation; T7N R1W S23; 24 Jun 1982; \textit{Lichvar 5177} (RM).
Boysen Dam, on the Wind River. 4,800' elevation; 26 Jun 1960;
\textit{Porter and Porter 8205} (RM).
Wind River Basin; slopes on west side of Boysen Reservoir, ca. 0.25 mi. W of
Power House at Boysen Dam. 5,200' elevation; T5N R6E S8	SE1/4 SW1/4 SE1/4;
12 Jun 1993; \textit{Fertig 13849} (RM).
Cedar Ridge and Barren Hills to the N. 5,600-6,000' elevation; T39N R92W S9,
S16, and S21; 28 Jun 1981; \textit{Hartman and Dueholm 13232} (RM).
Bridger Mountains; E end of Copper Mountain N of Point of Mountain RD, ca. 16
air mi. N of U.S. HWY 20-26. 6,200-6,600' elevation; T40N R91W S32 S1/2 NW1/4;
6 Jun 1996; \textit{Fertig 16544} (RM).
Cottonwood Creek, 7 air mi. NE of Lysite. 5,400-5,600' elevation; T39N R90W S14
and S15; 28 May 1985; \textit{Hartman 19963} (RM).
Lysite Badlands, Badlands draining north into Alkali Creek; ca. 2.5 air mi. W of
Lysite. 5,350' elevation; T38N R91W S15	N1/2; 17 Jun 1986;
\textit{Marriott 10133} (RM).
Pony CR \#1 gas line site just W of Moneta-Lysite RD, 3.3 road mi. N of Moneta.
5,800' elevation; T37N R91W S2	E1/2; 16 Jun 1986; \textit{Marriott 10089} (RM).
Edge of (badlands) Moneta Hills; ca. 3.75 air mi. NNE of Moneta.
5,630-5,730' elevation; T37N R91W S1; 20 May 1986; \textit{Haines 5939} (RM).
E of Moneta-Lysite RD ca. 5 air mi. ENE of Moneta, badlands and ridge to south.
5,650' elevation; T37N R90W S9	E1/2; 18 Jun 1986; \textit{Marriott 10170} (RM).
Burma RD, ca. 5.8 air mi. N of junction U.S. HWY 26 and WY HWY 789 in Riverton.
5,000-5,200' elevation; T2N R4E S34; \textit{Hartman and Haines 20054} (RM).
Flats, across U.S. HWY 26 from Paradise Valley Road, ca. 6 air mi. WNW of
Riverton. 5,300' elevation; T1N R3E S11 and S14; 15 Jun 1986;
\textit{Haines and Haines 6480} (RM, GH).
Steep slopes below sandstone ledges along the Oil Springs RD, 7 mi. N of HWY 130.
5,700' elevation; T34N R95W S30; 30 May 1985; \textit{Scott 4159} (RM).
Gas hills mining district; ca. 2.5 air mi. E of the intersection of WY 136 and
Ore RD (Main road through gas hills mining district). 6,480' elevation;
T33N R90W S24; 21 Jun 1985; \textit{Haines 4281} (RM).
On the eroded badlands hills near Lysite. 5,500' elevation; 10 Jul 1951;
\textit{Porter 5741} (RM).
% Natrona
  \textbf{Natrona County:}
SE Bighorn Mountains, north slope of Cedar Ridge, S of Badwater Creek, ca, 1.5-2
air mi. NW of Badwater. 6,200-6,700’ elevation; T39N R89W S24; 13 Jun 1993;
\textit{Fertig 13877} (RM).
South Fork of Sand Creek, ca. 7.25 air mi. E of Lost Cabin.
5,920-6,000’ elevation; T38N R89W S11 and S14; 1 Jun 1986;
\textit{Haines and Haines 6063} (RM, GH).
Cedar Ridge, ca. 13 air mi. NNW of Hiland. 6,100-6,825’ elevation;
T38N R88W S5 and S6, T39N R88W S31 and S32; 10 Jul 1985;
\textit{Haines 4792} (RM).
SE Bighorn Mountains, southeast slope of Cedar Ridge, ca. 2.5 air mi. S of
Badwater. 6,500-6,740’ elevation; T38N R88W S8 and S9; 13 Jun 1993;
\textit{Fertig 13861} (RM).
Southern Powder River Basin, ca. 6 air mi. NW of Arminto. 6,230’ elevation;
T38N R87W S19 SE4 and S30 NE4; 6 Jul 1993;
\textit{Hartman and Nelson 40363} (RM, MONTU).
Bad Water. 6 Jun 1910; \textit{Nelson 9403} (RM).
5 air mi. NNE of Arminto. 6,300’ elevation; T38N R87W S24; 12 Jul 1979;
\textit{Hartman 10048} (RM).
SE foothills Big Horn Mountains, draw due W of E-K Creek and ridge on east side
of creek, ca. 5.5 air mi. NNE of Arminto. 6,500’ elevation; T38N R87W S13 SE1/4;
8 Jun 1992; \textit{Fertig 12639} (RM).
E-K Creek, ca. 7.5 air mi. NNE of Arminto. 6,520’ elevation; T38N R86W S7;
1 Jun 1986; \textit{Haines and Haines 6012} (RM, GH).
Southern Powder River Basin, along Baker Cabin or County RD 108 between North
Fork Cottonwood Creek and Gray Wall, ca. 12 air mi. NE of Arminto, ca. 51 air
mi. NW of Casper. 6,340-6,490’ elevation; T39N R86W S25 W1/2; 23 May 1994;
\textit{Nelson 30628} (RM).
Ca. 24 air mi. N Powder River. 6,400’ elevation; T39N R85W S4 and S9;
3 Jul 1979; \textit{Dueholm 7727} (RM).
6,500' elevation; T40N R85W S11 S1/2; 43º27'N 107º01'W; 29 May 1999;
\textit{Dorn 7931} (RM).
NW Rattlesnake Hills, slopes adjacent to Poison Spider RD, ca. 1 mi. SE of
junction with Gas Hills RD. 6,800-7,000’ elevation; T34N R88W S32 E1/2;
15 Jun 1993; \textit{Fertig 13926} (RM).
Wallace Creek; 29 Jul 1894; \textit{Nelson 674} (NY).
Southern Powder River Basin; Blue Canyon at south end of Pine Mountain, ca. 28
air mi. WNW of Casper.	5,750-5,900' elevation; T34N R84W S13	N1/2 and
S24 NE1/4; 24 Jun 1993; \textit{Hartman 38283} (RM, UC).
Ca. 14.5 mi. SE Powder River. 5,700’ elevation; T33N R83W S33; 5 Jul 1979;
\textit{Dueholm 7850} (RM).
6,400; T30N R83W S19 N1/2; 20 Jun 1979; \textit{Dorn 3227} (RM).
33 mi. SW of Casper. 5,900’ elevation; 2 Jun 1963;
\textit{Mulligan and Mosquin 2791} (RSA-POM).
Satanka Formation, 4 mi. W of Alcova. 9 May 1948;
\textit{Porter 4427} (RM, RSA-POM).
Near Alcova, WY.  5,300’ elevation; 10 May 1980; \textit{Tresler 484} (RM).
North Platte River Basin, slopes on N side of Platte River, 0.5 mi. S of WY HWY
220, ca. 1.25 air mi. NE of Gray Reef Dam. 5,500’ elevation;
T30N R82W S8 NE1/4 SW1/4 and SE1/4 NW1/4; 5 Jun 1993;
\textit{Fertig 13791} (RM).
Slopes on west side of Platte River, ca. 21 air mi. SW of Casper.
5,400’ elevation; T31N R82W S23 W1/2; 5 Jun 1993; \textit{Fertig 13790} (RM).
Satanka formation, about 10 mi. SE of Alcova. 28 Jun 1950;
\textit{Porter 5400} (RM, NY).
One mi. N of Alcova. 21 May 1979; \textit{Rollins and Rollins 7919} (NY).
Near top of pass, between Alcova and Independence Rock, WY HWY 220.
21 May 1979; \textit{Rollins and Rollins 7920} (NY).
North of Casper Mtn., 20.5 mi. NE of Alcova on WY HWY 220. 21 May 1979;
\textit{Rollins 7914} (RM, NY, UC, US, MO).
13 mi. SW of Casper. 5,200’ elevation; 4 Jun 1963;
\textit{Mulligan and Mosquin 2792} (UC, UTC).
N. Laramie Range, south slope of Casper Mountain along ridge adjacent to
tributary of Little Red Creek and Casper Mountain Loop RD.
7,000-7,200’ elevation; T32N R79W S32; 9 Jul 1993; \textit{Fertig 14050} (RM).
Along road cut on road to Muddy Mountain. 7,000’ elevation; T31N R79W S4;
20 Jul 1987; \textit{Nelson and Nelson 8681} (RM).
29 mi. S of Casper. 8 Jul 1965; \textit{Mulligan and Crompton 3083} (RM).
Twin Buttes. 6,500’ elevation; T30N R80W S34 NW1/4; 9 Jun 1981;
\textit{Lichvar 4334} (RM).
Bates Hole. 20 Jun 1920; \textit{Payson and Payson 4781} (RM, MO).
North Platte River Basin; low ridges ca. 4.5 air mi. E of Alcova-Seminoe Scenic
Byway, ca. 1.25 air mi. N of Carbon County line. 7,000’ elevation;
T29N R82W S34 NE1/4; 11 Jul 1993; \textit{Fertig 14070} (RM).
East rim of Bates Hole, "Castle Rock" cliffs by Rimo Point ca. 2 mi. E of WY
HWY 487, ca. 3 mi. N of junction with HWY 77. 7,350’ elevation; T29N R79W S3
NW1/4 SW1/4; 21 Jun 1997; \textit{Fertig 17550} (RM).
West slope Laramie Range, divide between Mud Springs Draw and Chalk Creek,
ca. 1.2 mi. S of the confluence of Chalk and Bates Creeks, ca. 8.5 air mi. E of
WY HWY 487. 6,940-7,000' elevation; T29N R79W S3 NW1/4 SW1/4; 20 Jun 1997;
\textit{Fertig 17547} (RM).
Bates Creek. 5 Jul 1901; \textit{Goodding s.n.} (RM).
% Sublette
  \textbf{Sublette County:}
Wyoming Range; Ridge on E side of LaBarge Creek just N of confluence of
Packsaddle Creek, ca. 1 mi. E of Scaler Cabin (Guard Station).
8,000’ elevation; T28N R115W S29 E1/2 NW1/4; 10 Aug 1995;
\textit{Fertig 16258} (RM).
Formation just NW of Squaw Teat, ca. 4 air mi. S of Elkhorn JCT.
7,200-7,537' elevation; T28N R104W S27 NW1/4; 20 Jun 1994;
\textit{Cramer 960} (RM).
North rim of Dry Basin just off Calpet Road, ca. 10 air mi. SW of Big Piney.
7,480-7,600’ elevation; T29N R113W S27 E1/2; 28 Jun 1993;
\textit{Nelson and Nelson 26618, 26591} (RM).
Cretaceous Mountain; a branch of Deloney Canyon above Dry Basin, ca. 13.5 air
mi. SW of Big Piney. 7,800-8,160’ elevation; T28N R114W S12 NE1/4; 23 Jun 1993;
\textit{Nelson 26208} (RM).
Fogarty Canyon at the south end of Cretaceous Mountain, ca. 14 air mi. SW of
Big Piney. 7,480-7,680’ elevation; T28N R113W S18 SW1/4; 28 Jun 1993;
\textit{Nelson and Nelson 26681} (RM).
North end of Big Mesa above Dry Piney Creek, ca. 13.5 air mi. SW of Big Piney.
7,680-7,930' elevation; T28N R113W S28	SE1/4; 28 Jun 1993;
\textit{Nelson and Nelson 26706} (RM).
Lower end of Wildcat Canyon, ca. 3 air mi. WSW of Clara Birds Nipple
(Bird Nipple); ca. 15.5 air mi. SSW of Big Piney. 7,150’ elevation;
T27N R113W S10 SE1/4; 28 Jun 1993; \textit{Nelson and Nelson 26777} (RM).
Saddle ridge. 7,400’ elevation; T27N R113W S8 NE1/4; 7 Jun 1993;
\textit{Kass and Kass 3716} (RM).
Hogsback Ridge Area; northeast side of the Hogsback, ca. 18 air mi. SW of
Big Piney. 7,900-8,600’ elevation; T27N R113W S7 SC; 25 Aug 1993;
\textit{Hartman 44960} (RM).
East side of Hogsback Ridge above Calpet, ca. 7.5 air mi. W of La Barge.
7,600-8,300’ elevation; T27N R113W S19 S1/4; 12 Jun 1994;
\textit{Cramer 492} (RM).
Green River Basin; ca. 11.5 air mi. SW of Boulder. 7,200-7,300’ elevation;
T31N R109W S19 SW1/4; 7 Jul 1995; \textit{Cramer 7446} (RM).
North side of Ross Butte overlooking New Fork River, ca. 10 air mi. E of
Big Piney. 6,800-7,480’ elevation; T30N R110W S13 SE1/4 and S14 W1/4;
11 Jun 1994; \textit{Cramer 310} (RM).
North end of Ross Butte, ca. 0.5 mi. S of the New Fork River and ca. 20 air mi.
S of Pinedale. 7,100-7,460’ elevation; T30N R110W S13 N1/2 SW1/4 and
S14 N1/2 SE1/4; 9 Jul 1995; \textit{Fertig 15910} (RM).
Badlands at base of east slope of southern lobe of Ross Butte, ca. 1 mi. E of
the New Fork River, ca. 3.2 mi. ENE of confluence of the New Fork and Green
Rivers. 7,300’ elevation; T30N R110W S24 NW1/4 SW1/4 NW1/4; 30 May 1997;
\textit{Fertig 17415} (RM).
North Alkali Draw, ca. 15 air mi. E of Big Piney. 7,200’ elevation;
T30N R109W S33 SE1/4; 25 Jun 1995; \textit{Cramer and Hartman 6979} (RM).
Ca. 1.5 air mi. E of Marbleton. 6,900-7,000’ elevation; T30N R111W S28 NW1/4;
21 Jun 1995; \textit{Cramer and Kellett 6545} (RM).
Ca. 5 air mi. ESE of Big Piney. 6,900-7,000’ elevation; T29N R111W S11 E1/4;
8 Jul 1995; \textit{Cramer 7472} (RM).
Ca. 9 air mi. SW of Ross Butte. 7,100-7,200’ elevation; T29N R110W S30;
25 Jun 1995; \textit{Hartman 51227} (RM).
Lower part of Chapel Canyon, ca. 11 air mi. NE of La Barge.
6,850-7,130’ elevation; T28N R111W S27 W1/4; 12 Jun 1994;
\textit{Cramer 403} (RM).
Ca. 4 mi. N of LaBarge on crest of ridge, Bird Canyon. 2,040m elevation;
T27N R112W S15; 6 Jun 1993; \textit{Kass and Kass 3706} (RM).
Bess Canyon. 6,800’ elevation; T27N R112W S22 SE1/4; 15 Jun 1993;
\textit{Kass 3757} (RM).
20 mi. W of Big Piney. 9 Jul 1922;
\textit{Payson and Payson 2618} (RM, NY, RSA-POM, MO, F, UC).
Red Canyon, ca. 18.5 air mi. WNW of Big Piney. 8,150-8,200’ elevation;
T31N R114W S15 W1/4; 22 Jul 1995; \textit{Cramer and Kellett 9011b} (RM).
% Lincoln
  \textbf{Lincoln County:}
Cokeville. 11 Jun 1898; \textit{Nelson 4637} (RM, UC).
Southern Salt River Range and Vicinity; Snow Hollow, ca. 1.5 air mi. E of Idaho,
ca. 5.5 air mi SW of Cokeville; ca. 29 air mi. NW of Kemmerer.
6,800-6,900' elevation; T24N R120W S35 NW1/4; 8 Jul 1995;
\textit{Nelson and Refsdal 36568} (RM).
La Barge Creek. 7,200' elevation; T27N R114W S32	NW1/4; 27 May 1981;
\textit{Lichvar 4207} (RM).
Cretaceous Mountain / Hogsback Ridge Area; southeastern corner of Hogsback
Ridge, ca. 22 air mi. SW of Big Piney. 7,200-7,600'elevation; T26N R113W S8
NW1/4; 13 Aug 1993; \textit{Hartman 43802} (RM).
Green River Basin; ca. 2 air mi. E of La Barge. 7,200' elevation;
T26N R112W S9 NE1/4; 8 Jul 1995; \textit{Cramer 7519} (RM).
LaBarge Creek, on slope on northeast side of drainage, ca. 7.5 air mi.
WSW of LaBarge. 7,000’ elevation; T26N R113W S18 NW1/4 SW1/4; 18 Jun 1988;
\textit{Marriott and Horning 10827} (RM);
7,300’ elevation; T26N R113W S19 NE1/4 NW1/4; 18 Jun 1988;
\textit{Marriott and Horning 10828} (RM).
Ca. 5.9 mi. W of HWY 189, on LaBarge Creek RD. 6,200' elevation;
T26N R113W S20; 5 Jun 1987; \textit{Atwood 12830} (NY).
Ca. 3.5 mi. SE of LaBarge. 6,800' elevation; T26N R112W S21 SE1/4 NE1/4;
1 Jul 1993; \textit{Kass 3794} (RM).
Green River Basin; ca. 15.5 air mi. W of La Barge at the south end of
Fontenelle Hogbacks. 8,160-8,280' elevation; T26N R115W S28	SE1/4; 3 Aug 1995;
\textit{Cramer, Kellett, and Laster 10269} (RM).
Shale cliff near the Green River, 6 mi. S of Labarge. 6,500' elevation;
19 Jul 1939; \textit{Rollins and Muñoz 2867} (CAS).
Green River Basin; ca. 7 air mi. W of La Barge. 7,000-7,100' elevation;
T25N R112W S1 NW1/4 and S2 NE1/4, T25N R112W S35 SW1/4 and S36 SE1/4;
22 May 1994; \textit{Hartman, Cramer, and Refsdal 45172} (RM, NY).
Green River Basin; ca. 15.5 air mi. SSW of La Barge. 6,900' elevation;
T24N R114W S14 W1/2; 21 Jun 1995; \textit{Cramer and Kellett 6602} (RM).
Green River Basin; south end of Fontenelle Reservoir, ca. 2 air mi. NW of
Fontenelle Dam. 6,550' elevation; T24N R112W S23 SE1/4; 5 Jul 1995;
\textit{Cramer 7312} (RM).
Green River Basin; Slate Creek Butte, ca. 5 air mi. SW of Fontenelle Dam.
6,750' elevation; T23N R112W S14 W1/2; 21 Jun 1995;
\textit{Cramer and Kellett 6653} (RM).
Green River Basin; ca. 13 air mi. N of Opal. 6,720-6,760' elevation;
T23N R113W S20 SE1/4; 11 Jul 1995; \textit{Cramer and Kellett 7754} (RM).
Overthrust Belt; Hams Fork Plateau on south side of Robinson Creek Canyon, ca.
2.5 air mi. W of Kemmerer Reservoir, ca. 13 air mi. NW of Kemmerer, ca. 1 mi. N
of Hams Fork Road (near emigrant graves). 7,500' elevation;
T23N R117W S29 NW1/4 SE1/4; 2 Jul 1995; \textit{Fertig 16740} (RM).
Basins and Mountains of Southwest Wyoming; ca. 4.5 air mi. SW of Opal; 4.7 mi. S
on Wagon Wheel RD off of U.S. HWY 30; 6,680-6,720' elevation;
T20N R115W S14 E1/2; 1 Jul 1995; \textit{Refsdal and Refsdal 4755} (RM).
Ca. 12 air mi. SE of Opal, 18.3 mi. E of junction U.S. HWY 30 / WY HWY 240.
6,400-6,500' elevation; T20N R112W S28 E1/2; 30 Jun 1995;
\textit{Refsdal 4638} (RM, NY).
Benchland between Zieglers Wash and Dry Muddy Creek ca. 13 air mi. WNW of
Granger, ca. 22 air mi. SE of Kemmerer. 6,590-6,680' elevation;
T20N R113W S32 SE1/4 and S33 SW1/4; 3 Jul 1995;
\textit{Nelson and Refsdal 36196} (RM).
Ca. 9 air mi. WNW of Granger, ca. 24.5 air mi. SE of Kemmerer.
6,460-6,500' elevation; T19N R113W S12 SW1/4; 3 Jul 1995;
\textit{Nelson and Refsdal 36072} (RM).
Ca. 9 air mi. WNW of Granger, ca. 24.5 air mi. SE of Kemmerer.
6,460-6,500' elevation; T19N R113W S12 SW1/4; 3 Jul 1995;
\textit{Nelson and Refsdal 36073} (RM).
Little Muddy Creek, 5 mi. W of U.S. 189, 14 mi. SW of Kemmerer. 7,000' elevation;
T19N R117W S29; 17 Jul 1982; \textit{Atkins, Neely, and Carpenter 8238} (UTC).
14 mi. S of Kemmerer. 6,500' elevation; T19N R116W S33 E1/2; 27 May 1994;
\textit{Dorn 5592} (RM, MO).
% Uinta
  \textbf{Uinta County:}
Overthrust Belt; south slopes of Hanks Hill at north end of Woodruff Narrows
Reservoir, ca. 1 mi. E of the WY-UT state line, ca. 10 air mi. N of Evanston.
6,500-6,760' elevation; T18N R120W S29 SW1/4 SW1/4, S30 SE1/4, S31 N1/2 NE1/4,
and S32 NW1/4 NW1/4; 22 May 1996; \textit{Fertig 16460} (RM).
Ca. 16 air mi. N of Evanston, 8.5 mi. E on County RD 101.
6,460-6,750' elevation; T18N R120W S32 NW1/4, S31 NE1/4, S29 SE1/4, and
S28 SW1/4; 22 Jun 1995; \textit{Refsdal 4228} (RM).
18.5 mi. SSW of Diamondville. 6,650’ elevation; 11 Jul 1965;
\textit{Mulligan and Crompton 3095} (MONT).
Ca. 6 air mi. SE of Cumberland Gap, ca. 14.2 road mi. N of I-80 on east side of
WY HWY 412. 6,700-6,860' elevation; T18N R116W S28 W1/2; 19 May 1994;
\textit{Refsdal and Atwood 200, 201, 202} (RM).
SW end of the "Carter Cedars" along WY HWY 412, ca. 4 air mi. NW of Carter,
ca. 6.5 mi. E of U.S. HWY 189. 6,800' elevation;
T17N R116W S2 NE1/4 SW1/4; 18 Jun 1997; \textit{Fertig 17520, 17521} (RM).
U.S. 189, 8.7 road mi. NE of I-80. 7,000' elevation; T17N R117W S34; 6 Jul 1983;
\textit{Hartman 15736} (RM).
6 mi. N Ft. Bridger. 6,500' elevation; 13 Jun 1938; \textit{Rollins 2316} (RM).
S of Carter. 7,000' elevation; T17N R115W S34 NE1/4; 4 Jun 1980;
\textit{Lichvar 2777} (RM).
2 mi. W of Fort Bridger. 6,700’ elevation; 7 Jul 1965;
\textit{Mulligan and Crompton 3079} (MONTU).
2 mi. W of Fort Bridger. 6,700’ elevation; 21 Jul 1963;
\textit{Mulligan and Crompton 2785} (UTC).
Ca. 5 mi. SSW of Carter. 6,600' elevation; T16N R116W S13; 12 Jun 1980;
\textit{Lichvar 2866} (RM).
2 mi. W of Fort Bridger. 2,134m; 41º19'51"N, 110º25'10"W; 3 Jun 1996;
\textit{O'Kane 3785} (ISTC).
Foothills of Bridger Butte, 3 mi. W Ft. Bridger. 6,500’ elevation; 24 Jun 1938;
\textit{Rollins 2387} (NY).
Fort Bridger, Wyoming Territory. July 1873, \textit{Porter 10462} (NY);
\textit{Porter s.n.} (NY, NY, F).
Fort Bridger. 9 Jun 1898; \textit{Nelson 4602} (RM, F).
3 mi. W of Ft. Bridger, topotype.	7,000' elevation; T16N R116W S35; 24 May 1979;
\textit{Lichvar 1704} (RM).
About 3 mi. W of Ft. Bridger. 7,000' elevation;	7 Jul 1977;
\textit{Dorn 2974} (RM).
28 mi. WSW of Green River. 6,625’ elevation; 11 Jul 1965;
\textit{Mulligan and Crompton 3096} (NY).
Wildcat Butte between Church Butte RD and I-80 at Sweetwater County,
ca. 14.8 air mi. NE of Lyman, ca. 49 air mi. ENE of Evanston.
6,820-6,980' elevation; T17N R112W S22 NW1/4; 18 Jun 1995;
\textit{Nelson and Refsdal 35212} (RM).
Ca. 12 air mi. NE of Lyman. 6,940-7,000' elevation; T16N R113W S1 S1/2;
22 Jun 1995; \textit{Refsdal 4313} (RM).
6 mi. E of Lyman. 6,600’ elevation; 19 Jun 1956; \textit{Porter 7005} (RM).
6 mi. E of Lyman. 3 Jun 1970; \textit{Rollins 79152} (NY, US).
9 mi. ENE of Fort Bridger. 6,500’ elevation; 7 Jul 1965;
\textit{Mulligan and Crompton 3080} (CAS).
Sandy ravine near Blacks Fork River, 3 mi. N of Lyman. 6,500' elevation;
10 Jun 1937; \textit{Rollins 1653} (RM, NY, UC, MO).
Lyman. 15 Jun 1932; \textit{Rollins 182} (RM, MO).
Along Leavitt Creek below the south end of Cottonwood Bench, ca. 7 air mi. ESE
of Mountain View, ca. 39.5 air mi. E of Evanston.	6,700-6,860' elevation;
T15N R114W S36; 18 Jun 1995; \textit{Nelson 35163} (RM).
8 air mi. SE of Mountainview, Leavitt Creek. 6,800' elevation;
T15N R114W S36 SE1/4; 30 Jun 1982; \textit{Goodrich and Atwood 17162} (RM, NY).
Flat above barren cliffs overlooking Leavitt Creek, 1 km (0.6 mi.) S of WY HWY
414, 11.5 km (7mi.) air distance east-southeast of Mountain View.
6,800' elevation; 2,075m; 41º13'45"N, 110º12'56"W; 23 May 1999;
\textit{Holmgren and Holmgren 13447} (ISTC, NY, UTC).
Grizzly Buttes, Canyonlands and erosional badlands near Mountainview.
6,800' elevation;	T14N and T15N R114W S2 and S36; 13 Jul 1973;
\textit{Hill 881} (RM).
Sage Creek Mountain, ca. 12 air mi. SE of Mountain View. 7,200' elevation;
T14N R113W S20 SE and S21 SW; 12 Jun 1981; \textit{Dueholm 11434} (RM, NY).
East end of Sage Creek Mountain, ca. 5.3 air mi. N of Lonetree.
8,200-8,420' elevation; T13N R113W S2 NW1/4, T14N R113W S35 S1/2; 23 Jul 1995;
\textit{Refsdal 5887} (RM).
North Slope Uinta Mountains; Hickey Mountain, ca. 5.5 air mi. NW of Lonetree.
7,480-8,000' elevation; T13N R114W S12; 22 Jun 1994;
\textit{Refsdal and Fertig 1047} (RM).
5 mi. N 25 degrees W of Lonetree, E side Hickey Mtn. 7,800' elevation;
T13N R113W S18 SE1/4; 30 Jun 1982; \textit{Goodrich and Atwood 17171} (RM, NY).
Hickey Mountain, one mi. N of State HWY 414; 20 Jun 1986;
\textit{Rollins and Rollins 8670} (RM, NY, GH, UTC, MONTU).
Clay knolls and hillsides, County Road 290, 4 mi. W of Lonetree. 19 Jun 1986;
\textit{Rollins and Rollins 8666} (RM).
Uinta County RD 290, 3.7 air mi. W of Lonetree. 7,800' elevation;
T12N R114W S1; 7 Jul 1983; \textit{Hartman 15766} (RM).
Cedar Mountain, ca. 3 air mi. NE of Lonetree; ca. 3.3 road mi. E of Cedar
Mountain RD from WY HWY 414, west flank of the mountain.
7,700-7,800' elevation; T13N R113W S22 W1/2 and S15 S1/2; 11 Jun 1994;
\textit{Refsdal and Lathrop 725} (RM).
SW side of Cedar Mtn. 7,700' elevation; T13N R113W S24 S1/2; 28 Jun 1999;
\textit{Dorn 7997} (RM).
Ca. 6.5 air mi. NNW of Lonetree, ca. 1.2 road mi. SW of WY HWY 414.
7,260-7,410' elevation; T13N R113W S26 W1/2; 22 Jun 1994;
\textit{Refsdal and Fertig 1023} (RM).
Ca. 2 air mi. E of Lonetree, ca. 8.0 road mi. E of the junction of County RD 1
with WY HWY 414 on south side of WY HWY 414. 7,400-7,600' elevation;
T12N R113W S1 SW1/4 and S2 SE1/4; 7 Jun 1994; \textit{Refsdal 517} (RM).
Hoop Lake RD (Uinta County RD 295), 4 air mi. S of Lonetree.
7,800' elevation; T12N R113W S21; 7 Jul 1983; \textit{Hartman 15761} (RM).
Ca. 4 air mi. S of Lonetree, just N of Utah on Hoop Lake RD.
7,900-8,000' elevation; T12N R113W S21 S1/2; 7 Jun 1994;
\textit{Refsdal 549} (RM).
% Sweetwater
  \textbf{Sweetwater county:}
Top of Cedar Mtn. T13N R112W S28 NE1/4; 1 Jul 1982;
\textit{Atwood and Goodrich 9053} (NY).
Cedar Mountain, ca. 5 air mi. NW of McKinnon, ca. 0.5 road mi. N of Turtle Bluff
Rim RD. 8,380-8,560' elevation; T13N R112W S22 NW1/4 and S21 NE1/4;
22 Jun 1994; \textit{Refsdal 1010} (RM).
Cedar Mountain, ca. 5 air mi. NW of McKinnon; ca. 0.5 road mi. N of Turtle Bluff
Rim RD. 8,340-8,500' elevation; T13N R112W S22 NW1/4 and S21 NE1/4;
22 Jun 1994; \textit{Refsdal and Fertig 986} (RM).
Ca. 0.5 to 1 air mi. N of Henry's Fork W of County RD 1 on lower southeast
slopes of Cedar Mountain, ca. 40 air mi. SW of Green River, ca. 11 air mi. NW
of Manila, Utah. 7,000-7,100' elevation; T13N R111W S32; 18 Jun 1995;
\textit{Nelson and Refsdal 35135} (RM).
Ca. 6.2 air mi. N of McKinnon, ca. 6.1 mi. N on County RD 1 from WY HWY 414.
7,000-7,100' elevation; T13N R111W S14 SW1/4 and S15 SE1/4; 13 Jul 1995;
\textit{Refsdal 5289} (RM).
Southeast slope of Cedar Mountain. 8,000' elevation; 28 Jun 1951;
\textit{Rollins and Porter 5136} (RM, NY, US).
Flaming Gorge; N of Linwood Canyon, ca. 6 air mi. NE of Manila, Utah. 6,660'
elevation; T12N R108W S18 SE1/4; 9 Jun 2011; \textit{Heidel 3520} (RM).
Flaming Gorge National Recreation Area; ca. 8 air mi. NW of Dutch John, 2.1 mi.
W of the Forest Boundary. 6,400' elevation; T12N R108W S14 NE1/4; 7 Jun 1995;
\textit{Refsdal and Goodrich 3782} (RM).
Just across state line from Utah on HWY 191. 2,179m; 41º01'21"N, 109º25'09"W;
2 Jun 1996; \textit{O'Kane 3778} (ISTC, MO).
Ca. 3 air mi. NNW of The Gap, NW of Dutch John. 3 Jun 1980;
\textit{Atwood 7542} (UTC).
Lower Henry's Fork, 10 mi. N of Manila, Utah. 16 May 1966;
\textit{Tresler 268} (RM).
Green River Basin; ridge system between south bank of North Fork Anvil Wash and
summit of NE end of Black Mountain, ca. 2 mi. NE of Twin Buttes, ca. 5.25 air mi.
W of WY HWY 530. 6,840-7,300' elevation; T14N R109W S33 NE1/4 and S28 E1/2;
2 Jun 1995; \textit{Fertig 15718} (RM).
Black Mountain. 7,500' elevation; T14N R109W S21 SE1/4; 13 Jun 1988;
\textit{Atwood 13322} (RM).
North end of Black Mountain and Pine Spring area, ca. 26 air mi. S of Green
River. 6,800-7,900' elevation; T14N R110W S13 S1/2, S24, and S25 NE1/4;
T14N R109W S19 W1/2 and S30; 24 May 1994;
\textit{Hartman, Cramer, Refsdal 45215} (RM).
Ca. 19 air mi. SW of Green River, 1.1 mi. W on BLM RD 4315 (Burnt Fork RD).
6,460-6,680' elevation; T15N R110W S26 S1/2 and S35 N1/2;
14 Jun 1995; \textit{Refsdal 3975} (RM).
25 mi. SW of Green River on road to Manila, Utah. 25 Jun 1950;
\textit{Ownbey 3250} (UC).
Ca. 13.5 air mi. SW of Green River, 3.9 mi. W of WY HWY 530 on 2 track.
6,240-6,360' elevation; T16N R109W S27 NW1/4; 10 Jun 1995;
\textit{Refsdal 3818} (RM).
Ca. 11 air mi. W of Green River; ca. 2.4 road mi. S of I-80;
6,120-6,200' elevation; T18N R109W S26; 10 Jun 1994; \textit{Refsdal 647} (RM).
Granger. 13 Jun 1898; \textit{Nelson 4688} (RM).
Northeastern slopes and draws of Richards Mountain,
ca. 2 air mi. NW of Richards Gap, ca. 37 air mi. SSW of Rock Springs.
6,850-7,500' elevation; T12N R105W S8 S1/2, S9 SW1/4, and S17 NE1/4; 5 Jul 1996;
\textit{Ward 1960} (RM).
Red Creek Basin; adjacent to Richard's Gap RD at junction with pipeline road ca.
0.25 mi. N of Daniels Creek, ca. 2.5 mi. N of Utah State Line. 6,760' elevation;
T12N R105W S3 SE1/4 SW1/4 and S10 NE1/4 NW1/4; 27 May 1993;
\textit{Fertig 13671, 13648} (RM).
Red Creek Basin. 6,700' elevation; T13N R105W ca. S36 NW1/4; 3 Jul 1999;
\textit{Dorn 8013} (RM).
E of Little Mountain, ca. 35 air mi. S of Rock Springs. 6,960' elevation;
T13N R105W S26 NE1/4 NE1/4; 12 Jun 1979; \textit{Aldrich 103} (RM).
Red Creek Badlands. 6,900' elevation; T13N R105W S24; 10 Jul 1981;
\textit{Dueholm 11728} (RM).
Washakie Basin; ridges and flats between Richard's Gap and Tepee Mountain.
6,900-7,300' elevation; T12N R104W S18; T12N R105W S13, S14, and S15;
23 May 1981; \textit{Hartman and Dueholm 12592} (RM).
Red Creek Badlands along rim above Red Creek Ranch.	7,500-7,750' elevation;
T13N R104W S23 and S26; 19 Jun 1997; \textit{Atwood 22750, 22761} (RM).
Red Creek badlands on north and east slopes of Telephone Canyon, ca. 6 air mi.
N of Utah/Wyoming/Colorado border. 7,500' elevation; T13N R104W S24 SE1/4;
27 May 1993; \textit{Fertig 13627} (RM).
Potter Mountain. 7,400-8,000' elevation; T14N R103W S34 and S28,
T13N R103W S2; 7 Jul 1980; \textit{Dueholm 10415, 10416} (RM).
Green River Basin; slope from Currant Creek Ridge draining northward to Currant
Creek, ca. 5 air mi. E of Flaming Gorge Reservoir, ca. 26.5 air mi. SSW of Rock
Springs. 6,770-7,050' elevation; T14N R107W S2 SW1/4 and S11 NW1/4; 4 Jul 1996;
\textit{Ward 1932} (RM).
Sage Creek RD (Sweetwater CO 36) 3.7 road mi. W of WY HWY 430.
6,600-7,300' elevation; T15N R105W S7 and S18, R106W S12 and S13; 14 Jun 1981;
\textit{Dueholm 11511} (RM).
S of Rock Springs on east side of U.S. HWY 191. 6,400' elevation;
T16N R105W S18 NE1/4; 16 May 1994; \textit{Refsdal 153} (RM).
South peninsula at the confluence of Blacks Fork and Green River, ca. 17 air mi.
S of Green River. 6,040-6,080' elevation; T15N R108W S13 NW1/4; 20 Jun 1995;
\textit{Nelson, Refsdal, and Welp 35506} (RM).
Small cove on east side ca. 1.5 mi. below Blacks Fork, ca. 4 air mi. NE of
Buckboard Crossing, ca. 17.5 air mi. S of Green River. 6,040-6,290' elevation;
T15N R108W S13 SE1/4 and T15N R107W S18 SW1/4; 20 Jun 1995;
\textit{Nelson, Refsdal, and Welp 35277} (RM).
South peninsula at the confluence of Blacks Fork and Green River, ca. 17 air mi.
S of Green River. 6,120-6,264' elevation; T15N R108W S14 NE1/4; 20 Jun 1995;
\textit{Nelson, Refsdal, and Welp 35449} (RM).
White semibarren knoll of exposed Green River Shale. 6,200' elevation;
T16N R107W S33 SE1/4 SW1/4; 26 May 1999; \textit{Goodrich 26006} (RM, NY).
West side of Flaming Gorge Reservoir between Firehole Canyon and Sage Creek
Basin, ca. 13 air mi. S of Green River. 6,040-6,240' elevation; T16N R107W S27
W1/4 and S28 E1/4; 20 Jun 1995; \textit{Nelson, Refsdal, and Welp 35418} (RM).
18 mi. 167 degrees from Green River, Flaming Gorge National Recreation Area, 
South Chimney Rock. 6,400-6,600' elevation; T16N R107W S23 and S24; 15 Jun 1988;
\textit{Goodrich and Atwood 22529} (RM).
Firehole Canyon, ca. 18 air mi. SW of Rock Springs, southwest side of South
Chimney Rock. 6,000-6,600' elevation; T16N R107W S23 SE1/4; 16 Jun 1994;
\textit{Refsdal 881} (RM).
Canyon N of North Chimney Rock and E of the Green River, ca. 11 air mi. S of
Green River. 6,040-6,890' elevation; T16N R107W S12 S1/2 and S13 N1/2;
20 Jun 1995; \textit{Nelson, Refsdal, and Welp 35337} (RM).
Green River Basin; Slippery Jim Bottom on Flaming Gorge Reservoir, ca. 1 air mi.
N of Little Firehole Canyon; ca. 14 air mi. SSW of Rock Springs.
6,040-6,140' elevation; T17N R106W S30 N1/2 and S19 S1/2; 4 Jul 1996;
\textit{Ward 1852} (RM).
Rock Springs Uplift; Point 6465 on the west side of the Green River, ca. 0.75
mi. S of Cordwood Bottom, ca. 2.25 mi. ESE of Whalen Butte, ca. 3 air mi. SE of
the city of Green River. 6,400' elevation; T17N R106W S8 NE1/4 SW1/4 SW1/4;
6 Jun 1997; \textit{Fertig 17450} (RM).
Ca. 4 air mi. SE of Green River; E of FMC picnic area on old O \& G dirt road.
6,300-6,645' elevation; T18N R106W S32 SW1/4, T17N R106W S5 NW1/4; 3 Jun 1994;
\textit{Refsdal 473} (RM).
Just west of Green River, near the edge of the city. 3 Jun 1979;
\textit{Rollins 79151} (NY, MO, US).
Limey and rocky slope, 1 mi. S of Green River. 1 Jun 1938;
\textit{Rollins 2241} (NY).
WY HWY 530, 1.6 mi. S of Green River (bridge south of town). 6,600' elevation;
T18N R107W S28; 5 Jun 1971; \textit{Holmgren and Holmgren 5034} (NY, MONTU).
Green River. 14 Jun 1898; \textit{Nelson 4714} (MONT).
Green River. 30 May 1897; \textit{Nelson 3032} (RM).
Green River. 6,000' elevation; 23 Jun 1896; \textit{Jones s.n.} (RSA-POM).
Green River. 9 Jul 1897; \textit{Williams s.n.} (RM).
Green River. 24 Jun 1895; \textit{Shear 4364} (RM).
Green River. 25 Jun 1895; \textit{Rydberg s.n.} (NY, NY).
The Towers; ca. 0.5 air mi. N of Green River, northwest side of White Mountain
RD. 6,460-6,500' elevation; T18N R107W S12 N1/2; 29 Jun 1994;
\textit{Refsdal 1295} (RM).
4 mi. E of Green River. 6,225' elevation; 11 Jul 1965;
\textit{Mulligan and Crompton 3097} (RM).
Green River Basin; drainage into Scott Canyon, ca. 6 air mi. WNW of Pilot Butte,
ca. 12.5 air mi. WNW of Rock Springs. 6,660-6,950' elevation;
T19N R107W S3 NE1/4 and S2 NW1/4, T20N R107W S34 SE1/4; 10 Jul 1996;
\textit{Ward 2194} (RM).
Vicinity of Alkali Creek including Alkali Spring, ca. 3.5 air mi. E of southern
extent of Blue Rim, ca. 18 air mi. NW of Rock Springs. 6,470-6,550' elevation;
T21N R107W S35 N1/2 and S26 S1/4; 10 Jul 1996; \textit{Ward 2217} (RM).
6 mi. NW of Green River, WY. 13 Jun 1971; \textit{Hatch 1257} (NY, UTC).
Ca. 3 air mi. E of Fontenelle.  6,500-6,550' elevation;
T23N R111W S9 N1/2; 22 May 1994; \textit{Hartman, Cramer, and Refsdal 45145} (RM).
East end of Fontenelle Reservoir Dam, ca. 3 air mi. N of Fontenelle. 6,550'
elevation; T24N R111W S30 NE1/4 and S19 SE1/4; 4 Aug 1994; \textit{Cramer 2717} (RM).
Lower Eighteen Mile Canyon, ca. 21 air mi. W of Farson. 6,500-6,750' elevation;
T25N R109W S31 N1/2; 14 Jun 1994; \textit{Cramer 578} (RM, RSA-POM).
Ca. 26.5 air mi. WNW of Farson. 7,020-7,120' elevation; T26N R110W S3 W1/4;
2 Aug 1995; \textit{Cramer and Kellett 10241} (RM).
Sand knolls ca. 1.5 air mi. N of The Wells; ca. 5 air mi. W of White Mountain;
ca. 23.5 air mi. NNW of Rock Springs. 6,960-7,065' elevation; T23N R105W S31
SW1/4, T23N 106W S36 SE1/4; 10 Jul 1996; \textit{Ward 2255} (RM).
Rock Springs Uplift; drainages from White Mountain to Killpecker Creek, ca. 4
air mi. E of The Wells, ca. 20.5 air mi. NNW of Rock Springs. 7,010-7,300'
elevation; T22N R105W S11 S1/2 and S14; 23 Jun 1996; \textit{Ward 1352} (RM).
East Flaming Gorge RD 5 mi. S of I-80. 29 May 1982;
\textit{Atkins s.n.} (UTC-36377).
Common near Rock Springs. 6,600' elevation; 30 May 1947; \textit{Larsen 21} (RM).
Southerly drainages on highlands between Crooked Canyon and North Baxter Basin,
ca. 2 air mi. ESE from Twin Rocks, ca. 15.5 air mi. NNE of Rock Springs.
6,930-7,120' elevation; T21N R103W S16 NW1/4; 23 Jun 1996; \textit{Ward 1377} (RM).
7,200' elevation; T25N R102W S23; 4 Jul 1977; \textit{Dorn 2969} (RM).
Great Divide Basin; ridge on N side of Alkali Draw, S of County RD 21, ca. 4.5
air mi. E of Bush Rim. 7,100-7,200' elevation; T24N R100W S6 SE1/4 SE1/4,
S5 SW1/4, and S8 NW1/4 NW1/4; 27 Jul 1995; \textit{Fertig 16125} (RM).
Oregon Buttes, ca. 34 mi. NE of Farson. 8,500' elevation; T26N R101W S3 and S10;
27 Jun 1981; \textit{Dueholm 11642} (RM).
Slopes of the North Oregon Butte on the edge of the Red Desert. 8,000-8,600'
elevation; T26N R101W S2; 24 May 1985; \textit{Scott 4105} (RM).
47 mi. E of Rock Springs along HWY 430. 7,300' elevation; T13N R101W S17 SE;
18 Jun 1997; \textit{Atwood 22699} (RM).
Washakie Basin; from Cooper Ridge to Salt Wells Creek E of the confluence of
Pretty Water and Salt Wells creeks, ca. 21 air mi. SSE of Rock Springs.
6,705-6,880' elevation; T16N R102W S20 NW1/4 and S19 NE1/4; 5 Jul 1996;
\textit{Ward 2142} (RM).
Mesa W of north end of Cooper Ridge, S of Cutthroat Draw, ca. 1-2 air mi. E of
WY HWY 430, ca. 18 air mi. SE of Rock Springs. 6,620-7,230' elevation;
T17N R102W S16 N1/4 and S9 S1/4; 7 Jun 1996; \textit{Ward 1} (RM).
Black Buttes, ca. 9 air mi. S of Point of Rocks. 7,500' elevation; T18N R101W S9;
22 Jun 1980; \textit{Dueholm 10246} (RM).
Rock Springs Uplift; north facing drainages into Scheggs Draw, ca. 1 air mi. W
of Rifes Rim, ca. 33 air mi. SE of Rock Springs. 7,160-7,380' elevation;
T15N R101W S35 SW1/4 and S34 SE1/4, T14N R101W S3 NE1/4 and S2 NW1/4;
27 Jul 1996; \textit{Ward 3281} (RM).
Ca. 2 air mi. SW of Pine Butte. 7,400' elevation; T15N R100W S7 and S18;
16 Jul 1980; \textit{Dueholm 10582} (RM).
Washakie Basin; small butte above Pine Butte Basin between Pine and Sand butes,
ca. 33 air mi. SE of Rock Springs, ca. 24.5 air mi. SSE of Point of Rocks.
7,760-8,170' elevation; T16N R100W S33 N1/4 and S28 S1/4; 7 Jun 1996;
\textit{Ward 72} (RM).
East side of Kinney Rim, ca. 1 air mi. NW of County RD 19, ca.
43.5 air mi. SW of Wamsutter. 7,280-7,740' elevation; T14N R99W S8, S7 NE1/4,
S9 SW1/4; 13 Jun 1996; \textit{Ward 489} (RM).
Ca. 6.5 air mi. E of Sand Butte, ca. 2 air mi. SW of the
confluence of Pine Creek Wash and Antelope Creek, ca. 36 air mi. SW of Mansutter.
7,190-7,330' elevation; T16N R99W S22 SW1/4 and S21 E1/4; 13 Jun 1996;
\textit{Ward 516} (RM).
Butte N of Red Wash, ca. 1 mi. N of confluence with Bitter Creek,
ca. 1.5 mi. E of County RD 19, ca. 37 air mi. ESE of Rock Springs, ca. 28.5
air mi. WSW of Wamsutter. 6,890-7,270' elevation; T18N R98W S31 NE1/4,
S32 NW1/4; S29 SW1/4, and S30 SE1/4; 12 Jun 1996; \textit{Ward 248} (RM).
Great Divide Basin; south of Bitter Creek on County HWY 19 on E-side of
Antelope Creek. 7,000-7,200' elevation; T17N R99W S36; 14 Jun 1994;
\textit{Fertig 14865} (RM).
Ca. 13 air mi. SSW of Red Desert. 7,000' elevation; T17N R95W S7 and
T17N R96W S12; 11 Jun 1980; \textit{Dueholm 9958} (RM).
South end of Delaney Rim just N of North Barrel Springs Draw, ca. 6 air mi. E of
Man and Boy Buttes, ca. 13 air mi. SSW of Wamsutter. 6,760-6,880' elevation;
T17N R94W S6 NW1/4; 7 Jun 1996; \textit{Nelson 37905} (RM).
Ridge between Barrel Springs Draw and Mulligan Draw, ca. 5 air mi. E of The
Haystacks, ca. 17.5 air mi. SSW of Wamsutter. 6,880-6,920' elevation;
T17N R95W S25 SW1/4; 7 Jun 1996; \textit{Nelson 37954} (RM).
Bluffs above Windmill Draw, ca. 4 air mi. N of Courthouse Butte, ca. 24.5 air
mi. S of Wamsutter, ca. 24 air mi. NW of Baggs. 6,780-6,860' elevation;
T15N R94W S4 NE1/4; 7 Jun 1996; \textit{Nelson 37980} (RM).
West Flat Top Mountain, ca. 30 air mi. S of Wamsutter, ca. 18 air mi. NW of
Baggs. 7,400-7,420' elevation; T14N R94W S1 NE1/4; 10 Jul 1996;
\textit{Nelson 38423} (RM).
Southeast side of butte overlooking Sand Creek, ca. 2 air mi. W of McPherson
Springs, ca. 17.5 air mi. WNW of Baggs. 6,520-6,828' elevation;
T13N R94W S21 SW1/4 and S20 SE1/4; 28 Jun 1996; \textit{Ward 1728} (RM).
Between Cherokee Basin and Cherokee Rim, ca. 14.5 air mi. W of Baggs.
6,350-6,520' elevation; T12N R94W S11; 27 Jun 1996; \textit{Ward 1582} (RM).
From Powder Rim N to Sand Creek, ca. 7.5 air mi. ENE of Powder Mountain, ca.
22.5 air mi. WNW of Baggs. 7,010-7,170' elevation; T13N R95W S28; 27 Jun 1996;
\textit{Ward 1616} (RM).
White clay-like soil, 48 mi. S of Rock Springs. 19 Jun 1981;
\textit{Rollins and Rollins 81358} (RM, NY, UC, US).
% Carbon
  \textbf{Carbon County:}
Dyer’s Ranch. 21 Jun 1901; \textit{Goodding 80} (RM, MO-3833614, MO-5120378).
Saratoga, Wyoming. 23 May 1924; \textit{Nelson 10092} (RM, NY, UC).
Seminoe Reservoir, ca. 10 air mi. SSE of Seminoe Dam. 6,360’ elevation;
T24N R84W S14 and S24; 11 July 1979; \textit{Hartman 9925} (RM, NY).
E end Shirley Mountains, canyon at head of First Ranch Creek, ca. 6.5 air mi.
SW of WY HWY 77. 8,200-8,400’ elevation; T25N R81W S12 SW1/4; 11 Jun 1996;
\textit{Fertig 16588} (RM).
Shirley Basin, foothill ridges at NE end of Shirley Mountains, ca. 0.5 air mi.
SE of Sullivan Creek, ca. 4 air mi. W of Pine Hill, ca. 5.25 mi. S of County RD
102 (Leo Road). 7,700-7,800’ elevation; T25N R81W S1 SW1/4; 11 Jun 1996;
\textit{Fertig 16573} (RM).
West side of BLM RD 3115 ca. 3.5 air mi. S of junction with County RD 102,
between Sullivan Creek and First Ranch Creek, ca. 2 air mi. NE of base of
Shirley Mountains, ca. 4 air mi. SE of WY HWY 77. 7,160-7,200’ elevation;
T26N R80W S31 NE1/4 NW1/4; 11 Jun 1996; \textit{Fertig 16560} (RM).
13.6 mi. N of Rawlins on HWY 287. 2,277m elevation; 22 Jun 1996;
\textit{Salywon and Dierig 3119} (MO).
22 mi. N of Rawlins. 6,500’ elevation; T23N R88W S15 NW1/4; 30 May 1981;
\textit{Lichvar 4302} (RM).
Fort Steele. 18 Jun 1898; \textit{Nelson 4834} (RM).
Fort Steele. 6,500’ elevation; 25 May 1901; \textit{Tweedy 4488} (NY).
Gumbo flats 2 mi. S of Sinclair. 6,500’ elevation; 25 May 1947;
\textit{Porter 4145} (RM, RSA-POM).
Ca. 1/2 mi. E of Little Sage Creek Reservoir, ca. 13 air mi. S of Rawlins.
7,220’ elevation; T19N R88W S22 S1/2 NE1/4; 23 Jun 1983;
\textit{Warren 573} (RM).
Ca. 0.5 mi. SW of Bridger Pass on the southeast edge of Atlantic Rim, ca. 20 air
mi. SW of Rawlins. 7,600-7,840’ elevation; T18N R89W S8; 4 Jun 1996;
\textit{Nelson 37698} (RM).
Loco Creek Upland, Stratton Hydrological Study Area. 7,000-8,000’ elevation;
T17N R87W S24; \textit{Schroeder and Ranch 4200-1} (RM).
Sage Creek Basin, ca. 27 air mi. S of Rawlins. 7,800’ elevation; T17N R88W S32;
22 Jun 1980; \textit{Mastrella 22} (RM).
Open juniper forest with \textit{Arabis, Cercocarpus}. 7,100’ elevation;
T16N R92W S10; 1 Jul 1979; \textit{Dorn 3301} (RM).
Ridge E of Muddy Creek and WY HWY 789, ca. 36.5 air mi. SW of Rawlins, ca. 23
air mi. N of Baggs. 6,680-7,015’ elevation; T16N R92W S10 S1/4; 5 Jun 1996;
\textit{Nelson and Ward 37782} (RM).
On ridge between Muddy, Cow, and Wild Cow creeks and Wild Cow or County RD
608, ca. 40 air mi. SW of Rawlins, ca. 19.5 air mi. N of Baggs.
6,506-6,730’elevation; T15N R91W S8, S17, and S18; 5 Jun 1996;
\textit{Nelson and Ward 37848} (RM).
Southwest facing slope of Coal Gulch draining from Browns Hill to Savery Creek,
ca. 2.5 air mi. from the confluence of Coal Gulch with Savery Creek, ca. 16.5
air mi. NE of Baggs.	7,280-7,690' elevation; T14N R89W S9 NE1/4 and S10 NW1/4;
20 Jun 1996; \textit{Ward 961} (RM).
13 mi. N of Baggs, 2 mi. off WY HWY 789. 22 May 1979;
\textit{Rollins 7932} (NY, MO).
East end of Little Robber Gulch, ca. 12 air mi. NNW of Baggs. 6,500-6,800’
elevation; T14N R92W S11; 1 Jun 1996; \textit{Hartman and Ward 54230} (RM).
Deep Creek Rim and draws on west-facing slope, ca. 3 air mi. E of WY HWY 789,
ca. 8.5 air mi. NNE of Baggs. 6,680-7,210’ elevation; T14N R91W S26;
17 Jun 1996; \textit{Ward 860} (RM).
Washakie Basin, draw into Cottonwood Creek from the south between Streckfus Draw
and WY HWY 789, ca. 4 air mi. NNW of Baggs. 6,480-6,620’ elevation;
T13N R91W S7; 17 Jun 1996; \textit{Ward 824} (RM).
Southeast slope of headland between Middle Prong Red and North
Prong Red Creeks, ca. 3 air mi. S of Flat Top Mountain, ca. 10 mi. NW of Baggs.
6,510-6,885’ elevation; T13N R93W S1 N1/2 and T14N R93W S36 SW1/4; 17 Jun 1996;
\textit{Ward 785} (RM).
Tree Draw leading from Hangout Ridge W to Hangout Wash, ca.
2-2.5 air mi. from the confluence of Hangout Wash and Sand Creek, ca. 12 air mi.
WNW of Baggs. 6,260-6,620’ elevation; T13N R93W S16 W1/2, S17 N1/2, and
S18 SE1/4; 16 Jun 1996; \textit{Ward 673} (RM).
Rim N and W of Red Creek, ca. 2.5-3 air mi. above the confluence
of Red and Sand creeks, ca. 9.5 air mi. WNW of Bagggs. 6,310-6,530’ elevation;
T13N R93W S26 NW1/4 and S23 SW1/4; 17 Jun 1996; \textit{Ward 722} (RM).
Poison Buttes. 6,800’ elevation; T12N R93W S2 SW1/4; 14 Jun 1979;
\textit{Lichvar 1756} (NY).
Southern extension of the bluffs NW of Baggs between the Little Snake River
plain and Devils Canyon, ca. 3 air mi. W of Baggs. 6,420-6,880’ elevation;
T12N R92W S2; 16 Jun 1996; \textit{Ward 634} (RM).
Ca. 3 air mi. SE of Dixon. 6,600’ elevation; T12N R90W S15 SE1/4; 14 Jun 1979;
\textit{Hartman and Coffey 8970} (RM, NY).
% Albany
  \textbf{Albany County:}
Near Centennial, WY. July 1938; \textit{Mauzy s.n.} (NY).
State HWY 230, 4 mi. E of Jelm. 15 Jun 1986;
\textit{Rollins and Rollins 8621} (RM, NY, UTC).
Stony canyon, Camel Rock. 21 Jun s.n., \textit{Schwartz 51} (UC).
Canyon of the North Fork of Sybelle Creek, 14 mi. E of Bosler Junction (U.S. 30
and WY State Route 34). 7,000’ elevation; 24 Jun 1951;
\textit{Rollins and Porter 5114} (RM, NY, UC, US).
Morton’s Pass, NE of Bosler. 6,500’ elevation; 18 Jul 1950;
\textit{Ripley and Barneby 10543} (NY, CAS).
Sybille Canyon. 6,600’ elevation; T21N R72W S2 NW1/4; 2 Jun 1981;
\textit{Lichvar 4306} (RM).
Sand Creek. 1 Jun 1900; \textit{Nelson 7026} (RM, NY, RSA-POM, MO).
%%% COLORADO STATE SPECIMENS
  \textbf{Colorado:}
% Moffatt
  \textbf{Moffatt County:}
CO HWY 13, ca. 34 air mi. N of Craig. 6,400' elevation; T12N R91W S15 W1/2;
6 Jun 2000; \textit{Hartman 67312} (RM).
1.7 mi. S of confluence of Shell Creek and Hells Canyon. 6,800' elevation;
T11N R99W S5 SW1/4 NE1/4; 21 Jun 1983;
\textit{Baker, Wiley-Eberle, and Deardorff 83-72} (GH).
Browns Park; Spitzie place. 6,000' elevation; 23 May 1965;
\textit{MacLeod 358} (UTC).
Bull Canyon, ca. 2 mi. WSW of south entrance to Irish Canyon. 6,400' elevation;
T10N R101W S32 SW1/4; 8 Jun 1983; \textit{Peterson and Wiley-Eberle 83-171} (NY).
Sheep Head Basin. 6,500' elevation; 31 Jul 1936; \textit{Nielson N-73} (RM).
Crossing of Colorado Route 318 and the Little Snake River. 6,000' elevation;
5 May 1981; \textit{Lichvar 3950} (RM).
Warm Spring Cedars. 6,700' elevation; T7N R103W S13 SE1/4 SE1/4 SE1/4;
16 Jun 1987; \textit{O'Kane 3126} (NY).
West rim of Lodore Canyon. 7,500' elevation; 9 Jul 1945;
\textit{Porter 3656} (RM).
Sand Canyon, Dinosaur National Monument. 5,500' elevation; 14 May 1948;
\textit{Porter 4445} (RM, RSA-POM).
Blue Mountain Plateau, road to Pat's Hole, 3.8 mi. above Echo Park Campground,
17.5 airline mi. N of Dinosaur. 5,500' elevation; T6N R103W S5; 11 Jun 1971;
\textit{Holmgren and Holmgren 5142} (NY).
Steep ravine between Anderson Hole and Happy Hollow, N side of the Yampa River.
Dinosaur National Monument. 6,000' elevation; T6N R99W S18 NE1/4 NW1/4;
18 May 1988; \textit{Naumann 134} (RM).
0.5 mi. S of West Cactus Flat, S of Yampa River, SW of Benchmark 7029. 7,000'
elevation; T6N R100W S28 SW1/4 NW1/4; 15 May 1987; \textit{Neely 3961} (RM).
WNW of Massadona along U.S. 40 on ridge near Skull Creek ca. 5.94 km E of Blair
Spring. 5,675' elevation; T3N R101W S1; 40º15'22"N, 108º40'26"W;
3 Jun 2003; \textit{Windham 2750} (MO).
HWY 40, ca. 15 mi. NE of Elk Springs. 6,320' elevation; 1,926m;
40º22.564'N, 108º22.390'W; 9 Jun 2004; \textit{Grady 40} (ISTC).
CO HWY 13, ca. 13 air mi. N of Craig. 6,480-6,520' elevation; T9N R90W S29;
6 Jun 2000; \textit{Hartman 67231} (RM).
Steep clay bluff of Yampa River, 4 mi. E of Craig. 22 May 1979;
\textit{Rollins 7938} (RM, NY, US, F).
Flat Tops / White River Plateau; Monument Butte, ca. 6 air mi. SW of Hamilton.
6,700-7,250' elevation;  T4N R92W S24 and S25; 22 Jun 1990;
\textit{Hartman 25690} (RM).
% Routt
  \textbf{Routt County:}
Clay Hillside, bluff of Yampa River, 2.5 mi. W of Steamboat Springs.
22 May 1979; \textit{Rollins 7940} (NY).
Sulphur spring formation, Steamboat Springs. 6,000' elevation; 20 Jul 1903;
\textit{Goodding 1623} (RM, NY).
Off Sage Creek Canyon opposite Hillbilly Mountain, ca. 6.5 air mi. S of Hayden;
ca. 22 air mi. WSW of Steamboat Springs. 7,600' elevation; 18 Jun 1990;
\textit{Nelson 18597} (RM).
Above Williams Fork, ca. 13 air mi. SSW of Hayden. 6,800-7,200' elevation;
T4N R88W S18; 22 Jun 1990; \textit{Hartman 25601} (RM).
Williams Fork Mountains; above East Fork Williams Fork, ca. 13 air mi. SSW of
Hayden; ca. 28 air mi. WSW of Steamboat Springs. 6,900' elevation; T4N R88W S18;
19 Jun 1990; \textit{Nelson 18717} (RM, MONTU).
Along HWY 134 west of Gore Pass and along Toponas Creek. 2,530m elevation;
40º05'06"N, 106º42'50"W; 28 May 1996; \textit{O'Kane, Jr. 3760} (MO, ISTC).
% Rio Blanco
  \textbf{Rio Blanco County:}
Raven Ridge, ca. 12 mi. WSW of Rangely; end of ridge NW of Mormon Gap.
5,800' elevation; T2N R104W S13; 27 May 1979;
\textit{Weber, Johnston, Wingate, and Kelso 1925} (MONTU).
Between Rangely and Blue Mountain, ca. 3 mi. S of Blue Mountain.
1,905m elevation; 40º10'51"N, 108º50'30"W; 30 May 1996;
\textit{O'Kane, Jr. 3769} (RM).
Clay hills, 3 mi. NW of Rangely. 5,200' elevation; 21 Jun 1983;
\textit{Rollins and Rollins 83103} (RM, NY).
High hills N of White River, ca. 15 mi. E of Rangely. 5,600' elevation;
T3N R100W S31; 22 Jun 1983; \textit{Rollins and Rollins 83108} (RM, NY, US).
Growing 25 mi. E of Rangely on U.S. 64. 7 Jun 1965;
\textit{Collotzi and Collottzi 451} (NY, UTC).
On ridge between Boise Creek and Hammond Draw, 3.5 mi. S of State HWY 64.
1,880m elevation; T2N R99W S24; 26 May 1982;
\textit{Baker and Riefler 82-102} (NY).
Roadcut, Trail Canyon RD (County RD 24X), 0.4 mi. E of Calamity Ridge and
County RD 103; ca. 15 mi. ESE of Rangely. 2,378m elevation;
40º01'57"N, 108º32'25"W; 6 Jun 1996; \textit{Salywon 3104} (MO).
Ca. 2 mi. SW of junction of Stake Springs Draw and Corral Gulch. 2,015m
elevation; T2S R98W S6; 27 May 1982; \textit{Baker and Riefler 82-103} (GH).
Piceance Basin, ridge west of Black Sulphur Creek, County RD 26.
2,316m elevation; 39º49'42"N, 108º26'37"W; 25 May 1996;
\textit{O'Kane, Jr. 3763} (ISTC, MO).
Cathedral Bluffs. 8,400' elevation; T3S R99W S32 NE1/4; 10 Jul 1982;
\textit{Smith, Neese, Baker, Snyder, and Trent 1802} (UTC).
Off County Route 64, 24 mi. W of Meeker. 5,800' elevation; 21 Jun 1983;
\textit{Rollins and Rollins 83101} (RM, NY).
Dry knoll, 20 mi. N of Rifle. 8,000' elevation; 27 May 1938;
\textit{Rollins 2209} (NY).
2 mi. E of Calamity Ridge RD junction with Yellow Creek, Piceance Basin.
6,400' elevation; 22 Jun 1989; \textit{Flock 2091} (RM, NY).
7.8 air mi. WNW of junction of Ryan Gulch and Piceance Creek. 20 Jun 1983;
\textit{Rollins and Rollins 8396} (NY, GH).
Tongue of the Green River Formation. T2S R97W S4; 20 Jun 1983;
\textit{Rollins and Rollins 8392} (NY).
County RD 142, off of CO HWY 64, Scenery Gulch, ca. 6.5 mi. S of Moffat
County line. 6,200' elevation; 40º07'35"N, 108º07'38"W; 19 May 2003;
\textit{Snow, Atchley, Brasser, and Kamal 9121} (NY).
Danforth Hills; Devils Hole Gulch, ca. 10 air mi. N of Meeker.
7,300-7,500' elevation; T2N R94W S4; 22 Jun 1990; \textit{Hartman 25753} (RM).
Meeker. 6,200' elevation; 8 Jun 1901; \textit{Osterhout 2720} (RM).
Rocky hills near Meeker. 7,000' elevation; 28 May 1938;
\textit{Rollins 2220} (RM).
Off County Routes 13 and 789, 6 mi. S of Meeker. 6,500' elevation; 19 Jun 1983;
\textit{Rollins and Roads 8386} (RM, NY, US).
LO 7 Hill, ca. 5 air mi. S of Meeker. 6,800-7,400' elevation; T1S R94W S14;
25 May 1990; \textit{Hartman, Vanderhorst, and Fertig 24795} (RM).
Oak Ridge State Wildlife Area, ca. 12 air mi. SE of Meeker. 7,000' elevation;
T1S R92W S21; 9 Jun 1991; \textit{Vanderhorst 2531b} (RM).
White River, 2.5 mi. W of Piceance Creek, ca. 20 mi. W of Meeker.
6,500' elevation; 1 May 1982; \textit{Neese 11265} (NY).
% Garfield
  \textbf{Garfield County:}
Glenwood Springs, CO. 5,800' elevation; 20 Jun 1899;
\textit{Diahl s.n.} (RSA-POM-94733).
% Mesa
  \textbf{Mesa County:}
Rabbit Valley, 1 mi. S of Interstate HWY 70. 1,550m elevation; 39º10'18"N,
109º01'12"W; 29 May 1981; \textit{Wilken and Painter 13689} (NY).
Badger Wash Experimental Area. 5,000' elevation; 7 Jun 1972;
\textit{Reid and Ranck 4200} (RM).
Badger Wash. 5,200' elevation; 17 May 1955; \textit{Turner 203} (RM).
Low rocky hogbacks beneath northwest escarpment, Colorado National Monument,
3 mi. S of Fruita, west of highway approach to Monument. 1,530m elevation;
21 May 1948; \textit{Weber 3798} (UTC).
N side of Black Ridge, 6 air mi. S of Fruita. 6,700' elevation;
T11S R102W S26 NE NW; 24 May 1983; \textit{Neese 13373} (NY).
Colorado National Monument. 5,700-5,900' elevation; 24 Jun 1938;
\textit{Pennell and Schaeffer, Jr. 22150} (NY).
West of Grand Junction. 10 Jun 1940; \textit{Porter and Porter 2669} (RM).
8 mi. W of Grand Junction. 6,500' elevation;
26 May 1938; \textit{Rollins 2176} (RM, NY).
Book Cliff Road, Grand Junction. 18 May 1916; \textit{Eastwood 5186} (CAS).
Grand Junction. 4,500' elevation; 23 May 1911; \textit{Osterhout 4541} (RM).
Grand Junction - hills south. 4,500' elevation; 11 Jun 1920;
\textit{Osterhout 6018} (RM).
Grand Junction. 4,500' elevation; 21 Jun 1912; \textit{Osterhout 4719} (RM).
Grand Junction. 22 May 1891; \textit{Jones s.n.} (RSA-POM-94231).
Grand Junction. 28 May 1894; \textit{Crandall s.n.} (NY).
Mack. 27 May 1908; \textit{Jones s.n.} (CAS-150260, MO-3833615).
Glad Park, ca, 1.5 mi. SE of Lane Reservoir. 7,100' elevation;
T12S R102W S34 E1/2; 13 Jun 1983; \textit{Peterson and Kennedy 83-190} (RM, NY).
Roadcut, 9 mi. up Uncompahgre Plateau RD, 14.5 mi. SW of HWY 50 on HWY 141.
2,073m elevation; 38º50'10"N, 108º33'43"W; 5 Jun 1996;
\textit{Salywon 3085} (MO).
Collected 18.7 mi. S of junction of HWY 50 and 141 on HWY 141. 5,200' elevation;
19 May 1973; \textit{Morris, Dunn, and LeDoux 113} (NY).
Salt Creek Canyon, 7 mi. SE of Gateway. 5,600' elevation; T26S R104W S16;
26 May 1983; \textit{Atwood 9247} (NY).
% Delta
  \textbf{Delta County:}
4 mi. ESE of Hotchkiss, 1 mi. E of HWY 92. 5,640' elevation; T15S R92 S3 NE1/4;
11 May 1983; \textit{Neese 13240} (NY).
Escalante Canyon road between Whitewater and Delta, east slopes of Uncompahgre
Plateau; shaded ravine 14 mi. SE of junction with HWY 50,
above Cottonwood Spring. 15 May 1978; \textit{Weber 15306} (RM).
% Montrose
  \textbf{Montrose County:}
Clay hillside 0.8 mi. N of entrance road to Black Canyon National Monument,
off U.S. HWY 50. 28 May 1979; \textit{Rollins and Rollins 7986} (NY, GH).
Ca. 0.5 mi. NWW of junction of Dry Cedar Creek and S Canal, 0.2 mi. S of
Kinikin RD, on W side of road. 6,250' elevation; T48N R9W S13; 20 Jun 1985;
\textit{Neely 2938} (UTC).
Ca. 4 mi. S of Montrose and 2 mi. E of Highway. 2,012m elevation; 38º24'27"N,
107º48'43"W; 18 May 1996 \textit{O'Kane, Jr. and Anderson 3726} (ISTC).
Dry, rocky hills near Montrose Colo. 6,800' elevation; 3 May 1913;
\textit{Payson 75} (RM).
Ca. 8 mi. due W of Montrose. 6,000' elevation; T49N R10W S19; 15 May 1984;
\textit{Welsh, Welsh, and Kass 22767} (NY).
Steep road-cut on canyon wall of East Fork of Dry Creek, State HWY 90, 13 mi.
SW of Montrose. 16 Jun 1986; \textit{Rollins and Rollins 8636} (NY, GH, MONTU).
Uncompahgre National Forest, Dry Park. 7,000' elevation; T46N R13W S8;
14 May 1963; \textit{Gierisch 2626} (RM).
Naturita. 5,400' elevation; 22 Apr 1914; \textit{Payson 247} (RM, MONT, F).
Near road to Paradox, 1.1 mi. S of junction of CO HWY 141 and 90.
5,400; elevation; 16 May 1981; \textit{Rollins and Rollins 8135} (RM, NY).
San Miguel and lower Dolores River Drainages; Sinbad Valley, south end.
4,700-4,800' elevation; T49N R19W S22 and S27; 30 May 1995;
\textit{Hartman 50902} (RM).
5.4 mi. NE of Bedrock; road from Uravan to Bedrock; 8 Jun 1969;
\textit{Cox and Dunn 1312} (NY).
Ca. 5 mi. E of Paradox, Martin Mesa. 6,400' elevation; T48N R18W S28 N;
1 Jun 1986; \textit{Franklin 3892} (NY).
2.3 air mi. N of Paradox; ca. 25 air mi. NNW of Nucla, small draw across County
RD 6.00, W of old Paradox dump. 1,700m elevation; T48N R19W S22, S20 NW1/4,
and S21; 24 Apr 1994; \textit{Moore 1268} (RM).
Paradox, Montrose Co. 5,400' elevation; 13 Jun 1912;
\textit{Walker 89} (RM, RSA-POM).
Paradox, Montrose Co. 22 Jun 1912; \textit{Walker 169} (RM).
Along County RD U5, ca. 2 air mi. W to NW of Paradox; ca. 23 air mi. NNW of
Nucla, intersection of Manti-La Sal National Forest and BLM. 1,750m elevation;
T48N R19W S32 NW1/2; 23 Apr 1994; \textit{Moore 1255} (RM).
Ca. 4 air mi. WSW of Paradox off of County RD XZ; ca. 25 air mi. from Nucla.
2,100m elevation; T47N R20W S12 NW1/4; 23 Apr 1994; \textit{Moore 1239} (RM).
Ca. 24 air mi. WNW of Naturita, N of CO HWY 90, ca. 3/4 road mi. E of Utah.
6,000-6,200' elevation; T47N R20W S14 and S15; 17 Jun 1995;
\textit{Moore 5590} (RM).
Lake Creek Spring, ca. 2.5 road mi. E of Utah border, N of CO HWY 90, ca.
28 air mi. W of Nucla. 6,000' elevation; T47N R20W S14; 21 May 1995;
\textit{Moore 4255} (RM).
Ca. 18 air mi. NW of Naturita, ca. 2.5 air mi. S of Bedrock, east side of
Dolores River Canyon. 5,000' elevation; T47N R19W S31; 17 Jun 1995;
\textit{Moore 5473} (RM).
Ca. 18 air mi. W of Nucla at Bedrock Boat Launch along Dolores River, County
RD Y9. 5,200' elevation; T47N R18W S30; 21 May 1995; \textit{Moore 4279} (RM).
Paradox Valley area, ca. 16 air mi. W of Nucla on Davis Mesa along County RD
DD16. 6,100-6,400' elevation; T47N R18W S35 and S34, T46N R18W S2; 21 May 1995;
\textit{Moore 4223} (RM).
Skein Mesa, Muleshoe Reservoir area, head of Spring Canyon, County RD DD15 and
County RD DD9, ca. 18 air mi. W of Nucla. 6,300-6,400' elevation;
T46N R18W S18; 22 May 1995; \textit{Moore 4320} (RM).
Slick Rock Canyon, ca. 19 air mi. W of Naturita, at mouth of Spring Canyon and
surounding areas on north side of Dolores River. 5,100-5,200' elevation;
T46N R18W S19, T46N R19W S25; 5 Jul 1995; \textit{Moore 6372} (RM).
Wild Steer Mesa, off County RD FF16 to County RD GG11, ca. 16 air mi. SW of
Nucla. 2,000m elevation; T46N R18W S29 SE1/4 and S28 SW1/8; 2 Jun 1994;
\textit{Moore 2000} (RM).
Wild Steer Mesa, off County RD FF16 to County RD EE16, ca. 15 air mi. SW of
Nucla. 2,050m elevation; T46N R18W S27 NW1/4 and S22 SW1/8; 2 Jun 1994;
\textit{Moore 1967} (RM).
County RD FF13, ca. 17 air mi. WSW of Nucla. 7000' elevation; T46N R18W S27;
21 May 1995; \textit{Moore 4244} (RM).
Bull Canyon area, ca. 13 air mi. SE of Paradox; County RD DD19 to County RD
FF16 to Bull Canyon. 1,900m elevation; T46N R18W S36 SE1/4; 24 Apr 1994;
\textit{Moore 1275} (RM).
Upper Little Gypsum Valley, County RD 115 and County RD GG6, ca. 22 air mi.
WSW of Naturita. 5,700-5,800' elevation; T45N R19W S4; 18 Jun 1995;
\textit{Moore 5633} (RM).
Island Mesa, unmarked road just N of the county line, ca. 24 air mi. SW of Nucla.
6,800' elevation; T45N R19W S7; 25 May 1995; \textit{Moore 4462} (RM).
Lower Little Gypsum Valley, County RD 20.R, ca. 20 air mi. WSW of Naturita.
5,500' elevation; T45N R19W S14; 18 Jun 1995; \textit{Moore 5651} (RM).
Silveys Pocket below Slickrock Canyon ca. 11 mi. due S of Bedrock.
1,643m elevation; 27 May 1998; \textit{Atwood and Trotter 23619} (NY).
% Ouray
  \textbf{Ouray County:}
Billie Creek State Wildlife Area. 2,134m elevation; 38º17'29"N,
107º45'32"W; 18 May 1996; \textit{O'Kane, Jr. and Anderson 3721} (RM).
% San Miguel
  \textbf{San Miguel County:}
Norwood Hill. 7,000' elevation; 17 Aug 1912; \textit{Walker 490} (RM).
East Island Mesa Spring, County RD Z1, ca. 24 air mi. SW of Nucla.
7,000' elevation; T45N R19W S21; 25 May 1995; \textit{Moore 4436} (RM).
Slick Rock area, ca. 37 air mi. SW of Nucla, side road off E side of CO HWY
141. 1,700m elevation; T44N R18W S31 SW1/2; 9 May 1994;
\textit{Moore and Rambo 1472} (RM).
Small unnamed canyon, W of Slick Rock gas station, off County RD S9, ca. 44
air mi. SW of Nucla. 1,750m elevation; T44N R19W S36 N1/2; 9 May 1994;
\textit{Moore and Rambo 1486} (RM).
3.3 mi. S of Slickrock on switchbacks. 6,400' elevation; T43N R19W S1;
31 May 1990; \textit{Anderson 90-100} (RM).
Near CO HWY 141, 17 mi. N of junction with U.S. HWY 666. 7,000' elevation;
15 May 1981; \textit{Rollins and Rollins 8131} (RM, NY, US).
Slick Rock area off CO HWY 141, ridge top before highway drops down into
Slick Rock, ca. 40.5 air mi. SW of Nucla. 2,150m elevation; T43N R19W S11 NW1/4;
10 May 1998; \textit{Moore 1427} (RM).
Nichols Wash, 5 mi. E of Slickrock Post Office. 25 May 1987;
\textit{Barneby 18255} (NY).
Ca. 6 road mi. N of Egnar W of CO HWY 141, County RD 7N, ca. 14 air mi. WNW
of Dove Creek. 7,100-7,400' elevation; T43N R19W S21; 21 Jun 1995;
\textit{Moore 5902} (RM).
County Raod 5.H, ca. 4 road mi. NW of Egnar in Bishop Canyon. 6,700-7,000'
elevation; T42N R19W S29, S30, and S32; 25 May 1995; \textit{Moore 4492} (RM).
Ca. 15 air mi. N of Dove Creek in Disappointment Valley, County RD K20.
5,860' elevation; T43N R17W S20; 6 Jun 1995; \textit{Moore 5140} (RM).
Disappointment Valley, ca. 13 air mi. NE of Dove Creek on Mineral Mountain or
County RD K20. 6,400-6,800' elevation; T43N R17W S31; 6 Jun 1995;
\textit{Moore 5103} (RM).
Ca. 11 air mi. N of Dove Creek in Dolores River Canyon, County RD 10.00.
6,000' elevation; T42N R18W S14; 22 Jun 1995; \textit{Moore 6019} (RM).
% Dolores
  \textbf{Dolores County:}
Ca. 4 air mi. NE of Dove Creek near Dove Creek Pump Station along the Dolores
River, County RD 10.00. 6,000-6,100' elevation; T41N R18W S12, S14, and S23;
22 Jun 1995; \textit{Moore 5955} (RM).
Upper east end of Disappointment Valley, County RD D.00, Warden Draw, ca. 26
air mi. N of Dolores. 7,300-7,400' elevation; T42N R14W S33; 16 Jun 1995;
\textit{Moore 5455} (RM).
County RD 31.00, ca. 22 air mi. NNE of Dolores in the Belmear Lake drainage
area. 8,200-8,400' elevation; T41N R14W S24; 13 Jul 1995;
\textit{Moore and Smith 7013} (RM).
On low gray flats north of USFS RD 514 1.2 mi. E of Plateau Creek, 3.0 mi. W
of County RD 13/USFS RD 526. 7,700' elevation; T49N R15W S1 NE1/4;
37º40'23"N, 108º27'00"W; 18 Jun 2008; \textit{Reveal and Broome 8916} (NY, ISTC).
% Montezuma
  \textbf{Montezuma County:}
Montezuma National Forest, Montezuma Ranger Station. 7,200' elevation;
15 Aug 1918; \textit{Kauffman 973} (RM).
Montezuma Ranger Station. 7,250' elevation; 19 Jun 1914; \textit{Bowen 795} (RM).
Disappointment Ranger Station. 6,500' elevation; 17 May 1914;
\textit{Wilson 801} (RM).
San Juan National Forest; Bradfield Bridge area, off Forest RD 504 and
surrounding old gravel pit, ca. 9 air mi. SE of Dove Creek. 7,500' elevation;
T39N R16W S11 NW1/4; 21 May 1994; \textit{Moore and Judkins 1691} (RM).
Cabin Canyon Campground area, off Forest RD 504, ca. 12 air mi. SE of
Dove Creek. 6,900' elevation; T39N R16W S23 NE1/4; 21 May 1994;
\textit{Moore 1677} (RM).
Ca. 15 air mi. S of Dove Creek on the upper rim of Hovenweep Canyon,
County RD 10. 6,660' elevation; T38N R18W S19 and S18; 8 Jun 1995;
\textit{Moore 5320} (RM).
4.3 road mi. N of U.S. HWY 160 on County RD L, 7.5 road mi. NE of Cortez.
2,000m elevation; T36N R15W S15 NE1/4; 2 May 1994;
\textit{Moore, Rambo, and Defontes 1374} (RM).
Ca. 12 air mi. S of Dove Creek along County RD CC, W of Lowry Ruins.
2,000m elevation; T38N R19W S33 SE1/4, S34 S1/2; 11 May 1994;
\textit{Moore and Rambo 1559} (RM).
Ca. 17.5 air mi. S of Dove Creek, side road ca. 1/4 road mi. W of County RD Y,
on ridge between Hovenweep Canyon and Negro Canyon. 2,000m elevation;
T38N R19W S35 SE1/4; 23 May 1994; \textit{Moore 1735} (RM).
Negro Canyon area, off County RD Y, ca. 19 air mi. S of Dove Creek.
1,950m elevation; T37N R19W S1 SW1/4; 23 May 1994; \textit{Moore 1706} (RM).
Mockingbird Mesa, ca. 1.5 road mi. S of the intersection of County RD 11 and
County RD Y, ca. 18.3 air mi. S of Dove Creek. 1,900m elevation;
T37N R18W S5 NW2/3; 23 May 1994; \textit{Moore 1769} (RM).
Ca. 14 air mi. NW of Cortez in Sanstone Canyon, County RD W. 6,000-6,600'
elevation; T37N R18W S4; 2 Jun 1995; \textit{Moore 4770} (RM).
County RD 14.0 and County RD W, ca. 13 air mi. W of Cortez at the overlook
of Rock Creek Canyon N of Woods Canyon. 6,600' elevation; T37N R18W S9;
26 May 1995; \textit{Moore 4560} (RM).
Ca. 12 air mi. NW of Cortez in Woods Canyon, County RD 15 and County RD U.
6,400-6,600' elevation; T37N R18W S11; 2 Jun 1995; \textit{Moore 4803} (RM).
Hills about Dolores. 7,000' elevation; 16 Jun 1892; \textit{Crandall 53b} (RM).
McElmo Canyon Area; Yellow Jacket Canyon, 3.3 road mi. N of Ismay Trading Post;
ca. 21 air mi. W of Cortez. 1,550m elevation; T36N R19W S20 SW1/4; 26 Apr 1994;
\textit{Moore and Rambo 1305} (RM).
Ca. 12 air mi. NW of Cortez at Burro Canyon, unmarked road W of County RD N.
6,000-6,400' elevation; T36N R18W S6 and S7, T36N R19W S1; 1 Jun 1995;
\textit{Moore 4670} (RM).
Ca. 3 air mi. W of Cortez, northwest side of ridge overlooking Alkali Canyon,
Indian Camp Ranch, jeep trail W of County Road 23. 6,200' elevation;
T36N R18W S30 and S31; 26 May 1995; \textit{Moore 4528} (RM).
McElmo Canyon area, 13.5 air mi. W of Cortez; ca. 13 road mi. W of U.S. HWY 160
to trailhead of East Fork and Sand Canyon. 1,700m elevation; T36N R18W
S27 NW1/4 and S28 SE1/4; 4 May 1994; \textit{Moore and Rambo 1398} (RM).
Ca. 13 air mi. W of Cortez, County RD N, north end of Sand Canyon.
5,600-5,800' elevation; T36N R18W S13, S14 and S23; 31 May 1995;
\textit{Moore 4603} (RM).
Ca. 9 air mi. W of Cortez, ca. 1 mi. W of County RD 16, south side of County
RD N, at Sand Canyon Pueblo. 6,600-6,800' elevation; T36N R18W S12;
1 Jun 1995; \textit{Moore 4711} (RM).
Goodman Point at Hovenweep National Monument, ca. 8 air mi. NW of Cortex,
County RD P, between County RD 17 and 18. 6,700-6,800' elevation;
T36N R17W S4 and S5; 24 May 1994; \textit{Moore 4354} (RM).
Ca. 5 air mi. WSW of center of Cortez. 5,900-6,200' elevation; T35N R17W S1 S1/4
and S12 N1/4; 26 May 1995; \textit{Hartman 50638} (RM, NY).
Cortez Colorado. 6,000' elevation; 10 May 1925; \textit{Nelson 10383b} (RM).
Ca. 2 air mi. E of Cortez, NW of Montezuma County Fairgrounds, along McElmo
Creek. 6,300' elevation; T36N R15W S28; 7 Jun 1995; \textit{Moore 5260} (RM),
\textit{Moore 5257} (RM).
4.3 road mi. N of U.S. HWY 160 on County RD L, 7.5 road mi. NE of Cortez.
2,000m elevation; T36N R15W S15 NE1/4; 2 May 1994; \textit{Moore 1383} (RM).
Ca. 6 air mi. E of Mancos, north side of County RD L. 6,700' elevation;
T36N R15W S11; 7 Jun 1995; \textit{Moore 5199} (RM).
0.5 road mi. N of U.S. HWY 160 off County RD 34, ca. 8.5 air mi. E of Cortez.
2,050m elevation; T36N R14W S19 E1/4 and S20 W1/2; 14 May 1994;
\textit{Moore and Judkins 1612} (RM).
Mancos, common locally along flanks of foothills near town.
7,000' elevation; 23 Jun 1898;
\textit{Baker, Earle, and Tracy 75} (RM-168745, RM-13475, NY, RSA-POM, MO).
Sliding shale, 2 mi. E of Mancos. 14 Jul 1946; \textit{Ownbey 3025} (NY).
4 mi. E. of Mancos, between post 60-61, along HWY 160. 17 May 2000;
\textit{Atwood 25645} (NY).
Near Yucca House National Monument, top of shale slope on east side of the
highway. 6,800' elevation; UTM Zone 12S 708876E 4122452N NW; 30 Apr 2004;
\textit{Rink and Martinez 3044} (NY).
Ca. 25 air mi. WSW of center of Cortez. 5,100-5,200' elevation; T35N R20W S34
E1/4 and S35 W1/2; 27 May 1995; \textit{Hartman 50837} (RM).
Spruce Tree House, Mesa Verde National Park. 6,800' elevation; 26 May 1925;
\textit{Schmoll and Nusbaum 1566} (RM-123773, RM-105319).
Spruce Tree Camp, Mesa Verde National Park. 6,500' elevation; 29 Jun 1930;
\textit{Goodman and Hitchcock 1368} (RM).
Spruce Tree House, Mesa Verde National Park. Jul 1918; \textit{Bechel 7434} (RM).
Mesa Verde National Park, Colorado. \textit{Haas 40} (RM).
Mesa Verde Park, Colorado. 12 May 1925; \textit{Nelson 10425} (RM, NY).
Mancos Canyon, 1.1 mi. W of confluence with Ute Canyon. 1,707m elevation;
T33N R16W S34 S1/2; 5 May 1986; \textit{O'Kane, Jr. 2316} (NY).
Rim south of Mancos Canyon, ca. 1.5 mi. SE of Wing Spring, Tanner Mesa.
5,900' elevation; T32N R17W S1; 2 May 1987; \textit{Neely 3870} (RM, UTC).
% La Plata
  \textbf{La Plata County:}
8.5 mi. E of Mancos. 7,550' elevation; 4 Jul 1965;
\textit{Mulligan and Crompton 3056} (RM, NY).
Along HWY 160 0.5 mi. E of Cherry Creek Picnic Ground. 2,387m elevation;
37º19'41"N, 108º06'18"W; 21 May 1996; \textit{O'Kane, Jr. 3743} (ISTC).
Wildcat Canyon, 2 mi. W. of Durango. 6,500' elevation; 29 Jun 1974;
\textit{Stolze 1649} (F).
Durango. 6,500' elevation; 27 Jun 1898; \textit{Crandall s.n.} (NY).
Durango. 21 May 1916; \textit{Eastwood 5299} (CAS).
Southwestern outskirts of Durango. 15 May 1981;
\textit{Rollins and Rollins 8120} (RM).
2 mi. S of Durango. 6,600' elevation; 19 May 1943;
\textit{Ripley and Barneby 5326} (CAS).
Above Long Hollow near Iron Springs Gulch. 2,125m elevation;
37º07'13"N, 108º02'52"W; 16 Jun 2003;
\textit{O'Kane, Jr., Heil and Schlesser 7054} (ISTC).
% Hinsdale
  \textbf{Hinsdale County:}
Southern Gunnison Basin; ca. 0.5 air mi. WSW of Grassy Mountain along and around 
East Fork Williams Creek to ca. 0.3 air mi. N of confluence of East Fork and
Williams Creeks; ca. 5.5-7 air mi. SSW of Lake City. 9,800-12,000' elevation;
T43N R4W S30 SW1/4, S31 and T42N R4W S5 W1/4; 30 Aug 1999;
\textit{Arnett 7984} (RM).
2 mi. S of the town of Lake City. Above the city on the switchbacks, along 
CO HWY 149. Above the Packer Victim Memorial Plaque. BLM land, between 
Gunnison and Uncompahgre National Forests. 2,804m elevation; 23 Jul 1992; 
\textit{Ricketson 4626} (MO).
% Archuleta
  \textbf{Archuleta County:}
Junction of HWY 151 and Carracas road, just above the Piedra River. San Jose
Formation, clay hillside. 6,130' elevation; 37.05048ºN 107.4047ºW;
26 Apr 2005; \textit{O'Kane Jr., Heil, and Kenneth 7891} (ISTC).
% Mineral
  \textbf{Mineral County:}
Rio Grande National Forest, near Creede. 11 Jun 1939; \textit{Gierisch 947} (NY).
Steep south slope, bank of Caldwell Creek. 4 Jun 1911;
\textit{Murdoch Jr. 4542} (NY, MO, F).
% Saguache
  \textbf{Saguache County:}
Cochetopa Hills, Gunnison National Forest; northeast end of Razor Creek Park,
ca. 1.8 air mi. NW of West Baldy; ca. 3 air mi. SSW of Needle Creek Reservoir.
9,700-9,920' elevation; T47N R4E S19 SW1/4 and NW1/4 S30;
7 Aug 1999; \textit{Arnett 6548} (RM).
Near Sargents, Colorado; National Forest. 8,475' elevation; 29 May 1929;
\textit{Shoop 202} (F).
%%% UTAH STATE SPECIMENS
  \textbf{Utah:}
  \textbf{Unknown County:}
Green River Bottom. 12 Sep 1850; \textit{Stansbury s.n.} (NY).
% Daggett
  \textbf{Daggett County:}
North Slope Uinta Mountains; Ashley National Forest, ca. 7 air mi. SW of
McKinnon; ca. 1.9 road mi. S of Forest RD 221 from Forest boundary.
8,800' elevation; T2N R17E S10; 25 Jul 1994; \textit{Refsdal 2073} (RM).
Slope W of Birch Creek, on windswept slope. 9,300' elevation; T2N R17E S4 SW1/4;
19 Jul 1995; \textit{Goodrich 25185} (RM).
Uinta Mtns., ca. 15 mi. WSW Manila. 3,105m elevation; T2N R17E S11 SW1/4;
25 Aug 1983; \textit{Tuhy 1311} (UTC).
22.5 km 253 degrees SW of Manila. 9,300' elevation; T2N R17E S2 S1/4;
11 Aug 1983; \textit{Goodrich 19649} (RM).
Top of Phil Pico Mtn., W of Manila. 9,350' elevation; T3N R18E S31 NE1/4;
1 Jul 1982; \textit{Atwood and Goodrich 9080} (NY).
Ca. 5.5 air mi. W of Manila; ca. 0.2 mi. west of UT HWY 44. 7,320-7,440'
elevation; T3N R18E S24 SE1/4; 15 Jul 1995; \textit{Refsdal 5460} (RM).
2.25 mi. NW of Manila. 19 Jun 1986; \textit{Rollins and Rollins 8663} (NY).
Uinta Mtns., the loop road through the Sheep Creek Geological Area; 20.7 km
(12.9 mi.) up from the northern junction with UT HWY 44 (in Sheep Creek) and
7.6 km (4.7 mi.) from the southern junction. 7,900' elevation; T2N R19E S22;
40º53'35"N, 109º46'51"W; 18 Jun 1998;
\textit{Holmgren and Holmgren 13176} (UTC, ISTC).
Ca. 8.5 air mi. S of Manila; ca. 4.5 road mi. W of UT HWY 44 on both sides of
Forest RD 539. 7,780-7,900' elevation; T2N R19E S35 E1/2; 30 Jun 1994;
\textit{Refsdal 1390} (RM).
12.5 mi. SSE of Manila. 8,070' elevation; 20 Jun 1963;
\textit{Mulligan and Mosquin 2779} (RM).
State HWY 44, near Dowd Spring, 13 mi. N of junction with U.S. HWY 191.
19 Jun 1986; \textit{Rollins and Rollins 8657} (RM, NY).
UT HWY 44, 6 km (3.7 mi.) S of Manila (from the junction with UT HWY 43).
6,500' elevation; 1,980m elevation; T2N R20E S6;
40º56'19"N, 109º42'45"W; 18 Jun 1998;
\textit{Holmgren and Holmgren 13165} (NY, UTC, ISTC).
Utah Route 44, 5.5 km (3.4 mi) S of Manila (from the junction with Utah Route
43). 5,560' elevation; T2N R20E S6; 40º56'33"N, 109º42'47"W;
19 Jun 1999; \textit{Holmgren and Holmgren 13536} (UTC).
Sheep Creek Campground, ca. 4 mi. S of Manila. 6,100' elevation;
T2N R20E S8 SW1/4; 27 Jun 1979; \textit{Neese and Moore 7849} (RM).
UT HWY 44, 4.5 mi. S of Manila. 6,750' elevation; T2N R20E S5; 6 Jun 1971;
\textit{Holmgren and Holmgren 5056} (NY).
3.6 mi. S of Manila, along HWY 45, E side of HWY. T3N R20E S32; 6 Jun 1979;
\textit{Atwood 7275} (NY).
Near State HWY 44, 6 mi. SW of Manila. 19 Jun 1986;
\textit{Rollins and Rollins 8661} (RM, NY, UTC).
Flaming Gorge National Recreation Area; ca. 4 air mi. E of Manila.
6,080-6,280' elevation; T3N R20E S26 NW1/4; 5 Jun 1995;
\textit{Refsdal and Goodrich 3634} (RM).
Vicinity of Flaming Gorge. 5,500' elevation; 30 May 1932;
\textit{Williams 458} (RM, NY, MONT, UTC, MO).
Ca. 1 air mi. W of Flaming Gorge Dam; Cart Creek drainage into Flaming Gorge.
6,160-6,600' elevation; T2N R22E S16 S1/2; 23 Jun 1994;
\textit{Refsdal, Altenhofen, and Talbott 1115} (RM).
Dripping Spring, ca. 2 air mi. SE of Dutch John. 6,000-6,200' elevation;
T2N R23E S8; 24 May 1994; \textit{Hartman, Cramer and Refsdal 45321} (RM).
Red Creek S side of Clay Basin between Goslin Mtn. and Mountain Home. 5,740'
elevation; T2N R24E S8; 30 May 1988; \textit{Thorne and Zupan 6320A} (RM).
Brown's Park, 2.5 mi. S of Green River, mountain foot slopes, below Toliver's
Canyon. 6,800' elevation; T2N R24E S33; 29 May 1988;
\textit{Thorne and Zupan 6304} (RM, NY, RSA-POM).
Browns Park; ca. 14.7 air mi. ESE of Dutch John. 5,600-5,780' elevation;
T2N R25E S32 N1/2 and S29 S1/2; 31 May 1995; \textit{Refsdal 3392} (RM).
Brown's Park RD, steep grade into Browns Park from the north, 2.9 km (1.8 mi)
north of the John Jarvis Historic Site turnoff, which is near the Green River,
21 km (13 mi.) air distance E of Dutch John. 6,235' elevation; 1,900m;
40º55'36"N, 109º08'25"W; 28 May 2002;
\textit{Holmgren and Holmgren 14597} (NY, UTC, ISTC).
Ca. 13 air mi. E of Dutch John; 1.4 mi. N of junction of Jarvie Ranch and
Brown's Park RD on east side of Brown's Park RD. 6,100-6,300' elevation;
T2N R25E S7 W1/2; 31 May 1995; \textit{Refsdal 3438} (RM).
Ca. 12 air mi. E of Dutch John on both sides of Clay Basin RD. 6,990-7,200'
elevation; T3N R24E S36 S1/2; 5 Jul 1995; \textit{Refsdal 4859} (RM).
Clay Basin; Teepee Mountains, ca. 11.5 air mi. ENE of Dutch John. 6,600-6,760'
elevation; T3N R24E S15 SW1/4; 5 Jul 1995; \textit{Refsdal 4816} (RM).
Three Corners; ca. 20 air mi. ENE of Dutch John. 8,400-8,520' elevation;
T3N R25E S14 N1/2; 5 Jul 1995; \textit{Refsdal 4879} (RM).
About one mi. S of Flaming Gorge Dam. 2 Jun 1979;
\textit{Rollins and Rollins 79144} (RM).
Steep hillside, near Mountain Sheep Creek, about 14 mi. NW of Manila.
19 Jun 1946; \textit{Rollins and Rollins 3080} (RSA-POM).
West end of Brown's Park. 5,800' elevation; T6N R25E; 5 May 1981;
\textit{Lichvar 3957} (RM).
Frequent, dry rocky slope 3/4 mi. below Hole in Rock G. S. 8,000' elevation;
25 Jun 1953; \textit{Holmgren and Tillett 9514} (UTC).
% Summit
  \textbf{Summit County:}
Birch Creek, ca. 6 air mi. S of McKinnon, WY on east side of Forest RD 211.
6,820-7,040' elevation; T2N R17E S3 NW1/2; 2 Jun 1994; \textit{Refsdal 455} (RM).
West flank of Bald Range, ca. 4 air mi. S of Lonetree; 4.5 mi. S on Uinta County
RD 295 on east side of the road. 7,960-8,120' elevation; T3N R16E S17;
3 Aug 1995; \textit{Refsdal 6508} (RM).
Hole in the Rock, ca. 8.5 air mi. S of Lonetree; ca. 2.6 road mi. W of Hoop Lake
Campground on north side of Hoop Lake RD. 8,500-8,800' elevation;
T2N R16E S5 NE1/4; 7 Jun 1994; \textit{Refsdal 597} (RM).
Willow Park, ca. 23.5 air mi. SSE of Mountain View, WY on both sides of
Forest RD 082. 9,100-9,200' elevation; T2N R15E S3; 4 Aug 1995;
\textit{Refsdal 6586} (RM).
Ca. 1 mi. W of Upton, along UT HWY 133. 20 Jun 1967;
\textit{Murdock and Welsh 6263} (NY).
Echo, Utah. 7 May 1890; \textit{Jones s.n.} (NY).
Near Hole-in-the-Rock and entrance to Wasatch National Forest, County RD 295.
20 Jun 1986; \textit{Rollins and Rollins 8678} (RM, NY, MONTU, UTC).
Burntfork Creek. 24 Jul 1942; \textit{Jensen s.n.} (UTC).
% Salt Lake
  \textbf{Salt Lake County:}
Albion Basin, Wasatch Mts. 9,200' elevation; 5 Jul 1956;
\textit{Vickery, Jr. 665} (F).
Emigration Canyon. 17 Jul 1909; \textit{Garrett 2464} (MO).
Salt Lake. 14 Jun 1913; \textit{Garrett 2730} (NY).
Vicinity of Salt Lake City. 1 Aug 1908; \textit{Russell s.n.} (F-240726).
% Duchesne
  \textbf{Duchesne County:}
Blackhawk Mesa, 2 mi. S of Neola. T4S R17E S15; 8 Jun 1975;
\textit{Shultz 1677} (UTC).
21 mi. NW of Duchesne. 18 Jun 1981; \textit{Rollins and Rollins 81344} (NY).
6 mi. N of Duchesne, along Pioneer Canal. 5,400' elevation; T2S R5W S28 NE1/4;
6 Jun 1979; \textit{Neese and Welsh 7500} (NY-00181874, NY-00181876).
Chip rock, off U.S. HWY 40, 1 mi. E of Duchesne. 23 Jun 1983;
\textit{Rollins and Rollins 83112} (RM, NY, US).
Steep shaley-limy roadcut along HWY 40, 13.4 road mi. W of Duchense.
2,087m elevation; 40º10.896'N, 110º39.002'W; 7 Jun 2007;
\textit{Grady 137} (ISTC).
Uintah Basin, along U.S. HWY 40, 4.2 km (2.6 mi.) air distance W of downtown
Duchesne (from the junction with U.S. HWY 191). 5,800' elevation;
1,770m elevation; T4S R5W S4; 40º09'49"N, 110º27'03"W; 25 May 1999;
\textit{Holmgren and Holmgren 13465} (NY, UTC, ISTC).
Near summit E of Lake Canyon. 7,000' elevation; T4S R6W S3; 17 Jun 1980;
\textit{Neese and Welsh 8945} (RM, NY).
Steep clay bank of roadcut, 25.2 mi. W of Duchesne, Utah Route 33, Indian Canyon.
30 May 1979; \textit{Rollins and Rollins 79112} (NY).
Ridge between Bear Gulch and Bills Draw, RF Indian Drainage, 0.25 mi. N of
Gray Head Peak. 9,100' elevation; T6S R7W S7 NE1/4 SE1/4 UBM; 15 Jun 1999;
\textit{Huber 4003} (NY).
19 mi. SW of Duchesne, Indian Canyon. 7,375' elevation; T6S R7W S15 SW1/4;
4 May 1994; \textit{Huber 501} (RM).
Tavaputs Plateau, Argyle Canyon, Bad Land Cliffs. 8,700' elevation;
T11S R12E S1 NW1/4 NW1/4; 1 Jul 1998; \textit{Huber 3738} (RM).
Tavaputs Plateau, Anthro Mountain, head of Bad Land Cliffs. 9,000' elevation;
T11S R13E S5 NE1/4 NE1/4; 3 Jun 1998; \textit{Huber and Wedig 3629} (NY).
Wire Fence Canyone, Sowers drainage, UBM. 7,600' elevation;
T6S R5W S15 SW1/4 SW1/4; 10 Jun 1998; \textit{Huber and Weding 3671} (RM, NY).
Nine Mile Canyon RD between Wellington and Myton at junction with Wrinkle RD,
21 km NE of Carbon Co. line, ca. 64 km NE of Wellingtone, 48 km SE of Myton.
2,060m elevation; 39º51'42.4"N, 110º14'57.9"W; 11 Jun 2005;
\textit{Spellenberg 13631} (NY).
West Tavaputs Plateau, 11.6 km (7.2 mi.) E of the Myton-Wellington RD from the
summit above the Bad Lands Cliffs, on a pipeline road, 34 km (21 mi.) air
distance south-southeast of Myton. 7,000' elevation; 2,135m elevation;
T11S R16E S33; 39º53'53"N, 110º07'19"W; 22 Jun 1998;
\textit{Holmgren and Holmgren 13218} (NY, UTC, ISTC).
Theodor, 10 mi. S. 6,000' elevation; 19 May 1908;
\textit{Jones s.n.} (NY, RSA0010663, RSA0010647, CAS).
Above Little Mud Spring, South Fork Avintaquin drainage, West Tavaputs Plateau.
8,120' elevation; UTM, NAD83 UTM 0520951E 4419038N; 3 Jun 2003;
\textit{Huber 4545} (RM).
% Uintah
  \textbf{Uintah County:}
36 km 299 degrees from Vernal, Ice Cave Peak. 10,000' elevation; T2N R1E S5;
11 Jul 1985; \textit{Goodrich 21706} (RM).
1 mi. N of Tridell on red sandy clay hills. T1N R1E S23; 4 Jun 1980;
\textit{Atwood 7565} (NY, UTC).
23 mi. N of Vernal. 8,000' elevation; 20 Jun 1963;
\textit{Mulligan and Mosquin 2777} (NY, UTC).
20 mi. N of Vernal. 13 Jul 1938; \textit{Goodman 3101} (NY).
Off UT HWY 44, 18.8 mi. N of Vernal. 2 Jun 1979;
\textit{Rollins and Rollins 79140} (NY).
15.2 mi. N of Vernal. 6,900' elevation; T2S R22E S20; 9 Jun 1971;
\textit{Holmgren and Holmgren 5107} (RM, NY).
12 Mi. N of Vernal on State Route 44. T4N R22E; 30 May 1976;
\textit{Roberts 4582} (RM, UTC).
10 mi. N of Vernal on HWY 191. 1,692m elevation;
40º35'50"N, 109º27'53"W; 2 Jun 1996; \textit{O'Kane, Jr. 3774} (ISTC, MO).
Off UT HWY 44, 11 mi. N of Vernal. 2 Jun 1979;
\textit{Rollins and Rollins 79132} (NY).
N of Vernal. 6,600' elevation; T2S R22E S34 SW1/4; 21 Jun 1981;
\textit{Lichvar 4360} (RM).
10.5 mi. due N Maeser, N point Red Mtn. 7,400' elevation; T2S R21E S34 SW1/4;
10 Jun 1993; \textit{Huber 53} (UTC).
One mi. N of Steinaker dam, 5 mi. N of Vernal. 12 Jun 1963;
\textit{Holmgren and Reveal 14418} (NY-00182298, NY-00182256, UTC).
Stienaker Reservoir Management Area. 5,700' elevation; T3S R21E S35;
13 May 1980; \textit{Neese and White 8625} (NY).
Diamond Mountain Plateau, Jones Hole RD, 4.5 mi. W of Jones Hole Fish Hatchery,
26 airline mi. NE of Vernal. 7,600' elevation; T2S R25E S24; 8 Jun 1971;
\textit{Holmgren and Holmgren 5091} (NY).
Jones Hole. 5,900' elevation; 40º35.7'N, 109º04.1'W; 2 Jun 2002;
\textit{Dorn 9128} (RM, NY).
Blue Mtn., Iron Springs Wash, Vivas Cake Hill; 0.25 mi. S of Dinosaur National
Monument. 6,600' elevation; T4S R25E S14; 12 Jun 1982;
\textit{Neely and Thorne 908} (UTC).
18 mi. E of Vernal. 7,000' elevation; 26 Jun 1951;
\textit{Rollins and Porter 5120} (RM).
Moenkopi Shale, Split Mountain Gorge Campground, Dinosaur National Monument,
Utah. 27 May 1966; \textit{Brotherson and Yoyngberg 992} (NY).
Ca. 0.5 mi. S of 2785 E Spillway Drive, Vernal. 5,060' elevation;
40.412ºN, 109.473ºW; 30 Apr 1993; \textit{Hunting 7} (UTC).
2.4 mi. SW of Vernal, off U.S. Highway 40. 30 May 1979;
\textit{Rollins and Rollins 79120} (RM, NY, F).
Hills 5 mi. W of Vernal, on U.S. 40. 5,700' elevation; 20 Jun 1967;
\textit{Porter and Porter 10378} (RM, NY).
9 mi. SW of Vernal. T5S R20E; 22 May 1970; \textit{Cronquist 11119} (NY).
Hills east of the Green River, along Utah 264, 3 mi. SW of U.S. HWY 40, 11
airline mi. SE of Vernal. 4,800' elevation; T5S R23E S32; 12 Jun 1971;
\textit{Holmgren and Holmgren 5166} (RM, NY).
Bourdette Draw area E of Jensen. 5,183' elevation;
40º21.40'N, 109º14.60'W; 9 Jun 2005; \textit{Atwood 31257} (NY).
Miner's Draw. 6,800' elevation; T5S R25E S26; 22 Jun 1982;
\textit{Smith, Neese, and Trent 1712} (UTC).
Along Miner's Draw, shortly south of Cliff Ridge and ca. 40 km airline
east-southeast of Vernal. 1,800m elevation; T5S R25E S33; 30 May 1976;
\textit{Cronquist 11489} (NY, UTC, RSA-POM, IDS).
Ca. 12 mi. ESE of Jensen. 5,700' elevation; 40º19.6'N, 109º07.0'W;
28 May 2001; \textit{Dorn 8738} (RM).
Ca. 25 km airline N of Bonanza. 1,600m elevation; T7S R24E S1; 20 May 1976;
\textit{Cronquist 11402} (NY, UTC, RSA-POM, IDS).
Uinta Basin, 0.5 mi. SW of Squaw Ridge, Deadman Bench Divide. 5,900' elevation;
T7S R25E S21; 13 May 1982; \textit{Neely and Neese 739} (UTC).
E of Kennedy Wash, ca. 7 mi. SSE of Red Wash Townsite. 5,100' elevation;
T8S R23E S25 SW1/4; 27 May 1979; \textit{Neese and Welsh 7393} (NY).
Coyote Basin, NE drainage, 2 mi. E of junction with HWY 45. 5,800' elevation;
T8S R25E S27; 12 May 1982; \textit{Thorne and Neely 1685} (NY).
Between Walsh Knolls on the White River, 7 mi. NE of Bonanza on road to Rangely
Colorado. 27 May 1981; \textit{Shultz and Shultz 5104} (UTC).
Oil Shale Tract U-B, S of Bonanza. 5,400-5,600' elevation; T10S R24E S23 NE;
28 May 1975; \textit{McKell s.n.} (UTC).
Ca. 3.5 mi. due SSW of Rainbow, Long Canyon. 5,400m elevation; T11S R24E S3;
18 Jun 1980; \textit{Welsh 19610} (NY).
1 mi. W of Rainbow. 6,000' elevation; T11S R24E; 3 Jun 1965;
\textit{Holmgren, Reveal, and La France 1790} (NY).
3.5 mi SW of the Rainbow site. 6,100' elevation; T12S R25E S11; 25 May 1982;
\textit{Smith 1603} (RM, UTC).
Dragon. 25 May 1906; \textit{Jones s.n.} (RSA-POM-0010649).
Long's Draw. 6,000' elevation; T12S R24E S34; 26 May 1982;
\textit{Neely and Thorne 823} (UTC).
Ca. 2 mi. N of Bitter Creek. T13S R24E S6; 13 May 1977;
\textit{England 526} (UTC).
Ca. 7 air mi. S of White River. 6,300' elevation; 30 May 1982;
\textit{Neese and Fullmer 11501} (NY).
East Tavaputs Plateau, northeast end of Big Pack Mountain, 24 km (15 mi.) air
distance S of Ouray. 5,200' elevation; 1,585m elevation; T11S R20E S3;
21 Jun 1998; \textit{Holmgren and Holmgren 13204} (NY).
18 mi. S of Ouray, Uintah County. 31 May 1979;
\textit{Rollins and Rollins 79129} (NY).
Southwest side of Big Pack Mtn. 5,880' elevation; T12S R20E; 13 May 1979;
\textit{Shultz, Shultz, and Mutz 3165} (UTC).
Desolation Canyon; Green River, below Dutches' Hole. T12S R18E S11; 6 May 1999;
\textit{Atwood and Evenden 24329} (MO).
East side of Big Pack Mountain, west of Willow Creek. 31 May 1979;
\textit{Rollins and Rollins 79126} (RM, NY).
Canyon above Willow Creek; about 25 air mi. SE of Ouray, 3 mi. S Santio
Crossing. 5,680' elevation; T12S R21E S34; 27 May 1979;
\textit{Shultz and Shultz 3269} (RSA-POM, UTC).
East Tavaputs Plateau, Ouray - Rainbow RD, 28.6 mi. S of Ouray.
6,300' elevation; T12S R21E S23; 13 Jun 1971;
\textit{Holmgren and Holmgren 5203} (RM, NY, UTC).
Limy shale of the Green River Eocene Formation, eastern slope of Big Pack Mtn.,
4 mi. W of Willow Creek, Uinta Basin. 5,500' elevation; 15 Jun 1937;
\textit{Rollins 1708} (RM).
Uinta Basin, ca. 50 mi. SE of Ouray, Utah near Flat Rock Mesa.
7,200' elevation; T14S R20E S15 NW; 20 Jun 1980; \textit{Hreha UB-15} (NY).
5 mi. N of mouth of Buck Canyon, Ouray Road. 6,000' elevation;
20 Jun 1946; \textit{Rollins 3091} (RSA-POM).
Tavaputs Plateau, upper reaches of Klondike Canyon (between Willow and
Sweetwater Creeks). 12 Jun 1979; \textit{Shultz and Mutz 550} (NY).
6.6 mi. NE on King's Well RD from Seep Ridge RD intersection.
5,745' elevation; 26 May 1979; \textit{Shultz and Shultz 3222} (NY).
4 mi. S of Little Brush Creek. 8,000' elevation; 18 Jun 1934;
\textit{Harrison and Larsen 7811} (MO).
Hillside, about 25 mi. E of Vernal, between Vernal and Manila. 8,000' elevation;
26 Jun 1951; \textit{Rollins and Porter 5122} (RM, NY, UTC).
1 mi. S of ranch at Willow Creek Bridge, E face of Big Pack Mountain.
5,320' elevation; T11S R20E; 13 May 1979; \textit{Shultz and Shultz 3155} (UTC).
Vernal; Manila, road N of Vernal. 6,500' elevation; 19 Jun 1933;
\textit{Graham 8171} (MO).
% Carbon
  \textbf{Carbon County:}
Desolation Canyon; Green River, at Peters Point. T12S R17E S26; 6 May 1999;
\textit{Atwood and Evenden 24339} (NY).
Green River, Jacks Creek. T13S R17E S15; 6 May 1999;
\textit{Atwood and Evenden 24366} (NY).
Green River, Rock Creek. T15S R17E S5; 7 May 1999;
\textit{Atwood and Evenden 24398} (NY).
Green River, Three Canyon. T15S R17E S29; 8 May 1999;
\textit{Atwood and Evenden 24409} (RM).
CC Range Creek, 10 mi. SW of junction of Nine Mile Canyon and Gate Canyon.
8,500' elevation; T13S R14E S22; 22 Jun 1977;
\textit{Welsh and Taylor 15108} (NY).
Sunnyside, Utah. 15 Sep 1901; \textit{Jones s.n.} (RSA-POM-094220).
CC Price River, 12 mi. due NE of Wellington. 6,600' elevation;  T13S R12E S34;
21 Jun 1977; \textit{Welsh 15082} (NY).
Headwaters, L. Fk. Minnie Maude Creek, W Tavaputs Plateau, 20 mi. N of
Wellington. 9 Jun 1940; \textit{Maguire 18516} (RM, UTC).
About 2 mi. S of Helper, Utah, found on a Mancos shale outcrop along U.S. HWY
50-6. 29 Apr 1965; \textit{Walker 505} (NY).
4 mi. N of Price on slopes in gravelly soil. 11 May 1940;
\textit{Maguire 18337} (NY, UTC).
Scattered plants on bare, rolling clay hills adjacent to the northern limits of
Price. 5,700' elevation; 10 Jun 1963; \textit{Mulligan and Mosquin 2721} (NY).
Price, Utah. 5,500' elevation; 12 Jun 1900; \textit{Stokes s.n.} (NY).
% Grand
  \textbf{Grand County:}
Westwater, Colo. 18 May 1891; \textit{Jones s.n.} (RSA-POM-0010664).
CC Book Mtn., East Canyon, Westwater Canyon. 7,800' elevation; T16S R25E S5;
8 Jun 1977; \textit{Welsh 14874} (UTC).
San Arroyo Canyon; ca. 20 mi. NNW of Westwater. 1,850m elevation; T16S R25E S32;
12 May 1997; \textit{Atwood and Welsh 22023} (RM, NY).
San Arroyo Canyon. 1,700m elevation; T17S R25E S18; 12 May 1997;
\textit{Atwood and Welsh 22029} (NY).
Bookcliff Ridge RD above Westwater Wash. 1,554m elevation;
39º13'00"N, 109º09'20"W; 16 May 1999; \textit{O'Kane, Jr. 4540} (ISTC).
Cisco, Utah. 2 May 1890; \textit{Jones s.n.} (RSA-POM).
Between mi. 2-5 up Thompson Canyon from junction with Sego Canyon.
5,600-6,000' elevation; T20S R20E S20 and S17; 28 May 1997;
\textit{Atwood 22284} (NY).
Book Cliffs, Sego Canyon near Sego Ghost Town. 14 Jun 1985;
\textit{Anderson 85-75} (UTC).
CC Book Mtn., Sego. 5,600' elevation; T20S R20E S26; 11 May 1977;
\textit{Welsh, Taylor, Neese, and White 14743} (NY).
Sego, ca. 2 mi. N of Thompson. 7 May 1965; \textit{Welsh 4019} (NY).
Bluffs along Colorado River, ca. 10 mi. S of U.S. HWY 50-6, along UT HWY 128.
2 May 1968; \textit{Welsh 6972} (NY).
Dewey Bridge. 4,600' elevation; T23S R24E S7; 18 May/1986;
\textit{Welsh 23759} (RM, NY).
Colorado River, 1.3 mi. below Dewey Bridge. 4,200' elevation; T23S R24E S18;
10 May 1994; \textit{Dorn 5528} (RM).
Seven-Mile Mesa, 14.1 mi. SE of Dewey Bridge. 5,780' elevation; 9 Jun 1976;
\textit{Shultz, Shultz, Holmgren, and Lowrey 1923} (UTC, NY).
Rocky hillside 20 mi. NW of Moab. 9 May 1933; \textit{Williams 5968} (MO).
Near Skyline Arch, Arches National Monument. 13 May 1955;
\textit{Anderson 58} (UTC).
Colorado River Canyon 16.6 road mi. W of Dewey Bridge along HWY 128.
1,242m elevation; 38º40.943'N, 109º28.553'W; 5 Jun 2007;
\textit{Grady 122} (ISTC).
Mat Marin Point above Colorado River. 1,550m elevation; T24S R22E S34 NW1/4;
10 May 1986; \textit{Thorne, Chandler, and Franklin 4586} (NY).
Slopes on a bluff west of the low summit on the Castle Valley Road (2.4 km or
1.5 mi. from State Route 128). 4,625' elevation; 1,425m elevation; T25S R23E S6;
38º39'55"N, 109º25'10"W; 19 May 2001;
\textit{Holmgren and Holmgren 14222} (UTC, NY).
NE of Moab, Richardson Amphitheater. 4,480' elevation; T24S R23E S21 SE;
19 Apr 1986; \textit{Franklin 2716} (RM).
Near Castleton, Utah. 11 May 1945; \textit{Rattle s.n.} (UTC).
Fisher Mesa, NE of Moab. 7,200' elevation; T25S R24E S4; 24 May 1986;
\textit{Franklin 3237} (RSA-POM, MONTU).
LaSal Mountains, LaSal Mountain Loop Road, overlooking Castle Valley, 18.8 km
(11.7 mi.) SE of UT HWY 128 (Colorado River), 20 km (12.5 mi.) airline
distance E of Moab. 6,300' elevation; 1,920m elevation; T25S R23E S36;
8 Jun 1977; \textit{Holmgren and Holmgren 8385} (NY, UTC).
E of Moab, Porcupine Rim. 6,320' elevation; T25S R23E S33; 10 May 1986;
\textit{Chandler, Thorne, and Franklin 2864} (RM, NY).
Porcupine Rim, near head of Negro Bill Canyon. 6,320' elevation; T25S R23E S33;
10 May 1986; \textit{Thorne, Chandler, and Franklin 4560} (MONTU).
Porcupine Rim. 6,000' elevation; T25S R22E S24; 10 May 1986;
\textit{Thorne, Chandler, and Franklin 4569} (RM).
Dry rocky hillside, W of Moab. 4,200' elevation; 16 Apr 1933;
\textit{Williams 5979} (MO).
Arches National Monument, 5 mi. NW of Moab. 4,500' elevation; T25S R21E S21;
21 May 1984; \textit{Atwood, Goodrich, and Frates 9698} (RSA-POM, NY).
Ca. 5 mi. NW of Moab. 4,900' elevation; T25S R20E S18; 3 May 1983;
\textit{Neese 13181} (NY).
10 mi. N of Dead Horse Point. 11 Jun 1955;
\textit{Holmgren, Anderson, and Witte 10930} (UTC, NY).
Collected two mi. up road to Dead Horse Point, growing along intermittent stream
coming out of Box Canyon, Utah. 4,700' elevation; 18 May 1957;
\textit{Vickery, Jr. 749} (RM).
Canyonlands National Park; Island in the Sky District, White Rim Trail,
Gooseneck Overlook. 4,400' elevation; T27S R19E; 21 Apr 1991;
\textit{Floyd-Hanna 1} (RSA-POM).
Southeast of Moab, along La Sal Mountain Loop Road, 3.5 mi. from its junction
with Sand Flats Road. 7,770' elevation; 16 May 2002; \textit{Hufford 3935} (RM).
La Sal Mountains. 12 Jun 1913; \textit{Jones s.n.} (NY-00281169).
La Sal Mountains. 13 Jun 1913; \textit{Jones s.n.} (NY-00182277).
Western slope of La Sal Mountains, near Little Springs. 2,000-2,200m elevation;
5 Jul 1911; \textit{Rydberg and Garrett 8573, 8574} (NY).
% San Juan
  \textbf{San Juan County:}
Low summit along highway, 13 mi. S of Moab.
5,600' elevation; T27S R23E S30; 12 May 1961;
\textit{Cronquist 8995} (NY-00281167, NY-00281168, UTC).
12 mi. E of La Sal. 5 Jun 1985; \textit{Barneby 18009} (NY).
Near the Needles Overlook, 34.9 km (21.7 mi.) W of U.S. HWY 191.
6,465' elevation; 1,970m elevation; T29S R20E S20; 7 Jun 1997;
\textit{Holmgren and Holmgren 12757} (NY, UTC, ISTC).
Lockhart Basin RD, near Indian Creek Campground, 3.5 km (2.2 mi.) north of
the Needles RD (UT HWY 211). 4,760' elevation; 1,450m elevation;
T30S R20E S1; 38º12'21"N, 109º40'27"W; 8 Jun 1997;
\textit{Holmgren and Holmgren 12767} (NY, UTC).
Ca. 1 mi E of South Sixshooter Peak. 5,040' elevation; T31S R21E S5;
10 May 1982; \textit{Welsh, Welsh, and Chatterley 21077} (NY).
Rone Bailey Mesa, ca. 10 mi. S and 10 mi. W of La Sal. 1,900m elevation;
T30S R23E S19; 4 Jun 1985; \textit{Welsh and Neese 23502} (RM, NY).
14 mi. due SW of La Sal, summit of Rone Bailey Mesa. 1,900m elevation;
T30S R23E S19; 4 Jun 1985; \textit{Neese and Welsh 16991} (NY).
Roan Bailey Mesa about 15 mi. N of Monticello on HWY 163, west of the highway.
T30S R23E; 13 Jun 1986; \textit{Grimes and Meurer-Grimes 2958} (NY).
Ten mi. N of Monticello. 4,500' elevation; 7 May 1933;
\textit{Harrison 5892} (MO, UTC).
Ucolo-Libson Valley RD, 33.6 km (20.9 mi.) N of U.S. HWY 666.
6,860' elevation; 2,090m elevation; T31S R26E S18; 15 Jun 1993;
\textit{Holmgren and Holmgren 11888} (NY, UTC, ISTC).
Gypsum Canyon. 4,100-4,400' elevation; T32S R17E S23 SW1/4 and S22 NE1/4;
29 Apr 1987; \textit{Tuhy, Holland, and Ferguson 2996} (UTC).
Dark Canyon, La Sal forest. 7,000' elevation; 15 Jun 1916;
\textit{Locke 91} (RM).
Head of Gravel Canyon in coarse gravelly soil. 6,030' elevation; 12 Apr 1966;
\textit{Wilson 237} (UTC).
47 mi. S of Moab. 25 Jun 1933;
\textit{Maguire, Richards, Maguire, and Hammond 5809} (RM, UTC).
Abajo Mtns.; 1.95 mi. 58ºW of Mt. Linnaeus summit on ridge W of Dry Wash across
from cliff dwellings. 8,050' elevation; T34S R21E; 15 Jun 1985;
\textit{Morefield 2746} (NY).
Clay bank, near UT HWY 163, 3.7 mi. S of Monticello. 17 May 1981;
\textit{Rollins and Rollins 8142} (NY, US).
Mesa south of Devil's Canyon, 10 mi. N of Blanding. 5,500' elevation;
1 Jul 1932; \textit{Maguire and Redd 1837} (UTC).
14.8 mi. S of Monticello. 17 May 1981; \textit{Rollins and Rollins 8149} (RM, NY).
Coal Bed Canyon. 1,725m elevation; 37º39'29"N, 109º09'39"W; 16 May 1996;
\textit{O'Kane, Jr., Anderson, and Heil 3706} (ISTC).
Bottom of gulch SW of Blanding. 7 May 1933;
\textit{Harrison and Larsen s.n.} (MO-1040163).
5.5 mi. SE of Blanding, slopes adjacent to Recapture Creek. 1,700m elevation;
T37S R23E S19; 3 May 1985; \textit{Neese 16630} (NY).
Canyon 20 mi. W of Blanding. 7 May 1933; \textit{Harrison s.n.} (MO-1040353).
Gravelly soil of wash entering Comb Wash from Arch Canyon, 18 mi. SW of Blanding.
5,000' elevation; T37S R20E S25; 3 May 1961; \textit{Cronquist 8954} (NY, UTC).
Near UT HWY 95, 25 mi. W of Blanding. 17 May 1981;
\textit{Rollins and Rollins 8156} (NY).
Fish Canyon, 2.5 airmiles S of HWY 95. 5,600' elevation; T38S R19E S2;
4 Jun 1983; \textit{Neely and Carpenter 1242} (UTC).
Off UT HWY 276, Hall's Crossing Road; 5 mi. from junction of HWY 276 and 95.
5,700' elevation; T38S R17E S6; \textit{Neely 2599} (RM, UTC).
Valley of the Gods; first butte south of Castle Butte on the W side of the
valley road, ca. 3.2 air mi. E of UT State HWY 126, ca. 11 mi. N of Mexican Hat.
5,100' elevation; T40S R19E S18 S1/2 NE1/4; 21 May 1998;
\textit{Fertig and Welp 18117} (UTC).
Colrado Plateau; Cedar mesa, cliffs along south side of Road Canyon, ca. 5.5 air
mi. E of UT HWY 126, ca. 5.5 air mi. N of Castle Butte. 5,500' elevation;
T39S R19E S17 NW1/4 SE1/4; 19 May 1998; \textit{Fertig and Welp 18097} (UTC).
Valley of the Gods, base of Red Mesa of HWY 261, 6.4 road mi. N of junction with
HWY 191. 1,614m elevation; 37º15.865'N, 109º55.942'W; 4 Jun 2007;
\textit{Grady 116} (ISTC).
Salt Brush desert, 5 mi. NW Mexican Hat. 21 Apr 1936;
\textit{Maguire 20362} (UTC).
San Juan River, above mouth of side canyon at 71.7 river mi. below Sand Island
boat launch area. 3,940' elevation; 1,200m elevation; T40S R15E S23;
15 Jun 1991; \textit{Holmgren and Holmgren 11504} (NY, UTC).
Summit of Raplee Ridge, north of microwave tower. 1,707m elevation;
37º9'16"N, 109º48'47"W; 11 May 1999; \textit{O'Kane, Jr. 4530} (MO).
San Juan River, wash below Mule Ear Diatreme, entering the river at 9.3 river mi.
(15 km) below Sand Island Boat Launch (near Bluff), left bank. 4,400' elevation;
1,340m elevation; T41S R20E S33; 9 Jun 1993;
\textit{Holmgren and Holmgren 11869} (NY).
Rimrocks 15 mi. S of Blanding. 8 May 1944; \textit{Holmgren 3135} (UTC).
North end of Montezuma Creek, ca. 6 mi. E of U.S. 47. 8 May 1969;
\textit{Welsh, Atwood, and Higgins 8913} (NY).
Rocky north slopes of divide into Monument Valley. 19 May 1944;
\textit{Holmgren 3238} (NY, UTC).
Copper Canyon, 8 mi. NW of Oljato Post. 16 Jun 1938; \textit{Cutler 2253} (MO).
42 mi. N of Monticello, 9 mi. N of La Sal jct. 15 Jun 1944;
\textit{Holmgren and Hansen 3333} (NY, UTC).
Manti-La Sal N.F., Bears Ears Pass, SE of Abajo Mtns. T36S R19E; 14 Jun 1980;
\textit{Neely and Stockton 31} (UTC).
Lower Blue Notch Canyon. 5,200' elevation; 29 Mar 1966; \textit{Wilson 217} (UTC).
Growing in loam soil at Unnamed Mesa directly behind Wedding Cake Knob.
6,800' elevation; 26 Jun 1965; \textit{Wilson 78} (UTC).
