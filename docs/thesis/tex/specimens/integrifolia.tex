\section*{\textit{Physaria integrifolia}}

%%% WYOMING STATE SPECIMENS
  \textbf{Wyoming:}
  \textbf{Uncertain Locality:}
1.5 mi. W of Station. 8,300' elevation; 17 Jul 1930; \textit{Fest F-31} (RM).
Head Waters Cliff Creek. 9,000' elevation; 9 Aug 1900;
\textit{Curtis s.n.} (NY, NY).
  \textbf{Wyoming: Teton County:}
Teton Peaks. \textit{Davis s.n.} (NY, IDS-0030427).
Jackson Hole. 6 Jul 1860; \textit{Hayden s.n.} (MO-3833625).
Teton National Forest, 2 mi. above the mouth of Game Creek. 6,400' elevation;
11 Jul 1932; \textit{Buchenroth 159} (NY).
Infrequent in rocks on switchbacks up to "The Wall" in South Cascade Canyon.
5 Sep 1955; \textit{Shaw 1011} (UTC).
Plants of Grand Teton National Park, possibly on Mt. Woodring.
\textit{Woodring s.n.} (MO-1066430).
On ridge west of The Wall, N of Alaska Basin, Targhee National Forest.
10,500' elevation; 3 Aug 1955; \textit{Anderson 230} (NY, UTC).
Grand Teton National Park, 1/2 mi. E of Mount Hunt Divide Trail.
9,500' elevation; 11 Jul 1965; \textit{Shaw 1457} (UTC).
Teton Range East Slope; Rendevzvous Mountain, just below ridge between
Granite Canyon and Jackson Hole, ca. 3/4 mi. W of Apres Vous Peak, ca. 1.5 air
mi. NW of Teton Village; ca. 8 air mi. NW of Jackson. 9,200-9,500' elevation;
T42N R117W S14; 31 Jul 1996; \textit{Markow 11323} (RM).
Targhee National Forest, West Slope Teton Range; ca. 6-7 air mi. E of Victor,
Idaho. 8,400-9,400' elevation; T42N R118W S10, S11, S12, S14, and S15;
19 Aug 1991; \textit{Markow 6550} (RM).
Targhee National Forest, Teton Range; E-facing slope of Taylor Mtn, ca. 100 yds.
below summit ridge, ca. 12 air mi. NW of Jackson. 9,800-10,200' elevation;
T41N R118W S11; 25 Sep 1995; \textit{Markow 11199} (RM).
Targhee National Forest, Northwest Slope Snake River Range; Mail Cabin Creek,
ca. 11 air mi. W of Jackson. 7,600-8,000’ elevation; T41N R118W S21 and S22;
22 Jul 1991; \textit{Markow 3781} (RM).
In Pumice formation on hills near Adam's Ranch, Jackson. 6,200' elevation;
T40N R117W S4; 10 Jul 1901; \textit{Merrill and Wilcox 960} (RM).
Teton National Forest, big point ending in Snake River below Fall Creek.
6,000' elevation; T39N R116W S27; 12 Jul 1928; \textit{McDonald 816} (RM).
Gros Ventre Area, 1.5 air mi. S to 1 air mi. SE of Pinnacle Peak.
9,200-10,000' elevation; T39N R114W S3 NW1/4 and S4 SE1/4; 7 Jul 1994;
\textit{Hartman 47330} (RM, RSA-POM).
Hillside, ca. 3.5 mi. E of Kelly Warm Springs. 2,053m elevation;
43º38'02"N, 110º33'57"W; 25 Jun 1996; \textit{Salywon and Dierig 3146} (ISTC).
Gros Ventre Fork. 10 Jun 1860, \textit{Hayden s.n.} (MO-3833631).
Slide at Gros Ventre Lake. 7,000' elevation; 10 Sep 1951;
\textit{Munz 16997} (NY, RSA-POM).
Gros Ventre River. 16 Aug 1894; \textit{Nelson 927} (RM, MO).
Gros Ventre slide area, Teton National Forest. 7,000' elevation; 10 Jul 1959;
\textit{Porter and Porter 7891} (UC, CAS).
Gros Ventre Slide Road, 1/4 mi. W of Forest Service Exhibit on Gros Ventre
Slide. 6,800' elevation; 8 Jul 1971; \textit{Shaw 1812} (UTC).
Lower Slide Lake. 7,100' elevation; T42N R114W S5; 24 May 1977;
\textit{Lichvar 99} (RM).
Mount Leidy Highland Area, Gros Ventre River Road, ca. 2 air mi. E of Forest
boundary; just W to overlooking Lower Slide Lake, on the north side.
6,900-7,400' elevation; T42N R114W S5; 26 Jun 1995;
\textit{Hartman 51256} (RM).
Bridger-Teton Natlional Forest; 1/3 mi. E of Atherton Creek Campground, Gros
Ventre Canyon RD. 7,000' elevation; 18 Jun 1979; \textit{Shaw 2482} (UTC).
Crystal Creek. 6,950' elevation; R113W T42N S18; 7 Jul 1977;
\textit{Lichvar 691a} (RM).
Teton Forest - Miner Creek. 8,000' elevation; T42N R113W S17; 20 May 1913;
\textit{Maris 58} (RM).
Mount Leidy Highland Area, Gray Hills; N of Gros Ventre River, ca. 2 air mi.
NW of west end of Uppper Slide Lake, ca. 21 air mi. ENE of Jackson.
7,600' elevation; T42N R113W S14; 6 Jul 1990; \textit{Nelson 19288} (RM).
Gros Ventre Wilderness Area; ridge E of Crystal Creek, ca. 12 air mi
SE of Kelly. 7,200-8,000' elevation; T42N R113W S34; 24 Jun 1994;
\textit{Hartman 46465} (RM).
Mount Leidy Highland Area, Gray Hills; ridge and adjacent area overlooking
Slate Creek, ca. 4.5 air mi. NW of west end of Upper Slide Lake. 7,900-8,100'
elevation; T42N R113W S35; 6 Jul 1990; \textit{Hartman 26343} (RM).
Gros Ventre Area, Gros Ventre Wilderness Area; ridge E of Crystal Creek,
ca. 14 air mi. SE of Kelly. 8,600-9,450' elevation;
T41N R113W S1, S2, and S13; 24 Jun 1994; \textit{Hartman 46566} (RM).
Mount Leidy Highland Area, Burnt Creek and ridge to east.
7,200-8,200' elevation; T42N R112W S30 and S31; 24 Jul 1995;
\textit{Hartman 52772} (RM, RSA-POM).
Gros Ventre Area; upper Gros Ventre River Road, ca. 1.5 air mi. SE of
Goosewing Guard Station. 7,100' elevation; T41N R112W S3 NE1/4; 26 Jun 1994;
\textit{Hartman 46738} (RM).
West Slope Wind River Range; hills on eastern bank of Cottonwood Creek, ca.
28 air mi. E of Jackson. 7,600-8,280' elevation; T42N R111W S29, T42N R111W
S30 and S31, and T42N R112W S36; 6 Jul 1990; \textit{Fertig 3025} (RM).
West Slope Wind River Range; Fish Creek/Moccasin Basin area, along Fish Creek
just below confluence of North and South Forks, ca. 28.5-29.5 air mi. ENE of
Jackson. 7,600-7,700' elevation; T42N R111W S28 and S29; 25 Aug 1990;
\textit{Nelson 20322} (RM).
West Slope Wind River Range; South Fork Fish Creek between Hackamore and Bell
Creeks. 7,800-7,850' elevation; T42N R111W S36; 25 Aug 1990;
\textit{Hartman 28350} (RM).
N Wind River Range; N bank of South Fork Fish Creek just W of confluence of
Devils Basin Creek, ca. 4.25 air mi. NNW of Union Pass RD. 8,000' elevation;
T41N R110W S8 NW1/4 SE1/4; 11 Aug 1995; \textit{Fertig 16262} (RM).
West Slope Wind River Range; Fish Creek / Moccasin Basin Area, North Fork Fish
Creek between Packsaddle Creek and Harness Gulch. 7,800-7,900' elevation;
T42N R111W S11 and S15; 12 Jul 1990; \textit{Hartman 27148} (RM).
West Slope Wind River Range; North Fork of Fish Creek, ca. 32 air mi. NE of
Jackson. 8,000-8,530' elevation; T42N R111W S2; 11 Jul 1990;
\textit{Fertig 3535} (RM).
West Slope Wind River Range; Cottonwood and Moosehorn Creeks.
8,100-8,300' elevation; T42N R111W S5 and T43N R111W S32; 23 Aug 1990;
\textit{Hartman 28213} (RM).
Mount Leidy Highland Area; Grouse Mountain, southern ridge.
8,800-9,800' elevation; T43N R112W S2 and NE1/4 S11; 15 Aug 1995;
\textit{Hartman 53644} (RM).
Mount Leidy Highland Area; summit and upper slopes of Mount Leidy proper.
9,800-10,200' elevation; T43N R113W S3 and NW1/4 S3; 13 Aug 1995;
\textit{Hartman 53435} (RM).
Mt. Leidy. 10,000' elevation; \textit{Tweedy 391} (NY).
Mount Leidy Highland Area; Spread Creek, ca. 5 air mi. SW of Black Rock Ranger
Station. 7,300-7,500' elevation; T44N R113W S16; 14 Jul 1995;
\textit{Hartman 51830} (RM).
Mount Leidy Highland Area; 0.5-1 air mi. NW of Gunsight Pass on flank of
plateau. 9,100-9,200' elevation; T42N R112W S10 W1/2; 12 Aug 1995;
\textit{Hartman 53278} (RM).
Steep shale cliffs above Spread Creek. 7,500' elevation; T44N R113W S19;
13 Jun 1948; \textit{Reed and Reed 2302} (RM).
1.5 mi. E of Elk Ranch Reservoir. 7,300' elevation; T44N R114W S3; 22 Jun 1971;
\textit{Dorn 1281} (RM).
Grand Teton National Park and Vicinity, Jackson Hole; "Wolff Ridge" on the N
side of Spread Creek Valley, ca. 3.5 air mi. SW of Moran; ca. 25 air mi.
NE of Jackson. 6,900-7,160' elevation; T44N R114W S9 E1/2; 15 Jun 2006;
\textit{Nelson and Scott 68918} (NY).
  \textbf{Lincoln County:}
Targhee National Forest, West Slope Snake River Range; ridge leading to
Observation Peak, ca. 8 air mi. N of Alpine Junction. 9,600-10,000' elevation;
T38N R118W S13 and S14; 1 Aug 1991; \textit{Markow 4616} (RM).
Grand Canyon of the Snake River, 6.5 mi. E of the Targhee National Forest
Boundary. 5,744' elevation; 13 Jun 2014;
\textit{Ratcliff and O'Kane, Jr. 44} (ISTC).
Grand Canyon of Snake River. 5,500' elevation; T38N R116W S29; 8 Jul 1932;
\textit{Williams 809} (RM, MO).
Salt River Range; Star Peaks, peak 9,988 and slope immediately to W.
9,000-9,988' elevation; T35N R118W S23 E1/2; 22 Aug 1992;
\textit{Hartman 36731} (RM).
Man Peak, S of summit to 0.5 mi; ca. 9 air mi. E of Thayne.
9,400-9,900' elevation; T34N R117W S5; 16 Aug 1992; \textit{Hartman 35954} (RM).
Roadcut, 1.1 mi. N of Deadman Creek, FS RD 10138, ca. 12 air mi. E of Freedom.
1,986m elevation; 42º57'39"N, 110º44'37"W; 26 Jun 1996;
\textit{Salywon and Dierig 3148} (ISTC).
Wyoming Range; Mount McDougal between Peaks 10,682 and 10,742 (feet) and N top of
ridge overlooking Gunsight Pass; 2-3 air mi. NNW of McDougal Gap. 10,200-10,682'
elevation; T34N R115W S29 and S32; 19 Aug 1992; \textit{Hartman 36317} (RM).
McDougal Gap. 8,600' elevation; T33N R115W S8; 18 Aug 1982;
\textit{Dorn 3822} (NY).
11 mi. N of Afton. 6,000' elevation; 26 Jul 1957;
\textit{Rollins 57251} (NY, US, MO, UC).
Phillips Creek, ca. 5 air mi. N of Afton. 6,280-7,380' elevation;
T33N R118W S27, S33 and S34; 19 Jun 1992;
\textit{Hartman and Embury 32454} (RM).
Grover Park, 1/2 mi. above Grover town. 7,000' elevation; T32N R118W S6;
7 May 1930; \textit{Chas, McDonald, and Harrison 510} (RM).
Above Lake Barstow on exposed ridge, E side near summit.
10,100' elevation; T32N R117W S7 SE1/4; 5 Aug 1979;
\textit{Shultz and Shultz 3722} (GH, UTC-00254382, UTC-00232456).
North Fork Swift Creek on E side of canyon, ca. 4 air mi. NE of Afton.
7,200-7,700' elevation; T32N R118W S15 NW1/4; 19 Jul 1992;
\textit{Fertig 13086} (RM).
Swift Creek Canyon, north side of Scout Canyon bridge.
6,500' elevation; T32N R118W S29; 27 Jun 1950; \textit{Call and Call 385} (RM).
Slopes on west side of Swift Creek, ca. 2.5 air mi. NE of Afton.
6,800' elevation; T32N R118W S21; 19 Jul 1992; \textit{Fertig 13075} (RM).
1/2 mi. E of Afton. 6,400' elevation; 10 Jul 1965;
\textit{Mulligan and Mosquin 3088} (UC, NY).
East of Afton, roadcut along Swift Creek Canyon 0.1 mi. E of forest boundary.
6,461' elevation; 42º43.5424'N, 110º54.5635'W \pm 28 ft.;
13 Jun 2014; \textit{Ratcliff and O'Kane, Jr. 43} (ISTC).
Star Valley-Greys River in the Afton area, 0.4 mi. E of Afton along Swift Creek
at North Lower Reservoir. 6,350' elevation; T32N R118W S29; 20 Jun 1975;
\textit{Harrison 160} (RM).
Hills E of Afton. 6,500' elevation; 28 Jun 1923;
\textit{Payson and Armstrong 3825} (MO, RSA-POM).
Swift Creek Canyon, Star Valley. 6,600-6,700' elevation; T32N R118W S28;
24 Jun 1932; \textit{Rollins 239} (RM, MO).
Gannett Hills, 1-2 air mi. E of Idaho, ca. 6 air mi. SW of Afton.
6,250-7,000' elevation; T31N R119W S16 and S17; 8 May 1992;
\textit{Hartman and Embury 31962} (RM).
Gannett Hills, limestone exposure above Second Creek. T30N R119W S9 SW;
22 Jun 1979; \textit{Shultz 560} (UTC).
Lower Cottonwood Creek E of Smoot. 19 Jun 1969;
\textit{Barneby 15087} (NY, CAS).
East of Smoot. 6,800' elevation; T30N R117W S3; 1 Jun 1982;
\textit{Lichvar 4814} (RM).
East of Smoot. 6,700' elevation; T30N R117W S4; 8 Jun 1981;
\textit{Lichvar 4329} (RM).
Shale Creek drainage, E slope of the Wyoming Range. 8,100' elevation;
T30N R116W S17 SE1/4; 27 Jun 1979; \textit{Shultz and Shultz 3443} (UTC).
Shale Creek ca. 1.5 mi. SW of summit of Wyoming Peak, ca. 16.5 air mi. SE of
Afton. 8,600-8,800' elevation; T30N R116W S16; 26 Jul 1992;
\textit{Fertig 13171, 13172} (RM).
E face of ridge just S of Wyoming Peak, Wyoming Range. 10,300' elevation;
T30N R116W; 24 Aug 1978; \textit{Shultz and Shultz 2972} (GH, UTC).
Ca. 1.8 mi. due east of ID-WY state line. As confluence of Shale Hollow with
Salt Canyon on talus slope. 6,400' elevation; T28N R119W S19 SE and S20 SW;
1 Jul 1986; \textit{Franklin 3667} (RM, NY).
Salt River Range: along Smiths Fork at and above the mouth of Hobble Creek,
ca. 20 air mi. N of Cokeville; ca. 44 air mi. NNW of Kemmerer.
6,950-7,400' elevation; T28N R118W S27 SW1/4 S27; also E1/4 S28; 30 Jun 1995;
\textit{Nelson 35958} (RM).
Southern Sublette Range: ca. 7 air mi. N of Cokeville. 6,140-7,600' elevation;
T25N R119W S4 S1/4, also S8 NE1/4 and S9 N1/4; 28 Jun 1994;
\textit{Cramer and Kellett 1227} (RM).
Steep shaley road-cut, U.S. HWY 30 and 89, 7 mi. S of the Idaho State Line.
6,200-6,300' elevation; T25N R119W S21; 22 Jun 1986;
\textit{Rollins 8683} (RM, NY, GH, UTC).
Underwood Canyon. 7,000' elevation; T23N R118W S6 NE1/4; 30 May 1982;
\textit{Lichvar 4813} (RM).
West slope of Rock Creek Ridge, ridges on N and E side of southern tributary of
Horse Creek, ca. 1.3-1.5 mi. N of North Fork Leeds Creek, ca. 4.5 mi. E of U.S.
HWY 30. 7,000-7,200' elevation; T23N R119W S24 SW1/4 NE1/4, N1/2 SE1/4,
and E1/4 SE1/4 SE1/4 and T23N R118W S19 W1/2 SW1/4; 17 Jun 1997;
\textit{Fertig 17505} (RM).
West slope of Rock Creek Ridge, on ridge on N side of North Fork Leeds Creek,
ca. 4 air mi. E of U.S. HWY 30 and ca. 8 air mi. SE of Cokeville.
6,800-7,100' elevation; T23N R119W S25 S1/2 SE1/4 SW1/4; 17 Jun 1997;
\textit{Fertig 17502} (RM).
Rock Ridge, N of Haystack Peak. 7,200' elevation; T23N R118W S30 SW1/4;
5 Jul 1982; \textit{Lichvar 5206} (RM).
Southern Tunp Range, W slope of Rock Creek Ridge, S side of ridge on N side of
southern tributary of North Fork of Leeds Creek, ca. 3.6 mi. E of U.S. HWY 30.
6,700-6,900' elevation; T23N R119W S36 N1/2 SW1/4; 8 Jun 1997;
\textit{Fertig 17493} (RM).
North-South trending ridge between South Fork of Leeds Creek and the southern
branch of the North Fork of Leeds Creek, ca. 3.5 air mi. E of U.S. HWY 30.
6,600-6,900' elevation; T22N R119W S2 S1/2 NE1/4, NE1/4 SE1/4 NE1/4,
S1 W1/2 NW1/4; T23N R119W S36 SW1/4 SW1/4; 8 Jun 1997;
\textit{Fertig 17491} (RM).
West side of Rock Creek Ridge at base of north-south running ridge on north
side of triutary of Antelope Creek, ca. 3 air mi. E of U.S. HWY 30.
6,620' elevation; T22N R119W S11 NE1/4 NE1/4 NW1/4; 8 Jun 1997;
\textit{Fertig 17489} (RM).
Rock Creek Ridge, ridge along upper reaches of Antelope Creek. 7,000-7,500'
elevation; T22N R119W S13; 27 Jun 1992; \textit{Hartman 33607} (RM).
Rock Ridge. 6,900' elevation; T22N R119W S14 NE1/4; 6 Jul 1982;
\textit{Lichvar 5209} (RM, NY).
Rock Ridge. 7,000' elevation; T23N R118W S24 SW1/4; 5 Jul 1982;
\textit{Lichvar 5205} (NY, GH).
West end of Fossil Ridge, ca. 16 air mi. W of Kemmerer; 7.3 mi. SW on Northwest
Pipeline access road. 7,400-7,760' elevation; T21N R118W S29 N1/2; 10 Jul 1995;
\textit{Refsdal and Nelson 5200} (RM).
  \textbf{Sublette County:}
Gros Ventre Area, Tepee Creek Ridge W to Red Hills; 6-9 air mi. SW of
Mosquito Lake. 9,600-10,400' elevation; T39N R111W S7 E1/4, S8, S9 S1/2,
S10 S1/4; 9 Jul 1994; \textit{Hartman 47530} (RM).
Hoback Canyon, U.S. HWY 189/191, road cut just N of the Granite Creek turnoff,
16.5 km (10 mi.) air distance ESE of Hoback Junction. 6,315' elevation;
1,925m elevation; T38N R114W S4; 43º17'01"N, 110º32'05"W; 27 May 2002;
\textit{Holmgren and Holmgren 14589} (UTC, NY, ISTC).
Roadcut, 10.4 mi. E of HWY 89 on HWY 189, just W of Bull Creek.
2,142m elevation; 43º17'10"N, 110º33'03"W; 26 Jun 1996;
\textit{Salywon and Dierig 3147} (MO).
Near the Hoback River, 11 mi. NW of Bondurant. 7,200-7,400' elevation;
T38N R114W S6; 20 Jun 1979;
\textit{Rollins and Rollins 79301} (RM, NY, GH, MO, F, UC, US).
Ramshorn Mountain. 9,000' elevation; T37N R114W S32 SE1/4; 26 Aug 1980;
\textit{Lichvar 3630} (RM).
Jamb Creek and lower slopes of ridge to N; ca. 2.5 air mi. SE of Hoback Peak.
9,200-9,600' elevation; T36N R115W S26; 20 Jul 1992;
\textit{Hartman 34241} (RM).
Lookout Mountain at W end of Beaver Ridge on North Fork Dry Beaver Creek; ca.
25 air mi. WNW of Daniel Junction. 9,400-9,800' elevation; T35N R115W S25;
30 Jul 1992; \textit{Hartman 35242} (RM).
Wyoming Range, 15 mi. W of Merna. 9,200-10,300' elevation; T35N R115W S28;
18 Jul 1922; \textit{Payson and Payson 2777} (RM, MO).
Southeast end of Mount McDougal from Kleinstick Mine NW to county line; ca. 0.3
air mi. SE of peak 10,780 (feet); 1-1.3 air mi. N to NW of McDougal Gap.
8,900-9,800' elevation; T33N R115W S4; 19 Aug 1992; \textit{Hartman 36208} (RM).
Roadcut, ca. 0.5 mi. W of McDougal Gap, Wyoming Range, 38.5 air mi. W of Daniel.
1,985m elevation; 42º50'20"N, 110º35'04"W; 26 Jun 1996;
\textit{Salywon and Dierig 3149} (ISTC).
Wyoming Range, mountain to the SW of Triple Peaks. T32N R115W S7; 18 Aug 1978;
\textit{Shultz 347} (UTC).
South Cottonwood Creek crossing of Bridger Forest RD 050, Wyoming Range,
3 mi. E of Soda Lake. 8,100' elevation; T32N R115W S12; 26 Jun 1978;
\textit{Shultz and Shultz 2653} (NY).
Wyoming Range, Cottonwood Creek. 8,100' elevation; T32N R115W S12 NW1/4;
8 Jul 1982; \textit{Lichvar 5223} (NY, NY, GH).
Red Canyon, ca. 18.5 air mi. WNW of Big Piney. 8,150-8,200' elevation;
T31N R114W S15 W1/4; 22 Jul 1995; \textit{Cramer and Kellett 9011a} (RM).
Wyoming Range; Middle Piney Lake. 8,500' elevation; T30N R115W S8 NE1/4;
8 Jul 1982; \textit{Lichvar 5222} (NY).
30 mi. W of Big Piney, on ridge E of Wyoming Peak. 10,000' elevation;
T30N R115W S7; 30 Jul 1974; \textit{Landon 229} (RM).
Middle Piney Lake. 8,800' elevation; 42º0.35'N, 110º0.35'W; 13 Jul 1965;
\textit{Lankester 934} (CAS).
Wyoming Range, Coal Creek. 6,900' elevation; T29N R115W S10; 8 Jul 1982;
\textit{Lichvar 5221} (NY, NY).
Wyoming Range. 9,400' elevation; T29N R115W S8; 16 Aug 1978;
\textit{Smith 1166} (UTC).
%%% IDAHO STATE SPECIMENS
  \textbf{Idaho:}
  \textbf{Fremont County:}
Ca. 0.8 mi. S of Martin Creek, along Gray's River RD, and 51 mi. SE of
Alpine JCT, Bridger-Teton National Forest. T32N R116W; 4 Jun 1987;
\textit{Atwood 12812} (NY).
  \textbf{Madison County:}
Targhee National Forest, Big Hole Range; summit of Red Butte above and W of
Thousand Springs Valley, ca. 16 air mi. SW of Driggs. 8,100' elevation;
T4N R43E S20; 21 Aug 1995; \textit{Markow 11187a} (RM).
  \textbf{Bonneville County:}
Trail along North Fork of Rainey Creek, ca. 6 air mi. NE of Swan Valley.
6,000-6,400' elevation; T2N R45E S7, S18, and S19; 28 Jun 1991;
\textit{Markow 1642} (RM-694043, RM-695860).
Fall Creek NW of Palisades Reservoir. 22 Jul 1975;
\textit{Ertter BJE 75-312} (NY, UTC, IDS).
Gibson Creek NW of Palisades Reservoir. 5,800' elevation; T1N R42E S34 SE1/4;
7 Jul 1975; \textit{Brown CGB 75-110} (NY).
Snake River Range, Sheep Creek Peak. 8,400-9,950' elevation;
T1N R45E S34 and S35; 22 Jul 1991; \textit{Hartman and Molina 30187} (RM).
Big Elk Mountain, south side of Bear Creek. 5,700' elevation;
T1S R45E S31 SE1/4; 14 Jun 1978; \textit{Dieffenbach 615} (IDS).
Elk Mountain. 8,700' elevation; T2S R44E S23 SW1/4; 24 Jul 1978;
\textit{Dieffenbach 914} (IDS).
Elk Mountain, open east slope. 8,800' elevation; T2S R45E S31 NW1/4;
26 Jul 1978; \textit{Dieffenbach 964} (IDS).
Elk Mountain. 9,400' elevation; 26 Jul 1915;
\textit{Forest Service (\#21609) 342} (RM).
Targhee National Forest. 8,500' elevation; 2S 44E S16 NE; 16 Jul 1979;
\textit{Dieffenbach, Glennon, Golte, Mel, Pearson, and Vieth 335} (RM, UTC).
Caribou Range, southeastern end of Big Elk Mountain, 9 airline mi. S of
Palisades Dam. 8,800' elevation; T2S R45E S31; 22 Jul 1971;
\textit{Holmgren and Marttala 5584} (NY, GH).
Palisade Reservoir, where McCoy Creek meets reservoir. 5,700' elevation;
T3S R46E S6 NE1/4; 12 Jul 1978; \textit{Dieffenbach 773} (IDS).
Bald Mountain, ridges to shale slopes. 7,600' elevation; T3S R45E S30 SW1/4;
23 Aug 1978; \textit{Dieffenbach 1183} (IDS).
Caribou National Forest, Fox Springs. 8,200' elevation; 10 Jun 1913;
\textit{Anderson 71a} (RM).


