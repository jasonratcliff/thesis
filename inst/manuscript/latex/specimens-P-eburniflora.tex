\section*{\textit{Physaria eburniflora}}

  \textbf{Wyoming:}
% Fremont
  \textbf{Fremont County:}
% W Beaver Rim P. saximontana sympatry
W Wind River Basin, Beaver Rim Divide, south end of Cedar Rim. 6,800' elevation;
T31N R95W S27 SE1/4 SW1/4; 13 Jun 1994; \textit{Fertig 14841} (RM).
Sweetwater River Plateau, ridge system on N side of Government Meadow Draw, ca.
1-1.5 mi. E of WY Hwy 135, ca. 4.75-5 air mi. NNE of Sweetwater Station.
6,700-6,800' elevation; T31N R95W S34 NE1/4 of SE1/4, SE1/4 of NE1/4, and
E1/4 of SE1/4 of SE1/4, S35 SW1/4 of SW1/4 and S1/2 of SE1/4 of SW1/4;
13 Jul 1997; \textit{Fertig 17696} (RM).
Wind River Basin, Beaver Divide, adjacent to WY Hwy 135, ca. 28 air mi. S of
Riverton. 6,760-6,800' elevation; T31N R95W S3 NE1/4 NW1/4; 13 Jul 1992;
\textit{Fertig 13016} (RM).
Sweetwater River Plateau; Granite Mountains, west slope and summit knob of
Murphee Peak on N side of Sage Hen Creek, west of Agate Flat Road ca. 4 air mi.
N of confluence of Sage Hen Creek and the Sweetwater River.
6,500-6,800' elevation; T30N R90W S8 E1/2 SE1/4, S9 SW1/4 SW1/4, and
S16 N1/2 NW1/4; 3 Jun 1996; \textit{Fertig 16486} (RM).
Sweetwater Rocks; granite knob on E side of Agate Flat Road, ca. 1 mi. NE of
jct. with US Hwy. 287, ca. 1.2 mi. SW of the crossing of the Sweetwater River.
6,500' elevation; T29N R91W S12 SE1/4 NE1/4; 3 Jun 1996;
\textit{Fertig 16490} (RM).
Ca. 9 mi. S of Jeffrey City. 7,000' elevation; T28N R92W S28 SW1/4; 17 May 2001;
\textit{Dorn 8635} (RM).
Green Mountains, Wild Horse Overlook. 9,000' elevation; T27N R91W S1 or S2;
8 Jul 1978; \textit{Hartman 8402} (RM).
ESE end of Green Mountan; ATV road heading WSW from Low pass (towards N
Stratton Rim). 8,450' elevation; 10 Jun 2014;
\textit{Ratcliff and O'Kane, Jr. 37} (ISTC).
Stratton Rim on the southeast end of Green Mountain ca. 4 air mi. WSW of
Whiskey Peak; ca. 16.5 air mi. SE of Jeffrey City; ca. 5.5 air mi. NW of
Bairoil. 8,200-8,700' elevation; T27N R91W S24 NW1/4; 27 Jun 1995;
\textit{Nelson 35669} (RM).
Stratton Rim, ca. 16 air mi. SE of Jeffrey City, ca. 63 air mi. SE of Riverton.
8,660' elevation; T27N R91W S24 SW1/4 NW1/4; 27 Jul 1983;
\textit{Blomquist 70} (RM).
% Natrona
  \textbf{Natrona County:}
S foothills of the Rattlesnake Hills; south and east slope of Rocky Peak adj.
to county road ca. 6.5 air mi. NW of WY Hwy 220. 6,600-7,000' elevation;
T30N R87W S1 NE1/4 NE1/4; 6 Jun 1993; \textit{Fertig 13828, 13829} (RM).
13.1 mi. NE of Muddy Gap jct. 6,075' elevation; 10 Jun 2014;
\textit{Ratcliff and O'Kane, Jr. 38} (ISTC).
13.5 mi. E of Muddy Gap (Hwy 287) on Hwy. 220. 2,267m elevation; 23 Jun 2000;
\textit{Salywon and Dierig 3121} (ISTC).
Near Devil's Gate, 13.5 mi. NE of Muddy Gap Junction. 6,500' elevation;
21 May 1979; \textit{Rollins 7922} (NY, UC).
Sweetwater River Valley; Sweetwater Rocks just east of Devil's Gate, ca. 13.5
air mi. NE of Muddy Gap Junction. 6,200' elevation; T29N R87W S36 N1/4;
6 Jun 1993; \textit{Fertig 13825} (RM).
Near Devil's Gate, western Natrona County near Carbon County line, 13.5 mi. ENE
of Muddy Gap Junction. 27 Jun 1979; \textit{Rollins and Rollins 79344} (NY).
% East rim
Twin Butte. 6,600' elevation; NAm elevation; s.n. NA; 27919;
\textit{R. W. Lichvar 4336} (RM).
Shirley Basin; East rim of Bates Hole, Butte adjacent to Sand Draw Road, ca. 4
air mi. E of WY Hwy. 487. 6,800' elevation; T29N R80W S2 NE1/4 SW1/4;
12 Jul 1993; \textit{Fertig 14095} (RM, NY).
West foothills Laramie Range; summit crest and upper SW-facing slopes of ridge
along the divide between Mud Springs Draw and Chalk Creek, ca. 1.2 mi. S of the
confluence of Chalk and Bates creeks, ca. 8.5 air mi. E of WY Hwy 487.
6,940-7,000' elevation; T29N R79W S3 NW1/4 SW1/4; 20 Jun 1997;
\textit{Fertig 17541} (RM).
% Carbon
  \textbf{Carbon County:}
% NW Carbon
Sweetwater River Valley; ridge on NE side of WY Hwy. 287, 5 rd. mi. NW of Three
Forks (Muddy Gap) jct. 6,560-6,600' elevation; T28N R89W S8 NW1/4; 28 May 1993;
\textit{Fertig 13731} (RM).
Muddy Gap. 6,200' elevation; T28N R89W S33; 10 May 1981;
\textit{Hartman and Fonken 12547A} (RM, GH),
\textit{Hartman and Fonken 12547B} (RM).
Muddy Gap. 6,500' elevation; T28N R89W S33; 18 Jun 1977;
\textit{Dorn 2931} (RM).
Muddy Gap. 6,400' elevation; 12 Jun 1948;
\textit{Ripley and Barneby 9137} (CAS).
1.3 mi. S Muddy Gap jct. on Hwy. 287, E side of rd. on red rocky, gravelly
hillsides. 2,267m elevation; 22 Jun 1996;
\textit{Salywon and Dierig 3120} (ISTC).
3 mi. S of Muddy Gap. 27 Jun 1979;
\textit{Rollins and Rollins 79345} (NY, GH, MO, US).
3 mi. S of Muddy Gap. 21 May 1979;
\textit{Rollins and Rollins 7926} (NY, GH, MO, US).
Foothills of the Ferris Mountains; ridge ca. 1 mi. N of Bairoil junction
(intersected by WY Hwy 287), ca. 9 air mi. S of Three Forks. 6,750' elevation;
T26N R89W S15 N1/2; 10 Jun 1993; \textit{Fertig 13834} (RM).
Northern side of the Ferris Mountains, along Little Cherry Creek and slopes
above Little Cherry Creek, ca. 6 air mi. SE of Muddy Gap Junction (Three Forks).
7,200-8,200' elevation; T27N R88W S15, S16, S21, and S22; 2 Jul 1985;
\textit{Haines 4599} (RM).
Ferris Mtns., Cherry Creek Drainage. 7,500' elevation; 13 Jul 1981;
\textit{Lichvar 4594} (NY).
Ferris Mountains, Pete Creek. 7,100' elevation; T27N R87W S16; 15 Jul 1981;
\textit{Lichvar 4610} (RM).
N slope Ferris Mountains, ridge system on W side of Pole Canyon, from base of
mountain to ridge just below summit, ca. 10.5 air mi. S of WY Hwy. 220.
7,900-8,800' elevation; T27N R87W S28 W1/4, S32 NE1/4, S33 NW1/4 NW1/4;
24 Jun 1995; \textit{Fertig 15779} (RM).
Ferris Mountains. 9,000' elevation; T27N R87W S33; 8 Jun 1980;
\textit{Hartman 11620} (RM).
Ferris Mountains. 9,000' elevation; T26N R87W S4; 15 Jul 1981;
\textit{Lichvar 4617} (NY).
Whiskey Gap, Ferris Mts. 6,500' elevation; 20 Jun 1979;
\textit{Dorn 3228} (NY, GH).
South rim of Bear Mountain, ca. 1 mi. E of Sand Creek and 20 mi. ESE of
Muddy Gap. 7,440' elevation; T26N R86W S10 NE1/4 NW1/4 of SE1/4; 26 Jun 2001;
\textit{Fertig, Evans, and Evans 19643} (RM).
% Shirley Basin W
Shirley Basin; NE rim of Chalk Mountain ca. 3.5 mi. W of WY Hwy. 77 and 1.25
mi. SE of Horse Peak. 7,400-7,600' elevation; T28N R81W S24 SE1/4 of NE1/4,
T28N R80W S19 SW1/4 of NW1/4; 30 May 1999; \textit{Fertig and Welp 18640} (RM).
East slope of Chalk Mountain ca. 1.8 mi. S of Horse Peak and 4 mi. SW of
jct. of WY Hwy. 77 and 487. 7,680-7,740' elevation; T28N R81W S25 NE1/4 NE1/4
of NW1/4; 4 Jul 1999; \textit{Fertig 18746} (RM).
Chalk Mtn. NW edge of Shirley Basin. 7,900' elevation; 15 Jul 2010;
\textit{Dorn 10680} (RM).
W side of Chalk Mtn., NW edge of Shirley Basin. 7,900' elevation; 20 Jun 2008;
\textit{Dorn 10410} (RM).
SW end of Bates Hole; toe of SE end of Chalk Mountain, ca. 3.25 mi. W of WY
Hwy. 77. 7,400-7,500' elevation; T27N R80W S6 SW1/4 NW1/4,
T27N R81W S1 NE1/4 SE1/4; 21 Aug 1996; \textit{Fertig 17173} (RM).
% Shirley Basin E
White chalky outcrops near Hwy. 487. 7,217' elevation; T28N R79W S21;
2 Aug 1988; \textit{Locklear 114} (RM).
Shirley Basin. 7,200' elevation; T28N R79W S21; 1 Jul 1986;
\textit{Dorn 4357} (RM, NY).
North side of Shirley Basin off Shirley Ridge Road. 7,300' elevation;
16 Jul 2010; \textit{Dorn 10681} (RM).
Shirley Basin; N-S trending low ridges ca. 1.5 mi. E of the rim of Bates Hole
and ca. 2 mi. N of Moss Agate Ridge and WY Hwy. 487. 7,400-7,450' elevation;
T28N R79W S10 NE1/4 NW1/4; 21 Jun 1997; \textit{Fertig 17565} (RM).
