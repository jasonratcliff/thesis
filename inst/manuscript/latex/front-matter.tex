%%%%%% COPYRIGHT PAGE
\thispagestyle{empty}
\vspace*{\fill}         % Fill vertical space within the margins
% Center the text within the margins
\begin{center}          
  Copyright by\\
  JASON RATCLIFF\\
  2021\\
  All Rights Reserved
\end{center}
\vspace*{\fill}
\clearpage

%%%%%% ABSTRACT TITLE PAGE
\clearpage
\thispagestyle{empty}
\begin{center} 
THE SYSTEMATICS OF \textit{PHYSARIA} SENSU STRICTO FROM WYOMING 
AND SURROUNDING AREAS
\end{center}
\vspace{120pt}
\begin{center}
  An Abstract of a Thesis

  Submitted

  in Partial Fulfillment

  of the Requirements of the Degree
  
  Master of Science
\end{center}
\vspace{120pt}
\begin{center}
  Jason Ratcliff
  
  University of Northern Iowa
  
  July 2021
\end{center}
\clearpage

%%%%%% ABSTRACT PAGE
\thispagestyle{empty}
\begin{center}
ABSTRACT
\end{center}

\paragraph{} Two genera of Brassicaceae – \textit{Lesquerella} and
\textit{Physaria} – were recently united on the basis of monophyly.
Classically, these genera have been distinguished based on morphology with
"traditional" \textit{Physaria} species having doubly inflated (didymous)
fruits.  Molecular data consisting of ribosomal internal transcribed spacer
(ITS) DNA sequences indicate \textit{Physaria} has evolved from the larger
Lesquerella genus, thus making \textit{Lesquerella} a paraphyletic grouping as
previously described. A revised treatment of the combined \textit{Physaria}
recognized 106 species within the genus, of which 26 taxa occur in Wyoming.
Of the Wyoming taxa, those traditionally described as belonging to
\textit{Physaria} (i.e. \textit{Physaria} sensu stricto) have been identified
as needing taxonomic investigation. The evolutionary relationships of these
species is considered based on molecular, geographical, and morphological data.
Three groups of particular interest include the status of \textit{P. vitulifera}
in south-central Wyoming; the relationships of reportedly sympatric
\textit{P. saximontana} and \textit{P. eburniflora} to \textit{P. didymocarpa}
subspecies; and the status of regional and state endemics
\textit{P. integrifolia}, \textit{P. condensata}, and \textit{P. dornii}.
1665 herbarium specimens were reviewed for taxonomy and morphological data for
discrete and continuous characters were recorded. Geographical data in the form
of TRS or specific location descriptions from specimen vouchers were converted
to degree decimals and species distributions were mapped using open source R
packages. Nuclear ribosomal ITS DNA and plastid \textit{rps} intron and
\textit{ycf1} were sequenced from \textit{Physaria} sensu stricto individuals
throughout the respective distribution of each species. Phylogenetic analysis
of individual gene and multi-locus concatenated data by Bayesian inference did
not recover strong support for monophyletic groupings of these taxa as
previously described. Furthermore, a high degree of intragenic sequence
similarity between individuals supports a recent divergence of these taxa.
Geographic data reveal a high elevation distribution of contested
\textit{P. vitulifera} specimens from south-central Wyoming with an apparent
allopatric relationship to proximal species \textit{P. floribunda} and
\textit{P. vitulifera}. This lineage is hypothesized here as \textit{Physaria}
'medicinae' on the basis of phylogenetic support, reduced fruit size and growth
habit, and elevation gradient. The primary distinguishing
fruit characteristic of \textit{P. saximontana} subsp. \textit{saximontana}
is shown to be prevalent among \textit{P. didymocarpa} subspecies
\textit{didymocarpa} specimens of Montana, supporting a status change with
precedence to the earlier described taxa. Molecular data also support close
relationships between the regional and state endemics of south-western Wyoming,
suggesting a polyphyletic relationship as currently described.

\clearpage

%%%%%% THESIS TITLE PAGE
\clearpage
\thispagestyle{empty}
\begin{center} 
THE SYSTEMATICS OF \textit{PHYSARIA} SENSU STRICTO FROM WYOMING 
AND SURROUNDING AREAS
\end{center}
\vspace{120pt}
\begin{center}
  A Thesis

  Submitted

  in Partial Fulfillment

  of the Requirements of the Degree
  
  Master of Science
\end{center}
\vspace{120pt}
\begin{center}
  Jason Ratcliff
  
  University of Northern Iowa
  
  July 2021
\end{center}

%%%%%% APPROVAL PAGE
\clearpage
\pagestyle{contents}  % set page style
\pagenumbering{roman}  % Set style of page numbering
\setcounter{page}{2}  % Set page count to ii

This study by: Jason Ratcliff

Entitled: The Systematics of \textit{Physaria} Sensu Stricto in Wyoming and
Surrounding Areas

\vspace*{24pt}

has been approved as meeting the thesis requirement for the

Degree of Master of Science

\vspace*{24pt}

\singlespacing

\begin{noindent}
\begin{tabbing}
\rule{1.25in}{0.4pt}\hspace{0.75in}\=\rule{4in}{0.4pt}\\
Date \> Dr. Steve O'Kane, Jr., Chair, Thesis Committee\\
\end{tabbing}

\begin{tabbing}
\rule{1.25in}{0.4pt}\hspace{0.75in}\=\rule{4in}{0.4pt}\\
Date \> Dr. Theresa Spradling, Thesis Committee Member\\
\end{tabbing}

\begin{tabbing}
\rule{1.25in}{0.4pt}\hspace{0.75in}\=\rule{4in}{0.4pt}\\
Date \> Dr. James Demastes, Thesis Committee Member\\
\end{tabbing}

\begin{tabbing}
\rule{1.25in}{0.4pt}\hspace{0.75in}\=\rule{4in}{0.4pt}\\
Date \> Dr. Jennifer Waldron, Dean, Graduate College\\
\end{tabbing}
\end{noindent}

\doublespacing

%%%%%% DEDICATION
\clearpage
\thispagestyle{empty}
\begin{center}
DEDICATION
\end{center}

\paragraph{} To my parents, who provided every opportunity for success.

%%%%%% TABLE OF CONTENTS
\clearpage
\renewcommand{\contentsname}{TABLE OF CONTENTS} % ToC name
\setcounter{tocdepth}{3}
\tableofcontents
\addtocontents{toc}{~\hfill{PAGE}\par}

%%%%%% LIST OF TABLES
\clearpage
\phantomsection \label{listoftab}
\addcontentsline{toc}{chapter}{LIST OF TABLES}
\renewcommand{\listtablename}{LIST OF TABLES} % LoT name
\listoftables
\addtocontents{lot}{TABLE~\hfill{PAGE}\par}

%%%%%% LIST OF FIGURES
\clearpage
\phantomsection \label{listoffig}
\addcontentsline{toc}{chapter}{LIST OF FIGURES}
\renewcommand{\listfigurename}{LIST OF FIGURES} % LoF name
\listoffigures
\addtocontents{lof}{FIGURE~\hfill{PAGE}\par}

