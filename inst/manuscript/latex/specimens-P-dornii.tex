\section*{\textit{Physaria dornii}}

%%% WYOMING STATE SPECIMENS
  \textbf{Wyoming:}
  \textbf{Lincoln County:}
Overthrust Belt; W slope of Rock Creek Ridge, N-S ridge N of Trail Creek and
Underwood Canyon, between Trail Creek Road and powerline corridor, ca. 7 air mi.
SE of Cokeville. 6,800' elevation; T24N R118W S31 NW1/4 of SW1/4; 17 Jun 1997;
\textit{Fertig 17506} (RM).
North end of Rock Creek Ridge on ridge system on north side of
Trail Creek just north of the Cokeville-Ham's Fork Road, ca. 6 air mi. SE of
Cokeville. 6,600-7,000' elevation; T23N R118W S6 NW1/4 of NW14; T24N R118W S31
S1/2 of SW1/4 and NW1/4 of SW1/4; 3 Jul 1996; \textit{Fertig 16752} (RM).
West slope of Rock Creek Ridge, ridges on N and E side of southern tributary of
Horse Creek, ca. 1.3-1.5 mi. N of North Fork Leeds Creek, ca. 4.5 mi. E of U.S.
Hwy 30. 7,000-7,200' elevation; T23N R119W S24 SW1/4 of NE1/4, N1/2 of SE1/4, &
E1/4 of SE1/4SE1/4 and T23N R118W S19 W1/2 of SW1/4; 17 Jun 1997;
\textit{Fertig 17504} (RM).
West slope of Rock Creek Ridge, on ridge on N side of North Fork Leeds Creek,
ca. 4 air mi. E of U.S. Hwy. 30 and ca. 8 air mi. SE of Cokeville.
6,800-7,100' elevation; T23N R119W S25 S1/2 of SE1/4 of SW1/4; 17 Jun 1997;
\textit{Fertig 17503} (RM).
Southern Tunp Range, W slope of Rock Creek Ridge, S side of ridge on N side of
southern tributary of North Fork of Leeds Creek, ca. 3.6 mi. E of U.S. Hwy. 30.
6,700-6,900' elevation; T23N R119W S36 N1/2 of SW1/4; 8 Jun 1997;
\textit{Fertig 17492} (RM, NY)
North-South trending ridge between South Fork of Leeds Creek and the southern
branch of the North Fork of Leeds Creek, ca. 3.5 air mi. E of U.S. Hwy 30.
6,600-6,900' elevation; T22N R119W S2 S1/2 of NE1/4, NE1/4 of SE1/4 of NE1/4,
S1 W1/2 of NW1/4; T23N R119W S36 SW1/4 of SW1/4; 8 Jun 1997;
\textit{Fertig 17490} (RM).
Rock Ridge. 7,000' elevation; T22N R119W S12 SE1/4; 31 May 1982;
\textit{Lichvar 4809} (GH).
Rock Ridge. 7100' elevation; T22N R119W S14 SE1/4; 31 May 1982;
\textit{Lichvar 4807} (NY).
Rock Ridge. 6,900' elevation; T22N R119W S14 NE1/4; 6 Jul 1982;
\textit{Lichvar 5207} (NY-182085, NY-182084)
Southern end of Rock Creek Ridge, ca. 3/4 mi. NNE of Gooseberry Spring,
ca. 16 air mi. SSE of Cokeville; ca. 19 air mi. WNW of Kemmerer.
7,000-7,200' elevation; T22N R119W S23 S1/4 and N1/4 S26; 10 Jul 1995;
\textit{Nelson and Refsdal 36682} (RM).
West side of Rock Creek Ridge on ridge system on north side of east tributary
draw of Antelope Creek, ca. 4 air mi. E of U.S. Hwy. 30 and ca. 4 air mi. N of
Twin Creek and Orr. 6,800-7,100' elevation; T22N R119W S23 NE1/4 of NE1/4 and
S14 SE1/4; 2 Jul 1996; \textit{Fertig 16736} (RM).
Southwest slope of Rock Creek Ridge, Point 7085 and E-W trending ridges to
south, ca. 0.2-0.4 mi. E of Gooseberry Spring, ca. 2.5 air mi. N of U.S. Hwy. 30
at site of Orr, ca. 4.5 air mi. NE of Sage Junction. 6,820-7,040' elevation;
T22N R119W S26 SE1/4 of NE1/4 of SW1/4 and SE1/4 of SE1/4 of NW4; 18 Jun 1997;
\textit{Fertig 17511} (RM).
Rock Creek Ridge on west side of summit, ca. 14 air mi. WNW of Kemmerer,
Center of Sec. 30. 7,200' elevation; T22N R118W S30; 27 Jun 1988;
\textit{Marriott and Horning 10825} (RM).
West slope Rock Creek Ridge, roadcut 1.0 rd. mi. N of Hwy. 30 on BLM rd. 4211.
6,890' elevation; 12 Jun 2014; \textit{Ratcliff and O'Kane, Jr. 42} (ISTC).
Rocky Ridge. 6,900' elevation; T22N R119W S36 SE1/4; 3 Jul 1982;
\textit{Dorn 3734} (NY).
South end of Rock Creek Ridge, ca. 1 mi. N of U.S. Hwy. 30, ca. 5.5 air mi.
ENE of Sage Junction. 6,900' elevation; T22N R119W S36 SW1/4 of SE1/4;
1 Jul 1996; \textit{Fertig 16732} (RM).
White shale ridge west of Fossil Butte National Monument, north of Highway 30.
2042m elevation; Lat 41º49'24"N, Long 110º52'10"W; 4 Jun 1996;
\textit{O'Kane, Jr. 3789} (ISTC).
Rock Creek Ridge. 6,600-6,760' elevation; T21N R119W S1; 12 May 1993;
\textit{Refsdal 2} (RM).
Rock Ridge. 6,800' elevation; T21N R119W S12 NW1/4; 4 Jul 1981;
\textit{Lichvar 5198} (GH).
Collett Creek, on slope above and east of creek, ca. 0.5 air mi. S of Highway
30, ca. 18.5 air mi. W of Kemmerer. 6,500' elevation; T21N R119W S11 NE1/4
SE1/4; 17 Jun 1988; \textit{Marriott and Horning 10826} (RM).
Rock Ridge. 6,600' elevation; T21N R117W S1 E1/2; 4 Jul 1982;
\textit{Lichvar 5197} (NY).
  \textbf{Uinta County:}
Overthrust Belt: ridge ca. 1 mi. E of "The Boilers", ca. 1.5 air mi. S of
Interstate 80, ca. 2.5 mi. west of Ragan. 7,550-7,600' elevation;
T15N R118W S15 NW4 of SE4 of SW4; 5 Jul 1996; \textit{Fertig 16775} (RM).
Ca. 13 air mi. E of Evanston on north side of dirt road. 7,300-7,520' elevation;
T15N R118W S22 NW4 S22; also S1/2 S15 and S14; 24 Jun 1995;
\textit{Refsdal and Refsdal 4376} (RM).
W end of knoll on summit of ridge on NW side of Antelope Creek, ca. 1 mi. NW of
the Union Pacific Railroad and 4.5 mi. S of jct of Interstate 80 & US Hwy 189,
ca. 12 miles E of Evanston. 7600' elevation;
T15N R118W S28 S1/2 of SE1/4 of SW1/4 & S1/2 of SW1/4 of SE1/4; 19 Jun 1997;
\textit{Fertig 17533} (RM).

