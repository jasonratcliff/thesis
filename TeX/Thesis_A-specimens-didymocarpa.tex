\begin{center}
\textit{Physaria saximontana}
\end{center}

%%% WYOMING STATE SPECIMENS
\textbf{Wyoming: Park County:} Absaroka Mountains, North Fork Shoshone River Drainage; north side of Crow Peak ca. 1-2 mi. N of U.S. Hwy 14, 16, \& 20. 7,500-8,500’ elevation; T52N R109W S2; 28 Jun 1987; \textit{Evert 12677} (RM).  Ca. 1/2 mi. E of Pahaska along US Hwy 14, 16, \& 20. 6,800’ elevation; T52N R109W S2 SW4; 13 Jul 1981; \textit{Evert 3123} (RM).  Northeast-southwest trending ridge ca 1/4 mi E of Grinnell Creek and ca 1/8 -3/4 mi N of U.S. Hwy 14, 16, \& 20. 6,800-7,600’ elevation; T52N R109W S12, T52N R108W S6; 23 Jun 1987; \textit{Evert 12460} (RM, NY).  Ridge ca. 0.5 mi. W of Mormon Creek; ca. 0.5-2 mi. N of U.S. Hwy 14, 16, \& 20. 7,600-8,600’ elevation; T52N R108W S8; 18 Jul 1985; \textit{Evert 8392} (RM).  Absaroka Range, South Absaroka Wilderness; Fishhawk Creek, 0.5-1 mi. SSW of confluence with the North Fork Shoshone River, Sec. 27 \& 34. 6,500’ elevation; T52N R108W S27; 7 Jul 1978; \textit{Hartman 8361} (RM).  Along Elk Fork Creek trail, ca. 5-6 mi. S of U.S. Hwy 14, 16 \& 20. 6,500-6,700’ elevation; T51N R107W S13, S7; 26 Jul 1985; \textit{Evert 8918} (RM).  Ca. 3-4 mi. S of US Hwy 14, top of Clayton Mountain. 10,100’ elevation; T51N R107W S10 NW4; 20 Jul 1982; \textit{Evert 4362} (RM).  Slopes on S side of summit rim of Clayton Mountain and northernmost end of saddle connecting Clayton Mountain and Double Mountain, ca, 1.1 air mi. E of East Fork Blackwater Creek and 3.2 mi. S of US Hwy 14, 16, \& 20. 9,500-10,000’ elevation; T51N R107W S10 S2 NW4; 17 Jul 1996; \textit{Fertig 16871} (RM).  Top of Clayton Mountain, ca. 3 mi. S of U.S. Hwy 14, 16, \& 20. 9,800’ elevation; T51N R107W S10; 24 Jul 1985; \textit{Evert 8782} (RM).  N-S trending ridge immed. E of June Creek, ca. 1/4 mi. S of N fork Shoshone River. 6,400-6,600’ elevation; T52N R107W S10 \& S3; 4 Jun 1989; \textit{Evert 16391} (RM).  East of Newton Creek, ca. 1.5 mi. S of US Hwy 14. 6,800’ elevation; T52N R107W S21 SW4; 7 Aug 1982; \textit{Evert 4694} (RM).  Ridge/divide between Moss Creek and Clearwater Creek ca. 1.5-2.5 mi. N of Hwy 14, 16, \& 20. 7,600-8,300’ elevation; T52N R107W S10 \& S3; 6 Jul 1989; \textit{Evert 17578} (RM).  Many pinnacled ridge E of Clearwater Creek, ca. 3-4 mi. N of U.S. Hwy 14, 16, \& 20. 7,200-8,000’ elevation; T52N R107W S2; 9 Jun 1986, \textit{Evert 10137} (RM).  Ridge, 0.5 mi NW of Signal Peak, ca. 1 mi. N of U.S. Hwy 14, 16, \& 20. 7,000’ elevation; T52N R106W S18; 13 Jun 1986; \textit{Evert 9930} (RM).  N-S trending ridge immed. W of Clearwater Creek ca. 1/8 mi. N of Hwy. 14, 16, \& 20. 6,200’ elevation; T52N R106W S20; 4 Jun 1989; \textit{Evert 16363} (RM).  Elk Fork Creek ca. 3/4 mi. S of US Hwy 14. 6,000’ elevation; T52N R106W S29 NE4; 15 Jul 1982; \textit{Evert 4247} (RM).  Ridge E of Sweetwater Creek, ca. 4-5 mi. N of U.S. Hwy 14, 16, \& 20. 7,500-8,000’ elevation;  T52N R106W S5 \& S33; 16 Jun 1986; \textit{Evert 10040} (RM).  Ridge between Horse and Sweetwater creeks, ca. 1.5-3 mi. N of U.S. Hwy 14, 16, \& 20. 6,800-7,500’ elevation; T52N R106W S4 \& S9; 16 Jun 1987; \textit{Evert 12281} (RM, NY).  North-south trending ridge between Horse Creek and Grizzly Creek ca. 2-5 mi. N of U.S. Hwy 14, 16, \& 20. 7,500-8,500’ elevation; T52N R106W S3, T53N R106W S35; 25 Jun 1987; \textit{Evert 12582} (RM).  North ridge of Ptarmigan Mountain on divide between Cougar and Pagoda Creeks, ca. 3.9 mi. E of Elk Fork Creek, ca. 5 mi. S of US Hwy 14, 16 \& 20. 10,200’ elevation; T51N R106W S15 SW4 NE4; 30 Jul 1996; \textit{Fertig 16970} (RM).  Divide between Cougar and Pagoda Creeks below N ridge of Ptarmigan Mountain, ca. 3 air mi. E of Elk Fork Creek, ca. 3.5 mi. S of US Hwy 14, 16 \& 20. 8,600’ elevation; T51N R106W S10 NE4 SW4; 30 Jul 1996; \textit{Fertig 16965} (RM).  On ridge between Cougar and Pagoda Creeks, ca. 4 mi. S of US Hwy 14, 16, \& 20. 8,600’ elevation; T51N R106W S10 SW4; 22 Jul 1981; \textit{Evert 3275} (RM).  Ridge system on N side of North Fork Shoshone River between Signal Peak and Anvil Rock, ca. 1 mi. N of US Hwy 14/16/20 \& ca. 5 air mi. W of Wapiti. 7,200-7,400’ elevation; T52N R105W S18 E2 SW4; 26 Jun 1997; \textit{Fertig 17583} (RM).  On ridge between Fishhawk and Mesa creeks, ca. 2-3 mi. S of U.S. Hwy 14, 16, \& 20. 7,000-8,000’ elevation; T51N R105W S2 \& S3; 8 Aug 1984; \textit{Evert 7451} (RM).  Top of Four Bear (Black) Mountain, ca. 2.5 mi. N of U.S. Hwy 14, 16, \& 20. 7,200-7,600’ elevation; T52N R104W S5; 14 Jun 1988; \textit{Evert 14434} (RM).  North side of Logan Mountain. 5,850’ elevation; T52N R104W S11 SE4; 19 Jun 1982; \textit{Evert 3939} (RM).  Eastern Absaroka Mts., ca. 6 air mi. N of Dead Indian Peak, ca. 0.25 air mi. N of Sulight Crk.  7,000’ elevation, T55N R106W S4 SW4 of NE4; 19 Jul 1995; \textit{Mills 39} (RM).  Northern Absarokas, Windy Mountain Summit and upper NW slope, 6-7 air mi. NNW of sunlight ranger station.  9,900’ elevation; T56N R106W S26 \& S35; 17 Aug 1985; \textit{Hartman \& Nelson 21835} (RM).  Bighorn Basin, Kimball Bench, ca. 10 air mi. SSW of Clark.  4,800-5,000’ elevation; T55N R103W S3 \& S4; 24 May 1983; \textit{Hartman \& Hamann 14354} (NY).  SW-facing slopes below summit cone of Heart Mountain, NE of WY Hwy 120, ca. 9 air mi. NNW of Cody.  7,100’ elevation; T54N R102W S15 SE4 of NW4; 30 Jun 1997; \textit{Fertig \& Lenard 17641} (RM).  SW flank of Rattlesnake Mtn. ca. 6-7 miles W of Cody \& ca. 1 mi N of Hwy 14, 16, \& 20. 6,400-7,000' elevation; T52N R103W S2; 15 Jun 1989; \textit{Evert 16761} (RM).  Dry gravelly soil, Middle Shoshone Canyon.  6,500’ elevation; 7 Jul 1921; \textit{Wiegand, Castle, Dann, \& Douglas 1008} (F).  Bighorn Basin, Spring Creek, 2 air mi. WNW of Meeteetse.  5,800’ elevation; T48N R100W S6; 21 Jun 1983; \textit{Hartman \& Nelson 15525} (UTC).  Absaroka Mountains, ca. 22 air mi. SW of Meeteetse, in the vicinity and W of the junction of the Middle Fork Wood River and Beaver Creek.  7,900’ elevation; T46N R103W S35 \& 36; 26 Jul 1984; \textit{Kirkpatrick 4997} (RM).  Absaroka Mountains, ca. 33 air mi. SW of Cody at the junction of Boulder Creek and Little Boulder Creek. 7,300’ elevation; T49N R105W S30 \& S29; 1 Jul 1983; \textit{Kirkpatrick 613} (RM, GH).  Eleanor Creek N to ridge. 10,500’ elevation; T48N R105W S30 S32; 25 Aug 1984; \textit{Hartman 19364} (RM).  Jack Creek Trail ca. 0.2 mi. N of junction with Haymaker-Timber Creek Trail, ca. 14 air mi. W of Sunshine Reservoir Dam.  9,200’ elevation; T47N R104W S10 SE1/4 of NW1/4; 29 Jun 1988; \textit{Marriott 10872} (RM).  Upper Greybull River between Steer Creek and Cow Creek, 4.5-7 air mi. NW of Kirwin. 9,300’ elevation; T46N R104W S17, S18, S20, S29; 22 Aug 1983; \textit{Hartman 17305} (RM).  Southeastern Absaroka Mts., ca. 3 air mi. NE of Chief Mtn., ca. 0.5 air mi. N of Jojo Crk. and Wood River junction.  8,480’ elevation; T46 R103W S21 NE1/4 of SE1/4; 7 Jul 1995; \textit{Mills 16} (RM).  Ca. 4 air mi. NE of Chief Mtn., ca. 0.5 air mi. N of Jojo Crk. and Wood River junction.  8,560’ elevation; T46N R103W S22 SW1/4 of NW1/4; 7 Jul 1995; Mills 17 (RM).  Ca. 5 air mi. ENE of Chief Mtn., across road from Brown Creek Camp Ground. S23 SW4 of SE4. 7,600’ elevation; T46N R103W S23 SW1/4 of SE1/4; 8 Jul 1995; \textit{Mills 22} (RM).  Ca. 6 air mi. ENE of Chief Mtn., ca. 0.5 air mi. N of the Wood River.7,800’ elevation; T46N R103W S24 S1/2; 7 Jul 1995; \textit{Mills 20} (RM).  S. Fk. Shoshone River.  7,000’ elevation; T48N R106W S18 NW1/4; 24 Aug 1987; \textit{Dorn 4799} (RM).  Growing on shale slopes of southwest-facing slopes of mountains above the South Fork of the Wood River, Meeteetsee.  7,588’ elevation; Zone 12T WGS 84, Easting 0652708 Northing 4864916; 25 Aug 2012; \textit{Smith 551, 552, 553} (ISTC).  \textbf{Fremont County:}  NE Wind River Range: S end of Torrey Rim, ca. 0.2 mi. N of Trail Lake trailhead, ca. 1.75 mi. W of Trail Lake, ca. 7 air mi. S of Dubois. 7,700-7,850' elevation; T40N R106W S16 SE1/4SE1/4 S21 N1/2E1/4; 14 Jun 1996; \textit{Fertig 16632} (RM).  Absaroka Mountains, East fork of the Big Wind River, 7 air mi. ESE of Dubois. 7,000' elevation; T41N R105W S20; 23 Jun 1983; \textit{Hartman 15583} (RM).  Wind River Indian Reservation, along U.S. Hwy 287, ca. 14 mi. SE of Dubois. 6,600' elevation; 2 Jul 1983; \textit{Evert 5268} (RM).  Wind River Indian Reservation. 8,300' elevation; T7N R5W S13 NW1/4; 10 Jul 1985; \textit{Lichvar 4561} (RM).  Absaroka Range, Wind River Indian Reservation, ca. 20 air mi. E of Dubois. 9,700' elevation; T7N R4W S28; 31 Jul 1981; \textit{Day \& Berner 25} (RM).  Wind River Indian Reservation, gypsum formation. 6,000' elevation; T6N R3W S16; 23 Jun 1982; \textit{Lichvar 5168} (RM).  Wind River Reservation, Pasup Creek, Circle Ridge Road. 6,965' elevation; T7N 3W S35; 1 Aug 1942; \textit{Murphey s.n.} (RM-404206).  Heavy clay soil, hillside, 15 mi. NW of Fort Washakie. 22 Jul 1983; \textit{Rollins 83330} (NY).  Wind River Indian Reservation, ca. 25 mi. N NW of Morton.  22 Jun 1981; R. C. \& K. W. Rollins 81386 (RM, NY, GH, UC, US).  Shoshone N. F.; outcrop, ca. 1/2 mi WNW of USFS Sinks Canyon Campground, ca. 7 mi SW of Lander. 8,500' elevation; T32N R101W S13 SE1/4; 15 Jun 1978; \textit{Johnston \& Lucas 1680} (RM).  Southern Wind River Range, ca. 8 mi. SW of Lander, SE side of Fossil Hill. 7,840' elevation; T32N R100W S30 SW1/4 of SW/14; 28 Jun 1995; \textit{Mills 6} (RM).  E of Lander. 6,000' elevation; T32N R99W S11 SW1/4; 18 May 1981; \textit{Lichvar 4215} (RM).  SE edge of Wind River Range; E of Dry Lake \& NE of US 287/WY 28 jct., ca. 7 air mi. SE of Lander; Lee Ranch. 5,600' elevation; T32N R99W S23; 16 Jun 1989; \textit{Marriott 11013} (RM).  13 mi. SE of Lander. 5,800' elevation; T32N R98W S33; 1 Jul 1991; \textit{Dorn 5242} (RM, NY).  SE Wind River Range: Red Canyon Rim on E side of Red Canyon, ca. 11 air mi. SSE of Lander.  5,700-5,800' elevation; T31N R99W S10 SW1/4NW1/4 \& S9 NE1/4NE1/4; 7 Jun 1994; \textit{Fertig 14811} (RM, NY).  E slope Wind River Range: Red Canyon Rim, on east side of county road, ca. 1.25 mi. W of WY Hwy 28. 5,800-5,900' elevation; T31N R99W S10 NW1/4; 14 May 1994; \textit{Fertig 14672} (RM).  E Slope Wind River Range, Red Canyon,  slopes W of Red Canyon Creek ca. 13 air mi. SSE of Lander. 6,000-6,200' elevation; T31N R99W S15 SW1/4; 24 May 1993; \textit{Fertig \& Studenmund 13573} (RM).  Red Canyon Rim, ca. 14 air mi. SSE of Lander. 6,100' elevation; T31N R99W S23, S25, \& S26; 20 Jun 1986; \textit{Haines 6712} (RM).  E slope Wind River Range, Red Canyon Rim above Foster Draw ca. 15 air mi. SSE of Lander.  6,350' elevation; T31N R99W S26 NE1/4 SE1/4; 8 Jun 1994; \textit{Fertig 14829} (RM).  17.9 mi. S of Lander on Hwy 28; E side of rd. 2,059m elevation; 42° 36' 47" N, 108° 36' 20" W; 26 Jun 1996; \textit{Salywon \& Dierig 3153} (ISTC).  Near State Route 28, 18.4 mi. SW of Lander. 22 Jul 1983; \textit{R. C. \& K. W. Rollins 83331} (NY, GH, US).  21 mi. S of Lander. 20 Jun 1969; \textit{Barneby 15100} (NY, GH).  17.65 mi. S of Lander on Atlantic City Road. 8,100' elevation; T30N R99W S17; 19 May 1946; \textit{Wiegand 207} (RM).  18 mi. SSE of Lander. 6,800' elevation; T30N R98W S8 NE1/4 SW1/4; 12 Jun 1991; \textit{Dorn 5177} (RM, NY, MO).  Great Divide Basin Area, Popo Agie River Drainage, Box Spring, ca. 20 air mi. SE of Lander; ca. 17 air mi. W of Sweetwater Station. 6,220-6,600' elevation; T30N R98W S12; 5 Aug 1995; Welp 7397 (RM).  Hills along Twin Creek. 6,100' elevation; T31N R98W S36 NW1/4; 30 May 1990; \textit{Dorn 5054} (RM, NY).  Sheep Mountain, ca. 11 air mi. SE of jct US Hwy 287 and Wyo. Hwy 28. 6,200' elevation; T31N R98W S36, T31N R97W S31; 31 May 1985; \textit{Hartman 20103} (NY).  Great Divide Basin Area, Popo Agie River Drainage, above Red Bluff Canyon, ca. 6 air mi. N of Schoettlin Mountain; ca. 5.5 air mi. SE of Weiser Pass. 6,600-6,700' elevation; T30N R97W S4; 4 Aug 1995; \textit{Welp 7180} (RM).  Sweetwater River Plateau, south end of Beaver Rim, ridge on N side of Red Canyon on east bank of Beaver Creek, 2-2.5 mi. W of US Hwy 287. 5,800-6,100' elevation; T30N R97W S1 NE1/4 of SW1/4 \& N2 of SE1/4; 30 Jun 1995; \textit{Fertig \& Studenmund 15808} (RM).  Beaver Rim Divide, ca. 5 mi NW from Sweetwater Junction on Hwy 287, and over 1 mi. NW on abandoned highway. 6,720' elevation; T30N R96W S3 NE1/4 of SE1/4; 9 Jun 2003; \textit{Heidel 2300} (RM, NY).  Beaver Rim, ca. 6 air mi. NW of Sweetwater Station. 6,680' elevation; T30N R96W S2; 20 Jul 1986; \textit{Haines 6926} (RM).  Beaver Hill, about 35 mi. SE of Lander along Route 287. 6,400' elevation; T30N R96W S11; 8 Jun 1960; \textit{Wetherell 256} (RM).  Top of Beaver Rim, Devil's Gap, ca. 7 air mi. NNW of Sweetwater Station. 6,800' elevation; T31N R96W S25; 14 Jun 1986; \textit{Haines 6429 \& 6387} (RM).  W Wind River Basin, Beaver Rim; ridges in vicinity of Devil's Gap, ca. 1.5 air mi. W of Dishpan Butte. 6,900-7,000' elevation; T31N R95W S30 NW1/4; 13 Jun 1994; \textit{Fertig 14848} (RM).  Sweetwater River Plateau, Beaver Divide, SE end of Dishpan Butte, 1.25 air mi. W of junction of Dishpan Butte Road \& WY Hwy 135. 6,820-6,880' elevation; T31N R95W S29 NE1/4 of SE1/4 of NW1/4; 9 Jul 1997; \textit{Fertig \& Welp 17668} (RM).  Sw edge of Cedar Rim, ca. 0.7 mi. E of Rte 135, 28 air mi. SSE of Riverton.  T31N R95W S27 SW1/4 \& S34 NW1/4; 11 Jun 1993; \textit{Anderson 14371} (NY).  6 mi. N of Sweetwater Station. 6,700' elevation; T31N R95W S27 SW1/4 \& S34 NW1/4; 1 Jul 1991; \textit{Dorn 5246} (NY, MO).  W Wind River Basin, Beaver Rim Divide, south end of Cedar Rim. 6,800' elevation; T31N R95W S27 SE1/4 SW1/4; 13 Jun 1994; \textit{Fertig 14841} (RM).  Sweetwater River Plateau, ridge system on N side of Government Meadow Draw, ca. 1-1.5 mi. E of WY Hwy 135, ca. 4.75-5 air mi. NNE of Sweetwater Station. 6,700-6,800' elevation; T31N R95W S34 NE1/4 of SE1/4, SE1/4 of NE1/4, \& E1/4 of SE1/4 of SE1/4, S35 SW1/4 of SW1/4 \& S1/2 of SE1/4 of SW1/4; 13 Jul 1997; \textit{Fertig 17696} (RM).  Beaver Rim, ca. 9 mi. N of Sweetwater Station. 6,800' elevation; T31N R95W S9; 27 Jun 1981; \textit{Dueholm 11670} (RM).  Wind River Basin, Beaver Divide, adj to WY Hwy 135, ca. 28 air mi. S of Riverton. 6,760-6,800' elevation; T31N R95W S3 NE1/4 of NW1/4; 13 Jul 1992; \textit{Fertig 13016} (RM).  Ca. 3 air mi. SSW of Big Sand Draw oil and gas field. 6,000' elevation; T32N R95W S28; 3 Jul 1981; \textit{Hartman 13510} (RM).  Beaver Rim. 7,100' elevation; T32N R95W S26 S1/2 of S1/2 of NE1/4; 2 Jul 1991; \textit{Dorn 5252} (RM).

\begin{center}
\textit{Physaria didymocarpa}
\end{center}

\textbf{Montana: Lincoln County:}  Mt. sides, Midvale, Montana. 9 Jul 1903; \textit{Umbach 305} (RM- 86904, RM-168738, MONT, F, NY).  \textbf{Flathead County:}  Common in stony limestone soil of an exposed ridge south of Sphinx Peak. 8,300' elevation; T22N R12W S28; 30 Jul 2004; \textit{Lesica \& Hanna 8910} (NY).  River bar Shafer Ranger Station. 4,400' elevation; 17 Jul 1937; \textit{Root s.n.} (MONTU-103050).  0.5 mi. N of Pagoda Mt. approx. 32 mi. SE of Spotted Bear Ranger Station. 8,047' elevation; T22N R13W S3; 10 Jul 1996; \textit{Wirt 89} (ISTC).  \textbf{Glacier County:}  100 yd. west St. Mary bridge. 4,530' elevation; T35N R14W S33 NE1/4; 17 Jun 1947; \textit{McMullen GNP 2821, 2836} (RM).  2 mi. below summit of Siyeh Pass - Baring Basin. 15 Jul 1934; \textit{McLaughlin 3277} (F).  Divide Mtn., Glacier National Park; collected on east slope of peak in talus. 2,200m elevation; 9 Aug 1964; \textit{Harvey \& Pemble 7213} (MONTU).  Divide Mtn. 16 Jul 1897; \textit{Williams s.n.} (MONT-9298).  Glacier National Park, Two Medicine trail to Cutbank Pass, just below pass, in rocks. 7,500' elevation; 14 Aug 1939; \textit{H. Bailey \& V. Bailey 470} (UC).  Rising Wolf Ranch at U.S. 2. 1,220m elevation; 3 Jul 1950; \textit{Harvey 4174} (MONT).  Medicine Ridge at U.S. 89. 1,820m elevation; 13 Jul 1954; \textit{Harvey 5836} (MONT).  Many Glaciers Road, Glacier National Park. 27 Jun 1964; 27 Jun 1964; \textit{Harvey 7067} (MONTU).  \textbf{Pondera County:}  1.0 mi. S of Family Pk., 9.75 mi. E. Schafer Meadows Ranger Station - air strip area, ca. 2.5 mi. of Beaver Lk. 7,000' elevation; T28N R11W S33; 11 Aug 1995; \textit{Wirt 79} (ISTC).  \textbf{Teton County:}  East Front Mountains. 7,800' elevation; T26N R10W S14 N1/2; 28 Jul 1983; \textit{Lackschewitz 10565} (MONTU).  Front Range, MT. Wright, W. of Chouteau; SSE facing dry wash. Ca. 6,600' elevation; T26N R10W S36 NWSE1/4; 23 Jul 1982; \textit{Ramsden 1110} (MONTU).  East Front Mountains; Chouteau Mtn. 8,000' elevation; 7 Jul 1977; \textit{Lackschewitz 4432} (MONTU).  Bob Marshall Wilderness - Trail across Headquarters Pass. 7,650' elevation; T24N R9W S19 E1/2; 29 Jul 1978; \textit{Lackschewitz 8476} (MONTU).  Along road Chouteau-N-Fk. Teton R. T25N R8W S25 SW1/4; 30 Jun 1988; \textit{Lackschewitz 11452} (GH).  Two \& 1/2 mi. E below confluence of Teton River forks. 4,800' elevation; T25N R8W S25S S36N; 8 Jul 1973; \textit{Lackschewitz 4494} (MONTU).  Quarter mi. W of Pine Butte Swamp. 4,800' elevation; T24N R8W S11; 26 May 1995; \textit{Lesica 6580} (MONTU).  Common in shaly, barren soil on a steep west-facing slope at the head of Rierdon Gulch. 7,500' elevation; T24N R8W S23; 20 Aug 1995; \textit{Lesica 7104} (MONTU).  Foot of Rocky Mts., along Hwy. US-287 0.2 mi. N of Sevenmile Hill; 6.8 mi. SSW of Chouteau. 4,300' elevation; T23N R5W S27 SE1/4; 13 Jul 1975; \textit{Stickney 2261} (RM, MONT-66338, MONT-66339).  \textbf{Hill County:}  Bear's Paw Mtns., steep south-facing roadcut at the head of Big Sandy Creek, ca. 20 mi. S of Havre. 5,400' elevation; T28N R16E S20; 28 Jun 1988; \textit{Lesica 4609} (GH).  \textbf{Chouteau County:}  Eagle Creek Landing, floating the Missouri White Cliffs area. 2,500' elevation; T25N R13E S21; 27 Jun 1990; \textit{Lackschewitz 11656} (MONT).  \textbf{Missoula County:}  \textit{Kennedy s.n.} (MONT-9356).  Mouth of Lolo Canyon. 10 Apr 1938; \textit{Keilman 9} (MONT).  Miller Creek Canyon. 1,090m elevation; 9 May 1950; \textit{Harvey 4103} (MONT).  Growing on talous slope E of Missoula, Montana. 9 May 1960; \textit{Ruff s.n.} (UTC-00112173).  \textbf{Ravalli County:}  Bitterroot vicinity - Bitterroot Valley, tributary of Three Mile Creek; 10 mi. NE of Stevensville. 4,000' elevation; T10N R19W S24 SE1/4; 25 May 1965; \textit{Stickney 1195} (RM).  \textbf{Granite County:}  Shale banks ca. 32 mi. E of Missoula. 20 Jun 1944; \textit{Hitchcock \& Muhlick 9119} (RM, NY, MONT, UC, CAS, IDS, UTC).  Ca. 5.5 mi. W of Drummond between Rattlesnake Gulch and Mulkey Gulch. 1,250m elevation; 46º42'40"N, 113º14'43"W; 10 Jun 1996; \textit{O'Kane Jr. 3794} (MO).  Along N-side of Hwy 38 (Skalkaho Rd) between E-Fk. Rock Cr. Rd and the Middle Fk. Rd. 8 May 1977; \textit{Lackschewitz 7187} (RM, MONTU).  Roadcut - slide above old Hwy 10, Bearmouth Area. 3 Jun 1983; \textit{Lackschewitz 10433} (MONTU).  On roadcut above the old Hwy 90 in the Bearmouth Area. 13 May 1977; \textit{Lackschewitz 7176} (RM, MONTU).  Highway 10, 35 mi. E of Missoula. 3,220' elevation; 19 May 1938; \textit{Rose 107} (MONTU, MONT).  46 mi. E of Missoula. \textit{Gillett \& Moulds 12437} (NY).  On sandy bank near Hiway, ca. 40 mi. E. of Missoula. 5 May 1934; \textit{Hitchcock 2293} (RM, MONT, RSA-POM, MO).  \textbf{Powell County:}  Flathead National Forest; on top of Shale Mt., 5 mi. NE of Big Prairie Ranger Station. 22 Jul 1948; \textit{Hitchcock 18620} (RM, UC).  Flathead National Forest; Top of Gordon Mt., 6 mi. S of Big Prairie Ranger Station. 8,300' elevation; 22 Jul 1948; \textit{Hitchcock 18861} (RM, UC).  \textbf{Lewis \& Clark County:}  Bob Marshall Wilderness, Mountain S of Sock Lake, Upper E-Slope. 8,100' elevation; T24N R11W S31 SE1/4; 26 Jul 1979; \textit{Lackschewitz 9106} (RM).  Sawtooth Ranch. 4,200' elevation; T21N R7W S19 NE1/4; 24 May 1988; \textit{Lackschewitz 11372} (GH).  Flathead Range; uncommon in a limestone fellfield on a gentle southeast-facing slope of Flint Mtn. 8,900' elevation; T18N R10W S8; 25 Jul 2009; \textit{Lesica \& Hanna 10206} (MONTU).  Scapegoat Mountain. 7,600-8,400' elevation; 11 Aug 1975; \textit{Craighead 62} (MONTU).  Lewis and Clark National Forest. 5,500' elevation; T18N R8W S9; 25 Jul 1928; \textit{Harris 9} (RM).  Haystack Butte. 5,000' elevation; 27 May 1973; \textit{Lackschewitz 4263} (MONTU).  Low ridges along Mission Road 1 mi. W of the Dearborn River. 4,000' elevation; T17N R4W S27; 10 Jun 2012; \textit{Lesica 10808} (MONTU).  7.5 mi. NE of Wolf Creek. 3,800' elevation; 27 Jun 1963; \textit{Mulligan \& Mosquin 2820} (CAS).  York bridge on Missouri R. NE of Helena. T11N R2W S13 NW1/4; 6 May 1980; \textit{Ramsden 524} (MONTU).  About 2 mi. NE of Lakeside at mouth of canyon. 3,700' elevation; 4 Jul 1967; \textit{McKinney s.n.} (MONT-63192).  Prickly Pear Canon. 5 Aug 1887; \textit{Williams 515} (MONT).  Helena, Montana. \textit{Brandegee 368} (UC-117221, UC-117506).  \textbf{Cascade County:}  N. Fork of Sun River. 5,400' elevation; 2 Sep 1925; \textit{Kirkwood 2358} (UC).  Belt River (Milk Ranch) Montana. 12 Jun 1884; \textit{Anderson 41} (MONT).  \textbf{Judith Basin County:}  Arrow creek above mouth of Coffee Creek. 3 Jul 1901; \textit{Spragg 185} (MONT).  \textbf{Fergus County:}  Judith Mtns.; common in limestone scree on a steep west-facing slope above Maiden Canyon ca. 4 mi. NE of Giltedge. 4,600' elevation; T16N R20E S9; 20 Jun 1987; \textit{Lesica 4330} (GH).  Big Snowy Mtns.; Common in a limestone fellfield on top of Greathouse Peak. 8,600' elevation; T12N R19E S29; 19 Jul 2007; \textit{Lesica \& Hanna 9833} (ISTC, MONTU).  East-facing slope of Greathouse Pk. 8,200' elevation; T12N R19E S29; 24 Jul 1981; \textit{Lesica 1658} (MONTU).  Top of Grayhouse [sic] Peak, Big Snowy Mts. 3 Jul 1947; \textit{Hitchcock 16052} (NY, UC, RSA-POM, UTC).  On shale bank in Half Moon Canyon 5 mi. from mouth, Big Snowy Mts. 5 Jul 1945; \textit{Hitchcock \& Muhlick 12012} (NY, CAS, UTC).  Shale bank at western base of Little Snowy Mts. 4 Jul 1945; \textit{Hitchcock \& Muhlick 11931} (RM, NY, MONT, MO, CAS, UTC, UC).  \textbf{Golden Valley County:}  Ryegate, 8 mi. NE.  17 Jun 1957; Booth 57127 (MONT).  \textbf{Wheatland County:}  Harlowton, west of town in bunchgrass high cinder content soil. 22 May 1955; \textit{Booth 5514} (MONT).  Harlowton, 10-12 mi. S. 17 Jun 1957; \textit{Booth 5750} (MONT).  \textbf{Meagher County:}  Little Belt Mts., near the Pass.  7,000’ elevation; 10 Aug 1896; \textit{Flodman 496} (NY, MO), 596 (NY).  Cottonwood Creek.  5,000’ elevation; 30 Jul 1896; \textit{Flodman 495} (NY).  11.4 mi. N of White Sulphur Springs. 4 Jul 1963; \textit{Mosquin \& Gillett 5222} (RM, UC, UTC).  South-facing slope above Richardson Creek. 6,200’ elevation; 46º32.146'N, 110º42.330'W; 17 Jul 2012; \textit{Lesica 10820} (MONTU).  On moist ridge 1 mi. N of Baldy Mt., Big Belt Mts.  8,500’ elevation; 16 Jul 1945; \textit{Hitchcock \& Muhlick 12377} (CAS).  Belt Mountains. 8 Jul 1886; \textit{Anderson 411} (NY).  Belt Mountains. 8 Jul 1886; \textit{Anderson 42} (F).  Belt Mts. Montana.  12 Jul 1886; \textit{Anderson s.n.} (MONT-9302). Red shale outcrop near Four Mile Ranger Station, northeastern base of Castle Mtns. 8 Jul 1945; \textit{Hitchcock \& Muhlick 12075} (RM, NY, MO, UTC, UC, MONT, RSA-POM).  Red sandstone outcrop ca. 6 mi. W of Lennep. 2 Jul 1951; \textit{Hitchcock 5983} (RM, NY, MONT, RSA-POM, UC, UTC, IDS-030425, IDS-030426).  \textbf{Beaverhead County:}  Tendoy Mtns., locally common in barren gravelly soil on a steep, west-facing slope N of Poison Lakes. 8,000’ elevation; T11S R11W S27; 5 Jul 2002; \textit{Lesica 8475} (NY).  Tendoy Mountains, Bell Canyon; ca. 4 air mi. SW of Red Rock.  7,000’ elevation; T11S R11W S24; 26 Jun 1993; \textit{Vanderhorst 4983} (MONT).  Tendoy Mtns., common in stony limestone soils on a steep southwest-facing slope on the ridge W of Pileup canyon. 7,600’ elevation; T14S R10W S32; 10 Jul 1993; \textit{Lesica 6073} (NY, MONTU).  Mt. Lima, Mont. 30 Jun 1895; \textit{Shear 3406} (NY).  Beaverhead Mts., common in shallow calcareous soil on a steep, W-facing slope of the spur ridge of the unnamed peak 2 mi. E of Red Conglomerate Peak. 8,400’ elevation; T15S R8W S36; 8 Jul 1986; \textit{Lesica 3938} (GH).  Abundant in sandy, eroding soil on an east-facing slope above the East Fork Peet Creek. 6,850’ elevation; T14S R4W S27; 7 Jul 2005; \textit{Lesica 9393} (NY, MONTU).  Centennial Valley-Sheep Mountain, approx. 1.5 air mi. SE of Lakeview. 2,700m elevation; T14S R1W S28 E2; 20 Jun 1993; \textit{Culver \& Lavin 254} (MONT-72332, MONT-72333).  Dewey, Mont. 7,000’ elevation; 24 Jun 1902; Blankinship s.n. (MONT-9352). \textbf{Madison County:}  Mountains near Indian Creek, Montana. 8,000’ elevation; 22 Jul 1897; \textit{Rydberg \& Bessey 4166} (NY, F).  Ruby Peak, Ruby Mountains, East of Dillon, Montana. 9,300’ elevation; T6S R5W S16; 22 Aug 1982; \textit{Rosentreter 2836} (MONTU).  Gravelly Range; common in stony limestone-derived soil of the exposed summit of Baldy Mtn. 9,500’ elevation; T7S R3W S27; 17 Jun 2006; \textit{Lesica \& Kittelson 9636} (MONTU).  Madison Range - W. Slope - 8 mi. S of Jeffers. 17 Jun 1930; \textit{Young} (MONT-9350).  Cedar Mountain, Montana. 10,000’ elevation; 16 Jul 1897; \textit{Rydberg \& Bessey 4168} (NY).  Northern Rocky Mt'N., Beaverhead National Forest, Lewis Creek reseeding area. 6,300’ elevation; T9S R3W S28; 20 Jun 1937; \textit{Short \& Aicher S-525} (MONT).  Beaverhead National Forest - NRM Ranger Station; Schoolmarm Gulch. 6,050’ elevation; T9S R3W S17 SW4; 18 May 1952; \textit{Schmautz JES-28} (RM, MONT).  Uncommon on a rocky, south-facing slope of Cave Mtn. ca. 25 mi. S of Virginia City, plot 14. 9,600’ elevation; T10S R1W S32; 20 Jul 1989; \textit{Lesica \& Cooper 4914} (NY).  Madison Range; common in gravelly soil on top of a limestone ridge in Koch Basin, plot 59. 9,600’ elevation; T9S R2E S23; 2 Aug 1991; \textit{Lesica \& Cooper 5571} (NY).  Madison Range; common in limestone-derived soil on an exposed ridge in Koch Basin. 9,800’ elevation; T9S R2E S23; 28 Jul 1006; \textit{Lesica \& Cooper 9650} (MONTU).  On mountain 1/2 mi. N of Koch Peak; Taylor Mts. 2 Aug 1946; \textit{Hitchcock \& Muhlick 15225} (NY, UC, MO).  Madison Range, Gallatin National Forest; Koch Basin on the east side of Koch Peak at head of Tumbledown Creek, ca. 25 mi. SE of Ennis. 9,800’ elevation; T9S R2E S14; 7 Aug 1995; \textit{Evert 30558} (RM).  \textbf{Gallatin County:}  Bozeman. Jul-Aug 1906; \textit{Blankinship 63} (RM-89448, UC-311010, RSA-POM-143083, UTC-00159819).  Rocky Canon. 5,000’ elevation; 14 Jun 1905; \textit{Blankinship 63} (MO-3833621, F).  Mt. Bridger. 8,000’ elevation; 5 Jul 1905; \textit{Blankinship 64} (RM-89447, F-190110).  Bozeman. 24 May 1901; \textit{Moore s.n.} (RM-73253, MO-3833623).  Bozeman. 10 Jun 1921; \textit{Bohars s.n.} (MONT-26011).  Bozeman. 10 May 1904; Jackson s.n. (MONT-9295).  Bozeman. 20 Jun 1902; \textit{Jones s.n.} (UC-165173).  Bozeman. 24 May 1901; \textit{Jones s.n.} (RM-122279).  Bozeman. 16 May 1904; \textit{Maynard s.n.} (MONT-9311).  Bozeman. 14 May 1905; \textit{Flaherty s.n.} (UC-165233).  2.5 mi. NE Bozeman. 30 Apr 1921; Powers 14 (MONT).  Foothills 3 mi. NE of Bozeman. 30 Apr 1921; \textit{Savage 15} (MONT).  Bozeman.  10 May 1982; \textit{Blankinship s.n.} (MONT-9296).  Near Bozeman. 8 Jun 1883; \textit{Scribner 8} (NY).  East of Bozeman; near electric powerhouse. 23 May 1900; \textit{Wilcox 159} (NY).  Ft. Ellis to the Yellowstone. Jul 1891; \textit{Porter s.n.} (NY).  Shale bank at the south base of Bridger Mts., 8 mi. NE of Bozeman. 18 Jul 1945; \textit{Hitchcock \& Muhlick 12470} (RM, NY, UTC, UC).  Outside Gallatin National Forest. 6,100’ elevation; T2S R7E S13 E2; 20 Jun 1928; \textit{Swim 714} (RM, MONT).  Bridger foot-hills. 24 May 1921; \textit{Kindsley 33} (MONT).  Bridger Canyon. 5 Jun 1921; \textit{Hall s.n.} (MONT-26081).  Bridger Pass – Bozeman. 22 May 1898; \textit{Wilcox s.n.} (MONT-9303).  Bridger Pass. 22 May 1897; \textit{Blankinship s.n.} (MONT-9305).  Mt. Bridger. 9,000’ elevation; 3 Jul 1900; \textit{Blankinship s.n.} (MONT-9309).  Mt. Bridger. 7,000-8,000’ elevation; 11 Jul 1903; \textit{Blankinship s.n.} (MONT-9351).  Sedan. 30 May 1901; \textit{Jones s.n.} (MONT-9306).  Summit of Mt. Bridger, Bozeman. 9,000’ elevation; 26 Jun 1899; \textit{Blankinship s.n.} (MONT-9300).  Bridger Peak. Jun 1889; \textit{Kock s.n.} (MONT-9301).  Mt. Baldy, Bridger Mts.  7 Jul 1977; \textit{Forcella 69354} (MONT).  Bridger Peak. 8,900’ elevation; 25 Jun 1933; \textit{Young s.n.} (RM-139984, MONT-22915, UTC-15083).  Gallatin N. F., Bridger Mountains.  East slopes and summit of Mount Sacagawea, including the Bridger Divide (saddle between Mt. Sacagawea and Hardscrabble Peak); collected along trail from Fairy Lake to Sacajawea summit. 8,000’ elevation; T2N R6E S22 \& S27; 7 Aug 1990; \textit{Bayer, Lebedyk, \& Joncas MT-620} (RM).  Just N of the summit of Sacagawea Peak ca. 17 mi. N of Bozeman; Bridger Range. 9,000-9,500’ elevation; T2N R6E S27 NW4; 10 Aug 1989; \textit{Evert 18466} (RM).  Sacagawea Peak, Bridger Mts. 8,800’ elevation; 19 Jul 1969; \textit{Dorn 913} (RM). Gallatin National Forest, Sacagawea Peak. 9,800’ elevation; 8 Aug 1935; \textit{Whitham 1805} (RM, MONT).  Sedan, 3 mi. SW exposed dry slope. 26 Jun 1949; \textit{Metcalf s.n.} (MONT-77931).  Valley north of Sacajawea Peak, Bridger Range. 8,100-9,200’ elevation; 31 Jul 1938; \textit{Pennell, Cotner, \& Schaeffer 23832} (NY, US).  Manhattan, Mont.; 10 mi. N in Horseshoe Hills. 15 May 1949; \textit{s.n. s.n.} (MONT-42418).  Rocky hillside, near Maudlow. 1 Jun 1956; \textit{Denton s.n.} (MONT-52764).  Madison Range; common in limestone talus on a west-facing slope of Cone Peak. 9,600’ elevation; T10S R4E S26; 10 Jul 2007; \textit{Lesica \& Kittleson 9808} (MONTU, ISTC).  Bridger Mountains, Mont. 9,000’ elevation; 15 Jun 1897; \textit{Young 4167} (RM, MONT, F).  \textbf{Park County:}  Suksdorf's Gulch, 9 mi. NW of Wilsall. 19 Jul 1921; Suksdorf 513 (NY).  Ridge 1/8 - 1/4 mi. N of Cokedale Rd., 6 mi. W of Livingston; Gallatin Range foothills. 5,100-5,300’ elevation; T2S R8E S24; 14 Jun 1996; \textit{Evert 31157} (RM).  1 mi. NW Livingston. 20 May 1951; \textit{Wright 25} (MONT).  Livingston. 1901; \textit{Scheuber 363} (NY).  Livingston, Mont. 20 May 1901; \textit{Scheuber s.n.} (NY-2526, UC-991364).  Livingston. 20 May 1905; \textit{Scheuber s.n.} (MONT-9308).  About 3 mi. E of Livingston on a rocky slope, Cemetery road. 17 May 1959; \textit{Hoversten s.n.} (MONT-56052).  Absaroka Mtns., common in limestone talus near the summit of Elephanthead Mtn.  9,400’ elevation; T3S R11E S30; 14 Aug 1998; \textit{Lesica 7719} (NY, MONTU).  Chico Hot Springs. 15 Jul 1921; \textit{Suksdorf 443} (RM, NY, UTC, MO, RSA-POM).  Absaroka Forest. 5,000’ elevation; T6S R8E S27; 2 Jun 1924; \textit{Moir 73} (RM).  Electric Peak. 10,000’ elevation; 26 Jul 1902; \textit{Rev. Earnest \& Smith 25} (F).  \textbf{Sweet Grass County:}  Common in stony, calcareous soil on the rim of the saddle north of Picket Pin Mtn. 9,400’ elevation; T4S R14E S19; 28 Jul 1998; \textit{Lesica 7672} (NY).  Beartooth Mtns., common in gravelly limestone fellfield on the mountain ca. 1 mi. N of Picket Pin Mtn. 9,500’ elevation; T4S R14E S20; 10 Aug 1987; \textit{Lesica 4481} (GH).  Saddle between Picket Pin Mtn. \& unnamed pinnacled limestone ridge, Custer Natl. Forest, just N of Picket Pin Rd. ca. 12 mi. W of Nye; Beartooth Mtns. 9,000’ elevation; T4S R14E S29; 21 Jul 1994; \textit{Evert 28321} (RM).  Along Sliderock Mtn. - Grouse Ridge Rd. (F.S. \#482), Gallatin Natl. Forest; Beartooth Mtns. 7,100’ elevation; T3S R14E S14; 20 Jul 1991; \textit{Evert 22091} (RM).  Common in shallow, sparsely-vegetated metamorphic-derived soil of a ponderosa pine woodland on a south-facing slope above Jim's Gulch. 5,200’ elevation; T2S R15E S34; 12 Jun 2005; \textit{Lesica 9143} (NY).  Beartooth Mountains; in foothills, ca. 1/4 mi. W of Forest Road 482 (Grouse Ridge / Sliderock Mountain Road); ca 5 air mi. S of I-90. 5,000-5,200’ elevation; T2S R15E S7 \& S8; 4 Jun 1993; \textit{Evert 24792} (RM).  MacLead [sic]. 4 Jun 1923; \textit{Pope 145}.  \textbf{Stillwater County:}  Beartooth Mountains, ca. 1/4 mi. N of Castle Creek, in foothills ca. 7 mi. NW of Nye. 6,000’ elevation; T4S R15E S30 NE4; 9 Jul 1993; \textit{Evert 25821} (RM).  Midnight Canyon. 23 Apr 1923; \textit{s.n. s.n.} (MONT-34167).  Ca. 1/4-1/2 mi. S of Cliff Swallow fishing access site ca. 10 mi. W of Absarokee. 4,700’ elevation; T4S R17E S4; 12 Jun 1992; \textit{Evert 22811} (RM).  Absorokee, Mont. – Jackstone Creek. 26 Jul 1923; \textit{Hawkins s.n.} (MONT-34162).  Absorokee, Mont. 26 Jun 1922; \textit{Hawkins s.n.} (MONT-34163).  Absorokee, Mont. 21 Jun 1914; \textit{Hawkins s.n.} (MONT-34168).  Absorokee, Mont.  20 Jun 1924; \textit{Hawkins s.n.} (UC-372122).  Absorokee, Mont. 28 Jun 1922; \textit{Hawkins s.n.} (MONT-34166).  Fishtail, Mont. 26 May 1922; \textit{Hawkins s.n.} (MONT-34164).  2 mi. W of Columbus; Hwy. 10. 25 May 1925; \textit{J.C. Wright \& A. Wright s.n.} (MONT-43983).  4 mi. N of Columbus, Mont.; breaks of the Yellowstone River. 31 May 1948; \textit{Payne s.n.} (MONT-38787).  Along the road from Mouat Mill to Horseman’s Flat. 5,300’ elevation; 20 Jun 1976; \textit{Robertson 1119} (RM).  
