***Physaria integrifolia***

%%% WYOMING STATE SPECIMENS
\textbf{Wyoming} 

\textit{Teton}

Gros Ventre Fork. 10 Jun 1860, 	\textit{Hayden s.n.} (MO-3833631).  Teton Peaks. \textit{Davis s.n.} (NY, IDS-0030427).  Jackson Hole. 6 Jul 1860; \textit{Hayden s.n.} (MO-3833625).  Teton National Forest, 2 mi. above the mouth of Game Creek. 6,400' elevation; 11 Jul 1932; \textit{Buchenroth 159} (NY).  Infrequent in rocks on Switchbacks up to "The Wall" in South Cascade Canyon. 5 Sep 1955; \textit{Shaw 1011} (UTC).  Plants of Grand Teton National Park, possibly on Mt. Woodring. \textit{Woodring s.n.} (MO-1066430).  On ridge west of The Wall, N of Alaska Basin, Targhee Nat'l Forest. 10,500' elevation; 3 Aug 1955; \textit{Anderson 230} (NY, UTC).  Grand Teton National Park, 1/2 mi. E of Mount Hunt Divide Trail. 9,500' elevation; 11 Jul 1965; \textit{Shaw 1457} (UTC).  Teton Range East Slope; Rendevzvous Mountain, just below ridge between Granite Canyon and Jackson Hole, ca. 3/4 mi. W of Apres Vous Peak, ca. 1.5 air mi. NW of Teton Village; ca. 8 air mi. NW of Jackson. 9,200-9,500' elevation; T42N R117W S14; 31 Jul 1996; \textit{Markow 11323} (RM).  Targhee National Forest, West Slope Teton Range; ca. 6-7 air mi E of Victor, Idaho.  8,400-9,400' elevation; T42N R118W S10, S11, S12, S14, \& S15; 19 Aug 1991; \textit{Markow 6550} (RM).  Targhee NF, Teton Range; E-facing slope of Taylor Mtn, ca. 100 yds. below summit ridge, ca. 12 air mi. NW of Jackson.  9,800-10,200' elevation; T41N R118W S11; 25 Sep 1995; \textit{Markow 11199} (RM).  Targhee National Forest, Northwest Slope Snake River Range; Mail Cabin Creek, ca. 11 air mi. W of Jackson.  7,600-8,000’ elevation; T41N R118W S21 \& S22; 22 Jul 1991; Markow 3781 (RM).  In Pumice formation on hills near Adam's Ranch, Jackson. 6,200' elevation; T40N R117W S4; 10 Jul 1901; \textit{Merrill \& Wilcox 960} (RM).  Teton National Forest, big point ending in Snake R. below Fall Creek. 6,000' elevation; T39N R116W S27; 12 Jul 1928; \textit{McDonald 816} (RM).  Gros Ventre Area, 1.5 air mi. S to 1 air mi. SE of Pinnacle Peak.  9,200-10,000' elevation; T39N R114W S3 NW1/4 \& S4 SE1/4; 7 Jul 1994; \textit{Hartman 47330} (RM, RSA-POM).  Hillside, ca. 3.5 mi. E of Kelly Warm Springs. 2,053m elevation; 43º38'02"N 110º33'57"W; 25 Jun 1996; \textit{Salywon \& Dierig 3146} (ISTC).  Slide at Gros Ventre Lake. 7,000' elevation; 10 Sep 1951; \textit{Munz 16997} (NY, RSA-POM).  Gros Ventre River. 16 Aug 1894; \textit{Nelson 927} (RM, MO).  Gros Ventre slide area, Teton National Forest. 7,000' elevation; 10 Jul 1959; \textit{C.L. \& M.W. Porter 7891} (UC, CAS).  Gros Ventre Slide Road, 1/4 mi. W of Forest Service Exhibit on Gros Ventre Slide. 6,800' elevation; 8 Jul 1971; \textit{Shaw 1812} (UTC).  Lower Slide Lake. 7,100' elevation; T42N R114W S5; 24 May 1977; \textit{Lichvar 99} (RM).  Mount Leidy Highland Area, Gros Ventre River Road, ca. 2 air mi. E of Forest boundary; just W to overlooking Lower Slide Lake, on the north side.  6,900-7,400' elevation; T42N R114W S5; 26 Jun 1995; \textit{Hartman 51256} (RM).  Bridger-Teton Natl. For.; 1/3 mi. E of Atherton Creek Campground, Gros Ventre Canyon Road. 7,000' elevation; 18 Jun 1979; \textit{Shaw 2482} (UTC).  Crystal Creek. 6,950' elevation; R113W T42N S18; 7 Jul 1977; \textit{Lichvar 691a} (RM).  Teton Forest - Miner Creek. 8,000' elevation; T42N R113W S17; 20 May 1913; \textit{Maris 58} (RM).  Mount Leidy Highland Area, Gray Hils; N of Gros Ventre River, ca. 2 air mi. NW of west end of Uppper Slide Lake, ca. 21 air mi. ENE of Jackson. 7,600' elevation; T42N R113W S14; 6 Jul 1990; \textit{Nelson 19288} (RM).  Gros Ventre Area, Gros Ventre Wilderness Area; ridge E of Crystal Creek, ca. 12 air mi SE of Kelly.  7,200-8,000' elevation; T42N R113W S34; 24 Jun 1994; \textit{Hartman 46465} (RM).  Mount Leidy Highland Area, Gray Hills; ridge and adjacent area overlooking Slate Creek, ca. 4.5 air mi. NW of west end of Upper Slide Lake. 7,900-8,100' elevation; T42N R113W S35; 6 Jul 1990; \textit{Hartman 26343} (RM).  Gros Ventre Area, Gros Ventre Wilderness Area; ridge E of Crystal Creek, ca. 14 air mi. SE of Kelly. 8,600-9,450' elevation; T41N R113W S1, S2, \& S13; 24 Jun 1994; \textit{Hartman 46566} (RM).  Mount Leidy Highland Area, Burnt Creek and ridge to east. 7,200-8,200' elevation; T42N R112W S30 \& S31; 24 Jul 1995; \textit{Hartman 52772} (RM, RSA-POM).  Gros Ventre Area; upper Gros Ventre River Road, ca. 1.5 air mi. SE of Goosewing Guard Station. 7,100' elevation; T41N R112W S3 NE1/4; 26 Jun 1994; \textit{Hartman 46738} (RM).  West Slope Wind River Range; hills on eastern bank of Cottonwood Creek, ca. 28 air mi. E of Jackson. 7,600-8,280' elevation; T42N R111W S29, T42N R111W S30 \& S31, \& T42N R112W S36; 6 Jul 1990; \textit{Fertig 3025} (RM).  West Slope Wind River Range; Fish Creek/Moccasin Basin area, along Fish Creek just below confluence of North and South Forks, ca. 28.5-29.5 air mi. ENE of Jackson. 7,600-7,700' elevation; T42N R111W S28 \& S29; 25 Aug 1990; \textit{Nelson 20322} (RM).  West Slope Wind River Range; South Fork Fish Creek between Hackamore and Bell Creeks. 7,800-7,850' elevation; T42N R111W S36; 25 Aug 1990; \textit{Hartman 28350} (RM).  N Wind River Range; N bank of South Fork Fish Creek just W of confluence of Devils Basin Creek, ca. 4.25 air mi. NNW of Union Pass Rd. 8,000' elevation; T41N R110W S8 NW1/4 of SE1/4; 11 Aug 1995; \textit{Fertig 16262} (RM).  West Slope Wind River Range; Fish Creek / Moccasin Basin Area, North Fork Fish Creek between Packsaddle Creek and Harness Gulch. 7,800-7,900' elevation; T42N R111W S11 \& S15; 12 Jul 1990; \textit{Hartman 27148} (RM).  West Slope Wind River Range; North Fork of Fish Creek, ca. 32 air mi. NE of Jackson. 8,000-8,530' elevation; T42N R111W S2; 11 Jul 1990; \textit{Fertig 3535} (RM).  West Slope Wind River Range; Cottonwood and Moosehorn Creeks. 8,100-8,300' elevation; T42N R111W S5 \& T43N R111W S32; 23 Aug 1990; \textit{Hartman 28213} (RM).  Mount Leidy Highland Area; Grouse Mountain, southern ridge. 8,800-9,800' elevation; T43N R112W S2 \& NE1/4 S11; 15 Aug 1995; \textit{Hartman 53644} (RM).  Mount Leidy Highland Area; summit and upper slopes of Mount Leidy proper. 9,800-10,200' elevation; T43N R113W S3 \& NW1/4 S3; 13 Aug 1995; \textit{Hartman 53435} (RM).  Mt. Leidy. 10,000' elevation; \textit{Tweedy 391} (NY).  Mount Leidy Highland Area; Spread Creek, ca. 5 air mi. SW of Black Rock Ranger Station. 7,300-7,500' elevation; T44N R113W S16; 14 Jul 1995; \textit{Hartman 51830} (RM).  Mount Leidy Highland Area; 0.5-1 air mi. NW of Gunsight Pass on flank of plateau. 9,100-9,200' elevation; T42N R112W S10 W1/2; 12 Aug 1995; \textit{Hartman 53278} (RM).  Steep shale cliffs above Spread Creek. 7,500' elevation; T44N R113W S19; 13 Jun 1948; \textit{J.F. \& M.S. Reed 2302} (RM).  1.5 mi. E of Elk Ranch Reservoir. 7,300' elevation; T44N R114W S3; 22 Jun 1971; \textit{Dorn 1281} (RM).  Grand Teton National Park and Vicinity, Jackson Hole; "Wolff Ridge" on the N side of Spread Creek Valley, ca. 3.5 air mi. SW of Moran; ca. 25 air mi. NE of Jackson. 6,900-7,160' elevation; T44N R114W S9 E1/2; 15 Jun 2006; \textit{Nelson 68918} (NY).  