\section*{\textit{Physaria didymocarpa ssp. didymocarpa}}

%%% UNKOWN LOCALITIES
  \textbf{Unkown State:}
  \textbf{Unknown County:}
Rocky Mountain Flora. \textit{Hall and Harbour 47} (NY).
Palliser's Brit. N. Am. Expl. Expedition, Rocky Mountains. 1858;
\textit{Bourgeau s.n.} (NY).
Franklin's Journey. \textit{Hooker s.n.} (NY).
Rocky Mts. \textit{Durand s.n.} (NY).
Platte Rocks. \textit{Durand s.n.} (NY).
%%% CANADIAN PROVINCIAL SPECIMENS
  \textbf{Canada:}
  \textbf{Alberta:}
Bow River Valley, Banff. 4,500' elevation; 9-18 Jun 1906;
\textit{Brown 123} (MO).
Bow River. Jul 1897; \textit{Van Brunt and Van Brunt 70} (NY).
Gravelly and rocky slopes, vicinity of Banff. 4,700-7,000' elevation;
5 Jul 1899; \textit{McCalla 2270} (NY).
Devil's Lake, Banff; Alberta, Canada. 5,800' elevation; 5 Jul 1907;
\textit{Butters and Holway 41} (RM, NY).
Morley, foothills of Rocky Mts. 24 Jun 1885; \textit{Macoun s.n.} (NY);
18 Jul 1885; \textit{Macoun s.n.} (NY).
Campsite Island Lake. 4,500' elevation; Lat 114º40'00"W, Long 49º38'00"N;
22 Jul 1969; \textit{Warrington and Nagy 551} (MONT).
5 mi. N of Pincher Creek, dry south facing slope above Oldman River.
30 May 1963; \textit{Mulligan and Mosquin 2668} (NY).
Dry, open, south-facing slope, Rowe Creek; Canadian zone. 5,800' elevation;
21 Jul 1953; \textit{Breitung 16347} (NY).
East end Lakeview Ridge. 5,400' elevation; 113º53'00"W 49º09'00"N; 14 May 1969;
\textit{Nagy and Blais 170} (MONT).
Flora of Waterton Lakes National Park; gravelly river bar, 1 mi. E of Waterton
Lake; transition zone. 4,100' elevation; 10 Jul 1953;
\textit{Breitung 15714} (NY).
Open south-facing slope, trail to Crypt Lake; Canadian Zone. 5,000' elevation;
31 Jul 1954; \textit{Breitung 17009} (F).
Pipestone Creek. 7 Jul 1904; \textit{Macoun 64432} (NY).
  \textbf{Saskatchewan:}
Saskatchewan, plants of Paliser's Brit. N. Amer. Expl. Exped. 1858;
\textit{Bourgeau s.n.} (NY).
Junction of North Fork of North Branch Saskatchewan. 15 Jun 1908;
\textit{Brown 917} (MO).
Kootany Plains North Branch Saskatchewan River. 17 Jun 1908.
\textit{Brown 970} (NY).
%%% MONTANA STATE SPECIMENS
  \textbf{Montana:}
  \textbf{Unknown County:}
Beer Creek. 12 Jun 1884; \textit{Anderson s.n.} (UC-10767).
Midway Station, M. Y. Line. 20 Jun 1899; \textit{Nelson and Nelson 5457} (RM).
W. Fork of Sun River, Mont. 5,400' elevation; 2 Sep 1923;
\textit{Fickwood 2358} (RM).
Montana. 12 Mar 1909; \textit{Rydberg s.n.} (NY).
Montana. \textit{Williams 10460} (NY).
% Glacier
  \textbf{Glacier County:}
Many Glaciers Road, Glacier National Park. 27 Jun 1964; 27 Jun 1964;
\textit{Harvey 7067} (MONTU).
100 yd. west St. Mary bridge. 4,530' elevation; T35N R14W S33 NE1/4;
17 Jun 1947; \textit{McMullen 2821} (RM), \textit{McMullen 2836} (RM).
East side of Divide Mountain in talus fields. Off BIA near cell tower.
7,193' elevation; 15 Aug 2014; 48º44.10'N, 113º23.64'W;
\textit{Ratcliff and Horsch 56} (ISTC).
Divide Mtn., Glacier National Park; collected on east slope of peak in talus.
2,200m elevation; 9 Aug 1964; \textit{Harvey and Pemble 7213} (MONTU).
Divide Mtn. 16 Jul 1897; \textit{Williams s.n.} (MONT-9298).
Above timber line St. Mary Lake area on Goat Mountain above Baring Creek on
steep rock west facing slides of talus about even with Sexton Glacier,
Glacier National Park, Montana. 3 Jul 1953; \textit{McMullen 1133} MONTU.
2 mi. below summit of Siyeh Pass - Baring Basin. 15 Jul 1934;
\textit{McLaughlin 3277} (F).
Two Medicine Ridge at U.S. 89. 1,820m elevation; 13 Jul 1954;
\textit{Harvey 5836} (MONT).
Mt. sides, Mt. Henry; Midvale, Montana. 9 Jul 1903;
\textit{Umbach 305} (RM-86904, RM-168738, MONT, F, NY).
Glacier National Park, Two Medicine trail to Cutbank Pass, just below pass, in
rocks. 7,500' elevation; 14 Aug 1939; \textit{H. Bailey and V. Bailey 470} (UC).
Rare in shale-derived soil on a steep, west-facing roadcut above Summit Creek
ca. 7 mi. SW of East Glacier. 5100' elevation; T31N R13W S15; 24 Jul 1986;
\textit{Lesica 3986} (MONTU).
Rising Wolf Ranch at U.S. 2. 1,220m elevation; 3 Jul 1950;
\textit{Harvey 4174} (MONT).
Gravel soil in alpine tundra on Mt. Baldy, 10 mi. S of East Glacier.
7,500' elevation; 27 Jun 1981; \textit{Shaw 3433} (MONTU).
% Flathead
  \textbf{Flathead County:}
Gravelly stream bank at the foot of Summit Mtn. just north of Maria's Pass.
22 Jun 1978; \textit{Lesica 511} (MONTU).
Common in barren shaly soil on the crest of a ridge on the north end of the
Blacktail Hills. 5,700' elevation; 31 May 1993; \textit{Lesica 5935} (MONTU).
River bar Shafer Ranger Station. 4,400' elevation; 17 Jul 1937;
\textit{Root s.n.} (MONTU-103050).
0.5 mi. N of Pagoda Mt. approx. 32 mi. SE of Spotted Bear Ranger Station.
8,047' elevation; T22N R13W S3; 10 Jul 1996; \textit{Wirt 89} (ISTC).
Common in stony limestone soil of an exposed ridge south of Sphinx Peak.
8,300' elevation; T22N R12W S28; 30 Jul 2004;
\textit{Lesica and Hanna 8910} (NY, MONTU).
% Pondera
  \textbf{Pondera County:}
Common in stony limestone-derived soil on a ridge crest 1/2 mi. S of Scarface
Mountain. 7600' elevation; T28N R11W S15; 15 Jul 2005;
\textit{Lesica 9438} (MONTU).
1.0 mi. S of Family Pk., 9.75 mi. E. Schafer Meadows Ranger Station - air strip
area, ca. 2.5 mi. of Beaver Lk. 7,000' elevation; T28N R11W S33; 11 Aug 1995;
\textit{Wirt 79} (ISTC).
% Teton
  \textbf{Teton County:}
East Front Mountains. 7,800' elevation; T26N R10W S14 N1/2; 28 Jul 1983;
\textit{Lackschewitz 10565} (MONTU).
Common on loose limestone-rubble slides on S- to S- slopes of the mountain.
7,800' elevation; T26N R10W S19 NE1/4; 27 Jun 1983;
\textit{Lackschewitz 10532} (MONTU).
Front Range, MT. Wright, W. of Chouteau; SSE facing dry wash. Ca. 6,600'
elevation; T26N R10W S36 NWSE1/4; 23 Jul 1982; \textit{Ramsden 1110} (MONTU).
East Front Mountains; Chouteau Mtn. 8,000' elevation; 7 Jul 1977;
\textit{Lackschewitz 4432} (MONTU).
Two and 1/2 mi. E below confluence of Teton River forks. 4,800' elevation;
T25N R8W S25S S36N; 8 Jul 1973; \textit{Lackschewitz 4494} (MONTU).
Along road Chouteau-N-Fk. Teton R. T25N R8W S25 SW1/4; 30 Jun 1988;
\textit{Lackschewitz 11452} (GH).
Roadside Teton Canyon Road, 27 mi. NW of Choteau. 4,150' elevation;
4 Jul 1964; \textit{Finley s.n.} (MONTU).
Approx. 1 mi. E of bridge across the Teton River County Road from Ear Mtn.
Ranger Station to Choteau (20 mi.). 4,800' elevation; 9 Jun 1972;
\textit{Lackschewitz 3577} (MONTU).
Quarter mi. W of Pine Butte Swamp. 4,800' elevation; T24N R8W S11; 26 May 1995;
\textit{Lesica 6580} (MONTU).
Rocky Mountain Front Range, above and to the west of Our Lake at low pass.
7,760' elevation; T24N R9W S12 NW1/4; 9 Aug 1989;
\textit{Schassberger andEvenden 345} (MONTU).
Bob Marshall Wilderness; Trail across Headquarters Pass. 7,650' elevation;
T24N R9W S19 E1/2; 29 Jul 1978; \textit{Lackschewitz 8476} (MONTU).
Common in shaly, barren soil on a steep west-facing slope at the head of
Rierdon Gulch. 7,500' elevation; T24N R8W S23; 20 Aug 1995;
\textit{Lesica 7104} (MONTU).
Foot of Rocky Mts., along Hwy US-287 0.2 mi. N of Sevenmile Hill; 6.8 mi. SSW
of Chouteau. 4,300' elevation; T23N R5W S27 SE1/4; 13 Jul 1975;
\textit{Stickney 2261} (RM, MONT-66338, MONT-66339).
North Fork Teton River, 15 mi. W Chouteau. 4,000' elevation; 2 May 1981;
\textit{Shaw 3402} (MONTU).
Chouteau, about 12 mi. S. 11 May 1956; \textit{Booth 56261} (RM).
% Hill
  \textbf{Hill County:}
Bear's Paw Mtns., steep south-facing roadcut at the head of Big Sandy Creek,
ca. 20 mi. S of Havre. 5,400' elevation; T28N R16E S20; 28 Jun 1988;
\textit{Lesica 4609} (GH, MONTU).
% Chouteau
  \textbf{Chouteau County:}
Eagle Creek Landing, floating the Missouri White Cliffs area. 2,500' elevation;
T25N R13E S21; 27 Jun 1990; \textit{Lackschewitz 11656} (MONT).
% Missoula
  \textbf{Missoula County:}
Missoula County. \textit{Kennedy s.n.} (MONT-9356).
Miller Creek Canyon. 1,090m elevation; 9 May 1950; \textit{Harvey 4103} (MONT).
Mouth of Lolo Canyon. 10 Apr 1938; \textit{Keilman 9} (MONT).
Growing on talous slope E of Missoula, Montana. 9 May 1960;
\textit{Ruff s.n.} (UTC-00112173).
% Ravalli
  \textbf{Ravalli County:}
Bitterroot vicinity - Bitterroot Valley, tributary of Three Mile Creek; 10 mi.
NE of Stevensville. 4,000' elevation; T10N R19W S24 SE1/4; 25 May 1965;
\textit{Stickney 1195} (RM).
E St. Joseph Peak. 9,000' elevation; 4 Jul 1970;
\textit{Lackschewitz 2126} (NY).
% Granite
  \textbf{Granite County:}
Along N-side of Hwy 38 (Skalkaho Rd) between E-Fk. Rock Cr. Rd and the Middle
Fk. Rd. 8 May 1977; \textit{Lackschewitz 7187} (RM, MONTU).
Shale banks ca. 32 mi. E of Missoula. 20 Jun 1944;
\textit{Hitchcock and Muhlick 9119}
(RM, NY, MONT, UC, CAS-303267, CAS-303275, IDS, UTC).
Roadcut, slide above old Hwy 10, Bearmouth Area. 3 Jun 1983;
\textit{Lackschewitz 10433} (MONTU).
On roadcut above the old Hwy 90 in the Bearmouth Area. 13 May 1977;
\textit{Lackschewitz 7176} (RM, MONTU).
Highway 10, 35 mi. E of Missoula. 3,220' elevation; 19 May 1938;
\textit{Rose 107} (MONTU, MONT).
On sandy bank near Hiway, ca. 40 mi. E. of Missoula. 5 May 1934;
\textit{Hitchcock 2293} (RM, MONT, RSA-POM, MO).
Ca. 5.9 mi. W of Drummond. 3,862' elevation; 14 Jun 2014; 46º42.7216'N,
113º14.7275'W; \textit{Ratcliff and O'Kane, Jr. 46} (ISTC).
Ca. 5.5 mi. W of Drummond between Rattlesnake Gulch and Mulkey Gulch. 1,250m
elevation; 46º42'40"N, 113º14'43"W; 10 Jun 1996; \textit{O'Kane Jr. 3794} (MO).
46 mi. E of Missoula. \textit{Gillett and Moulds 12437} (NY).
Common in stony, calcareous soil of exposed ridge 1/2 mi. W of the Powell Mine.
8,400' elevation; T7N R12W S22; 2 Aug 2000; \textit{Lesica 8142} (MONTU).
% Powell
  \textbf{Powell County:}
Flathead National Forest; on top of Shale Mt., 5 mi. NE of Big Prairie Ranger
Station. 22 Jul 1948; \textit{Hitchcock 18620} (RM, UC).
Flathead National Forest; Top of Gordon Mt., 6 mi. S of Big Prairie Ranger
Station. 8,300' elevation; 22 Jul 1948; \textit{Hitchcock 18861} (RM, UC).
% Lewis and Clark
  \textbf{Lewis and Clark County:}
Bob Marshall Wilderness, Mountain S of Sock Lake, Upper E-Slope.
8,100' elevation; T24N R11W S31 SE1/4; 26 Jul 1979;
\textit{Lackschewitz 9106} (RM).
Sawtooth Ranch. 4,200' elevation; T21N R7W S19 NE1/4; 24 May 1988;
\textit{Lackschewitz 11372} (GH).
Haystack Butte. 5,000' elevation; 27 May 1973;
\textit{Lackschewitz 4263} (MONTU).
Common in limestone talus on an E-facing slope of Crown Mtn. near the pass.
7,200' elevation; T19N R9W S28; 6 Jun 1986; \textit{Lackschewitz 4043} (MONTU).
Flathead Range; uncommon in a limestone fellfield on a gentle southeast-facing
slope of Flint Mtn. 8,900' elevation; T18N R10W S8; 25 Jul 2009;
\textit{Lesica and Hanna 10206} (MONTU).
Scapegoat Mountain. 7,600-8,400' elevation; 11 Aug 1975;
\textit{Craighead 62} (MONTU).
Scapegoat plateau. 8,200' elevation; 30 Jul 1972;
\textit{Craighead 122} (MONTU).
Lewis and Clark National Forest. 5,500' elevation; T18N R8W S9; 25 Jul 1928;
\textit{Harris 9} (RM).
Low ridges along Mission Road 1 mi. W of the Dearborn River. 4,000' elevation;
T17N R4W S27; 10 Jun 2012; \textit{Lesica 10808} (MONTU).
7.5 mi. NE of Wolf Creek. 3,800' elevation; 27 Jun 1963;
\textit{Mulligan and Mosquin 2820} (CAS).
Prickly Pear Canon. 5 Aug 1887; \textit{Williams 515} (MONT).
York bridge on Missouri R. NE of Helena. T11N R2W S13 NW1/4; 6 May 1980;
\textit{Ramsden 524} (MONTU).
About 2 mi. NE of Lakeside at mouth of canyon. 3,700' elevation; 4 Jul 1967;
\textit{McKinney s.n.} (MONT-63192).
Helena, Montana. \textit{Brandegee 368} (UC-117221, UC-117506).
% Cascade
  \textbf{Cascade County:}
Valley of the Sun River in the Rocky Mts. 17 May 1884;
\textit{Doty 22} (MO-3930875).
Gravelly hills of Sun River near the Rocky Mts. 17 May 1884;
\textit{Doty 27} (MO-3833626).
N. Fork of Sun River. 5,400' elevation; 2 Sep 1925;
\textit{Kirkwood 2358} (RM, UC).
Belt-River. 2 Jun 1888; \textit{Williams s.n.} (MONT-9297).
Belt River (Milk Ranch) Montana. 12 Jun 1884; \textit{Anderson 41} (MONT).
% Judith Basin
  \textbf{Judith Basin County:}
Arrow creek above mouth of Coffee Creek. 3 Jul 1901; \textit{Spragg 185} (MONT).
% Fergus
  \textbf{Fergus County:}
Judith Mtns.; common in limestone scree on a steep west-facing slope above
Maiden Canyon ca. 4 mi. NE of Giltedge. 4,600' elevation; T16N R20E S9;
20 Jun 1987; \textit{Lesica 4330} (GH, MONTU).
Big Snowy Mtns.; Common in a limestone fellfield on top of Greathouse Peak.
8,600' elevation; T12N R19E S29; 19 Jul 2007;
\textit{Lesica and Hanna 9833} (ISTC, MONTU).
East-facing slope of Greathouse Pk. 8,200' elevation; T12N R19E S29;
24 Jul 1981; \textit{Lesica 1658} (MONTU).
Top of Grayhouse [sic] Peak, Big Snowy Mts. 3 Jul 1947;
\textit{Hitchcock 16052} (NY, UC, RSA-POM, UTC).
On shale bank in Half Moon Canyon 5 mi. from mouth, Big Snowy Mts. 5 Jul 1945;
\textit{Hitchcock and Muhlick 12012} (NY, CAS, UTC).
Shale bank at western base of Little Snowy Mts. 4 Jul 1945;
\textit{Hitchcock and Muhlick 11931} (RM, NY, MONT, MO, CAS, UTC, UC).
% Golden Valley
  \textbf{Golden Valley County:}
Ryegate, 8 mi. NE.  17 Jun 1957; Booth 57127 (MONT).
% Wheatland
  \textbf{Wheatland County:}
Harlowton, west of town in bunchgrass high cinder content soil. 22 May 1955;
\textit{Booth 5514} (MONT).
Harlowton, 10-12 mi. S. 17 Jun 1957; \textit{Booth 5750} (MONT).
% Meagher
  \textbf{Meagher County:}
King's Hill. 4 Jul 1948; \textit{Rose 4015} (MONTU).
Little Belt Mts., near the Pass. 7,000’ elevation; 10 Aug 1896;
\textit{Flodman 496} (NY, MO), 596 (NY).
11.4 mi. N of White Sulphur Springs. 4 Jul 1963;
\textit{Mosquin and Gillett 5222} (RM, UC, UTC).
Red shale outcrop near Four Mile Ranger Station, northeastern base of Castle
Mtns. 8 Jul 1945; \textit{Hitchcock and Muhlick 12075}
(RM, NY, MO, UTC, UC, MONT, RSA-POM). 
South-facing slope above Richardson Creek. 6,200’ elevation;
46º32.146'N, 110º42.330'W; 17 Jul 2012; \textit{Lesica 10820} (MONTU).
On moist ridge 1 mi. N of Baldy Mt., Big Belt Mts. 8,500’ elevation;
16 Jul 1945; \textit{Hitchcock and Muhlick 12377} (CAS).
Belt Mountains. 8 Jul 1886; \textit{Anderson 411} (NY).
Belt Mountains. 8 Jul 1886; \textit{Anderson 42} (F, MONTU).
Belt Mts. Montana.  12 Jul 1886; \textit{Anderson s.n.} (MONT-9302).
Red sandstone outcrop ca. 6 mi. W of Lennep. 2 Jul 1951;
\textit{Hitchcock 15983}
(RM, NY, MONT, RSA-POM, UC, UTC, IDS-030425, IDS-030426).
Cottonwood Creek. 5,000’ elevation; 30 Jul 1896; \textit{Flodman 495} (NY).
% Broadwater
  \textbf{Broadwater County:}
Big Belt Mountains, East Fork Cabin Gulch, ca. 1 mi. N of U.S. Hwy. 12, ca. 15
mi. E of Townsend. 5000' elevation; T7N R4E S22 N1/2; 12 Jun 1986;
\textit{Shelly 1100} (MONTU).
% Silver Bow
  \textbf{Silver Bow County:}
Near Butte Montana. 1893; \textit{Moore s.n.} (MO-3833622).
% Beaverhead
  \textbf{Beaverhead County:}
Tendoy Mountains, Bell Canyon; ca. 4 air mi. SW of Red Rock.  7,000’ elevation;
T11S R11W S24; 26 Jun 1993; \textit{Vanderhorst 4983} (MONT).
Tendoy Mtns., locally common in barren gravelly soil on a steep, west-facing
slope N of Poison Lakes. 8,000’ elevation; T11S R11W S27; 5 Jul 2002;
\textit{Lesica 8475} (NY, MONTU).
Dewey, Mont. 7,000’ elevation; 24 Jun 1902; Blankinship s.n. (MONT-9352).
Tendoy Mtns., common in stony limestone soils on a steep southwest-facing slope
on the ridge W of Pileup canyon. 7,600’ elevation; T14S R10W S32; 10 Jul 1993;
\textit{Lesica 6073} (NY, MONTU).
Tendoy Mtns.: Uncommon in limestone talus at the base of a steep south-facing
slope above Big Sheep Creek ca. 12 mi. SW of Lima. 6,800' elevation;
T15S R10W S4 SE1/4; 13 Jul 1985; \textit{Lesica 3522} (MONTU).
Mt. Lima, Mont. 30 Jun 1895; \textit{Shear 3406} (NY).
Lima Peaks: Locally common in gravelly, shale-derived soil on Ridge 9204 ca.
1.5 mi. SW of Garfield Mtn., ca. 8 mi. S of Lima. 9,200' elevation;
T15S R8W S21 SW1/4; 8 Jul 1988; \textit{Lesica 4640} (MONTU).
Beaverhead Mts., common in shallow calcareous soil on a steep, W-facing slope
of the spur ridge of the unnamed peak 2 mi. E of Red Conglomerate Peak.
8,400’ elevation; T15S R8W S36; 8 Jul 1986; \textit{Lesica 3938} (GH).
Abundant in sandy, eroding soil on an east-facing slope above the East Fork
Peet Creek. 6,850’ elevation; T14S R4W S27; 7 Jul 2005;
\textit{Lesica 9393} (NY, MONTU).
Centennial Valley-Sheep Mountain, approx. 1.5 air mi. SE of Lakeview.
2,700m elevation; T14S R1W S28 E2; 20 Jun 1993;
\textit{Culver and Lavin 254} (MONT-72332, MONT-72333).
% Madison
  \textbf{Madison County:}
Mountains near Indian Creek, Montana. 8,000’ elevation; 22 Jul 1897;
\textit{Rydberg and Bessey 4166} (NY, F).
Ruby Peak, Ruby Mountains, East of Dillon, Montana. 9,300’ elevation;
T6S R5W S16; 22 Aug 1982; \textit{Rosentreter 2836} (MONTU).
Gravelly Range; common in stony limestone-derived soil of the exposed summit of
Baldy Mtn. 9,500’ elevation; T7S R3W S27; 17 Jun 2006;
\textit{Lesica and Kittelson 9636} (MONTU).
Above the Ruby River 1 mi. N of the dam. 5,750' elevation; T7S R4W S4;
19 Jun 2003; \textit{Lesica 8648} (MONTU).
Madison Range; W. Slope, 8 mi. S of Jeffers. 17 Jun 1930;
\textit{Young 9350} (MONT).
Cedar Mountain, Montana. 10,000’ elevation; 16 Jul 1897;
\textit{Rydberg and Bessey 4168} (NY).
Northern Rocky Mt'N., Beaverhead National Forest; Lewis Creek reseeding area.
6,300’ elevation; T9S R3W S28; 20 Jun 1937;
\textit{Short and Aicher S-525} (RM, MONT).
Beaverhead National Forest; NRM Ranger Station,Schoolmarm Gulch.
6,050’ elevation; T9S R3W S17 SW4; 18 May 1952;
\textit{Schmautz JES-28} (RM, MONT).
Uncommon on a rocky, south-facing slope of Cave Mtn. ca. 25 mi. S of
Virginia City, plot 14. 9,600’ elevation; T10S R1W S32; 20 Jul 1989;
\textit{Lesica and Cooper 4914} (NY, MONTU).
Madison Range; common in gravelly soil on top of a limestone ridge in
Koch Basin, plot 59. 9,600’ elevation; T9S R2E S23; 2 Aug 1991;
\textit{Lesica and Cooper 5571} (NY, MONTU).
Madison Range; common in limestone-derived soil on an exposed ridge in
Koch Basin. 9,800’ elevation; T9S R2E S23; 28 Jul 1006;
\textit{Lesica and Cooper 9650} (MONTU).
On mountain 1/2 mi. N of Koch Peak; Taylor Mts. 2 Aug 1946;
\textit{Hitchcock and Muhlick 15225} (NY, UC, MO).
Madison Range, Gallatin National Forest; Koch Basin on the east side of
Koch Peak at head of Tumbledown Creek, ca. 25 mi. SE of Ennis.
9,800’ elevation; T9S R2E S14; 7 Aug 1995; \textit{Evert 30558} (RM).
% Gallatin
  \textbf{Gallatin County:}
Rocky hillside, near Maudlow. 1 Jun 1956; \textit{Denton s.n.} (MONT-52764).
Rocky Canon. 5,000’ elevation; 14 Jun 1905;
\textit{Blankinship 63} (F-190109, MO-3833621).
Manhattan, Mont.; 10 mi. N in Horseshoe Hills. 15 May 1949;
\textit{s.n. s.n.} (MONT-42418).
Sedan. 30 May 1901; \textit{Jones s.n.} (MONT-9306).
Sedan, 3 mi. SW exposed dry slope. 26 Jun 1949;
\textit{Metcalf s.n.} (MONT-77931).
Gallatin N. F., Bridger Mountains; East slopes and summit of Mount Sacagawea,
including the Bridger Divide (saddle between Mt. Sacagawea and Hardscrabble
Peak); collected along trail from Fairy Lake to Sacagawea summit.
8,000’ elevation; T2N R6E S22 and S27; 7 Aug 1990;
\textit{Bayer, Lebedyk, and Joncas MT-620} (RM).
Just N of the summit of Sacagawea Peak ca. 17 mi. N of Bozeman; Bridger Range.
9,000-9,500’ elevation; T2N R6E S27 NW4; 10 Aug 1989; \textit{Evert 18466} (RM).
Gallatin National Forest, Sacagawea Peak. 9,800’ elevation; 8 Aug 1935;
\textit{Whitham 1805} (RM, MONT).
Sacagawea Peak, Bridger Mts. 8,800’ elevation; 19 Jul 1969;
\textit{Dorn 913} (RM).
Collected from among rocks at ca. 9000 ft. from Sacagawea Pk. Bridger Range.
9,000' elevation; 16 Jul 1972; \textit{Schaack 675} (MONTU).
Valley north of Sacagawea Peak, Bridger Range. 8,100-9,200’ elevation;
31 Jul 1938; \textit{Pennell, Cotner, and Schaeffer 23832} (NY, US).
Mt. Bridger. 8,000’ elevation; 5 Jul 1905; \textit{Blankinship 64}
(RM-89447, F-190110, MONTU-3784).
Mt. Bridger. 9,000’ elevation; 3 Jul 1900;
\textit{Blankinship s.n.} (MONT-9309).
Mt. Bridger. 7,000-8,000’ elevation; 11 Jul 1903;
\textit{Blankinship s.n.} (MONT-9351).
Mt. Bridger. 8,500' elevation; 11 Aug 1903; \textit{Blankinship s.n.}
(MONTU-095452).
Bridger Peak. Jun 1889; \textit{Kock s.n.} (MONT-9301).
Bridger Peak. 8,900’ elevation; 25 Jun 1933;
\textit{Young s.n.} (RM-139984, MONT-22915, MONTU-28127, UTC-15083).
Summit of Mt. Bridger, Bozeman. 9,000’ elevation; 26 Jun 1899;
\textit{Blankinship s.n.} (MONT-9300).
Mt. Baldy, Bridger Mts. 7 Jul 1977; \textit{Forcella 69354} (MONT).
Bridger Mountains, Mont. 9,000’ elevation; 15 Jun 1897;
\textit{Young 4167} (RM, MONT, F).
Bridger Canyon. 5 Jun 1921; \textit{Hall s.n.} (MONT-26081).
Bridger Pass. 22 May 1898; \textit{Wilcox s.n.} (MONT-9303).
Bridger Pass. 22 May 1897; \textit{Blankinship s.n.} (MONT-9305).
Bridger foot-hills. 24 May 1921; \textit{Kindsley 33} (MONT).
Shale bank at the south base of Bridger Mts., 8 mi. NE of Bozeman. 18 Jul 1945;
\textit{Hitchcock and Muhlick 12470} (RM, NY, UC, MONT, UTC).
Foothills 3 mi. NE of Bozeman. 30 Apr 1921; \textit{Savage 15} (MONT).
2.5 mi. NE Bozeman. 30 Apr 1921; Powers 14 (MONT).
0.1 mi. SW of Kelly Canyon Rd. on Bridger Canyon Rd. (Hwy. 86). 5,013'
elevation; 14 Jun 2014; 45º42.1746'N, 110º55.8793'W;
\textit{Ratcliff and O'Kane, Jr. 47} (ISTC).
East of Bozeman; near electric powerhouse. 23 May 1900;
\textit{Wilcox 159} (NY).
Near Bozeman. 8 Jun 1883; \textit{Scribner 8} (NY).
Bozeman. Jul-Aug 1906; \textit{Blankinship 63}
(RM-89448, UC-311010, RSA-POM-143083, UTC-00159819).
Bozeman. 10 May 1892; \textit{Blankinship s.n.} (MONT-9296).
Bozeman. 10 May 1904; \textit{Jackson s.n.} (MONT-9295).
Bozeman. 14 May 1905; \textit{Flaherty s.n.} (UC-165233).
Bozeman. 16 May 1904; \textit{Maynard s.n.} (MONT-9311).
Bozeman. 24 May 1901; \textit{Moore s.n.} (RM-73253, MO-3833623).
Bozeman. 24 May 1901; \textit{Jones s.n.} (RM-122279).
Bozeman. 10 Jun 1921; \textit{Bohars s.n.} (MONT-26011).
Bozeman. 20 Jun 1902; \textit{Jones s.n.} (UC-165173).
Outside Gallatin National Forest. 6,100’ elevation; T2S R7E S13 E2;
20 Jun 1928; \textit{Swim 714} (RM, MONT).
Ft. Ellis to the Yellowstone. Jul 1891; \textit{Porter s.n.} (NY).
Madison Range; common in limestone talus on a west-facing slope of Cone Peak.
9,600’ elevation; T10S R4E S26; 10 Jul 2007;
\textit{Lesica and Kittleson 9808} (MONTU, ISTC).
% Park
  \textbf{Park County:}
Suksdorf's Gulch, 9 mi. NW of Wilsall. 19 Jul 1921; \textit{Suksdorf 513} (NY).
Several large colonies near the crest of the ridge; mountain N of Sunlight Lake.
9,200' elevation; T4N R11E S8 NW1/4; 30 Jul 1980;
\textit{Lackschewitz 9374} (MONTU).
Ridge 1/8 - 1/4 mi. N of Cokedale Rd., 6 mi. W of Livingston; Gallatin Range
foothills. 5,100-5,300’ elevation; T2S R8E S24; 14 Jun 1996;
\textit{Evert 31157} (RM).
Livingston, Mont. 20 May 1901; \textit{Scheuber s.n.}
(NY-2526, UC-991364, MONT-9308).
Livingston. 1901; \textit{Scheuber 363} (NY).
1 mi. NW Livingston. 20 May 1951; \textit{Wright 25} (MONT).
About 3 mi. E of Livingston on a rocky slope, Cemetery road. 17 May 1959;
\textit{Hoversten s.n.} (MONT-56052).
Common in open soil and in alpine turf on the ridge just east of Elephanthead
Mtn. Limestone Parent. 9,000' elevation; T3S R11E S30; 25 Jul 1985;
\textit{Antibus and Lesica 3569} (MONTU).
Absaroka Mtns., common in limestone talus near the summit of Elephanthead Mtn.
9,400’ elevation; T3S R11E S30; 14 Aug 1998; \textit{Lesica 7719} (NY, MONTU).
Chico Hot Springs. 15 Jul 1921;
\textit{Suksdorf 443} (RM, NY, UTC, MO, RSA-POM).
Absaroka Forest. 5,000’ elevation; T6S R8E S27; 2 Jun 1924;
\textit{Moir 73} (RM).
Electric Peak. 10,000’ elevation; 26 Jul 1902;
\textit{Rev. Earnest and Smith 25} (F).
% Sweet Grass
  \textbf{Sweet Grass County:}
Beartooth Mountains; in foothills, ca. 1/4 mi. W of Forest Road 482
(Grouse Ridge / Sliderock Mountain Road); ca 5 air mi. S of I-90. 5,000-5,200’
elevation; T2S R15E S7 and S8; 4 Jun 1993; \textit{Evert 24792} (RM).
McLeod. 4 Jun 1923; \textit{Pope 145}.
Common in shallow, sparsely-vegetated metamorphic-derived soil of a ponderosa
pine woodland on a south-facing slope above Jim's Gulch. 5,200’ elevation;
T2S R15E S34; 12 Jun 2005; \textit{Lesica 9143} (NY).
Along Sliderock Mtn. - Grouse Ridge Rd. (F.S. \#482), Gallatin Natl. Forest;
Beartooth Mtns. 7,100’ elevation; T3S R14E S14; 20 Jul 1991;
\textit{Evert 22091} (RM).
Common in stony, calcareous soil on the rim of the saddle north of Picket Pin
Mtn. 9,400’ elevation; T4S R14E S19; 28 Jul 1998; \textit{Lesica 7672} (NY).
Beartooth Mtns., common in gravelly limestone fellfield on the mountain ca.
1 mi. N of Picket Pin Mtn. 9,500’ elevation; T4S R14E S20; 10 Aug 1987;
\textit{Lesica 4481} (GH).
Saddle between Picket Pin Mtn. and unnamed pinnacled limestone ridge,
Custer Natl. Forest, just N of Picket Pin Rd. ca. 12 mi. W of Nye; Beartooth
Mtns. 9,000’ elevation; T4S R14E S29; 21 Jul 1994; \textit{Evert 28321} (RM).

% Stillwater
\textbf{Stillwater County:}
Beartooth Mountains, ca. 1/4 mi. N of Castle Creek, in foothills ca. 7 mi. NW
of Nye. 6,000’ elevation; T4S R15E S30 NE4; 9 Jul 1993;
\textit{Evert 25821} (RM).
Along the road from Mouat Mill to Horseman’s Flat. 5,300’ elevation;
20 Jun 1976; \textit{Robertson 1119} (RM).
Nye, Montana 19 Jun 1937; \textit{Peterson 736} (MONTU).
Midnight Canyon. 23 Apr 1923; \textit{s.n. s.n.} (MONT-34167).
Ca. 1/4-1/2 mi. S of Cliff Swallow fishing access site ca. 10 mi. W of
Absarokee. 4,700’ elevation; T4S R17E S4; 12 Jun 1992;
\textit{Evert 22811} (RM).
Absorokee, Mont. 20 Jun 1924; \textit{Hawkins s.n.} (UC-372122).
Absorokee, Mont. 21 Jun 1914; \textit{Hawkins s.n.} (MONT-34168).
Absorokee, Mont. 26 Jun 1922; \textit{Hawkins s.n.} (MONT-34163).
Absorokee, Mont. 28 Jun 1922; \textit{Hawkins s.n.} (MONT-34166).
Jackstone Creek; Absorokee, Mont. 26 Jul 1923;
\textit{Hawkins s.n.} (MONT-34162).
Fishtail, Mont. 26 May 1922; \textit{Hawkins s.n.} (MONT-34164).
2 mi. W of Columbus; Hwy 10. 25 May 1925;
\textit{J.C. Wright and A. Wright s.n.} (MONT-43983).
4 mi. N of Columbus, Mont.; breaks of the Yellowstone River. 31 May 1948;
\textit{Payne s.n.} (MONT-38787).
%%% IDAHO STATE SPECIMENS
  \textbf{Idaho:}
% Custer
  \textbf{Custer County:}
Challis National Forest; Morgan Creek NW. from Anderson's ranch. 10 Jun 1916;
\textit{Huffman 942} (RM).
% Lemhi
  \textbf{Lemhi County:}
4.0 rd. mi. W of Williams Creek Rd. from Hwy. 93, Southwest of Salmon.
4,535' elevation; 45º4.9216'N, 113º57.7358"W; 14 Jun 2014;
\textit{Ratcliff and O'Kane, Jr. 45} (ISTC).
Dry hill below Salmon. 17 May 1935; \textit{Davis 40-35} (IDS, NY).
Salmon National Forest; between Mud Spring and Atchinson Spring.
6,800' elevation; 5 Jun 1930; \textit{Phillips 50} (RM).
10 mi. W of Salmon, Idaho on road to Leesburg. 13 Jul 1945;
\textit{Christ and Ward 14727} (NY).
7 mi. SW of Salmon.  4,400' elevation; 23 Jun 1965;
\textit{Mulligan and Crompton 2962} (NY, GH).
In talus along E slope of Salmon Mts. near mouth of N. Fk. Williams Creek.
1 Jul 1945; \textit{Hitchcock and Muhlick 14305} (RM, MO, CAS, UC).
Williams Creek, 10 mi. SE of Salmon, Idaho. 4,050' elevation; 10 Jul 1982;
\textit{Hawiger 4} (MONTU).
Approximately 4.5 mi. up Williams Creek road in talus slopes and road cuts on
right hand side of road. 4,900' elevation; 25 May 1988;
\textit{Rittenhouse 102} (IDS).
Williams Creek, 10 mi. SE of Salmon. 4,050' elevation; T20N R21E S10;
10 May 1982; \textit{Rosentreter 2683} (MO, NY, IDS-77676).
Williams Creek, 10 mi. SE of Salmon. 4,050' elevation; T20N R21E S10;
22 Aug 1982; \textit{Rosentreter 2881} (MONTU, NY).
Upper Basin Creek, SW of Tendoy, Idaho. 5,000' elevation; T19N R23E S31 and
NW1/4 of NE1/4 S32; 21 May 1982; \textit{Rosentreter 2696} (MONTU, NY, NY).
Pattee Creek, 6 km NE of Tendoy. 5,200' elevation; T19N R24E S11 S1/2;
7 Jul 1986; \textit{Rosentreter 3855} (NY).
Agency Creek. 5,000' elevation; T19N R25E S18  SE1/4;
20 May 1982; \textit{Rosentreter 2689} (NY, MONTU, NY).
Agency Creek. 5,000' elevation; T19N R25E S18 SE1/4; 17 Jun 1982;
\textit{Rosentreter 2727} (MONTU).
West Fork Little Eightmile Creek talus, 1/2 mi. Maryland Mine, 10 mi. N of
Leadore in Lunistene talus. 7,200' elevation; T17N R25E S23 N1/2 NE1/4;
27 Jun 1990; \textit{Atwood and Moseley 13891} (UC).
Beaverhead Mountains, talus slope on the N side of Mahogany Canyon.
7,300' elevation; 1 Jul 1976; \textit{Henderson 3196} (NY).
Targhee National Forest, Beaverhead Range; above South Fork of Worthing Canyon,
ca. 10.5 air mi. SE of Nicholia. 8,000-8,600' elevation; T11N R30E S34;
18 Jun 1992; \textit{Markow 7784} (RM).
North side of Meadow Canyon, ca. 1 mi. W of junction. 16 Jul 1975;
\textit{Henderson 2641} (NY).
Lemhi Range, Peak 10,652 just N of Meadow Canyon. 24 Jul 1975;
\textit{Henderson 2867} (NY).
% Butte
  \textbf{Butte County:}
Cirque below N face of Diamond Peak, Lemhi Range, Targhee NF; T10N R28E
unserveyed sections, ca. 9 mi. W of Blue Dome. 9,400' elevation; 9 Jul 1989;
\textit{Moseley 1508} (NY).
Lemhi Range, Targhee National Forest, 0.5 mi. NE of Pk 10020, on divide between
Rocky Can and Sawmill Can, ca. 12 mi. NW of Blue Dome. 9,600' elevation;
T10N R28E S11 NE1/4; 22 Jul 1984; \textit{Moseley 420} (RM, NY).
% Clark
  \textbf{Clark County:}
Targhee National Forest, along and near top of ridge. 8,100' elevation;
T13N R34E S9; 10 Jul 1926; \textit{Pickett 303(P-56)} (RM).
Targhee National Forest, Beaverhead Range; slope W of Gallagher Peak, ca. 30
air mi. W of Dubois. 9,500-9,700' elevation; T10N R31E S27; 4 Aug 1992;
\textit{Markow 10270} (RM).

