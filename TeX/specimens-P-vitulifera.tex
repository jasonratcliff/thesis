\section*{\textit{Physaria vitulifera}}

%%% COLORADO STATE SPECIMENS
  \textbf{Colorado:}
  \textbf{Unknown Locality:}
Powell's Colorado Exploring Expedition. \textit{Vasey 46} (NY).
  \textbf{Weld County:}
Platte River; Evans, Colorado. 13 Jun 1910; \textit{Johnston 633} (NY),
\textit{Johnston 633b} (RM, NY).
  \textbf{Boulder County:}
South St. Vrain Canyon, 5 mi. W of Lyons. 6,000' elevation; 12 Jun 1922;
\textit{Wiegand and Upton 3321} (F).
US Hwy. 36 between Boulder and Lyons. 16 May 1977; \textit{Harmon 8792} (NY).
10 mi. S of Lyons, Boulder-Lyons road. 5,300' elevation; 3 Jul 1951;
\textit{Rollins 5145} (US).
1 mi. N of Boulder on Hwy. 36. 1,650m elevation; 11 Jul 1978;
\textit{Wagner 3713} (MO).
Near Boulder, Colo. \textit{Ramaley 27} (RM).
Gregory Canyon; 1 mi. SW of Boulder. 5,520' elevation; 30 May 1948;
\textit{Brewbaker s.n.} (GH).
Gregory Canyon foothills. 9 Apr 1939; \textit{Fullerton s.n.} (RM).
Boulder Mt. Parks, along Flagstaff Trail. 6,000' elevation; 5 Jun 1991;
\textit{Hogan 1060} (RM).
Boulder Mt. Parks, Gregory Canyon; foothill canyon with mosaic of open slopes.
6,000' elevation; 23 Apr 1992; \textit{Hogan 1567} (RM).
2.5 mi WSW of Boulder. 5,800' elevation; 18 Aug 1964;
\textit{Mulligan and Crompton 2844} (NY).
Mouth of Gregory Canyon, near Boulder. 5,500' elevation; 5 Jul 1951;
\textit{Rollins 5146} (RM, NY, GH, US, UTC).
Plains and Foothills near Boulder. 5,000-6,000' elevation;
\textit{Tweedy 5068} (RM, NY).
East face of Flagstaff Mt., outer foothills of Front Range just west of Boulder.
6,000-7,000' elevation; 8 May 1959; \textit{Graham 18} (NY).
Dry hills near mouth of Gregory Canyon; Boulder, CO. 2 Jul 1917;
\textit{Payson 1031} (RM).
Boulder Mt. Park., west side of Bear and S. Boulder Pks. 7,200' elevation;
19 May 1992; \textit{Hogan 1644} (RM).
Dry, rocky hillside in upper Bear Creek Canyon, 4.5 mi. SW of Boulder.
7,000' elevation; 13 Jul 1941; \textit{Robbins 788} (CAS).
Along Gross Dam Road. 7,521' elevation; Lat 39º55.73'N, Long 105º20.63'W;
14 Jun 2001; \textit{King and Garvey 11717} (NY, F, CAS, MO).
  \textbf{Gilpin County:}
Ca. 4 mi. S of Black Hawk. 5 Jun 1972;
\textit{Higgins 1494} (RM, NY, GH, UTC, MO, US).
1.5 mi. W of jct. of Hwy. 6 and Hwy. 119. 2,377m elevation; 24 May 1996;
\textit{O'Kane, Jr. 3754} (MO).
Eldora to Baltimore. 8,500-9,500' elevation; Jul 1903;
\textit{Tweedy 5585} (RM, NY).
Tunnel Road, 8 mi. SE of Central City. 2 Jun 1947; \textit{Hollister 536} (US).
  \textbf{Clear Creek County:}
From the head-waters of Clear Creek, and the alpine ridges lying east of
"Middle Park," Colorado Territory. \textit{Parry 101} (NY, F).
Mountains about the head waters of Clear Creek; Mountain sides near Empire.
8,600' elevation; \textit{Patterson s.n.} (F-107068).
Mountains about the head waters of Clear Creek; dry places in Clear Creek
Canyon, Empire. 8,500' elevation; \textit{Patterson s.n.} (F-107705).
6 mi. along Hwy. 279 south from Central City. 8,100' elevation; 18 Aug 1964;
\textit{Mulligan and Crompton 2849} (NY).
Idaho Springs. 7,650' elevation; 15 Jul 1950;
\textit{Ripley and Barnbey 10475} (NY, CAS).
Idaho Springs. 29 Aug 1895; \textit{Shear 3272} (NY).
Idaho Springs. 5,200' elevation; 15 May 1916;
\textit{Clokey 2753} (NY), \textit{Clokey 2675} (F).
Golden, at junction of the N and S Clear Creeks, around tunnel entrance and
above creek. 7,119' elevation; Zone 13S, WGS84 Easting 0465854,
Northing 4399748; 1 Oct 2011; \textit{Smith 482} (ISTC).
Clear Creek Canyon. Jun-Jul 1873; \textit{Wolfe 648} (NY).
  \textbf{Jefferson County:}
Northern outskirts of Golden. 5,925' elevation; 22 Aug 1964;
\textit{Mulligan and Crompton 2893} (GH).
1 mi. SE of Golden. 6,350' elevation; 23 Aug 1964;
\textit{Mulligan and Crompton 2894} (NY).
Golden. 1,730m elevation; 24 Jun 1920;
\textit{Duthie and Clokey 3777} (RM, NY, RSA-POM).
Golden. 24 Jun 1920; \textit{Osterhout 6028} (RM, RSA-POM).
Golden. 6,000' elevation; 30 Apr 1892; \textit{Crandall 45} (NY).
Hills about Golden. 6,000' elevation; 30 Apr 1892; \textit{Crandall 53a} (NY).
Vicinity of Golden. 6,000' elevation; Jun 1934;
\textit{Brother Cletus 179} (RM).
Lookout Mountain, just west of Golden. 7,300' elevation; 23 May 1979;
\textit{Rollins 7947} (US, GH, NY).
Golden, Colo. 4 mi. NW Road to Black Hawk. 6 Jul 1917;
\textit{Johnston and Hedgcock 959} (RM-168747, RM-101788).
Genesee Park. 3 May 1939; \textit{Zobel s.n.} (NY).
1.5 mi. N of Red Rock theatre, about 4 mi. W of Denver. 6,028' elevation;
18 Jun 1963; \textit{Mulligan and Mosquin 2761} (RM).
Bear Creek Canyon. \textit{Greene s.n.} (NY).
West flank of hogback at Morrison. Lat 39º39'03"N, Long 105º11'23"W;
14 May 1996; \textit{O'Kane, Jr. and Grimes 3701} (ISTC).
Pine wood, Shaffers Crossing. 7,000' elevation; 8 Jul 1937;
\textit{Beetle 2056} (RM).
0.25 mi. E of Pine; S-facing slope above North Fork S Platte River.
6,900' elevation; T7S R71W S26; 8 Jul 1991; \textit{Wood s.n.} (RM).
Dome Rock. 6,250' elevation; \textit{Schedin and Schedin 261} (RM).
Deckers Road, Highway CR 126; along the Hwy. road cuts between Pine and Deckers.
6,679' elevation; Zone 13S, NAD 83 Easting 0479200, Northing 434179;
20 Nov 2010; \textit{Smith s.n.} (ISTC).
Hwy. 126, 6.2 mi. N of Deckers, along Sixmile Creek. 7,500' elevation;
26 Jun 1985; \textit{Wilken 14502} (NY).
Along County Road 126 north of Deckers. 6,985' elevation; Zone 13S,
WGS 84 Easting 0477087, Northing 4345362; 25 Jun 2011; \textit{Smith 499} (RM).
2 mi. W of Deckers. 6,500' elevation; 6 Jul 1951;
\textit{Rollins and Livingston 5151} (RM).
Denver, Colorado. \textit{Newberry and Hulse s.n.} (RM).
Along roadside, Colorado State Hwy. 68, west of Denver. 6,000' elevation;
28 Apr 1972; \textit{Andrews 5368} (RSA-POM).
South Platte Burn, Pike National Forest. \textit{Mann 18} (RM).
  \textbf{Douglas County:}
Wolhurst. 1,616m elevation; 15 May 1920;
\textit{Duthie and Clokey 3775} (RM, GH, RSA-POM, UTC, CAS).
13.8 mi. S of Sedalia on Hwy. 67. 2,166m elevation;
Lat 39º20'36"N, Long 105º07'15"W; 19 Jul 1996; \textit{Salywon 3198} (MO).
Roadcut, 21.2 mi. S of Sedalia on Hwy. 67. 2,100m elevation;
39º20'36"N, 105º07'15"W; 29 Jul 1996; \textit{Salywon 3201} (MO).
15 mi. SW of Sedalia, between Sedalia and Deckers. 7,000' elevation;
6 Jul 1951; \textit{Rollins and Livingston 5149} (RM, NY, UTC).
1 mi. above South Fork of South Platte River, 6 mi. NE of Deckers.
6,500' elevation; 6 Jul 1951; \textit{Rollins and Livingston 5150} (RM, NY).
Slope above Sugar Creek. 6,500' elevation; Lat 39º18.4'N, Long 105º11.7'W;
28 Jun 2004; \textit{Dorn 9837} (RM, NY).
Canyon 1 mi. NE of Deckers just above Platte River Valey. 9 Jul 1949;
\textit{Weber 4930} (UTC, CAS).
  \textbf{El Paso County:}
Ruxton Ridge. 3,050m elevation; 9 Jul 1905;
\textit{Clements and Clements 97} (RM-51208, RM-34243, NY).
  \textbf{Park County:}
Lava slopes of East Buffalo Peaks along County Road 435. 9,351' elevation;
Zone 13S, WGS 84 Easting 0412147, Northing 4312454; 15 Jun 2011;
\textit{Smith 500} (RM).
South Park, Colorado 12,000' elevation; Jul 1872 \textit{Wolf 642} (F).
  \textbf{Fremont County:}
Spruce Basin Road west of County Road 12, 3 mi. N of Cotopaxi and Arkansas
River. 8,478' elevation; Zone 13S, WGS84 Easting 0436288, Northing 4254134;
25 Oct 2011; \textit{Smith 477} (RM).
Along Hwy. 50 on slopes above Arkansas River below Salida. 1,951m elevation;
38º28'32"N, 105º52'48"W; 21 May 1996; \textit{O'Kane, Jr. 3752} (ISTC, MO).
Cañon City. \textit{Brandegee 346} (RM).  
  \textbf{Costilla County:}
8.8 mi. E of Fort Garland and west of La Veta Pass. 2,522m elevation;
16 Jun 2000; \textit{O'Kane, Jr. 4925} (ISTC, MO).
Collected at mile marker No. 269, E of Fort Garland on U.S. 160.
7,700' elevation; 30 May 1982; \textit{Ricketson and Walten 437-25A} (MO).

