\section*{\textit{Physaria condensata}}

%%% WYOMING STATE SPECIMENS
  \textbf{Wyoming:}
  \textbf{Sublette County:}
7,200’ elevation; T28N R110W S6; 2 Jul 1983; \textit{Dorn 3839} (NY).
  \textbf{Lincoln County:}
Shaley hill, 14 mi. W of Kemmerer. 7,000' elevation; 27 Jun 1957;
\textit{Rollins 57254} (NY, GH).
West end of Fossil Ridge, ca. 16 air mi. W of Kemmerer; 7.3 mi. SW on Northwest
Pipeline access road. 7,400-7,760' elevation; T21N R118W S29 N1/2; 10 Jul 1995;
\textit{Refsdal and Nelson 5201} (RM).
Southwest end of Bear River Divide on east-west running ridge on NE side of
Bridger Basin, ca. 2 air mi. NE of Bridger Hill, ca. 10 air mi. S of Orr.
7,500' elevation; T20N R119W S30 SE1/4; 4 Jul 1996; \textit{Fertig 16765} (RM).
Bear River Divide overlooking Bridger Basin, ca. 1 mi. NE of Bridger Hill
junction and 0.25 mi. NNW. 7,740' elevation; T20N R119W S32 NW1/4 of NW1/4;
28 Jun 2003; \textit{Heidel 2318} (RM, NY).
West of Pittsburg and Midway Coal Mine, 7 air mi. SW of Kemmerer, exposed
ridgetop, Bear River Divide, Green River Formation. 7,700' elevation;
10 Jun 1984; \textit{Neely and Carpenter 2135} (UTC).
Ca. 7 air mi. WSW of Kemmerer; 4.6 mi. S on County Road 328, then 0.7 mi. W on
two track. 6,960-7,100' elevation; T20N R118W S26; 1 Sep 1995;
\textit{Refsdal 7732} (RM).
North Fork of Twin Creek, 3 air mi. E of Fossil Butte Nat. Mon.
7,300-7,400' elevation; T22N R117W S27; 18 Jun 1983;
\textit{Neely and Hall 1282} (UTC).
South end of Hams Fork Plateau, on ridge ca. 2 air mi. E of Fossil Butte,
on north side of Union Pacific Railroad, ca. 1.5 mi. N of U.S. Hwy. 30, ca. 7
air mi. WNW of Kemmerer. 7,200' elevation; T21N R117W S3 NE1/4 of NE1/4 and S2
NW1/4 of NW1/4; 3 Jul 1996; \textit{Fertig 16754} (RM).
Northwest end of the butte east of Hay Hollow, 1.6 rd. mi. from U.S. Hwy. 30 on
rd. on north side of Union Pacific Railroad, 3.9 rd. mi. E of Clear Creek Rd. on
Hwy. 30. 7,241' elevation; 12 Jun 2014;
\textit{Ratcliff and O'Kane, Jr. 41} (ISTC).
6 mi. W of Kemmerer on Hwy. 30, above railroad tracks. 2,286m elevation;
Lat 41º49'04"N, Long 110º38'40"W; 4 Jun 1996; \textit{O'Kane, Jr. 3788} (MO).
Butte on east side of Hay Hollow and on north side of Union Pacific Railroad
and U.S. Hwy 30, ca. 4.5 mi. E of Fossil Butte and 5-6 air mi. NW of Kemmerer.
7,320' elevation; T21N R117W S12 N1/2 of NE1/4; 1 Jul 1999;
\textit{Walford 3009} (RM).
Butte on east side of Hay Hollow and on north side of Union Pacific Railroad
and U.S. Hwy. 30, ca. 4.5 mi. E of Fossil Butte and ca. 5 air mi. NW of
Kemmerer. 7,320' elevation; T21N R116W S7 N1/2 of NW1/4 of NW1/4;
T21N R117W S12 N1/2 of NE1/4; 7 Jun 1997; \textit{Fertig 17475} (RM).
Benchtop between North Fork of Twin Creek and Hay Hollow, east of Fossil Butte
Nat. Mon. T22N R117W; 26 Jun 1983; \textit{Neely and Carpenter 1331} (UTC).
Ulrich Quarry, calcareous ridge. 6,700' elevation; T21N R117W S22; 19 Jun 1982;
\textit{Lichvar 5046} (GH).
10 mi. N of Kemmerer in southwest Wyoming. 7,000' elevation; 8 Jul 1982;
\textit{Williams 82-288-4} (NY).
10 mi. N of Kemmerer. 24 Jun 1979;
\textit{Rollins and Rollins 79304} (NY, US, F).
Green River Basin, ca. 13.5 air mi. NE of Kemmerer. 7,000-7,100' elevation;
T23N R115W S24 S1/4; 4 Jul 1995; \textit{Cramer 7118} (RM).
Badlands area ca. 1 air mi. N of Round Mountain; ca. 11 air mi. NE of Kemmerer.
7,000-7,480' elevation; T22N R114W S7; 28 May 1994; \textit{Hartman 45697} (RM).
0.4 mi. N of Hwy. 189 on unnamed road heading towards South Fork Slate Creek,
then 0.2 mi. on two-track road; hillside overlooking South Fork Slate Creek.
7,299' elevation; 12 Jun 2014; \textit{Ratcliff and O'Kane, Jr. 40} (ISTC).
E of Kemmerer. 6,800' elevation; T22N R115W S12 SE1/4; 15 Jun 1982;
\textit{Lichvar 5034} (NY-182071, NY-182074);
T22N R115W S14 W1/2; 15 Jun 1982; \textit{Lichvar 5032} (NY).
Ridge on south side of South Fork of Slate Creek, ca. 0.3 mi. N of U.S. Hwy.
189 and ca. 0.6 mi. NW of Round Mountain. 7,240-7,320' elevation;
T22N R115W S12 SW1/4; 7 Jun 1997; \textit{Fertig 17472} (RM).
10 mi. NE of Kemmerer along Rt. 189; "One of the largest specimens seen of this
species." 7,000' elevation; 8 Jul 1982; \textit{Lichvar 5220} (GH).
On a partially disturbed roadside west of U.S. Hwy. 189, 13.6 mi. SW of Wyoming
Hwy. 372 and 5 mi. NE of Fontenelle Creek Road, 10.3 mi. N of the junction of
U.S. Highway 30A in Kemmerer. 2,254m elevation;
Lat 41º52'25"N, Long 110º26'54"W; 30 Jun 2008; T22N R115W S23 NW1/4 NW1/4;
\textit{Reveal 8965} (NY, MO, ISTC).
Overthrust Belt; ca. 10.5 mi. E of Kemmerer, S of Hwy. 189.  Uplands above N.
Fk. Alkali Creek. 7,380' elevation; T22N R115W S23 NW1/4; 15 Jun 2004;
\textit{Heidel 2585} (RM).
Ca. 7 air mi. NE of Kemmerer; ca. 0.8 rd. mi. S of U.S. Hwy. 189.
7,350-7,380' elevation; T22N R115W S23 W1/2; 17 Jun 1994;
\textit{Refsdal 897} (RM).
NW of Kemmerer. 7,350' elevation; T22N R115W S23 NW1/4; 30 May 1982;
\textit{Lichvar 4785} (NY).
Kemmerer. 7,350-7,380' elevation; T22N R115W S23 NW1/4; 12 May 1993;
\textit{Refsdal 5} (RM).
E of Kemmerer. 6,900' elevation; T22N R115W S26 NW1/4; 30 May 1982;
\textit{Lichvar 4790} (NY).
E of Kemmerer. 6,900' elevation; T21N R115W S13 NW1/4; 29 May 1982;
\textit{Lichvar 4794} (GH).
Between Opal and Kemmerer. 19 Jun 1923; \textit{Payson and Armstrong 3219} (MO).
Barren whitish shale with Astragalus jejunus, Cryptanthd; matted.
6,840' elevation; T21N R115W S25; 3 Jul 1982; \textit{Dorn 3729} (GH).
U.S. Hwy. 30 N, about 9.5 mi. E of Kemmerer-Diamondville. 6,850' elevation;
T21N R115W S25; 5 Jun 1971; \textit{Holmgren and Holmgren 5030} (NY).
Ca. 3.5 mi. SE of LaBarge. 6800' elevation; T26N R112W SE4NE4 S21 1 Jul 1993;
\textit{Kass 3797} (RM).
  \textbf{Uinta County:}
2 mi. W of Fort Bridger. 2,134m elevation; Lat/Lon 41:19'51"N 110:25'10"W;
3 Jun 1996; \textit{O'Kane, Jr 3787} (MO).
3 mi. W of Fort Bridger. 17 Jun 1946; \textit{Rollins 3074} (RSA-POM, CAS).
Foothills of Bridger Butte, 3 mi. W of Fort Bridger. 6,800' elevation;
20 Jul 1939; \textit{Rollins and Munoz 2868} (CAS, UTC).
Bridger Butte. 7,300' elevation; T15N R116W S15 SE4; 17 Apr 1980;
\textit{Lichvar 3939} (NY, GH).
S end of Bridger Butte. 7,000' elevation; T15N R116W S15 SE4; 4 Jun 1980;
\textit{Lichvar 2776} (NY).
Ridge on divide between Soda Hollow and Piedmont Creek, ca. 1 mi. W of Piedmont
and 5 air mi. S of Interstate 80. 7420' elevation; T14N R117W S6 SE4 of NE4NE4;
30 Jun 1999; \textit{Walford 3003} (RM).

