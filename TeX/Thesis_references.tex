\chapter*{REFERENCES}
\addcontentsline{toc}{chapter}{REFERENCES}
\vspace*{-12pt}

\setlength{\parindent}{-4em}
\setlength{\leftskip}{4em}
\setlength{\parskip}{12pt}
\singlespacing

\hspace*{-4em}Abellán P, Ribera I. 2017. Using phylogenies to trace the geographical signal of diversification. Journal of Biogeography. 44(10):2236-2246.

Ackerfield J. 2015. Flora of Colorado. 1st ed. Fort Worth (TX): Botanical Research Institute of Texas.

Aho AV, Kernighan BW, Weinberger PJ. 1988. The AWK programming language. 1st ed. Murray Hill (NJ): AT\&T Bell Laboratories.

Alexander PJ, Rajanikanth G, Bacon CD, Bailey CD. 2007. Recovery of plant DNA using a reciprocating saw and silica-based columns. Mol Ecol Notes. 7(1):5-9.

Al-Shehbaz IA, O’Kane SL. 2002. \textit{Lesquerella} is united with \textit{Physaria} (Brassicaceae). Novon. 12(3):319-329.

Bache SM, Wickham H. 2014. magrittr: A forward-pipe operator for R. R package version 1.5. https://CRAN.R-project.org/package=magrittr

Bash manual page. [Computer Program]. 2018. München (Germany): Michael Kerrisk; [Cited 2018 August 29].  Available from http://man7.org/linux/man-pages/man1/getopts.1p.html

Becker RA, Wilks AR. 1993. Maps in S. [Internet]. Murray Hill (NJ): AT\&T Bell Laboratories Statistics Research Report; [cited 2018 Apr 27]. Available from http://ect.bell-labs.com/sl/doc/93.2.ps

Becker RA, Wilks AR. 1995. Constructing a geographical database. [Internet]. Murray Hill (NJ): AT\&T Bell Laboratories Statistics Research Report; [cited 2018 Apr 27]. Available from http://ect.bell-labs.com/sl/doc/95.2.ps

Becker RA, Wilks AR, Brownrigg R, Minka TP, Deckmyn A. 2018. maps: Draw geographical maps. R package version 3.3.0. https://CRAN.R-project.org/package=maps

Beitz E. 2000. TEXshade: Shading and labeling of multiple sequence alignments using \LaTeX 2$\backepsilon$. Bioinformatics. 16(2):135-139.

Bloomquist, Lemey, Suchard. 2010. Three roads diverged? Routes to phylogeographic inference. Trends in Ecology and Evolution. 25(11):626-632.

Bodenhofer U, Bonatesta E, Horej\v{s}-Kainrath C, Hochreiter S. 2015. msa: An R package for multiple sequence alignment. Bioinformatics. 31(24):3997-3999.

Comai L. 2005. The advantages and disadvantages of being polyploid. Nature Reviews Genetics. 6(11):836-846.

Cornish-Bowden A. 1985. Nomenclature for incompletely specified bases in nucleic acid sequences: recommendations 1984.  Nucleic Acids Research. 13(9):3021-3030.

Darwin. 1859. On the origin of species.

Darwin Core. c2009-2015. San Francisco (CA): Taxonomic Databases Working Group; [accessed 2018 Jan 9]. http://rs.tdwg.org/dwc/index.htm.

Dayrat B. 2005. Towards integrative taxonomy. Biological Journal of the Linnean Society. 85(3):407-415.

de Queiroz K. 1998. The general lineage concept of species, species criteria, and the process of speciation: A conceptual unification and terminological recommendations. In: Howard DJ, Berlocher SH, editors. Endless forms: species and speciation. New York (NY): Oxford Univ. Press. p. 57–75.

de Queiroz K. 2007. Species Concepts and Species Delimitation. Syst. Biol. 56(6):879-886.

Dorn R. 2001. Vascular Plants of Wyoming. 3rd edition. Cheyenne (WY): Mountain West Publishing. Key to Brassicaceae, \textit{Physaria}; p. 143-144.

Drabkova L, Kirschner J, Vlcek C. 2002. Comparison of seven DNA extraction and amplification protocols in historical herbarium specimens of Juncaceae. Plant Mol Biol Reporter. 20(2):161-175.

Edwards SM. 2019. lemon: Freshing up your 'ggplot2' plots. R package version 0.4.3. https://CRAN.R-project.org/package=lemon

Fairbairns R, Rahtz S, Barroca L. 2016. rotating - rotation tools, including rotated full-page floats [Internet]. Heidelberg (Germany): Comprehensive TeX Archive Network; [cited 2018 April 23]. Available from https://ctan.org/pkg/rotating?lang=en.

Fertig W. 1998. Status report on Dorn’s Twinpod (Physaria dornii) in Southwestern Wyoming. Report prepared for the Bureau of Land Management Wyoming State Office by the Wyoming Natural Diversity Database.

Fertig W. 2002. Status of Tufted Twinpod (Physaria condensata) in Southwest Wyoming. Report prepares for the Bureau of Land Management Wyoming State Office and Wyoming Natural Diversity Database by a Botanical Consultant.

Fois M, Fenu G, Cañadas EM, Bacchetta G. 2017. Disentangling the influence of environmental and anthropogenic factors on the distribution of endemic vascular plants in Sardinia. PLoS ONE. 12(8):e0182539.

Gillespie RM, Roderick GK. 2014. Geology and climate drive diversification. Nature. 509(7500):297-298.

GNU bash manual. [Internet]. 2016. Boston (MA): Free Software Foundation; [Cited 2018 August 25]. Available from https://www.gnu.org/software/bash/manual/

Google Inc. 2019. Google Earth [Internet]. Mountain View (CA): Google Inc.; [cited 2019 Sep 29]. Available from https://www.google.com/earth/

Gotoh O. 1995. A weighting system and algorithm for aligning many phylogenetically related sequences. Compu Appl Biosci. 11(5):543-51.

Graham CH, Carnaval AC, Cadena CD, Zamudio KR, Roberts TE, Parra JL, McCain CM, Bowie RCK, Moritz C, Baines SB, Schneider CJ, VanDerWal J, Rahbek C, Kozak KH, Sanders NJ. 2014. The origin and maintenance of montane diversity: Integrating evolutionary and ecological processes. Ecography. 37(8):711-719.

Gray A. 1895. Synoptical flora of North America. Vol. 1, Pt. 1. University Press: Cambridge. Cruciferae, p. 121.

Griekspoor A, Groothuis T. 2015. 4Peaks version 1.7.2. Nucleobytes.

Grolemund G, Wickham H. 2011. Dates and times made easy with lubridate. Journal of Statistical Software. 40(3):1-25.

Handley J, Heidel B. 2011. Status of \textit{Physaria didymocarapa} var. \textit{lanata} (wooly twinpod), Big Horn Mountains, north-central Wyoming. Report prepared for the Bighorn National Forest by the Wyoming Natural Diversity Database – University of Wyoming Laramie, Wyoming.

Heidel B. 2012. Wyoming plant species of concern list, Wyoming plant species of potential concern list, Wyoming plant species of undetermined status  [Internet]. Laramie (WY): University of Wyoming; [cited 2016 Apr 19]. Available from www.uwyo.edu/wyndd/

Heidel B, O’Kane SL. 2012. Wyoming \textit{Physaria} species key. \textit{Castilleja}. 31(3):1-10.

Henry L, Wickham H. 2019. purrr: Functional programming tools. R package version 0.3.0. https://CRAN.R-project.org/package=purrr

Henry L, Wickham H. 2020. rlang: Functions for base types and core R and 'Tidyverse' features. R package version 0.4.4. https://CRAN.R-project.org/package=rlang

Hey J, Waples RS, Arnold ML, Butlin RK, Harrison RG. 2003. Understanding and confronting species uncertainty in biology and conservation. Trends in Ecology and Evolution. 18(11):597-603.

Hijmans RJ. 2020. raster: Geographic data analysis and ,odeling. R package version 3.0-12. https://CRAN.R-project.org/package=raster

Hitchcock, C.L. 1964. \textit{Physaria}. Pages 529-532 In: Vascular plants of the Pacific Northwest, Part 2, By C.L. Hitchcock, A Cronquist, M. Ownbey, and J.W. Thompson. University of Washington Press, Seattle.

Hoffmann J, Heinz C, Moses B. 2015. Listings [Internet]. Heidelberg (Germany): Comprehensive TeX Archive Network; [cited 2018 September 2]. Available from https://ctan.org/pkg/listings?lang=en.

Hollister JW, Shah T. 2017. elevatr: Access elevation data from various APIs. R package version 0.1.3. https://github.com/usepa/elevatr

Hooker WJ. 1830. Flora Boreali-Americana. Vol. 1, pt. 1. Covent Garden, London: Henry G. Bohn. Cruciferae, p. 37-70; p. 49 and tab XVI.

Hooker WJ. 1847. London journal of botany. Vol. 6. London: Hippolyte Bailliere. Botanical information, p. 70; tab V.

Huelsenbeck JP, Ronquist F. 2001. MRBAYES: Bayesian inference of phylogeny. Bioinformatics. 17(8):754-755.

Isaac NJB, Mallet J, Mace GM. 2004. Taxonomic inflation: Its influence on macroecology and conservation. Trends in Ecology and Evolution. 19(9):464-469.

Ives JC, Newberry JS, Gray, Torrey, Thurber, Engelmann, Baird SF. 1861. Report upon the Colorado River of the west. Washington: Government Printing Office. Botany, part 4, page 6.

Jenner RA. 2004. Accepting partnership by submission? Morphological phylogenetics in a molecular millenium. Syst Biol. 53(2):333-342.

Kahle D, Wickham H. 2013. ggmap: Spatial visualization with ggplot2. The R Journal. 5(1):144-161.

Kates HR, Johnson MG, Gardner EM, Zerega NJC, Wickett NJ. 2018. Allele phasing has minimal impact on phylogenetic reconstruction from targeted nuclear gene sequences in a case study of \textit{Artocarpus}. American Journal of Botany. 105(3):404-416.

Katoh K, Kuma K, Toh H, Miyata T. 2005. MAFFT version 5: Improvement in accuracy of multiple sequence alignment. Nucleic Acids Research. 33(2):511-518.

Katoh K, Standley DM. 2013. MAFFT multiple sequence alignment software version 7: Improvements in performance and usability.  Mol Biol and Evol. 30(4):772-780.

Kluge J, Worm S, Lange S, Long D, Böhner J, Yangzom R, Miehe G. 2017. Elevational seed plants richness patterns in Bhutan, Eastern Himalaya. J. Biogeography. 44(8):1711-1722.

Knowles LL, Massatti R. 2017. Distributional shifts - not geographic isolation – as a probable driver of montane species divergence. Ecography. 40(12):1475-1485.

Körner, C. 2000. Why are there global gradients in species richness? Mountains might hold the answer. Trends Ecol Evol. 15(12):513–514.

Kumar S, Stecher G, Tamura K. 2016. MEGA7: Molecular evolutionary genetics analysis version 7.0 for bigger datasets. Mol Biol Evol. 33(7):1870-1874.

Laiolo P, Pato J, Obeso JR. 2018. Ecological and evolutionary drivers of the elevational gradient of diversity. Ecology Letters. 21(7):1022-1032.

Larkin MA, Blackshields G, Brown NP, Chenna R, McGettigan PA, McWilliam H, Valentin F, Wallace IM, Wilm A, Lopez R, Thompson JD, Gibson TJ, Higgins DG. 2007. Clustal W and Clustal X version 2.0. Bioinformatics. 23:2947-2948.

Lesica P. 2012. Manual of Montana Vascular Plants. Texas (Fort Worth): Botanical Research Institute of Texas.

Lichvar RW. 1983. A new species of \textit{Physaria} (Cruciferae) from Wyoming.  Brittonia. 35(2):150-155.

Lichvar RW. 1984. A re-evaluation of \textit{Physaria didymocarpa} var. \textit{integrifolia} (Cruciferae). Madroño. 31(4):203-207.

Lomolino MV. 2001. Elevation gradients of species-density: Historical and prospective views. Global Ecology and Biogeography. 10(1):3-13.

Maddison WP. 1997. Gene trees in species trees. Syst Biol. 46(3):523-536.

Maddison WP, Knowles LL. 2006. Inferring phylogeny despite incomplete lineage sorting. Syst Biol. 55(1):21-30.

Manton I. 1932. Introduction to the general cytology of the Cruciferae. Annals of Botany. 46(532):509-556.

McCain CM, Grytnes J-A. 2010. Elevational gradients in species richness. In: Encyclopedia of Life Sciences (ELS). Chichester (England):John Wiley \& Sons, Ltd.  DOI: 10.1002/9780470015902.a0022548

Miller JS, Venable DL. 2000. Polyploidy and the evolution of gender dimorphism in plants. Science. 289(5488):2335-2338.

Miller MA, Pfeiffer W, Schwartz T. 2010. Creating the CIPRES science gateway for inference of large phylogenetic trees. Paper presented at: GCE. Proceedings of the Gateway Computing Environments Workshop; New Orleans, LA.

Moseley RK, Mancuso M, Caicco SL. 1990. Field investigations of two sensitive plant species on the salmon national forest: \textit{Phycelia lyalli} and \textit{Physaria didymocarpa} var. \textit{lyrata}. Unpublished report; natural heritage section, nongame/endangered wildlife program, bureau of wildlife [internet]. Boise (ID): Idaho Department of Fish and Game; [Cited 2016 Jan 23]. Available from https://fishandgame.idaho.gov/ifwis/idnhp/cdc\_pdf /moser90h.pdf

Müller K. 2017. here: A simpler way to find your files. R package version 0.1. https://CRAN.R-project.org/package=here

Müller K, Wickham H. 2019. tibble: Simple data frames. R package version 2.0.1. https://CRAN.R-project.org/package=tibble

Mulligan GA. 1966. Two new species of \textit{Physaria} (Cruciferae) in Colorado. Canadian journal of botany. 44(12):1661-1665.

Mulligan GA. 1967. Diploid and autotetraploid \textit{Physaria vitulifera} (Cruciferae). Canadian Journal of Botany. 45:183-188.

Mulligan GA. 1967. Cytotaxonomy of \textit{Physaria acutifolia}, \textit{P. chambersii}, and \textit{P. newberryi} (Cruciferae). Canadian journal of botany. 45(10):1887-1898.

Myers N, Mittermeier RA, Mittermeier CG, da Fonseca GAB, Kent J. 2000. Biodiversity hotspots for conservation priorities. Nature. 403(6772):853-858.

Needleman S, Wunsch C. 1970. A general method applicable to the search for similarities in the amino acid sequence of two proteins. J Mol Biol. 48(3):443-53.

Nelson A. 1904. New plants from Wyoming, XV. Bulletin of the Torrey Botanical Club. 31(5):241.

Noroozi J, Talebi A, Doostmohammadi M, Rumpf SB, Linder HP, Schneeweiss GM. 2018. Hotspots within a global biodiversity hotspot - areas of endemism are associated with high mountain ranges. Scientific Reports. 8(1):1-10.

Nuttall T. 1838. Cruciferae: \textit{Vesicaria}, sect. \textit{Physaria}. In: Torrey J, Gray A., editors. A flora of North America. Vol. 1, pt. 1. New York: Wiley \& Putnam. p. 102-103.

O’Kane SL. 2007. \textit{Physaria scrotiformis} (Brassicaceae), a new high–elevation species from southwestern Colorado and new combinations in \textit{Physaria}. Novon. 17(3):376-382.

O’Kane SL. 2010. \textit{Physaria}. In: Flora of North America editorial committee. Flora of North America north of Mexico. Volume 7. Salicaceae to Brassicaceae. New York (NY): Oxford Univ. Press. p. 616-665.

Padial JM, Castroviejo-Fisher S, Köhler J, Vilá C, Chaparro JC, De la Riva I. 2009. Deciphering the products of evolution at the species level: The need for an integrative taxonomy. Zoologica Scripta. 38(4):431-447.

Padial JM, Miralles A, De la Riva I, Vences M. 2010. The integrative future of taxonomy. Frontiers in Zoology. 7(16):1-14.

Pagès H, Aboyoun P, Gentleman R, DebRoy S. 2019. Biostrings: Efficient manipulation of biological strings. R package version 2.50.2. https://www.bioconductor.org/packages/release/bioc/html/Biostrings.html

Paradis E, Schliep K. 2018. ape 5.0: an environment for modern phylogenetics
and evolutionary analyses in R. Bioinformatics. 35(3):526-528.

Payson EB. 1918. Notes on certain Cruciferae. Annals of the Missouri Botanical Garden. 5(2):143-147.

Payson EB. 1921. A monograph of the genus \textit{Lesquerella}. Annals of the Missouri Botanical Garden. 8(1):103-236.

R Core Team. 2019. R: A language and environment for statistical computing [Internet]. Vienna (Austria): R Foundation for Statistical Computing; [cited 2020 Feb 28]. Available from https://www.R-project.org/

Raposo MA, Stopiglia R, Brito GRR, Bockmann FA, Kirwan GM, Gayon J, Dubois A. 2017. What really hampers taxonomy and convservation? A riposte to Garnett and Christidis (2017). Zootaxa. 4317(1):179-184.

Rethwisch, K. 2014. A preliminary systematic analysis of the species of genus \textit{Physaria} (Brassicaceae) of Wyoming. Honors Program Theses. Paper 105.
http://scholarworks.uni.edu/hpt/105

Rieseberg LH, Brouillet L. 1994. Are many plant species paraphyletic? Taxon. 43:21-32.

Rokas A, Williams BL, King N, Carroll SB. 2003. Genome-scale approaches to resolving incongruence in molecular phylogenies. Nature. 425(6960):798-804.

Rokas A, Carroll SB. 2005. More genes or more taxa? The relative contribution of gene number and taxon number to phylogenetic accuracy. Mol Biol Evol. 22(5):1337-1344.

Rollins RC. 1939. The Cruciferous genus \textit{Physaria}. Rhodora. 41:391-414.

Rollins RC, Banerjee UC. 1979b. Trichome patterns in \textit{Physaria} (Cruciferae). Publ. Bussey Inst. Harvard Univ. 1979:65-77.

Rollins RC. 1981. Studies in the genus \textit{Physaria} (Cruciferae). Brittonia. 33(3):332-334.

Rollins RC. 1984. Studies in the Cruciferae of western North America II. Contributions from the Gray Herbarium of Harvard University. 214:1-18.

Rollins RC. 1993. The Cruciferae of Continental North America: Systematics of the Mustard Family from the Artic to Panama. 1st ed. Standford: Standford University Press.

Ronquist F, Teslenko M, Van der Mark P, Ayres DL, Darling A, Höhna S, Larget B, Liu L, Suchard MA, Huelsenbeck JP. 2012. MrBayes 3.2: Efficient Bayesian inference and model choice across a large model space. Syst Biol. 61(3):539-542.

Rostagi S, Liberles DA. 2005. Subfunctionalization of duplicated genes as a transistion state to neofunctionalization. BMC Evol Biol. 5(28):1-7.

Rydberg PA. 1901. Studies on the Rocky Mountain Flora - V. Bulletin of the Torrey Botanical Club. 28(5):278-280.

Rydberg PA. 1902. Studies on the Rocky Mountain Flora - VIII. Bulletin of the Torrey Botanical Club. 29(4):237.

Rydberg PA. 1912. Studies on the Rocky Mountain Flora – XXVII. Bulletin of the Torrey Botanical Club. 39(7):322.

Sanderson MJ, Donoghue MJ. 1996. The relationship between homoplasy and confidence in a phylogenetic tree. In: Sanderson MJ, Hufford L, editors. Homoplasy: The recurrence of similarity in evolution. San Diego (CA): Academic Press. p. 67-89.

Salinas NR, Little DP. 2014. 2Matrix: A utility for indel coding and phylogenetic matrix concatenation. Applications in Plant Sci. 2(1): 1300083.

Scotland RW, Olmstead RG, Bennett JR. 2003. Phylogeny reconstruction: The role of morphology. Syst Biol. 52(4):539-548.

Simmons MP, Ochoterena H. 2000. Gaps as characters in sequence-based phylogenetic analyses. Syst Biol. 49(2):369-381.

Simpson GG. 1951. The species concept. Evolution. 5(4):285–298.

Shaffer BH, Thomson RC. 2007. Delimiting species in recent radiations. Syst Biol. 56(6):896-906.

SOUTHWEST ENVIRONMENTAL INFORMATION NETWORK. c2018. Arizona Chapter: SEINet; [accessed 2018 Jan 8] http//:swbiodiversity.org/seinet/index.php.

Gray A. 1848. Cruciferae: \textit{Vesicaria}. In: Sprauge I, Gray A, editors. Genera florae Americae boreali-orientalis illustrata. Vol. 1. Boston (MA): James Munroe and Company. p. 162.

Steele R, Henderson DM, Johnson FD, Packard P. 1977. Endangered and threatened plants of Idaho. University of Idaho forest, wildlife and range experiment station [Internet]. Moscow (ID): University of Idaho; [Cited 2016 Jan 23]. Available from http://digital.lib.uidaho.edu/cdm/ singleitem/collection/fwres/id/154/rec/3

Steinbauer MJ, Field R, Grytnes, J-A, Trigas P, Ah-Peng C, Attorre F, Birks HJB, Borges PAV, Cardoso P, Chou C-H, De Sanctis M, de Sequeira MM,  Duarte MC, Elias RB, Fern\v{a}ndez-Palacios JM, Gabriel R, Gereau RE, Gillespie RG, Greimler J, Harter DEV, Huang T-J, Irl SDH, Jeanmonod D, Jentsch A, Jump AS, Kueffer C, Nogu\v{e} S, Otto R, Price J, Romeiras MM, Strasberg D, Stuessy T, Svenning J-C, Vetaas OR, Beierkuhnlein C. 2016. Topography-driven isolation, speciation and a global increase of endemism with elevation. Global Ecol Biogeogr. 25(9): 1097-1107.

Steinbauer MJ, Irl SDH, Beierkuhnlein C. 2013. Elevation-driven ecological isolation promotes diversification on Mediterranean islands. Acta Oecologica. 47(1):52-56.

Suksdorf W. 1906. Neue pflanzen aus Washington. West American Scientist. 15(130):58.

Tamura K, Peterson D, Peterson N, Stecher G, Nei M, Kumar S. 2011. MEGA5: Molecular evolutionary genetics analysis using maximum likelihood, evolutionary distance, and maximum parsimony methods. Mol Biol and Evol. 28(10): 2731-2739.

Tamura K, Stecher G, Peterson D, Filipski A, Kumar S. 2013. MEGA6: Molecular evolutionary genetics analysis version 6.0. Mol Biol and Evol. 30(12): 2725-2729.

Templeton AR. 1989. The meaning of species and speciation: A genetic perspective. In Otte D, Endler JA, editors. Speciation and its consequences. Sunderland (MA): Sinauer, p. 3-27.

Thiele K, Yeates D. 2002. Tension arises from duality at the heart of taxonomy. Nature. 419(6905):337.

Thomson SA, Pyle RL, Ahyong ST, Alonso-Zarazaga M, Ammirati J, Araya JF, et al. 2018. Taxonomy based on science is necessary for global conservation. PLoS Biol. 16(3):e2005075.

Trigas P, Panitsa M, Tsiftsis S. 2013. Elevational gradient of vascular plant species richness and endemism in Crete - the effect of post-isolation mountain uplift on a continental island system. PLoS ONE 8(3): e59425. doi:10.1371/journal.pone.0059425

Vetaas OR, Grytnes J-A. 2002. Distribution of vascular plant species richness and endemic richness along the Himalayan elevation gradient in Nepal. Global Ecology and Biogeography. 11(4):291-301.

Walker A. 2018. openxlsx: Read, write and edit XLSX files. R package version 4.1.0. https://CRAN.R-project.org/package=openxlsx

Wang LG, Lam TTY, Xu S, Dai Z, Zhou L, Feng T, Guo P, Dunn CW, Jones BR, Bradley T, Zhu H, Guan Y, Jiang Y, Yu G. treeio: an R package for phylogenetic tree input and output with richly annotated and associated data. 2019. Molecular Biology and Evolution. 37(2):599-603.

Watson S. 1882. Contributions to American botany. Proc. of the Amer. Acad. 17(9):363-364.

Watson S. 1888. Contributions to American botany. Proc. of the Amer. Acad. 23(2):249-287.

Weber WA, Brewbaker JL. 1950. \textit{Physaria vitulifera}, a tetraploid species of Cruciferae. Series in Biology. 3:24-28.

Weber WA, Wittman RC. 1996. Colorado FLora Easter Slope. 2nd Edition. Niwot (CO): University Press of Colorado. Key to Brassicaceae, \textit{Physaria}; p. 137-138.

Wefald M. 2003. Graphical Locator TRS-data [Internet]. Bozeman (MT): Montana State University; [cited 2016 Feb 12]. Available from http://www.esg.montana.edu/gl/trs-data.html

Weigelt P, Kreft H. 2013. Quantifying island isolation – insights from global patterns of insular plant species richness. Ecography. 36(4):417-429.

Welsh SL. 1986. New taxa and combinations in the Utah flora. Great Basin Naturalist. 46(2): 255.

Welsh SL, Reveal JL. 1977. Utah Flora: Brassicaceae (Cruciferae). Great Basin Naturalist. 37(3): 345.

Wickham H. 2016. ggplot2: Elegant graphics for data analysis. 1st ed. New York: Springer-Verlag.

Wickham H. 2019. stringr: Simple, consistent wrappers for common string operations. R package version 1.4.0. https://CRAN.R-project.org/package=stringr

Wickham H. 2011. The split-apply-combine strategy for data analysis. Journal of Statistical Software. 40(1):1-29.

Wickham H, Bryan J. 2019. readxl: Read excel files. R package version 1.3.1. https://CRAN.R-project.org/package=readxl

Wickham H, Francois R, Henry L, Müller K. 2019. dplyr: A grammar of data manipulation. R package version 0.8.0.1. https://CRAN.R-project.org/package=dplyr

Wickham H, Hester J, Francois R. 2018. readr: Read rectangular text data. R package version 1.3.1. https://CRAN.R-project.org/package=readr

Wickham H, Seidel D. 2019. scales: Scale functions for visualization. R package version 1.1.0. https://CRAN.R-project.org/package=scales

Wiens JJ. 2004. The role of morphological data in phylogenetic reconstruction. Syst Biol. 53(4):653-661.

Wiens JJ. 2004. What is speciation and how should we study it. The American Naturalist. 163(6):914-923.

Wilke CO. 2019. cowplot: Streamlined plot theme and plot annotations for 'ggplot2'. R package version 1.0.0. https://CRAN.R-project.org/package=cowplot

Wilke CO. 2020. ggtext: Improved text rendering support for 'ggplot2'. R package version 0.1.0. https://wilkelab.org/ggtext

Wünschiers R. 2013. Computational biology: A practical introduction to biodata processing and analysis with Linux, MySQL, and R. 2nd ed. Berlin (Heidelberg): Springer-Verlag.

Xie Y. 2018. bookdown: Authoring books and technical documents with R markdown. R package version 0.9. https://cran.r-project.org/package=bookdown

Xie Y. 2018. knitr: A general-purpose package for dynamic report generation in R. R package version 1.21. https://cran.r-project.org/package=knitr

Yu G, Smith D, Zhu H, Guan Y, Lam TT-Y. 2017. ggtree: An R package for visualization and annotation of phylogenetic trees with their covariates and other associated data. Methods in Ecology and Evolution. 8(1):28-36.

Zachos FE. 2016. An annotated list of species concepts. In: Species concepts in biology. Switzlerand: Springer International Publishing. p. 77-96.

Zachos FE. 2018. (New) Species concepts, species delimitation and the inherent limitations of taxonomy. Journal of Genetics. 97(4):811-815.

Zhu H. 2019. kableExtra: Construct complex table with 'kable' and pipe syntax. R package version 1.0.1. https://CRAN.R-project.org/package=kableExtra

\setlength{\parindent}{0em}
\setlength{\leftskip}{0em}
\setlength{\parskip}{6pt}
\doublespacing
