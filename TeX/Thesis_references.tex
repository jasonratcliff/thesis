\chapter*{REFERENCES}
\addcontentsline{toc}{chapter}{REFERENCES}
\vspace*{-12pt}

\setlength{\parindent}{-4em}
\setlength{\leftskip}{4em}
\setlength{\parskip}{12pt}
\singlespacing

\hspace*{-4em}

Abellán P, Ribera I. 2017. Using phylogenies to trace the geographical signal of diversification. Journal of Biogeography. 44(10):2236-2246.

Beitz E. 2000. TEXshade: Shading and labeling of multiple sequence alignments using \LaTeX 2$\backepsilon$. Bioinformatics. 16(2):135-139.

Bloomquist, Lemey, Suchard. 2010. Three roads diverged? Routes to phylogeographic inference. Trends in Ecology and Evolution. 25(11):626-632.

Bodenhofer U, Bonatesta E, Horej\v{s}-Kainrath C, Hochreiter S. 2015. msa: An R package for multiple sequence alignment. Bioinformatics. 31(24):3997-3999.

Comai L. 2005. The advantages and disadvantages of being polyploid. Nature Reviews Genetics. 6(11):836-846.


Fairbairns R, Rahtz S, Barroca L. 2016. rotating - rotation tools, including rotated full-page floats [Internet]. Heidelberg (Germany): Comprehensive TeX Archive Network; [cited 2018 April 23]. Available from https://ctan.org/pkg/rotating?lang=en.

Fois M, Fenu G, Cañadas EM, Bacchetta G. 2017. Disentangling the influence of environmental and anthropogenic factors on the distribution of endemic vascular plants in Sardinia. PLoS ONE. 12(8):e0182539.

Gillespie RM, Roderick GK. 2014. Geology and climate drive diversification. Nature. 509(7500):297-298.

Graham CH, Carnaval AC, Cadena CD, Zamudio KR, Roberts TE, Parra JL, McCain CM, Bowie RCK, Moritz C, Baines SB, Schneider CJ, VanDerWal J, Rahbek C, Kozak KH, Sanders NJ. 2014. The origin and maintenance of montane diversity: Integrating evolutionary and ecological processes. Ecography. 37(8):711-719.

Hijmans RJ. 2020. raster: Geographic data analysis and ,odeling. R package version 3.0-12. https://CRAN.R-project.org/package=raster

Kates HR, Johnson MG, Gardner EM, Zerega NJC, Wickett NJ. 2018. Allele phasing has minimal impact on phylogenetic reconstruction from targeted nuclear gene sequences in a case study of \textit{Artocarpus}. American Journal of Botany. 105(3):404-416.

Kluge J, Worm S, Lange S, Long D, Böhner J, Yangzom R, Miehe G. 2017. Elevational seed plants richness patterns in Bhutan, Eastern Himalaya. J. Biogeography. 44(8):1711-1722.

Knowles LL, Massatti R. 2017. Distributional shifts - not geographic isolation – as a probable driver of montane species divergence. Ecography. 40(12):1475-1485.

Körner, C. 2000. Why are there global gradients in species richness? Mountains might hold the answer. Trends Ecol Evol. 15(12):513–514.

Laiolo P, Pato J, Obeso JR. 2018. Ecological and evolutionary drivers of the elevational gradient of diversity. Ecology Letters. 21(7):1022-1032.

Lomolino MV. 2001. Elevation gradients of species-density: Historical and prospective views. Global Ecology and Biogeography. 10(1):3-13.

Maddison WP. 1997. Gene trees in species trees. Syst Biol. 46(3):523-536.

Manton I. 1932. Introduction to the general cytology of the Cruciferae. Annals of Botany. 46(532):509-556.

McCain CM, Grytnes J-A. 2010. Elevational gradients in species richness. In: Encyclopedia of Life Sciences (ELS). Chichester (England): John Wiley and Sons, Ltd.  DOI: 10.1002/9780470015902.a0022548

Miller JS, Venable DL. 2000. Polyploidy and the evolution of gender dimorphism in plants. Science. 289(5488):2335-2338.

Mulligan GA. 1967. Diploid and autotetraploid \textit{Physaria vitulifera} (Cruciferae). Canadian Journal of Botany. 45:183-188.

Myers N, Mittermeier RA, Mittermeier CG, da Fonseca GAB, Kent J. 2000. Biodiversity hotspots for conservation priorities. Nature. 403(6772):853-858.

Noroozi J, Talebi A, Doostmohammadi M, Rumpf SB, Linder HP, Schneeweiss GM. 2018. Hotspots within a global biodiversity hotspot - areas of endemism are associated with high mountain ranges. Scientific Reports. 8(1):1-10.

Paradis E, Schliep K. 2018. ape 5.0: an environment for modern phylogenetics
and evolutionary analyses in R. Bioinformatics. 35(3):526-528.

Rollins RC, Banerjee UC. 1979b. Trichome patterns in \textit{Physaria} (Cruciferae). Publ. Bussey Inst. Harvard Univ. 1979:65-77.

Rostagi S, Liberles DA. 2005. Subfunctionalization of duplicated genes as a transistion state to neofunctionalization. BMC Evol Biol. 5(28):1-7.

Steinbauer MJ, Field R, Grytnes, J-A, Trigas P, Ah-Peng C, Attorre F, Birks HJB, Borges PAV, Cardoso P, Chou C-H, De Sanctis M, de Sequeira MM,  Duarte MC, Elias RB, Fern\v{a}ndez-Palacios JM, Gabriel R, Gereau RE, Gillespie RG, Greimler J, Harter DEV, Huang T-J, Irl SDH, Jeanmonod D, Jentsch A, Jump AS, Kueffer C, Nogu\v{e} S, Otto R, Price J, Romeiras MM, Strasberg D, Stuessy T, Svenning J-C, Vetaas OR, Beierkuhnlein C. 2016. Topography-driven isolation, speciation and a global increase of endemism with elevation. Global Ecol Biogeogr. 25(9): 1097-1107.

Steinbauer MJ, Irl SDH, Beierkuhnlein C. 2013. Elevation-driven ecological isolation promotes diversification on Mediterranean islands. Acta Oecologica. 47(1):52-56.

Templeton AR. 1989. The meaning of species and speciation: A genetic perspective. In Otte D, Endler JA, editors. Speciation and its consequences. Sunderland (MA): Sinauer, p. 3-27.

Trigas P, Panitsa M, Tsiftsis S. 2013. Elevational gradient of vascular plant species richness and endemism in Crete - the effect of post-isolation mountain uplift on a continental island system. PLoS ONE 8(3): e59425. doi:10.1371/journal.pone.0059425

Vetaas OR, Grytnes J-A. 2002. Distribution of vascular plant species richness and endemic richness along the Himalayan elevation gradient in Nepal. Global Ecology and Biogeography. 11(4):291-301.

Wang LG, Lam TTY, Xu S, Dai Z, Zhou L, Feng T, Guo P, Dunn CW, Jones BR, Bradley T, Zhu H, Guan Y, Jiang Y, Yu G. treeio: an R package for phylogenetic tree input and output with richly annotated and associated data. 2019. Molecular Biology and Evolution. 37(2):599-603.

Weber WA, Brewbaker JL. 1950. \textit{Physaria vitulifera}, a tetraploid species of Cruciferae. Series in Biology. 3:24-28.

Weigelt P, Kreft H. 2013. Quantifying island isolation – insights from global patterns of insular plant species richness. Ecography. 36(4):417-429.

Wiens JJ. 2004. What is speciation and how should we study it. The American Naturalist. 163(6):914-923.

Yu G, Smith D, Zhu H, Guan Y, Lam TT-Y. 2017. ggtree: An R package for visualization and annotation of phylogenetic trees with their covariates and other associated data. Methods in Ecology and Evolution. 8(1):28-36.

\setlength{\parindent}{0em}
\setlength{\leftskip}{0em}
\setlength{\parskip}{6pt}
\doublespacing
