\chapter*{REFERENCES}
\addcontentsline{toc}{chapter}{REFERENCES}
\vspace*{-12pt}

\setlength{\parindent}{-4em}
\setlength{\leftskip}{4em}
\setlength{\parskip}{12pt}
\singlespacing

\hspace*{-4em}Ackerfield J. 2015. Flora of Colorado. 1st ed. Fort Worth (TX): Botanical Research Institute of Texas.

Aho AV, Kernighan BW, Weinberger PJ. 1988. The AWK programming language. 1st ed. Murray Hill (NJ): AT\&T Bell Laboratories.

Alexander PJ, Rajanikanth G, Bacon CD, Bailey CD. 2007. Recovery of plant DNA using a reciprocating saw and silica-based columns. Mol Ecol Notes. 7(1):5-9.

Al-Shehbaz IA, O’Kane SL. 2002. \textit{Lesquerella} is united with \textit{Physaria} (Brassicaceae). Novon. 12(3):319-329.

Becker RA, and Wilks AR. 1993. Maps in S. [Internet]. Murray Hill (NJ): AT\&T Bell Laboratories Statistics Research Report; [cited 2018 Apr 27]. Available from http://ect.bell-labs.com/sl/doc/93.2.ps

Becker RA, and Wilks AR. 1995. Constructing a geographical database. [Internet]. Murray Hill (NJ): AT\&T Bell Laboratories Statistics Research Report; [cited 2018 Apr 27]. Available from http://ect.bell-labs.com/sl/doc/95.2.ps

Beitz E. 2000. TEXshade: shading and labeling of multiple sequence alignments using \LaTeX 2$\backepsilon$. Bioinformatics. 16(2):135-139.

Bloomquist, Lemey, Suchard. 2010. Three roads diverged? Routes to phylogeographic inference. Trends in Ecology and Evolution. 25(11):626-632.

Bodenhofer U, Bonatesta E, Horej\v{s}-Kainrath C, Hochreiter S. 2015. msa: an R package for multiple sequence alignment. Bioinformatics. 31(24):3997-3999. 

Cornish-Bowden A. 1985. Nomenclature for incompletely specified bases in nucleic acid sequences: recommendations 1984.  Nucleic Acids Research. 13(9):3021-3030.

Darwin Core. c2009-2015. San Francisco (CA): Taxonomic Databases Working Group; [accessed 2018 Jan 9]. http://rs.tdwg.org/dwc/index.htm.

Dorn R. 2001. Vascular Plants of Wyoming. 3rd edition. Cheyenne (WY): Mountain West Publishing. Key to Brassicaceae, \textit{Physaria}; p. 143-144.

Drabkova L, Kirschner J, Vlcek C. 2002. Comparison of seven DNA extraction and amplification protocols in historical herbarium specimens of Juncaceae. Plant Mol Biol Reporter. 20(2):161-175.

Fairbairns R, Rahtz S, Barroca L. 2016. rotating - rotation tools, including rotated full-page floats [Internet]. Heidelberg (Germany): Comprehensive TeX Archive Network; [cited 2018 April 23]. Available from https://ctan.org/pkg/rotating?lang=en.

Fertig W. 1998. Status report on Dorn’s Twinpod (Physaria dornii) in Southwestern Wyoming. Report prepared for the Bureau of Land Management Wyoming State Office by the Wyoming Natural Diversity Database.  

Fertig W. 2002. Status of Tufted Twinpod (Physaria condensata) in Southwest Wyoming. Report prepares for the Bureau of Land Management Wyoming State Office and Wyoming Natural Diversity Database by a Botanical Consultant.

Bash manual page. [Computer Program]. 2018. München (Germany): Michael Kerrisk; [Cited 2018 August 29].  Available from http://man7.org/linux/man-pages/man1/getopts.1p.html

GNU bash manual. [Internet]. 2016. Boston (MA): Free Software Foundation; [Cited 2018 August 25]. Available from https://www.gnu.org/software/bash/manual/

Google Inc. 2015. Google Earth [Internet]. Mountain View (CA): Google Inc.; [cited 2016 Feb 12]. Available from https://www.google.com/earth/

Gotoh O. 1995. A weighting system and algorithm for aligning many phylogenetically related sequences. Compu Appl Biosci. 11(5):543-51.

Gray A. 1895. Synoptical flora of North America. Vol. 1, Pt. 1. University Press: Cambridge. Cruciferae, p. 121.

Griekspoor A, Groothuis T. 2015. 4Peaks version 1.7.2. Nucleobytes.

Guangchuang Yu, David Smith, Huachen Zhu, Yi Guan, Tommy Tsan-Yuk Lam. ggtree: an R package for visualization and annotation of phylogenetic trees with their covariates and other associated data. Methods in Ecology and Evolution 2017, 8(1):28-36
  
Hadley Wickham (2011). The Split-Apply-Combine Strategy for Data Analysis. Journal of Statistical Software, 40(1), 1-29. URL http://www.jstatsoft.org/v40/i01/.

Handley J, Heidel B. 2011. Status of *Physaria didymocarapa* var. *lanata* (wooly twinpod), Big Horn Mountains, north-central Wyoming. Report prepared for the Bighorn National Forest by the Wyoming Natural Diversity Database – University of Wyoming Laramie, Wyoming.  

Heidel B. 2012. Wyoming plant species of concern list, Wyoming plant species of potential concern list, Wyoming plant species of undetermined status  [Internet]. Laramie (WY): University of Wyoming; [cited 2016 Apr 19]. Available from www.uwyo.edu/wyndd/

Heidel B, O’Kane SL. 2012. Wyoming \textit{Physaria} species key. *Castilleja*. 31(3):1-10.

Hitchcock, C.L. 1964. \textit{Physaria}. Pages 529-532 In: Vascular plants of the Pacific Northwest, Part 2, By C.L. Hitchcock, A Cronquist, M. Ownbey, and J.W. Thompson. University of Washington Press, Seattle.

Hoffmann J, Heinz C, Moses B. 2015. Listings [Internet]. Heidelberg (Germany): Comprehensive TeX Archive Network; [cited 2018 September 2]. Available from https://ctan.org/pkg/listings?lang=en.

Hooker, WJ. 1830. Flora Boreali-Americana. Vol. 1, pt. 1. Covent Garden, London: Henry G. Bohn. Cruciferae, pp. 37-70; p. 49 and tab XVI.

Hooker, WJ. 1847. London journal of botany. Vol. 6. London: Hippolyte Bailliere. Botanical information, p. 70; tab V.  

Huelsenbeck JP, and F. Ronquist. 2001. MRBAYES: Bayesian inference of phylogeny. Bioinformatics 17:754-755.

Ives JC, Newberry JS, Gray, Torrey, Thurber, Engelmann, Baird SF. 1861. Report upon the Colorado River of the west. Washington: Government Printing Office. Botany, part 4, page 6. 

Katoh K, Kuma K, Toh H, Miyata T. 2005. MAFFT version 5: improvement in accuracy of multiple sequence alignment. Nucleic Acids Research. 33(2):511-518.

Katoh K, Standley DM. 2013. MAFFT multiple sequence alignment software version 7: improvements in performance and usability.  Mol Biol and Evol. 30(4):772-780.

Kumar S, Stecher G, Tamura K. 2016. MEGA7: molecular evolutionary genetics analysis version 7.0 for bigger datasets. Mol Biol Evol. 33(7):1870-1874.

Larkin MA, Blackshields G, Brown NP, Chenna R, McGettigan PA, McWilliam H, Valentin F, Wallace IM, Wilm A, Lopez R, Thompson JD, Gibson TJ, Higgins DG. 2007. Clustal W and Clustal X version 2.0. Bioinformatics. 23:2947-2948. 

Lesica P. 2012. Manual of Montana Vascular Plants. Texas (Fort Worth): Botanical Research Institute of Texas.

Lichvar RW. 1983. A new species of \textit{Physaria} (Cruciferae) from Wyoming.  Brittonia. 35(2):150-155.

Lichvar RW. 1984. A re-evaluation of *Physaria didymocarpa* var. *integrifolia* (Cruciferae). Madroño. 31(4):203-207.

Maddison WP, Knowles LL. 2006. Inferring phylogeny despite incomplete lineage sorting. Syst Biol. 55(1):21-30.

maps: draw geographical maps R package version 3.3.0 [Internet]. 2018. Original S code by Richard A. Becker, Allan R. Wilks. R version by Ray Brownrigg. Enhancements by Thomas P Minka and Alex Deckmyn. [cited 2018 Apr 27]. Available from https://CRAN.R-project.org/package=maps 

Miller MA, Pfeiffer W, Schwartz T. 2010. Creating the CIPRES science gateway for inference of large phylogenetic trees. Paper presented at: GCE. Proceedings of the Gateway Computing Environments Workshop; New Orleans, LA.

Moseley RK, Mancuso M, Caicco SL. 1990. Field investigations of two sensitive plant species on the salmon national forest: Phycelia lyalli and Physaria didymocarpa var. lyrata. Unpublished report; natural heritage section, nongame/endangered wildlife program, bureau of wildlife [internet]. Boise (ID): Idaho Department of Fish and Game; [Cited 2016 Jan 23]. Available from https://fishandgame.idaho.gov/ifwis/idnhp/cdc\_pdf /moser90h.pdf

Mulligan GA. 1966. Two new species of \textit{Physaria} (Cruciferae) in Colorado. Canadian journal of botany. 44(12):1661-1665.

Mulligan GA. 1967. Cytotaxonomy of *Physaria acutifolia*, *P. chambersii*, and *P. newberryi* (Cruciferae). Canadian journal of botany. 45(10):1887-1898.

Needleman S, Wunsch C.  1970. A general method applicable to the search for similarities in the amino acid sequence of two proteins. J Mol Biol. 48(3):443-53.

Nelson A. 1904. New plants from Wyoming, XV. Bulletin of the Torrey Botanical Club. 31(5):241.

O’Kane SL. 2007. *Physaria scrotiformis* (Brassicaceae), a new high–elevation species from southwestern Colorado and new combinations in \textit{Physaria}. Novon. 17(3):376-382.

O’Kane SL. 2010. \textit{Physaria}. In: Flora of North America editorial committee. Flora of North America north of Mexico. Volume 7. Salicaceae to Brassicaceae. New York (NY): Oxford Univ. Press. p. 616-665. 
Payson EB. 1918. Notes on certain Cruciferae. Annals of the Missouri Botanical Garden. 5(2):143-147.

Pagès H, Aboyoun P, Gentleman R and DebRoy S (2017). Biostrings: Efficient manipulation of biological strings. R package version 2.46.0. 

Payson EB. 1921. A monograph of the genus \textit{Lesquerella}. Annals of the Missouri Botanical Garden. 8(1):103-236.

Rieseberg LH, Brouillet L. 1994. Are many plant species paraphyletic? Taxon. 43:21-32.

Rethwisch, K. 2014. A preliminary systematic analysis of the species of genus \textit{Physaria} (Brassicaceae) of Wyoming. Honors Program Theses. Paper 105.
http://scholarworks.uni.edu/hpt/105

Rokas A, Carroll SB. 2005. More genes or more taxa? The relative contribution of gene number and taxon number to phylogenetic accuracy. Mol Biol Evol. 22(5):1337-1344.

Rollins RC. 1939. The Cruciferous genus \textit{Physaria}. Rhodora. 41:391-414.

Rollins RC, Banerjee UC. 1979b. Trichome patterns in \textit{Physaria} (Cruciferae). Publ. Bussey Inst. Harvard Univ. 1979:65-77.

Rollins RC. 1981. Studies in the genus \textit{Physaria} (Cruciferae). Brittonia. 33(3):332-334.

Rollins RC. 1984. Studies in the Cruciferae of western North America II. Contributions from the Gray Herbarium of Harvard University. 214:1-18.

Rollins RC. 1993. The Cruciferae of Continental North America: Systematics of the Mustard Family from the Artic to Panama. 1st ed. Standford: Standford University Press.

Ronquist F, Teslenko M, Van der Mark P, Ayres DL, Darling A, Höhna S, Larget B, Liu L, Suchard MA, Huelsenbeck JP. 2012. MrBayes 3.2: Efficient Bayesian inference and model choice across a large model space. Syst Biol. 61(3):539-542.

Rydberg PA. 1901. Studies on the Rocky Mountain Flora - V. Bulletin of the Torrey Botanical Club. 28(5):278-280.

Rydberg PA. 1902. Studies on the Rocky Mountain Flora - VIII. Bulletin of the Torrey Botanical Club. 29(4):237.

Rydberg PA. 1912. Studies on the Rocky Mountain Flora – XXVII. Bulletin of the Torrey Botanical Club. 39(7):322.

Salinas NR, Little DP. 2014. 2Matrix: a utility for indel coding and phylogenetic matrix concatenation. Applications in Plant Sci. 2(1): 1300083.

Simmons MP, Ochoterena H. 2000. Gaps as characters in sequence-based phylogenetic analyses. Syst Biol. 49(2):369-381.

SOUTHWEST ENVIRONMENTAL INFORMATION NETWORK. c2018. Arizona Chapter: SEINet; [accessed 2018 Jan 8] http//:swbiodiversity.org/seinet/index.php.

Sprague I, Gray A. 1848. Genera florae americae boreali-orientalis illustrata.  Vol. 1. Boston: James Munroe and Company. Cruciferae, pp. 161-162.  

Steele R, Henderson DM, Johnson FD, Packard P. 1977. Endangered and threatened plants of Idaho. University of Idaho forest, wildlife and range experiment station [Internet]. Moscow (ID): University of Idaho; [Cited 2016 Jan 23]. Available from http://digital.lib.uidaho.edu/cdm/ singleitem/collection/fwres/id/154/rec/3

Suksdorf W. 1906. Neue pflanzen aus Washington. West American Scientist. 15(130):58.

Tamura K, Peterson D, Peterson N, Stecher G, Nei M, Kumar S. 2011. MEGA5: molecular evolutionary genetics analysis using maximum likelihood, evolutionary distance, and maximum parsimony methods. Mol Biol and Evol. 28(10): 2731-2739.

Tamura K, Stecher G, Peterson D, Filipski A, Kumar S. 2013. MEGA6: molecular evolutionary genetics analysis version 6.0. Mol Biol and Evol. 30(12): 2725-2729.

Torrey J, Gray A. 1838. A flora of North America. Vol. 1, pt. 1. New York: Wiley \& Putnam. Cruciferae, pp. 102-103.
  
Watson S. 1882. Contributions to American botany. Proc. of the Amer. Acad. 17(9):363-364.

Watson S. 1888. Contributions to American botany. Proc. of the Amer. Acad. 23(2):249-287.

Weber WA, Wittman RC. 1996. Colorado FLora Easter Slope. 2nd Edition. Niwot (CO): University Press of Colorado. Key to Brassicaceae, \textit{Physaria}; p. 137-138.

Wefald M. 2003. Graphical Locator TRS-data [Internet]. Bozeman (MT): Montana State University; [cited 2016 Feb 12]. Available from http://www.esg.montana.edu/gl/trs-data.html

Welsh SL, Reveal JL. 1977. Utah Flora: Brassicaceae (Cruciferae). Great Basin Naturalist. 37(3): 345.

Welsh SL. 1986. New taxa and combinations in the Utah flora. Great Basin Naturalist. 46(2): 255.

Wickham H, Francois R, Henry L, Müller K. (2017). dplyr: A Grammar of Data Manipulation. R package version 0.7.4. https://CRAN.R-project.org/package=dplyr

Wünschiers R. 2013. Computational biology: a practical introduction to biodata processing and analysis with Linux, MySQL, and R. 2nd ed. Berlin (Heidelberg): Springer-Verlag.

\setlength{\parindent}{0em}
\setlength{\leftskip}{0em}
\setlength{\parskip}{6pt}
\doublespacing
