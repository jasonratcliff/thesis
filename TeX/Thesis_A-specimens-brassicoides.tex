\section*{\textit{Physaria brassicoides}}

%%% NORTH DAKOTA STATE SPECIMENS
  \textbf{North Dakota: Unknown County:}
24 May 1893; \textit{Williams s.n.} (MO). 
  \textbf{McKenzie County:} 
Gorham. 22 May 1938; \textit{Moran 399} (F), \textit{Moran 400} (MONT, F).  
  \textbf{Billings County:}
Medora. 30 May 1908; \textit{s.n. 2304} (RM).
Medora. 1 Jun 1912; \textit{Bergman s.n.} (MO).
Medora. 19 Aug 1891; \textit{Lee s.n.} (MONT-93139).
Medora. 17 Jul 1898; \textit{s.n. s.n.} (NY).
1 mi. S of Medora. 28 May 1969; \textit{Ward 118} (MO).

%%% SOUTH DAKOTA STATE SPECIMENS
  \textbf{South Dakota: Unknown County:}
Badlands, Dakota Territory. 1886; \textit{Hatcher s.n.} (NY).
Badlands. Jun 1886; \textit{Hatcher s.n.} (UC-10768).
  \textbf{Harding County:}
Slim Buttes. 14 Jun 1911; \textit{Visher 472} (RM).
  \textbf{Haakon County:}
9 mi. N, 11 E Billsburg Steep. 24 Jun 1967; \textit{Stephens 12190} (NY, CAS).
  \textbf{Pennington County:}
Badlands National Park, Hay Butte Overlook, Sage Creek Rim Road. 3,123' elevation; Lat 43º52.540'N, Long 102º15.363'W; 19 Jul 2005; \textit{O'Kane, Jr. and Grady 7902} (RM, NY, ISTC).
  \textbf{Ogalala Lakota County:}
Rockyford. 12 Jun 1940; \textit{McIntosh 7} (US).
  \textbf{Fall River County:}
7 mi. SW Hot Springs, Angusturo road. 21 May 1970; \textit{Stephens 38275} (NY).

%%% NEBRASKA STATE SPECIMENS
  \textbf{Nebraska: Unknown County:}
Badlands. 1853-1854; \textit{Hayden s.n.} (NY).
1842; \textit{Fremont s.n.} (NY).
  \textbf{Scotts Bluff County:}
Clay knolls and banks, 6 mi. W of Gering. 20 May 1979; \textit{Rollins and Rollins 7907} (RM, NY, US, F, MO).
Steep clay hillside, canyon wall just south of Scotts Bluff. 19 May 1979; \textit{Rollins and Rollins 7904} (NY, GH, MO).

%%% MONTANA STATE SPECIMENS
  \textbf{Montana: Unknown County:}
Sioux Forest, Ekalaka Division. 3,700' elevation; 28 Jun 1917; \textit{Flint 5} (RM).
  \textbf{Carter County:}
Newberry knob, ca. 9 air mi. SW of Ekalaka. 3,500' elevation; T1S R57E S3; 27 Jun 1997; \textit{Vanderhorst 5673a} (RM, MONT, MONTU).
Powderville Road badlands, ca. 17 air mi. WSW of Ekalaka. 3,000' elevation; T1S R55E S9; 12 Jul 1997; \textit{Vanderhorst 5731} (RM).

%%% WYOMING STATE SPECIMENS
  \textbf{Wyoming: Crook County:}
New Haven. 4,350’ elevation; 1925; \textit{Strusou 5} (RM).
Devil's Tower National Monument; near E boundary on hill above Redbeds. 4,050' elevation; T53N R65W S7; 30 Jun 1982; \textit{Marriott 1251} (RM).
0.8 mi. NW of Hulett. 24 Jun 1963; \textit{Mulligan and Mosquin 2810} (NY, UC).
N edge of Hulett. 3,800' elevation; T54N R64W S7; 26 Jul 1982; \textit{Dorn 3769} (RM).
1 mi. NW Hulett. 3,900' elevation; T54N R65W S11; 30 May 1935; \textit{Ownbey 610} (RM, NY, UTC, CAS, IDS, UC).
Black Hills, Government Valley, ca. 8 air mi. NNE of Sundance. 4,600' elevation; T52N R62W S5 W1/2; 6 Jun 1983; \textit{Marriott 2586} (RM).
Black Hills, Redwater Cr., Queen Ranch, ca. 7 air mi. SSW of Aladdin. 4,100' elevation; T53N R62W S25 and S26; 31 May 1983; \textit{Marriott 2511} (RM).
Black Hills, Sundance Cr., ca. 8.5 air mi. NE of Sundance. 4,300' elevation; T52N R62W S24 SE1/4; 12 Jul 1983; \textit{Marriott 4461} (RM).
  \textbf{Weston County:}
Ca. 4 (air) mi. SSE of Rochelle. 4,200' elevation; T41N R67W S7 and S8; 20 Jun 1978; \textit{Hartman, Dueholm, and Sanguinetti 6734} (NY).
  \textbf{Niobara County:}
Ca. 2.5 (air) mi. E of Cow Creek. 4,100' elevation; T38N R64W S29; 23 Jun 1979; \textit{Dueholm 7374} (RM).
Ca. 23 air mi. NW of Lusk, ca. 14 air mi. SW of Lance Creek. 5,200' elevation; T34N R67W S24 SW1/4; 2 Jun 1978; \textit{Nelson 1597} (RM, NY).
  \textbf{Campbell County:}
Red rocky hillside, 1.1 mi. E of Hwy. 59 and 6.5 mi. N of jct. with Hwy. 387. 1,688m elevation; Lat 42:49'34"N, Long 105:03'22"W; 23 May 1996; \textit{Salywon and Dierig 3130} (RM).
  \textbf{Converse County:}
Flat Top Hill, 6.5 mi. N of Douglas and 15.1 mi. E of Hwy. 59 on Bill Hall Rd. then Flat Top Rd. 1,688m elevation; Lat 42º49'34"N, Long 105º03'22"W; 23 Jun 1996; \textit{Salywon and Dierig 3127} (MO).
Flat Top Hill, ca. 17 air mi. ENE of Douglas. 5,400' elevation; T33N R68W S20; 18 Jun 1981; \textit{Dueholm 11608} (RM).
5 mi. SE of Douglas. 5,100' elevation; 20 May 1979; \textit{Rollins and Rollins 7913} (RM, NY, GH, US, MO).
Platte R. 9 mi. NW of Orin. 4,950' elevation; 20 Jul 1950; \textit{Ripley and Barneby 10551} (NY, NY-3507).
About 6 mi. S of Douglas. 5,000' elevation; 10 Jul 1951; \textit{Porter 5727} (CAS).
5.4 mi. S of I-25 on Hwy. 95. W side of rd. 1,465m elevation; Lat 42º40'43"N, Long 105º23'38"W; 23 Jun 1996; \textit{Salywon and Dierig 3122} (RM).
  \textbf{Platte County:}
Southern Powder River Basin, Southeastern Plains; ca. 5 air mi. NNW of Glendo. 4,900-5,000' elevation; T30N R68W S19 N1/2; 1 Jul 1993; \textit{Hartman 39479} (RM).
Ca. 5 air mi. NNW of Guernsey ca. 2 air mi. WNW of Hartville in Long Canyon. 4,600' elevation; T27N R66W S3 NE1/4; 22 Jun 1978; \textit{Nelson and Ehrmann 1827} (NY, UC).
Long Canyon, ca. 1.2 km N of Rocky Pass and ca. 1.3 km W of Webb Canyon Road. 1,460m elevation; DMC (5)21300 (46)88300; 10 Jul 1996; \textit{Hazlett, Arnett, and Popolizio 954} (RM).
Site ca. 1.2 km NNE of Rocky Pass, ca. 0.9 km W of Webb Canyon Road. 1,600m elevation; DMC (5)21750 (46)88200; 27 May 1997; \textit{Popolizio, and Hazlett 1199} (RM).
1 mi. above Gray Rocks Dam along road on S side. 4,600' elevation; T25N R66W S6; 8 Jun 1985; \textit{Dorn 4225} (NY).
Dickinson Hill on N rim of Goshen Hole, ca. 13 air mi. SE of Wheatland. 5,000' elevation; T23N R66W S12; 17 Jun 1981; \textit{Dueholm 11549} (RM).
Goshen Hole Rim; Dickinson Hill ca. 3.5 air mi. W of Goshen County, ca. 16 air mi. NNE of Chugwater; ca. 13.5 air mi. ESE of Wheatland. 4,950-5,120' elevation; T23N R66W S12 SW1/4; 29 Jun 1994; \textit{Nelson 32410} (RM).
  \textbf{Goshen County:}
  Common in the sandy hills between Wheatland and Veteran. 4,400' elevation; T24N R64W S24; 19 Jun 1953; \textit{Porter 6237} (RM, UC, MO).
  \textbf{Laramie County:}
Spring Creek. 6,000' elevation; T19N R68W S16 NE1/4; 6 Jun 1989; \textit{Dorn 4975} (NY).
6,000' elevation; T19N R68W S15 NW1/4; 9 Jun 1989; \textit{Dorn 4981} (RM).
Southeastern Plains; in the canyon along Spring Creek, ca. 34 air mi. NNW of Cheyenne; ca. 11 air mi. ENE of Farthing (Iron Mountain). 6,020-6,220' elevation; T19N R68W S21 NW1/4; 8 Jul 1994; \textit{Nelson 32550} (RM).
