\section*{\textit{Physaria acutifolia}}

%%% UNKOWN LOCALITIES
  \textbf{Unkown State:}
  \textbf{Unknown County:}
Fremont's (2nd) Expedition to California. 1843-1844;
\textit{Unknown Collector s.n.} (NY, NY, NY).
%%% MONTANA STATE SPECIMENS
  \textbf{MONTANA:}
  \textbf{Prairie County:}
Head of Powder River. Sep 1859; \textit{Hayden s.n.} (MO-3093900).
% Petroleum
  \textbf{Petroleum County:}
App. 13 mi. N and 7 mi. NE of Winnett. 3,000' elevation; T18N R27E S33 NW1/4;
10 Jul 1999; \textit{Heidel 1896} (ISTC).
% Garfield
  \textbf{Garfield County:}
9 mi. E. Mosby; Hwy. 18. 17 May 1952; \textit{Wright and Wright 102} (RM).
Hell Creek State Park; One colony on dry S-Slope of high ridge above meadow
with camping facilities. 10 Jun 1978;
\textit{Lackschewitz 8092} (MONTU, MONT).
% Musselshell
  \textbf{Musselshell County:}
Roundup, 10 mi. NW. 18 Jun 1957; \textit{Booth 57901} (RM).
% Treasure
  \textbf{Treasure County:}
Divide between Big Horn and Hysham; Hwy. 10. 24 May 1952;
\textit{Wright and Wright s.n.} (MONT-44019, MONT-44025).
% Rosebud
  \textbf{Rosebud County:}
Forsyth. 24 Jul 1901; \textit{Blankinship s.n.} (MONT-9307).
Fort Union Basin. T2N R41E S20; 23 May 1975; \textit{Lackschewitz 5999} (MONTU).
Custer Nat. Forest, Ashland District, East Fork (off O'Dell Cr.) ca. 1.5 air mi.
E of O'Dell Road. 3,750' elevation; T5S R44E S10 NE1/4 NE1/4; 13 Jun 1995;
\textit{Marriott 11510} (MONT).
Zook Creek Wilderness Study Area. 3,375' elevation; T6S R42E S2 NW1/4;
20 Jun 2001; \textit{Taylor 8669} (MONTU).
% Custer
  \textbf{Custer County:}
Miles City. 2,500' elevation; 16 Jun 1936; \textit{Roberts 257} (MONT).
% Powder River
  \textbf{Powder River County:}
Top of the hills at the head of Wilbur Creek. 3,800' elevation; T3S R46E S1;
20 Jun 2011; \textit{Lesica 10596} (RM).
On the north side of Cow Creek 1 mi. above Otter Creek. 3,700' elevation;
T6S R45E S25; 8 May 2005; \textit{Lesica 9029} (NY, MONTU).
0.5 mi. SW of Fort Howed Work Center. 3,500' elevation; T6S R45E S25 NE1/4;
23 Jun 1999; \textit{Heidel 1845} (ISTC).
Spear Hills, over 3 mi. S of Moorhead. 3,760' elevation; T9S R48E S30 SE1/4;
30 May 1999; \textit{Heidel 1811} (MONTU).
% Yellowstone
  \textbf{Yellowstone County:}
7 mi. S of Custer Station. 6 Jul 1890; \textit{Blankinship 117} (MONT, MO).
Billings vicinity. 19 Jun 2002; \textit{Clawson s.n.} (MONT-77887).
Grassland 5-10 mi. S.E. on Hwy. 87 toward Hardin from Billings, Montana.
14 May 1941; \textit{White s.n.} (MONT-54041).
Growing 14 mi. N of Billings along Montana 3. 9 Jun 1966;
\textit{Davidse and Collotzi 657} (NY).
5.6 mi S. Billings on \# 87. 20 May 1955; \textit{Scharff s.n.} (MONT-49106).
Billings, 5 mi. SE on \# 87. 2 Jun 1955; \textit{Scharff s.n.} (MONT- 49613).
% Carbon
  \textbf{Carbon County:}
Beartooth Forest. 5,500' elevation; T7S R18E S1; 3 Jul 1928;
\textit{Coster 10} (RM).
Just n. of N. Fk. Dry Creek on n-s trending ridge with sandstone outcroppings
ca. 8 mi. nw of Belfry. 4,060-4,280' elevation; T7S R22E S7 SW1/4; 5 Jun 1994;
\textit{Evert 26694} (RM).
SE of Belfry. T9S R23E; 15 Jun 1938; \textit{Shunk s.n.} (MONT-53767).
Ridge 1 mi. W of Belfry and 1/4 mi. N of Hwy. 308. 4,100-4,200' elevation;
T8S R22E S16 and S9; 6 Jun 1996; \textit{Evert 30967} (RM).
Bighorn Basin; Hollenbeck Road E of Mont Hwy. 72 and ca. 0.5-1 mi. S of Clarks
Fork Yellowstone River. 4,000-4,300' elevation; T9S R22E S9 E1/2 and S16 NE1/4;
22 May 2009; \textit{Hartman 88777} (MONTU).
Hollenbeck Road E of Mont. Hwy. 72 and S of Clarks Fork Yellowstone River.
4,000-4,300' elevation; T9S R22E S4 SE1/4 and S9 E1/2; 22 May 2009;
\textit{Hartman 88766} (NY).
East side of Long Draw. 4,100' elevation; T8S R23E S31; 20 May 1991;
\textit{Lesica 5294} (NY, MONTU).
Sage Creek Road (Pryor Mountain Road), ca. 7 air mi. SE of Bridger; South Fork
Bridger Creek, road to Depression Reservoir and to top of hogback to SW.
4,120-4,450' elevation; T7S R24E S32 N1/4; 23 May 2009;
\textit{Hartman 88842} (NY, MONTU).
Along Sage Creek Road, ca. 9.5 air mi. SE of Bridger. 4,540-4,740' elevation;
T7S R24E S21 SW1/4; 19 Jun 2008; \textit{Nelson 74468} (ISTC).
Bluewater Creek ca. 6 mi. E of Bridger. 4,200' elevation; T6S R24E S15;
14 Jun 1984; \textit{Lesica 3002} (MONTU).
South-facing slopes above Gypsum Creek ca. 1 mi. NW of Gyp. Springs.
4,700' elevation; T9S R27E S32; 21 May 1991; \textit{Lesica 5304} (NY, MONTU).
Southern end of Pryor Mountain range, ca. 1 mi. N of Gyp Springs Road. 12.7 air
mi. N of Lovell, WY. 4,773' elevation; 1454.81m elevation;
Lat 45º01'00"N, Long 108º27'00"W; 20 Jun 2004; \textit{Grady 67} (ISTC).
Red Desert area, 1 mi. NW of Gypsum Springs, Red Pryor Mtn. 1,643m elevation;
T9S R27E S29 SW 1/4; 5 Jun 1994; \textit{McCarthy 58} (MONT).
Tan-purple ridge E of Helt and Crooked Cr. Roads intersection, Red Pryor Mtn.
Quad. 1,565m elevation; 12 May 1992; \textit{Jacobsen 206} (MONT).
Along Crooked Creek Rd. ca. 13 mi. N of Lovell, WY. 4,300' elevation;
T9S R27E S33; 16 Jun 1990; \textit{Evert 18942} (RM).
Ridge east of Helt Rd. and Crooked Creek Rd. intersection. 4,716' elevation;
Lat 45º0.5243'N, Long 108º25.4114W; 15 Jun 2014;
\textit{Ratcliff and O'Kane, Jr. 48} (ISTC).
Slopes ca. 1 mi. N of Gyp. Springs. 4,800' elevation; T9S R27E S28 SW1/4;
10 Jun 1991; \textit{Lesica 5366} (MONTU).
Spring, NE 4.8 km on road, then SE 0.8 km on jeep road into canyon
below, Red Pryor Mtn. 1,368m elevation; T9S R27E S27 NW 1/4; 19 Jun 1994;
\textit{McCarthy 226} (MONT).
Southern footslopes of Pryor Mountains along Road 3085 west of Penney Peak.
1,524m elevation; Lat 45º03'22"N, Long 108º24'55"W; 1 Jul 1998;
\textit{O'Kane, Jr. and Prather-O'Kane 4510} (ISTC, MO).
Crooked Creek Road (BLM Road 1017) ca. 0.5 air mi. W of Demijohn Flat.
5,000-5,320' elevation; T9S R27E S9 SE1/4 and S10 SW1/4; 25 May 2009;
\textit{Hartman 88978} (MONTU, NY).
Near Gypsum Ck. 5,400' elevation; 1,645m elevation; T9S R27E; 15 may 1976;
\textit{Dorn 2547} (MONT).
Hills S. of Pryor Mtns., dry plains between scattered pines. 5,100' elevation;
14 Jun 2005; \textit{Hjalmarsson 5105} (MONT).
Wassin Canyon. 4,600' elevation; T8S R28E S24; 10 May 1983;
\textit{Lichvar 5586} (RM).
Medicine Creek Campground. 3,700' elevation; T8S R29E S16; 13 Jun 1983;
\textit{Lichvar 6093} (RM).
Barry's Landing Campground. 19 Jun 1982; \textit{Thompson 2300} (MONTU).
Big Horn Canyon. 3,500' elevation; 1,070m elevation; T8S R29E S31; 15 May 1976;
\textit{Dorn 2559} (MONT).
Bighorn Canyon Rd. 16 mi. North of Lovell, WY. 19 Jun 1982;
\textit{Thompson 2282} (MONTU).
Common on rocky limestone ridgetops on the east side of Crooked Creek ca. 10 mi.
NW of Lovell, WY. 4,500' elevation; T9S R28E S36; 18 Jun 1983;
\textit{Lesica 2607} (MONTU).
West of Yellow Hill. 4,800' elevation; T9S R28E S27; 14 May 1983;
\textit{Lichvar 5636} (RM).
Pryor Mountain Wild Horse Range; Sykes Ridge (BLM Road 1019) along spine E of
Big Coulee. 4,650-4,950' elevation; T9S R28E S28 and S21 SE1/4; 1 Jun 2009;
\textit{Hartman 89242} (MONTU).
Pryor Mountain National Wild Horse Range; on Sykes Ridge or BLM Road 1019 at
the state line, ca 37 air mi SE of Bridger. 4,640-4,720' elevation;
T9S R28E S33 NE1/4; 16 Jun 2008; \textit{Nelson 74315} (ISTC).
%%% IDAHO STATE SPECIMENS
  \textbf{Idaho:}
% Bannock
  \textbf{Bannock County:}
Soda Springs. 5,700' elevation; 18 Jun 1920;
\textit{Payson and Payson 1701} (RM, NY, MO).
Soda Springs. \textit{Christ 3400} (NY).
% Caribou
  \textbf{Caribou County:}
Crow Creek Road, 11.6 km (7.2 mi.) SW of the Wyoming border, 22.5 km (14 mi)
air distance southwest of downtown Afton. 6,560' elevation; 2,000m elevation;
Lat 42º35'10"N, Long 111º07'56"W; 23 Jun 2003;
\textit{Holmgren and Holmgren 14877} (UTC, NY, ISTC).
% Bear Lake
  \textbf{Bear Lake County:}
Georgetown Summit. 6,232' elevation; 11 May 1978;
\textit{Shultz and Shultz 2434} (UTC).
Preuss Range, Georgetown Canyon, 3.7 mi. E of Georgetown
(jct. with U.S. Hwy. 30N) 6,400' elevation; T11S R44E S4; 23 May 1971;
\textit{Holmgren and Holmgren 4782} (NY).
Near U.S. Hwy. 89, 3.8 mi. NE of Montpelier. 15 Jun 1981;
\textit{Rollins and Rollins 81317} (RM, NY, UC, F).
4 mi. E of Montpelier on Montpelier Canyon Road, just E of Home Canyon.
6,300' elevation; 14 Jul 1978; \textit{Shultz and Shultz 2765} (RM, UTC, IDS).
1/2 mi. W of Geneva Summit, off U.S. Hwy. 89. 24 Jun 1990;
\textit{Rollins and Rollins 8692} (RM, UTC, GH, NY).
One mi. SW of Geneva Summit. 8,000' elevation; 15 Jun 1981;
\textit{Rollins and Rollins 81319} (RM, NY, UC, US).
5.4 mi. N of Geneva. 15 Jun 1981;
\textit{Rollins and Rollins 81321} (RM, NY, US, UC).
Cache National Forest; B. M. Fish Haven Ridge. 8,000' elevation; T16S R43E S5;
26 May 1928; \textit{Averill A-8} (RM).
Top of hills east of Montpelier. 28 Jun 1930;
\textit{Davis s.n.} (IDS-30405, IDS-30406).
Between Geneva and Montpelier. 11 May 1987; \textit{Christ and Ward 7127} (NY).
8.5 mi. E of Montpelier. 6,800' elevation; 11 Jul 1965;
\textit{Mulligan and Crompton 3092} (NY).
% Oneida
  \textbf{Oneida County:}
Two-Mile Canyon. T14S R37E S32 SW1/4 SW1/4; 21 Jun 1992;
\textit{John 771} (UTC).
%%% WYOMING STATE SPECIMENS
  \textbf{Wyoming:}
  \textbf{Unknown County:}
Green River Mountains. \textit{Sheppard 26} (MO-1076929).
% Park
  \textbf{Park County:}
Bighorn Basin; on road between Silvertip and Elk Basin Oil Fields, SSE of Elk
Basin Oil Field; ca. 14.5 air mi. NW of Powell. 4,600' elevation;
T57N R100W S1; 21 Jun 1987; \textit{Nelson	13688} (NY).
Bighorn Basin; near the head of Spring Creek, ca. 6 air mi. S of Cody.
5,500' elevation; T52N R101W S31; 26 Jun 1983; \textit{Nelson 9944} (ISTC).
Bighorn Basin; north end of Oregon Basin, ca. 8 air mi. SE of Cody.
5,300-5,600' elevation; T52N R100W S8, S16, and S17; 25 May 1983;
\textit{Hartman and Hamann 14449} (ISTC).
Along W side of Oregon Basin Rd., ca. 1.5 mi. S of Hwy. 14, 16, 20.
5250' elevation; T52N R100W S15 and S16;	13 Jun 1990;
\textit{Evert 18832} (RM).
Bighorn Basin; Burlington Meeteetse Road, 16.5 mi. NE of intersection with
Wyo Hwy. 120. T51N R98W S14, S15; 24 May 1980;
\textit{Hartman with Dueholm 11146} (ISTC, UTC).
Bighorn Basin; on the divide between Gooseberry Creek and Renner Draw, ca.
1.5 air mi. S of Gooseberry Creek; ca. 11 air mi. S of Meeteetse.
6,400' elecation; T47N R100W S33; 27 Jun 1983; \textit{Nelson 9999} (ISTC).
% Big Horn
  \textbf{Big Horn County:}
Bighorn Basin, above Dry Creek, ca. 6 air mi. NNE of Cowley, ca. 8.5 air mi. NNW
of Lovell. 4,300' elevation; T58N R96W S34; 28 Jun 1987;
\textit{Nelson 14016} (UTC).
Sykes Mtn. 4,100' elevation; T57N R95W S13; 14 May 1983;
\textit{Lichvar 5619} (RM).
Sykes Mtn. 4,000' elevation; T57N R95W S13; 10 Jun 1983;
\textit{Lichvar 6037} (RM).
Big Horn Mountains; John Blue Canyon Road, ca. 5.9 mi. NNE of US 14A.
6,000' elevation; T57N R94W S12 and S13; 28 May 1980;
\textit{Hartman 11367} (RM, GH).
5.5 mi. NE of Kane. 13 Jun 1964; \textit{Wight 68} (RM).
Big Horn Mountains; Cottonwood Canyon, ca. 17 air mi. E of Lovell.
5,400' elevation; T56N R93W S4; 10 Jun 1980; \textit{Nelson 5367} (RM).
Bighorn Basin, NW end of Sheep Mountain, near Ribbon Canyon. 4,200' elevation;
T54N R94W S20; 28 May 1980; \textit{Hartman and Dueholm 11349} (UTC).
3-4 air mi. N of Greybull. T53N R93W S22; 22 May 1983;
\textit{Hartman and Hamann 14275} (NY).
Red Gluch, south of Trapper Canyon. 7 May 1926; \textit{Finley 10} (RM).
West flank of Bighorn Mountains; west side of Medicine Lodge Creek Canyon, ca.
7 air mi. NE of Hyattville. 5,200' elevation; T51N R89W S10 NE1/4 NW1/4;
22 Jun 1989; \textit{Marriott 11018} (RM).
Mouth of Dry Medicine Lodge Canyon (S9 and S16) and ridge above Medicine Lodge
Canyon (S10 and S15). 5,000' elevation; T50N R89W S9, S16, S10, and S15;
26 May 1980; \textit{Dueholm and Hartman 9487} (RM).
4 mi. SE of Hyattsville. 5,200' elevation; T49N R89W S22 S1/2; 21 May 1979;
\textit{Lichvar 1660} (RM).
% Wakashie
  \textbf{Wakashie County:}
Steep slope of red soil, near Wyo. Hwy. 436, ca. 5 mi. SE of Ten Sleep.
23 Jun 1981; \textit{Rollins and Rollins 81392} (NY).
Big Cedar Ridge, W of BLM Road 1411, ca. 24 air mi. SE of Worland.
5,300' elevation; T45N R89W S28; 9 Jul 2011; \textit{Heidel 3540b} (RM).
9 mi. WSW of Neiber. 4,500' elevation; T45N R95W S3 SW1/4; 6 Jun 1995;
\textit{Dorn 5924} (RM).
% Hot Springs
  \textbf{Hot Springs County:}
Wind River Valley. 29 May 1860; \textit{Hayden s.n.} (MO-1923237).
Gravelly hills in Wind River Valley. 5,500' elevation; 18 May 1860;
\textit{Hayden s.n.} (MO-137159).
Near Long Creek Tavern, Hwy. 287 approaching Wind River Valley from NW.
14 Jul 1942; \textit{Cantelow s.n.} (CAS).
Bighorn Basin; west side of Hillberry Rim, ca. 0.8 mi. E of WY Highway 120.
5,600' elevation; T47N R99W S13 SW4 of NE4; 1 Jul 1998; \textit{Welp 7860} (RM).
Ridge along north side of Grass Creek Road, ca. 6 mi. W of Wyo Hwy. 120.
5,800' elevation; T46N R99W S14; 27 Jun 1993; \textit{Evert 25349} (RM).
Along east side of Murphy Draw Road, ca. 3/4 mi. N of Wyo Hwy.
431. 5,200' elevation; T47N R97W S28; 11 Jun 1995; \textit{Evert 28855} (RM).
Off Wyo Hwy. 120 on Red Ridge, ca. 5.5 air mi. E of Grass Creek Post Office;
ca. 26.5 mi. NW of Thermopolis. 5,500' elevation; T46N R97W S19; 29 Jun 1983;
\textit{Nelson 10164} (NY).
Ca. 5 air mi. SE of Grass Creek; ca. 25 air mi. NW of Thermopolis.
5,500' elevation; T45N R98W S1; 3 Jun 1981; \textit{Nelson 7463} (ISTC).
North slope of Ilo Ridge, ca. 1 mi. S of Wyo Hwy. 171 and 1.2 mi.
S of Grass Creek.	5,500' elevation; T45N R98W S1 NW4 of SW4 of NW4,
T45N R98W S2 SW4 of NE4; 29 Jun 1998;
\textit{Welp 7851, 7855} (RM).
Ca. 11.5 air mi. SE of Grass Creek; ca. 19 air mi. NW of
Thermopolis on the Cottonwood Creek Road. 5,000' elevation; T45N R97W S14;
3 Jun 1981; \textit{Nelson 7505} (NY, UTC).
E Foothills Absaroka Mountains; Mount 7049, ca. 1.25 air mi. NW of summit of
Adam Weiss Peak, ca. 33.5 air mi. NW of Thermopolis. 6,600-6,800' elevation;
T45N R99W S6	SE4NW4; 9 Jul 1992; \textit{Fertig 12949a} (RM).
S Bighorn Basin; ridge on western rim of Wagonhound Bench, ca. 27 air mi. NW of
Themopolis. 5,740-6,060' elevation; T44N R99W S1, S2, S3; 10 Jul 1992;
\textit{Fertig 12980} (RM).
Foothills of the Absaroka and Owl Creek Mountains; west end of Padlock Rim, ca.
3 air mi. SE of Hamilton Dome, ca. 17 air mi. WNW of Thermopolis.
5,300' elevation; T44N R97W S29, S31 NE4, and S32 NW4; 6 Jul 1992;
\textit{Fertig 12896} (RM).
S. Big Horn Basin; Sand Draw Ridge, ca. 13.5 air mi. NNW of Thermopolis.
4,760-4,900' elevation; T44N R96W S3 SE4 and S10 NE4; 3 Jul 1992;
\textit{Fertig 12855} (RM).
Bighorn Basin; ca. 2 mi. W of Gebo (abandoned townsite), ca. 12 mi. N of
Thermopolis. 4,500' elevation; T44N R95W S9; 7 Jun 1995;
\textit{Evert 28732} (RM).
Ca. 1 mi. W of Gebo; ca. 10.5 air mi. NNW of Thermopolis. 4,500' elevation;
T44N R95W S10; 18 Jun 1987; \textit{Nelson 13540} (UTC).
Ca. 10 air mi. E of Thermopolis, chugwater cliffs and adjacent plains;
T42N R93W S3 and S10; 28 May 1981; \textit{Hartman and Dueholm 12821} (UTC).
Bridger Creek Road, 0.25 mi. N of Fremont County line. 5,740' elevation;
T41N R90W S34 SW, SE; 	28 Jun 1981; \textit{Martin 1575} (RM).
20 mi. SW of Grass Creek. 14 Jul 1964; \textit{Despain 28} (RM).
Red Canyon. 5,200' elevation; T42N R92W; 21 May 1979; \textit{Lichvar 1671} (NY).
Owl Creek Mountains; South Fork Owl Creek, north rim of canyon below Anchor Dam,
ca. 30 air mi. W of Thermopolis. 6,800' elevation; T43N R100W S26	SE1/4 NE1/4;
21 Jun 1991; \textit{Marriott 11356} (RM).
% Fremont
  \textbf{Fremont County:}
Wind River Indian Reservation, south flank of the Owl Creek Mountains.
6,400' elevation; T7N R1W S23; 24 Jun 1982; \textit{Lichvar 5177} (RM).
Boysen Dam, on the Wind River. 4,800' elevation; 26 Jun 1960;
\textit{Porter and Porter 8205} (RM).
Wind River Basin; slopes on west side of Boysen Reservoir, ca. 0.25 mi. W of
Power House at Boysen Dam. 5,200' elevation; T5N R6E S8	SE1/4 SW1/4 SE1/4;
12 Jun 1993; \textit{Fertig 13849} (RM).
Cedar Ridge and Barren Hills to the N. 5,600-6,000' elevation; T39N R92W S9,
S16, and S21; 28 Jun 1981; \textit{Hartman and Dueholm 13232} (RM).
Bridger Mountains; E end of Copper Mountain N of Point of Mountain Road, ca. 16
air mi. N of US Hwy. 20-26. 6,200-6,600' elevation; T40N R91W S32	S1/2 NW1/4;
6 Jun 1996; \textit{Fertig 16544} (RM).
Cottonwood creek, 7 air mi. NE of Lysite. 5,400-5,600' elevation; T39N R90W S14
and S15; 28 May 1985; \textit{Hartman 19963} (RM).
Lysite Badlands, Badlands draining north into Alkali Creek; ca. 2.5 air mi. W of
Lysite. 5,350' elevation; T38N R91W S15	N1/2; 17 Jun 1986;
\textit{Marriott 10133} (RM).
Pony CR \#1 gas line site just W of Moneta-Lysite rd, 3.3 rd. mi. N of Moneta.
5,800' elevation; T37N R91W S2	E1/2; 16 Jun 1986; \textit{Marriott 10089} (RM).
Edge of (badlands) Moneta Hills; ca. 3.75 air mi. NNE of Moneta.
5,630-5,730' elevation; T37N R91W S1; 20 May 1986; \textit{Haines 5939} (RM).
E of Moneta-Lysite Rd. ca. 5 air mi. ENE of Moneta, badlands and ridge to south.
5,650' elevation; T37N R90W S9	E1/2; 18 Jun 1986; \textit{Marriott 10170} (RM).
Burma Road, ca. 5.8 air mi. N of jct. US Hwy. 26 and Wyo Hwy. 789 in Riverton.
5,000-5,200' elevation; T2N R4E S34; \textit{Hartman and Haines 20054} (RM).
Flats, across US Hwy. 26 from Paradise Valley Road, ca. 6 air mi. WNW of
Riverton. 5,300' elevation; T1N R3E S11 and S14; 15 Jun 1986;
\textit{Haines and Haines 6480} (RM, GH).
Steep slopes below sandstone ledges along the Oil Springs Road, 7 mi. N of Hwy.
130. 5,700' elevation; T34N R95W S30; 30 May 1985; \textit{Scott 4159} (RM).
Gas hills mining district; ca. 2.5 air mi. E of the intersection of Wyo 136 and
Ore Road (Main Road through gas hills mining district). 6,480' elevation;
T33N R90W S24; 21 Jun 1985; \textit{Haines 4281} (RM).
On the eroded badlands hills near Lysite. 5,500' elevation; 10 Jul 1951;
\textit{Porter 5741} (RM).
% Natrona
  \textbf{Natrona County:}
SE Bighorn Mountains, north slope of Cedar Ridge, S of Badwater Creek, ca, 1.5-2
air mi. NW of Badwater. 6,200-6,700’ elevation; T39N R89W S24; 13 Jun 1993;
\textit{Fertig 13877} (RM).
South Fork of Sand Creek, ca. 7.25 air mi. E of Lost Cabin.
5,920-6,000’ elevation; T38N R89W S11 and S14; 1 Jun 1986;
\textit{Haines and Haines 6063} (RM, GH).
Cedar Ridge, ca. 13 air mi. NNW of Hiland. 6,100-6,825’ elevation;
T38N R88W S5 and S6, T39N R88W S31 and S32; 10 Jul 1985;
\textit{Haines 4792} (RM).
SE Bighorn Mountains, southeast slope of Cedar Ridge, ca. 2.5 air mi. S of
Badwater. 6,500-6,740’ elevation; T38N R88W S8 and S9; 13 Jun 1993;
\textit{Fertig 13861} (RM).
Southern Powder River Basin, ca. 6 air mi. NW of Arminto. 6,230’ elevation;
T38N R87W S19 SE4 and S30 NE4; 6 Jul 1993;
\textit{Hartman and Nelson 40363} (RM, MONTU).
Bad Water. 6 Jun 1910; \textit{Nelson 9403} (RM).
5 air mi. NNE of Arminto. 6,300’ elevation; T38N R87W S24; 12 Jul 1979;
\textit{Hartman 10048} (RM).
SE foothills Big Horn Mountains, draw due W of E-K Creek and ridge on east side
of creek, ca. 5.5 air mi. NNE of Arminto. 6,500’ elevation; T38N R87W S13 SE1/4;
8 Jun 1992; \textit{Fertig 12639} (RM).
E-K Creek, ca. 7.5 air mi. NNE of Arminto. 6,520’ elevation; T38N R86W S7;
1 Jun 1986; \textit{Haines and Haines 6012} (RM, GH).
Southern Powder River Basin, along Baker Cabin or County Road 108 between North
Fork Cottonwood Creek and Gray Wall, ca. 12 air mi. NE of Arminto, ca. 51 air
mi. NW of Casper. 6,340-6,490’ elevation; T39N R86W S25 W2; 23 May 1994;
\textit{Nelson 30628} (RM).
Ca. 24 air mi. N Powder River. 6,400’ elevation; T39N R85W S4 and S9;
3 Jul 1979; \textit{Dueholm 7727} (RM).
6,500' elevation; T40N R85W S11 S1/2; 43º27'N 107º01' W; 29 May 1999;
\textit{Dorn 7931} (RM).
NW Rattlesnake Hills, slopes adj. to Poison Spider Road, ca. 1 mi. SE of jct.
with Gas Hills Road. 6,800-7,000’ elevation; T34N R88W S32 E1/2; 15 Jun 1993;
\textit{Fertig 13926} (RM).
Wallace Creek; 29 Jul 1894; \textit{Nelson 674} (NY).
Southern Powder River Basin; Blue Canyon at south end of Pine Mountain, ca. 28
air mi. WNW of Casper.	5,750-5,900' elevation; T34N R84W S13	N1/2 and
S24 NE1/4; 24 Jun 1993; \textit{Hartman 38283} (RM, UC).
Ca. 14.5 mi. SE Powder River. 5,700’ elevation; T33N R83W S33; 5 Jul 1979;
\textit{Dueholm 7850} (RM).
6,400; T30N R83W S19 N1/2; 20 Jun 1979; \textit{Dorn 3227} (RM).
33 mi. SW of Casper. 5,900’ elevation; 2 Jun 1963;
\textit{Mulligan and Mosquin 2791} (RSA-POM).
Satanka Formation, 4 mi. W of Alcova. 9 May 1948;
\textit{Porter 4427} (RM, RSA-POM).
Near Alcova, WY.  5,300’ elevation; 10 May 1980; \textit{Tresler 484} (RM).
North Platte River Basin, slopes on N side of Platte River, 0.5 mi. S of WY Hwy.
220, ca. 1.25 air mi. NE of Gray Reef Dam. 5,500’ elevation;
T30N R82W S8 NE1/4 SW1/4 and SE1/4 NW1/4; 5 Jun 1993;
\textit{Fertig 13791} (RM).
Slopes on west side of Platte River, ca. 21 air mi. SW of Casper.
5,400’ elevation; T31N R82W S23 W1/2; 5 Jun 1993; \textit{Fertig 13790} (RM).
Satanka formation, about 10 mi. SE of Alcova. 28 Jun 1950;
\textit{Porter 5400} (RM, NY).
One mi. N of Alcova. 21 May 1979; \textit{Rollins and Rollins 7919} (NY).
Near top of pass, between Alcova and Independence Rock, Wyo Hwy. 220.
21 May 1979; \textit{Rollins and Rollins 7920} (NY).
North of Casper Mt., 20.5 mi. NE of Alcova on Wyo. Hwy. 220. 21 May 1979;
\textit{Rollins 7914} (RM, NY, UC, US, MO).
13 mi. SW of Casper. 5,200’ elevation; 4 Jun 1963;
\textit{Mulligan and Mosquin 2792} (UC, UTC).
N. Laramie Range, south slope of Casper Mountain along ridge adj. to tributary
of Little Red Creek and Casper Mountain Loop Rd. 7,000-7,200’ elevation;
T32N R79W S32; 9 Jul 1993; \textit{Fertig 14050} (RM).
Along road cut on road to Muddy Mountain. 7,000’ elevation; T31N R79W S4;
20 Jul 1987; \textit{Nelson and Nelson 8681} (RM).
29 mi. S of Casper. 8 Jul 1965; \textit{Mulligan and Crompton 3083} (RM).
Twin Buttes. 6,500’ elevation; T30N R80W S34 NW1/4; 9 Jun 1981;
\textit{Lichvar 4334} (RM).
Bates Hole. 20 Jun 1920; \textit{Payson and Payson 4781} (RM, MO).
North Platte River Basin; low ridges ca. 4.5 air mi. E of Alcova-Seminoe Scenic
Byway, ca. 1.25 air mi. N of Carbon County line. 7,000’ elevation;
T29N R82W S34 NE1/4; 11 Jul 1993; \textit{Fertig 14070} (RM).
East rim of Bates Hole, "Castle Rock" cliffs by Rimo Point ca. 2 mi. E of Wyo
Hwy. 487, ca. 3 mi. N of jct. with Hwy. 77. 7,350’ elevation; T29N R79W S3
NW1/4 SW1/4; 21 Jun 1997; \textit{Fertig 17550} (RM).
West slope Laramie Range, divide between Mud Springs Draw and Chalk Creek,
ca. 1.2 mi. S of the confluence of Chalk and Bates creeks, ca. 8.5 air mi. E of
Wyo Hwy. 487. 6,940-7,000' elevation; T29N R79W S3 NW1/4 SW1/4; 20 Jun 1997;
\textit{Fertig 17547} (RM).
Bates Creek. 5 Jul 1901; \textit{Goodding s.n.} (RM).
% Sublette
  \textbf{Sublette County:}
Wyoming Range; Ridge on E side of LaBarge Creek just N of confluence of
Packsaddle Creek, ca. 1 mi. E of Scaler Cabin (Guard Station).
8,000’ elevation; T28N R115W S29 E1/2 of NW/4; 10 Aug 1995;
\textit{Fertig 16258} (RM).
Formation just NW of Squaw Teat, ca. 4 air mi. S of Elkhorn Junction.
7,200-7,537' elevation; T28N R104W S27 NW1/4; 20 Jun 1994;
\textit{Cramer 960} (RM).
North rim of Dry Basin just off Calpet Road, ca. 10 air mi. SW of Big Piney.
7,480-7,600’ elevation; T29N R113W S27 E1/2; 28 Jun 1993;
\textit{Nelson and Nelson 26618, 26591} (RM).
Cretaceous Mountain; A branch of Deloney Canyon above Dry Basin, ca. 13.5 air
mi. SW of Big Piney. 7,800-8,160’ elevation; T28N R114W S12 NE1/4; 23 Jun 1993;
\textit{Nelson 26208} (RM).
Fogarty Canyon at the south end of Cretaceous Mountain, ca. 14 air mi. SW of
Big Piney. 7,480-7,680’ elevation; T28N R113W S18 SW1/4; 28 Jun 1993;
\textit{Nelson and Nelson 26681} (RM).
North end of Big Mesa above Dry Piney Creek, ca. 13.5 air mi. SW of Big Piney.
7,680-7,930' elevation; T28N R113W S28	SE1/4; 28 Jun 1993;
\textit{Nelson and Nelson 26706} (RM).
Lower end of Wildcat Canyon, ca. 3 air mi. WSW of Clara Birds Nipple
(Bird Nipple); ca. 15.5 air mi. SSW of Big Piney. 7,150’ elevation;
T27N R113W S10 SE1/4; 28 Jun 1993; \textit{Nelson and Nelson 26777} (RM).
Saddle ridge. 7,400’ elevation; T27N R113W S8 NE1/4; 7 Jun 1993;
\textit{Kass and Kass 3716} (RM).
Hogsback Ridge Area; northeast side of the Hogsback, ca. 18 air mi. SW of
Big Piney. 7,900-8,600’ elevation; T27N R113W S7 SC; 25 Aug 1993;
\textit{Hartman 44960} (RM).
East side of Hogsback Ridge above Calpet, ca. 7.5 air mi. W of La Barge.
7,600-8,300’ elevation; T27N R113W S19 S1/4; 12 Jun 1994;
\textit{Cramer 492} (RM).
Green River Basin; Ca. 11.5 air mi. SW of Boulder. 7,200-7,300’ elevation;
T31N R109W S19 SW1/4; 7 Jul 1995; \textit{Cramer 7446} (RM).
North side of Ross Butte overlooking New Fork River, ca. 10 air mi. E of
Big Piney. 6,800-7,480’ elevation; T30N R110W S13 SE1/4 and S14 W1/4;
11 Jun 1994; \textit{Cramer 310} (RM).
North end of Ross Butte, ca. 0.5 mi. S of the New Fork River and ca. 20 air mi.
S of Pinedale. 7,100-7,460’ elevation; T30N R110W S13 N1/2 of SW4 and S14 N1/2
of SE1/4; 9 Jul 1995; \textit{Fertig 15910} (RM).
Badlands at base of east slope of southern lobe of Ross Butte, ca. 1 mi. E of
the New Fork River, ca. 3.2 mi. ENE of confluence of the New Fork and Green
rivers. 7,300’ elevation; T30N R110W S24 NW1/4 of SW1/4 of NW1/4; 30 May 1997;
\textit{Fertig 17415} (RM).
North Alkali Draw, ca. 15 air mi. E of Big Piney. 7,200’ elevation;
T30N R109W S33 SE1/4; 25 Jun 1995; \textit{Cramer and Hartman 6979} (RM).
Ca. 1.5 air mi. E of Marbleton. 6,900-7,000’ elevation; T30N R111W S28 NW1/4;
21 Jun 1995; \textit{Cramer and Kellett 6545} (RM).
Ca. 5 air mi. ESE of Big Piney. 6,900-7,000’ elevation; T29N R111W S11 E1/4;
8 Jul 1995; \textit{Cramer 7472} (RM).
Ca. 9 air mi. SW of Ross Butte. 7,100-7,200’ elevation; T29N R110W S30;
25 Jun 1995; \textit{Hartman 51227} (RM).
Lower part of Chapel Canyon, ca. 11 air mi. NE of La Barge.
6,850-7,130’ elevation; T28N R111W S27 W1/4; 12 Jun 1994;
\textit{Cramer 403} (RM).
Ca. 4 mi. N of LaBarge on crest of ridge, Bird Canyon. 2,040m elevation;
T27N R112W S15; 6 Jun 1993; \textit{Kass and Kass 3706} (RM).
Bess Canyon. 6,800’ elevation; T27N R112W S22 SE1/4; 15 Jun 1993;
\textit{Kass 3757} (RM).
20 mi. W of Big Piney. 9 Jul 1922;
\textit{Payson and Payson 2618} (RM, NY, RSA-POM, MO, F, UC).
Red Canyon, ca. 18.5 air mi. WNW of Big Piney. 8,150-8,200’ elevation;
T31N R114W S15 W1/4; 22 Jul 1995; \textit{Cramer and Kellett 9011b} (RM).
% Lincoln
  \textbf{Lincoln County:}
Cokeville. 11 Jun 1898; \textit{Nelson 4637} (RM, UC).
Southern Salt River Range and Vicinity: Snow Hollow, ca 1.5 air mi E of Idaho,
ca. 5.5 air mi SW of Cokeville; ca. 29 air mi NW of Kemmerer.
6800-6900' elevation; T24N R120W S35 NW1/4 S35; 8 Jul 1995;
\textit{Nelson and Refsdal 36568} (RM).
La Barge Creek. 7,200' elevation; T27N R114W S32	NW1/4; 27 May 1981;
\textit{Lichvar 4207} (RM).
Cretaceous Mountain / Hogsback Ridge Area; southeastern corner of Hogsback
Ridge, ca. 22 air mi. SW of Big Piney. 7,200-7,600'elevation; T26N R113W S8
NW1/4; 13 Aug 1993; \textit{Hartman 43802} (RM).
Green River Basin; ca. 2 air mi. E of La Barge. 7,200' elevation;
T26N R112W S9 NE1/4; 8 Jul 1995; \textit{Cramer 7519} (RM).
LaBarge Creek, on slope on northeast side of drainage, ca. 7.5 air mi.
WSW of LaBarge. 7,000’ elevation; T26N R113W S18 NW1/4 SW1/4; 18 Jun 1988;
\textit{Marriott and Horning 10827} (RM);
7,300’ elevation; T26N R113W S19 NE1/4 NW1/4; 18 Jun 1988;
\textit{Marriott and Horning 10828} (RM).
Ca. 5.9 mi. W of Hwy. 189, on LaBarge Cr. Road. 6,200' elevation;
T26N R113W S20; 5 Jun 1987; \textit{Atwood 12830} (NY).
Ca. 3.5 mi. SE of LaBarge. 6,800' elevation; T26N R112W S21 SE1/4 NE1/4;
1 Jul 1993; \textit{Kass 3794} (RM).
Green River Basin; ca. 15.5 air mi. W of La Barge at the south end of
Fontenelle Hogbacks. 8,160-8,280' elevation; T26N R115W S28	SE1/4; 3 Aug 1995;
\textit{Cramer, Kellett, and Laster 10269} (RM).
Shale cliff near the Green River, 6 mi. S of Labarge. 6,500' elevation;
19 Jul 1939; \textit{Rollins and Munoz 2867} (CAS).
Green River Basin; ca. 7 air mi. W of La Barge. 7,000-7,100' elevation;
T25N R112W S1 NW1/4 and S2 NE1/4, T25N R112W S35 SW1/4 and S36 SE1/4;
22 May 1994; \textit{Hartman, Cramer, and Refsdal 45172} (RM, NY).
Green River Basin; ca. 15.5 air mi. SSW of La Barge. 6,900' elevation;
T24N R114W S14 W1/2; 21 Jun 1995; \textit{Cramer and Kellett 6602} (RM).
Green River Basin; south end of Fontenelle Reservoir, ca. 2 air mi. NW of
Fontenelle Dam. 6,550' elevation; T24N R112W S23 SE1/4; 5 Jul 1995;
\textit{Cramer 7312} (RM).
Green River Basin; Slate Creek Butte, ca. 5 air mi. SW of Fontenelle Dam.
6,750' elevation; T23N R112W S14 W1/2; 21 Jun 1995;
\textit{Cramer and Kellett 6653} (RM).
Green River Basin; ca. 13 air mi. N of Opal. 6,720-6,760' elevation;
T23N R113W S20 SE1/4; 11 Jul 1995; \textit{Cramer and Kellett 7754} (RM).
Overthrust Belt; Hams Fork Plateau on south side of Robinson Creek Canyon, ca.
2.5 air mi. W of Kemmerer Reservoir, ca. 13 air mi. NW of Kemmerer, ca. 1 mi. N
of Hams Fork Road (near emigrant graves). 7,500' elevation;
T23N R117W S29 NW1/4 of SE1/4; 2 Jul 1995; \textit{Fertig 16740} (RM).
Basins and Mountains of Southwest Wyoming; ca. 4.5 air mi. SW of Opal; 4.7 mi. S
on Wagon Wheel Road off of US Hwy. 30; 6,680-6,720' elevation;
T20N R115W S14 E1/2; 07/01/1995; \textit{Refsdal and Refsdal 4755} (RM).
Basins and Mountains of Southwest Wyoming; ca. 12 air mi. SE of Opal, 18.3 mi. E
of jct. US Hwy. 30 / Wyo Hwy. 240. 6,400-6,500' elevation; T20N R112W S28 E1/2;
30 Jun 1995; \textit{Refsdal 4638} (RM, NY).
Basins and Mountains of Southwest Wyoming; benchland between Zieglers Wash and
Dry Muddy Creek ca. 13 air mi. WNW of Granger, ca. 22 air mi. SE of Kemmerer.
6,590-6,680' elevation; T20N R113W S32 SE1/4 and S33 SW1/4; 3 Jul 1995;
\textit{Nelson and Refsdal 36196} (RM).
Ca. 9 air mi. WNW of Granger, ca. 24.5 air mi. SE of Kemmerer.
6,460-6,500' elevation; T19N R113W S12 SW1/4; 3 Jul 1995;
\textit{Nelson and Refsdal 36072} (RM).
Ca. 9 air mi. WNW of Granger, ca. 24.5 air mi. SE of Kemmerer.
6,460-6,500' elevation; T19N R113W S12 SW1/4; 3 Jul 1995;
\textit{Nelson and Refsdal 36073} (RM).
Little Muddy Creek, 5 mi. W of US 189, 14 mi. SW of Kemmerer. 7,000' elevation;
T19N R117W S29; 17 Jul 1982; \textit{Atkins, Neely, and Carpenter 8238} (UTC).
14 mi. S of Kemmerer. 6,500' elevation; T19N R116W S33 E1/2; 27 May 1994;
\textit{Dorn 5592} (RM, MO).
% Uinta
  \textbf{Uinta County:}
Overthrust Belt; south slopes of Hanks Hill at north end of Woodruff Narrows
Reservoir, ca. 1 mi. E of the WY-UT state line, ca. 10 air mi. N of Evanston.
6,500-6,760' elevation; T18N R120W S29 SW1/4 SW1/4, S30 SE1/4, S31 N1/2 of NE14,
and S32 NW1/4 of NW1/4; 22 May 1996; \textit{Fertig 16460} (RM).
Ca. 16 air mi. N of Evanston, 8.5 mi. E on County Road 101.
6,460-6,750' elevation; T18N R120W S32 NW1/4, S31 NE1/4, S29 SE1/4, and
S28 SW1/4; 22 Jun 1995; \textit{Refsdal 4228} (RM).
18.5 mi. SSW of Diamondville. 6,650’ elevation; 11 Jul 1965;
\textit{Mulligan and Crompton 3095} (MONT).
Basins and Mountains of Southwest Wyoming; ca. 6 air mi. SE of Cumberland Gap,
ca. 14.2 road mi. N of I-80 on east side of Wyo Hwy. 412.
6,700-6,860' elevation; T18N R116W S28 W1/2; 19 May 1994;
\textit{Refsdal and Atwood 200, 201, 202} (RM).
Overthrust Belt; SW end of the "Carter Cedars" along Wyo Hwy. 412,
ca. 4 air mi. NW of Carter, ca. 6.5 mi. E of US Hwy. 189. 6,800' elevation;
T17N R116W S2 NE1/4 of SW1/4; 18 Jun 1997; \textit{Fertig 17520, 17521} (RM).
US 189, 8.7 road mi. NE of I-80. 7,000' elevation; T17N R117W S34; 6 Jul 1983;
\textit{Hartman 15736} (RM).
6 mi. N Ft. Bridger. 6,500' elevation; 13 Jun 1938; \textit{Rollins 2316} (RM).
S of Carter. 7,000' elevation; T17N R115W S34 NE1/4; 4 Jun 1980;
\textit{Lichvar 2777} (RM).
2 mi. W of Fort Bridger. 6,700’ elevation; 7 Jul 1965;
\textit{Mulligan and Crompton 3079} (MONTU).
2 mi. W of Fort Bridger. 6,700’ elevation; 21 Jul 1963;
\textit{Mulligan and Crompton 2785} (UTC).
Ca. 5 mi. SSW of Carter. 6,600' elevation; T16N R116W S13; 12 Jun 1980;
\textit{Lichvar 2866} (RM).
2 mi. W of Fort Bridger. 2,134m; 41º19'51"N, 110º25'10"W; 3 Jun 1996;
\textit{O'Kane 3785} (ISTC).
Foothills of Bridger Butte, 3 mi. W Ft. Bridger. 6,500’ elevation; 24 Jun 1938;
\textit{Rollins 2387} (NY).
Fort Bridger, Wyoming Territory. July 1873, \textit{Porter 10462} (NY);
\textit{Porter s.n.} (NY, NY, F).
Fort Bridger. 9 Jun 1898; \textit{Nelson 4602} (RM, F).
3 mi. W of Ft. Bridger, topotype.	7,000' elevation; T16N R116W S35; 24 May 1979;
\textit{Lichvar 1704} (RM).
About 3 mi. W of Ft. Bridger. 7,000' elevation;	7 Jul 1977;
\textit{Dorn 2974} (RM).
28 mi. W SW of Green River. 6,625’ elevation; 11 Jul 1965;
\textit{Mulligan and Crompton 3096} (NY).
Basins and Mountains of Southwest Wyoming; Wildcat Butte between Church Butte
Road and I-80 at Sweetwater County, ca. 14.8 air mi. NE of Lyman, ca. 49 air mi.
ENE of Evanston.  6,820-6,980' elevation; T17N R112W S22 NW1/4; 18 Jun 1995;
\textit{Nelson and Refsdal 35212} (RM).
Ca. 12 air mi. NE of Lyman. 6,940-7,000' elevation; T16N R113W S1 S1/2;
22 Jun 1995; \textit{Refsdal 4313} (RM).
6 mi. E of Lyman. 6,600’ elevation; 19 Jun 1956; \textit{Porter 7005} (RM).
6 mi. E of Lyman. 3 Jun 1970; \textit{Rollins 79152} (NY, US).
9 mi. E NE of Fort Bridger. 6,500’ elevation; 7 Jul 1965;
\textit{Mulligan and Crompton 3080} (CAS).
Sandy ravine near Blacks Fork River, 3 mi. N of Lyman.	6,500' elevation;
10 Jun 1937; \textit{Rollins 1653} (RM, NY, UC, MO).
Lyman. 15 Jun 1932; \textit{Rollins 182} (RM, MO).
Along Leavitt Creek below the south end of Cottonwood Bench, ca. 7 air mi. ESE
of Mountain View, ca. 39.5 air mi. E of Evanston.	6,700-6,860' elevation;
T15N R114W S36; 18 Jun 1995; \textit{Nelson 35163} (RM).
8 air mi. SE of Mountainview, Leavitt Cr. 6,800' elevation;
T15N R114W S36 SE1/4; 30 Jun 1982; \textit{Goodrich and Atwood 17162} (RM, NY).
Flat above barren cliffs overlooking Laevitt Creek, 1 km (0.6 mi.) S of Wyo Hwy.
414, 11.5 km (7mi.) air distance east-southeast of Mountain View.
6,800' elevation; 2,075m; 41º13'45"N, 110º12'56"W; 23 May 1999;
\textit{Holmgren and Holmgren 13447} (ISTC, NY, UTC).
Grizzly Buttes, Canyonlands and erosional badlands near Mountainview.
6,800' elevation;	T14N and 15N R114W S2 and 36; 13 Jul 1973;
\textit{Hill 881} (RM).
Sage Creek Mountain, ca. 12 air mi. SE of Mountain View. 7,200' elevation;
T14N R113W S20 SE and S21 SW; 12 Jun 1981; \textit{Dueholm 11434} (RM, NY).
East end of Sage Creek Mountain, ca. 5.3 air mi. N of Lonetree.
8,200-8,420' elevation; T13N R113W S2 NW1/4, T14N R113W S35 S1/2; 23 Jul 1995;
\textit{Refsdal 5887} (RM).
North Slope Uinta Mountains; Hickey Mountain, ca. 5.5 air mi. NW of Lonetree.
7,480-8,000' elevation; T13N R114W S12; 22 Jun 1994;
\textit{Refsdal and Fertig 1047} (RM).
5 mi. N 25 dg W of Lonetree, E side Hickey Mtn. 7,800' elevation;
T13N R113W S18 SE1/4; 30 Jun 1982; \textit{Goodrich and Atwood 17171} (RM, NY).
Hickey Mountain, one mi. N of State Hwy. 414; 20 Jun 1986;
\textit{Rollins and Rollins 8670} (RM, NY, GH, UTC, MONTU).
Clay knolls and hillsides, County Road 290, 4 mi. W of Lonetree. 19 Jun 1986;
\textit{Rollins and Rollins 8666} (RM).
Uinta County Road 290, 3.7 air mi. W of Lonetree. 7,800' elevation;
T12N R114W S1; 7 Jul 1983; textit{Hartman 15766} (RM).
Cedar Mountain, ca. 3 air mi. NE of Lonetree; ca. 3.3 road mi. E of Cedar
Mountain Road from Wyo Hwy. 414, west flank of the mountain.
7,700-7,800' elevation; T13N R113W S22 W1/2 and S15 S1/2; 11 Jun 1994;
\textit{Refsdal and Lathrop 725} (RM).
SW side of Cedar Mtn. 7,700' elevation; T13N R113W S24 S1/2; 28 Jun 1999;
\textit{Dorn 7997} (RM).
Ca. 6.5 air mi. NNW of Lonetree, ca. 1.2 road mi. SW of Wyo Hwy. 414.
7,260-7,410' elevation; T13N R113W S26 W1/2; 22 Jun 1994;
\textit{Refsdal and Fertig 1023} (RM).
Ca. 2 air mi. E of Lonetree, ca. 8.0 road mi. E of the junction of County Road 1
with Wyo Hwy. 414 on south side of Wyo Hwy. 414. 7,400-7,600' elevation;
T12N R113W S1 SW1/4 and S2 SE1/4; 7 Jun 1994; \textit{Refsdal 517} (RM).
Hoop Lake Road (Uinta County Road 295), 4 air mi. S of Lonetree.
7,800' elevation; T12N R113W S21; 7 Jul 1983; \textit{Hartman 15761} (RM).
Ca. 4 air mi. S of Lonetree, just N of Utah on Hoop Lake Road.
7,900-8,000' elevation; T12N R113W S21 S1/2; 7 Jun 1994;
\textit{Refsdal 549} (RM).

