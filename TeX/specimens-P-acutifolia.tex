\section*{\textit{Physaria acutifolia}}

%%% IDAHO STATE SPECIMENS
  \textbf{Idaho:}
  \textbf{Bannock County:}
Soda Springs. 5,700' elevation; 18 Jun 1920;
\textit{Payson and Payson 1701} (RM, NY, MO).
Soda Springs. \textit{Christ 3400} (NY).
  \textbf{Caribou County:}
Crow Creek Road, 11.6 km (7.2 mi.) SW of the Wyoming border, 22.5 km (14 mi)
air distance southwest of downtown Afton. 6,560' elevation; 2,000m elevation;
Lat 42º35'10"N, Long 111º07'56"W; 23 Jun 2003;
\textit{Holmgren and Holmgren 14877} (UTC, NY, ISTC).
  \textbf{Bear Lake County:}
Georgetown Summit. 6,232' elevation; 11 May 1978;
\textit{Shultz and Shultz 2434} (UTC).
Preuss Range, Georgetown Canyon, 3.7 mi. E of Georgetown
(jct. with U.S. Hwy. 30N) 6,400' elevation; T11S R44E S4; 23 May 1971;
\textit{Holmgren and Holmgren 4782} (NY).
Near U.S. Hwy. 89, 3.8 mi. NE of Montpelier. 15 Jun 1981;
\textit{Rollins and Rollins 81317} (RM, NY, UC, F).
4 mi. E of Montpelier on Montpelier Canyon Road, just E of Home Canyon.
6,300' elevation; 14 Jul 1978; \textit{Shultz and Shultz 2765} (RM, UTC, IDS).
1/2 mi. W of Geneva Summit, off U.S. Hwy. 89. 24 Jun 1990;
\textit{Rollins and Rollins 8692} (RM, UTC, GH, NY).
One mi. SW of Geneva Summit. 8,000' elevation; 15 Jun 1981;
\textit{Rollins and Rollins 81319} (RM, NY, UC, US).
5.4 mi. N of Geneva. 15 Jun 1981;
\textit{Rollins and Rollins 81321} (RM, NY, US, UC).
Cache National Forest; B. M. Fish Haven Ridge. 8,000' elevation; T16S R43E S5;
26 May 1928; \textit{Averill A-8} (RM).
Top of hills east of Montpelier. 28 Jun 1930;
\textit{Davis s.n.} (IDS-30405, IDS-30406).
Between Geneva and Montpelier. 11 May 1987; \textit{Christ and Ward 7127} (NY).
8.5 mi. E of Montpelier. 6,800' elevation; 11 Jul 1965;
\textit{Mulligan and Crompton 3092} (NY).
  \textbf{Oneida County:}
Two-Mile Canyon. T14S R37E S32 SW1/4 SW1/4; 21 Jun 1992;
\textit{John 771} (UTC).
%%% WYOMING STATE SPECIMENS
  \textbf{Wyoming: }
  \textbf{Sublette County:}
Wyoming Range; Ridge on E side of LaBarge Creek just N of confluence of
Packsaddle Creek, ca. 1 mi. E of Scaler Cabin (Guard Station).
8,000’ elevation; T28N R115W S29 E1/2 of NW/4; 10 Aug 1995;
\textit{Fertig 16258} (RM).
Formation just NW of Squaw Teat, ca. 4 air mi. S of Elkhorn Junction.
7,200-7,537' elevation; T28N R104W S27 NW1/4; 20 Jun 1994;
\textit{Cramer 960} (RM).
North rim of Dry Basin just off Calpet Road, ca. 10 air mi. SW of Big Piney.
7,480-7,600’ elevation; T29N R113W S27 E1/2; 28 Jun 1993;
\textit{Nelson and Nelson 26618, 26591} (RM).
Cretaceous Mountain; A branch of Deloney Canyon above Dry Basin, ca. 13.5 air
mi. SW of Big Piney. 7,800-8,160’ elevation; T28N R114W S12 NE1/4; 23 Jun 1993;
\textit{Nelson 26208} (RM).
Fogarty Canyon at the south end of Cretaceous Mountain, ca. 14 air mi. SW of
Big Piney. 7,480-7,680’ elevation; T28N R113W S18 SW1/4; 28 Jun 1993;
\textit{Nelson and Nelson 26681} (RM).
North end of Big Mesa above Dry Piney Creek, ca. 13.5 air mi. SW of Big Piney.
7,680-7,930' elevation; T28N R113W S28	SE1/4; 28 Jun 1993;
\textit{Nelson and Nelson 26706} (RM).
Lower end of Wildcat Canyon, ca. 3 air mi. WSW of Clara Birds Nipple
(Bird Nipple); ca. 15.5 air mi. SSW of Big Piney. 7,150’ elevation;
T27N R113W S10 SE1/4; 28 Jun 1993; \textit{Nelson and Nelson 26777} (RM).
Saddle ridge. 7,400’ elevation; T27N R113W S8 NE1/4; 7 Jun 1993;
\textit{Kass and Kass 3716} (RM).
Hogsback Ridge Area; northeast side of the Hogsback, ca. 18 air mi. SW of
Big Piney. 7,900-8,600’ elevation; T27N R113W S7 SC; 25 Aug 1993;
\textit{Hartman 44960} (RM).
East side of Hogsback Ridge above Calpet, ca. 7.5 air mi. W of La Barge.
7,600-8,300’ elevation; T27N R113W S19 S1/4; 12 Jun 1994;
\textit{Cramer 492} (RM).
Green River Basin; Ca. 11.5 air mi. SW of Boulder. 7,200-7,300’ elevation;
T31N R109W S19 SW1/4; 7 Jul 1995; \textit{Cramer 7446} (RM).
North side of Ross Butte overlooking New Fork River, ca. 10 air mi. E of
Big Piney. 6,800-7,480’ elevation; T30N R110W S13 SE1/4 and S14 W1/4;
11 Jun 1994; \textit{Cramer 310} (RM).
North end of Ross Butte, ca. 0.5 mi. S of the New Fork River and ca. 20 air mi.
S of Pinedale. 7,100-7,460’ elevation; T30N R110W S13 N1/2 of SW4 and S14 N1/2
of SE1/4; 9 Jul 1995; \textit{Fertig 15910} (RM).
Badlands at base of east slope of southern lobe of Ross Butte, ca. 1 mi. E of
the New Fork River, ca. 3.2 mi. ENE of confluence of the New Fork and Green
rivers. 7,300’ elevation; T30N R110W S24 NW1/4 of SW1/4 of NW1/4; 30 May 1997;
\textit{Fertig 17415} (RM).
North Alkali Draw, ca. 15 air mi. E of Big Piney. 7,200’ elevation;
T30N R109W S33 SE1/4; 25 Jun 1995; \textit{Cramer and Hartman 6979} (RM).
Ca. 1.5 air mi. E of Marbleton. 6,900-7,000’ elevation; T30N R111W S28 NW1/4;
21 Jun 1995; \textit{Cramer and Kellett 6545} (RM).
Ca. 5 air mi. ESE of Big Piney. 6,900-7,000’ elevation; T29N R111W S11 E1/4;
8 Jul 1995; \textit{Cramer 7472} (RM).
Ca. 9 air mi. SW of Ross Butte. 7,100-7,200’ elevation; T29N R110W S30;
25 Jun 1995; \textit{Hartman 51227} (RM).
Lower part of Chapel Canyon, ca. 11 air mi. NE of La Barge.
6,850-7,130’ elevation; T28N R111W S27 W1/4; 12 Jun 1994;
\textit{Cramer 403} (RM).
Ca. 4 mi. N of LaBarge on crest of ridge, Bird Canyon. 2,040m elevation;
T27N R112W S15; 6 Jun 1993; \textit{Kass and Kass 3706} (RM).
Bess Canyon. 6,800’ elevation; T27N R112W S22 SE1/4; 15 Jun 1993;
\textit{Kass 3757} (RM).
20 mi. W of Big Piney. 9 Jul 1922;
\textit{Payson and Payson 2618} (RM, NY, RSA-POM, MO, F, UC).
Red Canyon, ca. 18.5 air mi. WNW of Big Piney. 8,150-8,200’ elevation;
T31N R114W S15 W1/4; 22 Jul 1995; \textit{Cramer and Kellett 9011b} (RM).
  \textbf{Lincoln County:}
Cokeville. 11 Jun 1898; \textit{Nelson 4637} (RM, UC).
Southern Salt River Range and Vicinity: Snow Hollow, ca 1.5 air mi E of Idaho,
ca. 5.5 air mi SW of Cokeville; ca. 29 air mi NW of Kemmerer.
6800-6900' elevation; T24N R120W S35 NW1/4 S35; 8 Jul 1995;
\textit{Nelson and Refsdal 36568} (RM).
La Barge Creek. 7,200' elevation; T27N R114W S32	NW1/4; 27 May 1981;
\textit{Lichvar 4207} (RM).
Cretaceous Mountain / Hogsback Ridge Area; southeastern corner of Hogsback
Ridge, ca. 22 air mi. SW of Big Piney. 7,200-7,600'elevation; T26N R113W S8
NW1/4; 13 Aug 1993; \textit{Hartman 43802} (RM).
Green River Basin; ca. 2 air mi. E of La Barge. 7,200' elevation;
T26N R112W S9 NE1/4; 8 Jul 1995; \textit{Cramer 7519} (RM).
LaBarge Creek, on slope on northeast side of drainage, ca. 7.5 air mi.
WSW of LaBarge. 7,000’ elevation; T26N R113W S18 NW1/4 SW1/4; 18 Jun 1988;
\textit{Marriott and Horning 10827} (RM);
7,300’ elevation; T26N R113W S19 NE1/4 NW1/4; 18 Jun 1988;
\textit{Marriott and Horning 10828} (RM).
Ca. 5.9 mi. W of Hwy. 189, on LaBarge Cr. Road. 6,200' elevation;
T26N R113W S20; 5 Jun 1987; \textit{Atwood 12830} (NY).
Ca. 3.5 mi. SE of LaBarge. 6,800' elevation; T26N R112W S21 SE1/4 NE1/4;
1 Jul 1993; \textit{Kass 3794} (RM).
Green River Basin; ca. 15.5 air mi. W of La Barge at the south end of
Fontenelle Hogbacks. 8,160-8,280' elevation; T26N R115W S28	SE1/4; 3 Aug 1995;
\textit{Cramer, Kellett, and Laster 10269} (RM).
Shale cliff near the Green River, 6 mi. S of Labarge. 6,500' elevation;
19 Jul 1939; \textit{Rollins and Munoz 2867} (CAS).
Green River Basin; ca. 7 air mi. W of La Barge. 7,000-7,100' elevation;
T25N R112W S1 NW1/4 and S2 NE1/4, T25N R112W S35 SW1/4 and S36 SE1/4;
22 May 1994; \textit{Hartman, Cramer, and Refsdal 45172} (RM, NY).
Green River Basin; ca. 15.5 air mi. SSW of La Barge. 6,900' elevation;
T24N R114W S14 W1/2; 21 Jun 1995; \textit{Cramer and Kellett 6602} (RM).
Green River Basin; south end of Fontenelle Reservoir, ca. 2 air mi. NW of
Fontenelle Dam. 6,550' elevation; T24N R112W S23 SE1/4; 5 Jul 1995;
\textit{Cramer 7312} (RM).
Green River Basin; Slate Creek Butte, ca. 5 air mi. SW of Fontenelle Dam.
6,750' elevation; T23N R112W S14 W1/2; 21 Jun 1995;
\textit{Cramer and Kellett 6653} (RM).
Green River Basin; ca. 13 air mi. N of Opal. 6,720-6,760' elevation;
T23N R113W S20 SE1/4; 11 Jul 1995; \textit{Cramer and Kellett 7754} (RM).
Overthrust Belt; Hams Fork Plateau on south side of Robinson Creek Canyon, ca.
2.5 air mi. W of Kemmerer Reservoir, ca. 13 air mi. NW of Kemmerer, ca. 1 mi. N
of Hams Fork Road (near emigrant graves). 7,500' elevation;
T23N R117W S29 NW1/4 of SE1/4; 2 Jul 1995; \textit{Fertig 16740} (RM).
Basins and Mountains of Southwest Wyoming; ca. 4.5 air mi. SW of Opal; 4.7 mi. S
on Wagon Wheel Road off of US Hwy. 30; 6,680-6,720' elevation;
T20N R115W S14 E1/2; 07/01/1995; \textit{Refsdal and Refsdal 4755} (RM).
Basins and Mountains of Southwest Wyoming; ca. 12 air mi. SE of Opal, 18.3 mi. E
of jct. US Hwy. 30 / Wyo Hwy. 240. 6,400-6,500' elevation; T20N R112W S28 E1/2;
30 Jun 1995; \textit{Refsdal 4638} (RM, NY).
Basins and Mountains of Southwest Wyoming; benchland between Zieglers Wash and
Dry Muddy Creek ca. 13 air mi. WNW of Granger, ca. 22 air mi. SE of Kemmerer.
6,590-6,680' elevation; T20N R113W S32 SE1/4 and S33 SW1/4; 3 Jul 1995;
\textit{Nelson and Refsdal 36196} (RM).
Ca. 9 air mi. WNW of Granger, ca. 24.5 air mi. SE of Kemmerer.
6,460-6,500' elevation; T19N R113W S12 SW1/4; 3 Jul 1995;
\textit{Nelson and Refsdal 36072} (RM).
Ca. 9 air mi. WNW of Granger, ca. 24.5 air mi. SE of Kemmerer.
6,460-6,500' elevation; T19N R113W S12 SW1/4; 3 Jul 1995;
\textit{Nelson and Refsdal 36073} (RM).
Little Muddy Creek, 5 mi. W of US 189, 14 mi. SW of Kemmerer. 7,000' elevation;
T19N R117W S29; 17 Jul 1982; \textit{Atkins, Neely, and Carpenter 8238} (UTC).
14 mi. S of Kemmerer. 6,500' elevation; T19N R116W S33 E1/2; 27 May 1994;
\textit{Dorn 5592} (RM, MO).
  \textbf{Uinta County:}
Overthrust Belt; south slopes of Hanks Hill at north end of Woodruff Narrows
Reservoir, ca. 1 mi. E of the WY-UT state line, ca. 10 air mi. N of Evanston.
6,500-6,760' elevation; T18N R120W S29 SW1/4 SW1/4, S30 SE1/4, S31 N1/2 of NE14,
and S32 NW1/4 of NW1/4; 22 May 1996; \textit{Fertig 16460} (RM).
Ca. 16 air mi. N of Evanston, 8.5 mi. E on County Road 101.
6,460-6,750' elevation; T18N R120W S32 NW1/4, S31 NE1/4, S29 SE1/4, and
S28 SW1/4; 22 Jun 1995; \textit{Refsdal 4228} (RM).
18.5 mi. SSW of Diamondville. 6,650’ elevation; 11 Jul 1965;
\textit{Mulligan and Crompton 3095} (MONT).
Basins and Mountains of Southwest Wyoming; ca. 6 air mi. SE of Cumberland Gap,
ca. 14.2 road mi. N of I-80 on east side of Wyo Hwy. 412.
6,700-6,860' elevation; T18N R116W S28 W1/2; 19 May 1994;
\textit{Refsdal and Atwood 200, 201, 202} (RM).
Overthrust Belt; SW end of the "Carter Cedars" along Wyo Hwy. 412,
ca. 4 air mi. NW of Carter, ca. 6.5 mi. E of US Hwy. 189. 6,800' elevation;
T17N R116W S2 NE1/4 of SW1/4; 18 Jun 1997; \textit{Fertig 17520, 17521} (RM).
US 189, 8.7 road mi. NE of I-80. 7,000' elevation; T17N R117W S34; 6 Jul 1983;
\textit{Hartman 15736} (RM).
6 mi. N Ft. Bridger. 6,500' elevation; 13 Jun 1938; \textit{Rollins 2316} (RM).
S of Carter. 7,000' elevation; T17N R115W S34 NE1/4; 4 Jun 1980;
\textit{Lichvar 2777} (RM).
2 mi. W of Fort Bridger. 6,700’ elevation; 7 Jul 1965;
\textit{Mulligan and Crompton 3079} (MONTU).
2 mi. W of Fort Bridger. 6,700’ elevation; 21 Jul 1963;
\textit{Mulligan and Crompton 2785} (UTC).
Ca. 5 mi. SSW of Carter. 6,600' elevation; T16N R116W S13; 12 Jun 1980;
\textit{Lichvar 2866} (RM).
2 mi. W of Fort Bridger. 2,134m; 41º19'51"N, 110º25'10"W; 3 Jun 1996;
\textit{O'Kane 3785} (ISTC).
Foothills of Bridger Butte, 3 mi. W Ft. Bridger. 6,500’ elevation; 24 Jun 1938;
\textit{Rollins 2387} (NY).
Fort Bridger, Wyoming Territory. July 1873, \textit{Porter 10462} (NY);
\textit{Porter s.n.} (NY, NY, F).
Fort Bridger. 9 Jun 1898; \textit{Nelson 4602} (RM, F).
3 mi. W of Ft. Bridger, topotype.	7,000' elevation; T16N R116W S35; 24 May 1979;
\textit{Lichvar 1704} (RM).
About 3 mi. W of Ft. Bridger. 7,000' elevation;	7 Jul 1977;
\textit{Dorn 2974} (RM).
28 mi. W SW of Green River. 6,625’ elevation; 11 Jul 1965;
\textit{Mulligan and Crompton 3096} (NY).
Basins and Mountains of Southwest Wyoming; Wildcat Butte between Church Butte
Road and I-80 at Sweetwater County, ca. 14.8 air mi. NE of Lyman, ca. 49 air mi.
ENE of Evanston.  6,820-6,980' elevation; T17N R112W S22 NW1/4; 18 Jun 1995;
\textit{Nelson and Refsdal 35212} (RM).
Ca. 12 air mi. NE of Lyman. 6,940-7,000' elevation; T16N R113W S1 S1/2;
22 Jun 1995; \textit{Refsdal 4313} (RM).
6 mi. E of Lyman. 6,600’ elevation; 19 Jun 1956; \textit{Porter 7005} (RM).
6 mi. E of Lyman. 3 Jun 1970; \textit{Rollins 79152} (NY, US).
9 mi. E NE of Fort Bridger. 6,500’ elevation; 7 Jul 1965;
\textit{Mulligan and Crompton 3080} (CAS).
Sandy ravine near Blacks Fork River, 3 mi. N of Lyman.	6,500' elevation;
10 Jun 1937; \textit{Rollins 1653} (RM, NY, UC, MO).
Lyman. 15 Jun 1932; \textit{Rollins 182} (RM, MO).
Along Leavitt Creek below the south end of Cottonwood Bench, ca. 7 air mi. ESE
of Mountain View, ca. 39.5 air mi. E of Evanston.	6,700-6,860' elevation;
T15N R114W S36; 18 Jun 1995; \textit{Nelson 35163} (RM).
8 air mi. SE of Mountainview, Leavitt Cr. 6,800' elevation;
T15N R114W S36 SE1/4; 30 Jun 1982; \textit{Goodrich and Atwood 17162} (RM, NY).
Flat above barren cliffs overlooking Laevitt Creek, 1 km (0.6 mi.) S of Wyo Hwy.
414, 11.5 km (7mi.) air distance east-southeast of Mountain View.
6,800' elevation; 2,075m; 41º13'45"N, 110º12'56"W; 23 May 1999;
\textit{Holmgren and Holmgren 13447} (ISTC, NY, UTC).
Grizzly Buttes, Canyonlands and erosional badlands near Mountainview.
6,800' elevation;	T14N and 15N R114W S2 and 36; 13 Jul 1973;
\textit{Hill 881} (RM).
Sage Creek Mountain, ca. 12 air mi. SE of Mountain View. 7,200' elevation;
T14N R113W S20 SE and S21 SW; 12 Jun 1981; \textit{Dueholm 11434} (RM, NY).
East end of Sage Creek Mountain, ca. 5.3 air mi. N of Lonetree.
8,200-8,420' elevation; T13N R113W S2 NW1/4, T14N R113W S35 S1/2; 23 Jul 1995;
\textit{Refsdal 5887} (RM).
North Slope Uinta Mountains; Hickey Mountain, ca. 5.5 air mi. NW of Lonetree.
7,480-8,000' elevation; T13N R114W S12; 22 Jun 1994;
\textit{Refsdal and Fertig 1047} (RM).
5 mi. N 25 dg W of Lonetree, E side Hickey Mtn. 7,800' elevation;
T13N R113W S18 SE1/4; 30 Jun 1982; \textit{Goodrich and Atwood 17171} (RM, NY).
Hickey Mountain, one mi. N of State Hwy. 414; 20 Jun 1986;
\textit{Rollins and Rollins 8670} (RM, NY, GH, UTC, MONTU).
Clay knolls and hillsides, County Road 290, 4 mi. W of Lonetree. 19 Jun 1986;
\textit{Rollins and Rollins 8666} (RM).
Uinta County Road 290, 3.7 air mi. W of Lonetree. 7,800' elevation;
T12N R114W S1; 7 Jul 1983; textit{Hartman 15766} (RM).
Cedar Mountain, ca. 3 air mi. NE of Lonetree; ca. 3.3 road mi. E of Cedar
Mountain Road from Wyo Hwy. 414, west flank of the mountain.
7,700-7,800' elevation; T13N R113W S22 W1/2 and S15 S1/2; 11 Jun 1994;
\textit{Refsdal and Lathrop 725} (RM).
SW side of Cedar Mtn. 7,700' elevation; T13N R113W S24 S1/2; 28 Jun 1999;
\textit{Dorn 7997} (RM).
Ca. 6.5 air mi. NNW of Lonetree, ca. 1.2 road mi. SW of Wyo Hwy. 414.
7,260-7,410' elevation; T13N R113W S26 W1/2; 22 Jun 1994;
\textit{Refsdal and Fertig 1023} (RM).
Ca. 2 air mi. E of Lonetree, ca. 8.0 road mi. E of the junction of County Road 1
with Wyo Hwy. 414 on south side of Wyo Hwy. 414. 7,400-7,600' elevation;
T12N R113W S1 SW1/4 and S2 SE1/4; 7 Jun 1994; \textit{Refsdal 517} (RM).
Hoop Lake Road (Uinta County Road 295), 4 air mi. S of Lonetree.
7,800' elevation; T12N R113W S21; 7 Jul 1983; \textit{Hartman 15761} (RM).
Ca. 4 air mi. S of Lonetree, just N of Utah on Hoop Lake Road.
7,900-8,000' elevation; T12N R113W S21 S1/2; 7 Jun 1994;
\textit{Refsdal 549} (RM).
  \textbf{Park County:}
Bighorn Basin; on road between Silvertip and Elk Basin Oil Fields, SSE of Elk
Basin Oil Field; ca. 14.5 air mi. NW of Powell. 4,600' elevation;
T57N R100W S1; 21 Jun 1987; \textit{Nelson	13688} (NY).
Bighorn Basin; near the head of Spring Creek, ca. 6 air mi. S of Cody.
5,500' elevation; T52N R101W S31; 26 Jun 1983; \textit{Nelson 9944} (ISTC).
Bighorn Basin; north end of Oregon Basin, ca. 8 air mi. SE of Cody.
5,300-5,600' elevation; T52N R100W S8, S16, and S17; 25 May 1983;
\textit{Hartman and Hamann 14449} (ISTC).
Along W side of Oregon Basin Rd., ca. 1.5 mi. S of Hwy. 14, 16, 20.
5250' elevation; T52N R100W S15 and S16;	13 Jun 1990;
\textit{Evert 18832} (RM).
Bighorn Basin; Burlington Meeteetse Road, 16.5 mi. NE of intersection with
Wyo Hwy. 120. T51N R98W S14, S15; 24 May 1980;
\textit{Hartman with Dueholm 11146} (ISTC, UTC).
Bighorn Basin; on the divide between Gooseberry Creek and Renner Draw, ca.
1.5 air mi. S of Gooseberry Creek; ca. 11 air mi. S of Meeteetse.
6,400' elecation; T47N R100W S33; 27 Jun 1983; \textit{Nelson 9999} (ISTC).

