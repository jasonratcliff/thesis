\chapter*{APPENDIX A: SPECIMENS REVIEWED}
\addcontentsline{toc}{chapter}{APPENDIX A: SPECIMENS REVIEWED}

***Physaria nova***

\textbf{Wyoming} % WYOMING STATE SPECIMENS

\textit{Carbon}

Hog Park, Sierra Madre Mts. 8400’ elevation; 4101.2’N 10651.1’W; 10 July 2005; \textit{Dorn 10105} (RM, NY).  Mts. of Southern Wyoming, Hog Park. 20 August 1894; \textit{Osterhout 411} (RM).  Sierra Madre Mts. 8300’ elevation; T12N R84W S9 SW4; 21 Jun 1981; \textit{Dorn 3695} (RM, NY).  Copper Creek W of Encampment. 8200’ elevation; T14N R84W S29 SE4; \textit{Morton 63} (RM).  Hillock, 3 mi. W of Encampment on Hwy 70, ca. 30 m S of rd. 2066m elevation; 4211’24”N 10651’03”W; 27 Jun 1996; \textit{Salywon \& Dierig 3154} (ISTC).  Ca. 3 air mi. WSW of Encampment. 8000’ elevation; T14N R84W S9; 1 Jun 1985; \textit{Williams 54} (RM).  Sierra Madre, ridge N of Miner Creek, ca. 2 air mi. S of Encampment. 7600’ elevation; T14N R84W S14; 25 May 1987; \textit{Williams 476} (RM).  W slope Medicine Bow Range, W foothills of Barrett Ridge on N \& W side of Jim Draw, ca. 0.9 mi. S of WY Highway 130 and 0.75 mi. E of Brush Creek. 8000’ elevation; T16N R82W S24; 30 May 2000; \textit{Fertig \& Welp 19075} (RM).  W slope Medicine Bow Range, ridge between Jim Draw and Francis Draw at far western edge of Barrett Ridge, ca. 0.3 mi. E of Brush Creek and 1.1 mi. S of WY Highway 130. 7900’ elevation; T16N R82W S23; 30 May 2000; \textit{Fertig \& Welp 19078} (RM).  W slope Medicine Bow Range, W foothills of Barrett Ridge ca. 0.1 mi. S of Jim Draw and 0.3 mi. E of Brush Creek, ca. 0.5 mi. S of WY Highway 130. 7760-7800’ elevation; T16N R82W S14; 30 May 2000; \textit{Fertig \& Welp 19073} (RM).  Sierra Madre, ca. 8 air mi. SSE of Miller Hill.  7,650’ elevation; T17N R88W S35; 28 Jun 1988; \textit{N. Kastning, Huston, \& S. Kastning 267} (RM).    Sierra Madre, ca. 14 air mi. NNW of Bridger Peak.  7,900’ elevation; T16N R87W S5, S6, S7, S8, \& S9; 14 Jun 1988; \textit{Hartman \& Kastning 23657} (RM).  Medicine Bow Range, ca. 2.5 air mi. N of Kennaday Peak, ca. 1.5 mi. E of Cedar Pass. 8,100’ elevation; T18N R81W S33 N1/2 NE1/4 of NW1/4; 25 Jun 1996; \textit{Fertig 16685} (RM).  Sierra Madre, ridge on E side of USFS Rd 830 \& on W side of North Fork Deep Creek, ca. 2.5 air mi. NW of Singer Peak. 8660-8720’ elevation; T15N R87W S33 E2SE4SW4 \& SW4SW4SE4; 28 June 1996; \textit{Fertig 16699} (RM, NY).  Sierra Madre, low ridge on E side of USFS Rd 830 ca. midway between Deep Creek and Big Sandstone Creek, ca. 1.75 air mi. NW of Singer Peak. 8500’ elevation; T14N R87W S4 SE4SE4; 28 June 1996; \textit{Fertig 16713} (RM).   East slope Sierra Madre, along Big Creek Road (FS Road 498), ca. 5.5 air mi. W of WY Highway 230 and 0.5 mi. E of South Fork Big Creek. 8250’ elevation; T14N R87W S28 SW4SE4; 6 June 2001; \textit{Fertig 19510} (RM).   Corral Ranger Station. 8000’ elevation; 23 May 1915; \textit{Peryam 6} (RM).  Battle Mt. Deer Ridge, NW corner of Mt.  7.5 air mi. E Savery.  8,300’ elevation; T12N R88W S4; 17 Jun 1980; \textit{Current 792} (RM).  Battle Mt., top of Buck Camp draw, N. of stockdam 8.5 air mi. E of Savery.  8,800’ elevation; T12N R88W S3; 2 Jul 1980; \textit{Current 951} (RM).  South-central Wyoming, ca. 2-2.5 air mi. E of Savery.  6,600-6,750’ elevation; T12N R89W S9 S1/2; 1 Jun 1996; \textit{Hartman \& Ward 54164} (RM).  Hayden National Forest.  7,300' elevation; 6 Jun 1934; \textit{Nelson 11363} (RM).  

\textit{Albany}

E slope Medicine Bow Range, S end of Sheep Mountain on ridge on E side of Fence Creek, ca. 1.5 mi. N of Woods Landing, ca. 26 air mi. SW of Laramie. 8,000' elevation; T14N R77W S34 SW1/4 SW1/4; 2 Jun 1996; \textit{Fertig \& Barlow 16475} (RM).

\textbf{Colorado} % COLORADO STATE SPECIMENS

\textit{Routt}

North-central Colorado, Park Range; at the junction of Bedrock and Tennessee creeks, along Beeler Gulch and vicinity, ca. 35.5 air mi. NNW of Steamboat Springs. 7,320-7,420' elevation; T12N R86W S27 W1/4; 27 Jun 2001; \textit{Nelson 53301} (RM).  Sierra Madre ca 11.1 mi. NNW of Columbine.  7,400' elevation; T12N R86W ca. S27; 14 Jun 1979; \textit{Hartman \& Coffey 8953} (RM). 

\textit{Jackson}

North-central Colorado, Medicine Bow Mountains; N of Camp Creek ca. 1.5 air mi. W of Colo Hwy 127, ca. 5 air mi. NE of Three Way; ca. 17 air mi. N of Walden.  8,550-8,650' elevation; T12N R79W S28 E1/4; 13 Jun 2000; \textit{Nelson 49286} (RM).

***Physaria floribunda***

**Grand**

Middle Park: clay cliffs just west of the Muddy, west end. 27 Jul 1875; *Patterson* (F-208917, F-980).  Along and above Eastern Gulch on County Road 27 (Forest Road 103), ca. 1/4 mi. NE of U.S. Hwy 40, ca. 1.5 air mi. SSE of Granny's Nipple; ca. 14 air mi. NNW of Kremmling. 7,720-7,960' elevation; T3N R81W S3 N1/4; 15 Jun 2000; *Nelson 49478* (RM).  


***Physaria saximontana***

\textbf{Wyoming} % WYOMING STATE SPECIMENS

\textit{Park}

Absaroka Mountains, North Fork Shoshone River Drainage; north side of Crow Peak ca. 1-2 mi. N of U.S. Hwy 14, 16, \& 20. 7,500-8,500’ elevation; T52N R109W S2; 28 Jun 1987; *Evert 12677* (RM).  Ca. 1/2 mi. E of Pahaska along US Hwy 14, 16, \& 20. 6,800’ elevation; T52N R109W S2 SW4; 13 Jul 1981; *Evert 3123* (RM).  Northeast-southwest trending ridge ca 1/4 mi E of Grinnell Creek and ca 1/8 -3/4 mi N of U.S. Hwy 14, 16, \& 20. 6,800-7,600’ elevation; T52N R109W S12, T52N R108W S6; 23 Jun 1987; *Evert 12460* (RM, NY).  Ridge ca. 0.5 mi. W of Mormon Creek; ca. 0.5-2 mi. N of U.S. Hwy 14, 16, \& 20. 7,600-8,600’ elevation; T52N R108W S8; 18 Jul 1985; *Evert 8392* (RM).  Absaroka Range, South Absaroka Wilderness; Fishhawk Creek, 0.5-1 mi. SSW of confluence with the North Fork Shoshone River, Sec. 27 \& 34. 6,500’ elevation; T52N R108W S27; 7 Jul 1978; *Hartman 8361* (RM).  Along Elk Fork Creek trail, ca. 5-6 mi. S of U.S. Hwy 14, 16 \& 20. 6,500-6,700’ elevation; T51N R107W S13, S7; 26 Jul 1985; *Evert 8918* (RM).  Ca. 3-4 mi. S of US Hwy 14, top of Clayton Mountain. 10,100’ elevation; T51N R107W S10 NW4; 20 Jul 1982; *Evert 4362* (RM).  Slopes on S side of summit rim of Clayton Mountain and northernmost end of saddle connecting Clayton Mountain and Double Mountain, ca, 1.1 air mi. E of East Fork Blackwater Creek and 3.2 mi. S of US Hwy 14, 16, \& 20. 9,500-10,000’ elevation; T51N R107W S10 S2 NW4; 17 Jul 1996; *Fertig 16871* (RM).  Top of Clayton Mountain, ca. 3 mi. S of U.S. Hwy 14, 16, \& 20. 9,800’ elevation; T51N R107W S10; 24 Jul 1985; *Evert 8782* (RM).  N-S trending ridge immed. E of June Creek, ca. 1/4 mi. S of N fork Shoshone River. 6,400-6,600’ elevation; T52N R107W S10 \& S3; 4 Jun 1989; *Evert 16391* (RM).  East of Newton Creek, ca. 1.5 mi. S of US Hwy 14. 6,800’ elevation; T52N R107W S21 SW4; 7 Aug 1982; *Evert 4694* (RM).  Ridge/divide between Moss Creek and Clearwater Creek ca. 1.5-2.5 mi. N of Hwy 14, 16, \& 20. 7,600-8,300’ elevation; T52N R107W S10 \& S3; 6 Jul 1989; *Evert 17578* (RM).  Many pinnacled ridge E of Clearwater Creek, ca. 3-4 mi. N of U.S. Hwy 14, 16, \& 20. 7,200-8,000’ elevation; T52N R107W S2; 9 Jun 1986, *Evert 10137* (RM).  Ridge, 0.5 mi NW of Signal Peak, ca. 1 mi. N of U.S. Hwy 14, 16, \& 20. 7,000’ elevation; T52N R106W S18; 13 Jun 1986; *Evert 9930* (RM).  N-S trending ridge immed. W of Clearwater Creek ca. 1/8 mi. N of Hwy. 14, 16, \& 20. 6,200’ elevation; T52N R106W S20; 4 Jun 1989; *Evert 16363* (RM).  Elk Fork Creek ca. 3/4 mi. S of US Hwy 14. 6,000’ elevation; T52N R106W S29 NE4; 15 Jul 1982; *Evert 4247* (RM).  Ridge E of Sweetwater Creek, ca. 4-5 mi. N of U.S. Hwy 14, 16, \& 20. 7,500-8,000’ elevation;  T52N R106W S5 \& S33; 16 Jun 1986; *Evert 10040* (RM).  Ridge between Horse and Sweetwater creeks, ca. 1.5-3 mi. N of U.S. Hwy 14, 16, \& 20. 6,800-7,500’ elevation; T52N R106W S4 \& S9; 16 Jun 1987; *Evert 12281* (RM, NY).  North-south trending ridge between Horse Creek and Grizzly Creek ca. 2-5 mi. N of U.S. Hwy 14, 16, \& 20. 7,500-8,500’ elevation; T52N R106W S3, T53N R106W S35; 25 Jun 1987; *Evert 12582* (RM).  North ridge of Ptarmigan Mountain on divide between Cougar and Pagoda Creeks, ca. 3.9 mi. E of Elk Fork Creek, ca. 5 mi. S of US Hwy 14, 16 \& 20. 10,200’ elevation; T51N R106W S15 SW4 NE4; 30 Jul 1996; *Fertig 16970* (RM).  Divide between Cougar and Pagoda Creeks below N ridge of Ptarmigan Mountain, ca. 3 air mi. E of Elk Fork Creek, ca. 3.5 mi. S of US Hwy 14, 16 \& 20. 8,600’ elevation; T51N R106W S10 NE4 SW4; 30 Jul 1996; *Fertig 16965* (RM).  On ridge between Cougar and Pagoda Creeks, ca. 4 mi. S of US Hwy 14, 16, \& 20. 8,600’ elevation; T51N R106W S10 SW4; 22 Jul 1981; *Evert 3275* (RM).  Ridge system on N side of North Fork Shoshone River between Signal Peak and Anvil Rock, ca. 1 mi. N of US Hwy 14/16/20 \& ca. 5 air mi. W of Wapiti. 7,200-7,400’ elevation; T52N R105W S18 E2 SW4; 26 Jun 1997; *Fertig 17583* (RM).  On ridge between Fishhawk and Mesa creeks, ca. 2-3 mi. S of U.S. Hwy 14, 16, \& 20. 7,000-8,000’ elevation; T51N R105W S2 \& S3; 8 Aug 1984; *Evert 7451* (RM).  Top of Four Bear (Black) Mountain, ca. 2.5 mi. N of U.S. Hwy 14, 16, \& 20. 7,200-7,600’ elevation; T52N R104W S5; 14 Jun 1988; *Evert 14434* (RM).  North side of Logan Mountain. 5,850’ elevation; T52N R104W S11 SE4; 19 Jun 1982; *Evert 3939* (RM).  Eastern Absaroka Mts., ca. 6 air mi. N of Dead Indian Peak, ca. 0.25 air mi. N of Sulight Crk.  7,000’ elevation, T55N R106W S4 SW4 of NE4; 19 Jul 1995; *Mills 39* (RM).  Northern Absarokas, Windy Mountain Summit and upper NW slope, 6-7 air mi. NNW of sunlight ranger station.  9,900’ elevation; T56N R106W S26 \& S35; 17 Aug 1985; *Hartman \& Nelson 21835* (RM).  Bighorn Basin, Kimball Bench, ca. 10 air mi. SSW of Clark.  4,800-5,000’ elevation; T55N R103W S3 \& S4; 24 May 1983; *Hartman \& Hamann 14354* (NY).  SW-facing slopes below summit cone of Heart Mountain, NE of WY Hwy 120, ca. 9 air mi. NNW of Cody.  7,100’ elevation; T54N R102W S15 SE4 of NW4; 30 Jun 1997; *Fertig \& Lenard 17641* (RM).  SW flank of Rattlesnake Mtn. ca. 6-7 miles W of Cody \& ca. 1 mi N of Hwy 14, 16, \& 20. 6,400-7,000' elevation; T52N R103W S2; 15 Jun 1989; *Evert 16761* (RM).  Dry gravelly soil, Middle Shoshone Canyon.  6,500’ elevation; 7 Jul 1921; *Wiegand, Castle, Dann, \& Douglas 1008* (F).  Bighorn Basin, Spring Creek, 2 air mi. WNW of Meeteetse.  5,800’ elevation; T48N R100W S6; 21 Jun 1983; *Hartman \& Nelson 15525* (UTC).  Absaroka Mountains, ca. 22 air mi. SW of Meeteetse, in the vicinity and W of the junction of the Middle Fork Wood River and Beaver Creek.  7,900’ elevation; T46N R103W S35 \& 36; 26 Jul 1984; *Kirkpatrick 4997* (RM).  Absaroka Mountains, ca. 33 air mi. SW of Cody at the junction of Boulder Creek and Little Boulder Creek. 7,300’ elevation; T49N R105W S30 & S29; 1 Jul 1983; *Kirkpatrick 613* (RM, GH).  Eleanor Creek N to ridge. 10,500’ elevation; T48N R105W S30 S32; 25 Aug 1984; *Hartman 19364* (RM).  Jack Creek Trail ca. 0.2 mi. N of junction with Haymaker-Timber Creek Trail, ca. 14 air mi. W of Sunshine Reservoir Dam.  9,200’ elevation; T47N R104W S10 SE1/4 of NW1/4; 29 Jun 1988; *Marriott 10872* (RM).  Upper Greybull River between Steer Creek and Cow Creek, 4.5-7 air mi. NW of Kirwin. 9,300’ elevation; T46N R104W S17, S18, S20, S29; 22 Aug 1983; *Hartman 17305* (RM).  Southeastern Absaroka Mts., ca. 3 air mi. NE of Chief Mtn., ca. 0.5 air mi. N of Jojo Crk. and Wood River junction.  8,480’ elevation; T46 R103W S21 NE1/4 of SE1/4; 7 Jul 1995; *Mills 16* (RM).  Ca. 4 air mi. NE of Chief Mtn., ca. 0.5 air mi. N of Jojo Crk. and Wood River junction.  8,560’ elevation; T46N R103W S22 SW1/4 of NW1/4; 7 Jul 1995; Mills 17 (RM).  Ca. 5 air mi. ENE of Chief Mtn., across road from Brown Creek Camp Ground. S23 SW4 of SE4. 7,600’ elevation; T46N R103W S23 SW1/4 of SE1/4; 8 Jul 1995; *Mills 22* (RM).  Ca. 6 air mi. ENE of Chief Mtn., ca. 0.5 air mi. N of the Wood River.7,800’ elevation; T46N R103W S24 S1/2; 7 Jul 1995; *Mills 20* (RM).  S. Fk. Shoshone River.  7,000’ elevation; T48N R106W S18 NW1/4; 24 Aug 1987; *Dorn 4799* (RM).  Growing on shale slopes of southwest-facing slopes of mountains above the South Fork of the Wood River, Meeteetsee.  7,588’ elevation; Zone 12T WGS 84, Easting 0652708 Northing 4864916; 25 Aug 2012; *Smith 551, 552, 553* (ISTC).

\textit{Fremont}

NE Wind River Range: S end of Torrey Rim, ca. 0.2 mi. N of Trail Lake trailhead, ca. 1.75 mi. W of Trail Lake, ca. 7 air mi. S of Dubois. 7,700-7,850' elevation; T40N R106W S16 SE1/4SE1/4 S21 N1/2E1/4; 14 Jun 1996; *Fertig 16632* (RM).  Absaroka Mountains, East fork of the Big Wind River, 7 air mi. ESE of Dubois. 7,000' elevation; T41N R105W S20; 23 Jun 1983; *Hartman 15583* (RM).  Wind River Indian Reservation, along U.S. Hwy 287, ca. 14 mi. SE of Dubois. 6,600' elevation; 2 Jul 1983; *Evert 5268* (RM).  Wind River Indian Reservation. 8,300' elevation; T7N R5W S13 NW1/4; 10 Jul 1985; *Lichvar 4561* (RM).  Absaroka Range, Wind River Indian Reservation, ca. 20 air mi. E of Dubois. 9,700' elevation; T7N R4W S28; 31 Jul 1981; *Day \& Berner 25* (RM).  Wind River Indian Reservation, gypsum formation. 6,000' elevation; T6N R3W S16; 23 Jun 1982; *Lichvar 5168* (RM).  Wind River Reservation, Pasup Creek, Circle Ridge Road. 6,965' elevation; T7N 3W S35; 1 Aug 1942; *Murphey s.n.* (RM-404206).  Heavy clay soil, hillside, 15 mi. NW of Fort Washakie. 22 Jul 1983; *Rollins 83330* (NY).  Wind River Indian Reservation, ca. 25 mi. N NW of Morton.  22 Jun 1981; R. C. \& K. W. Rollins 81386 (RM, NY, GH, UC, US).  Shoshone N. F.; outcrop, ca. 1/2 mi WNW of USFS Sinks Canyon Campground, ca. 7 mi SW of Lander. 8,500' elevation; T32N R101W S13 SE1/4; 15 Jun 1978; *Johnston \& Lucas 1680* (RM).  Southern Wind River Range, ca. 8 mi. SW of Lander, SE side of Fossil Hill. 7,840' elevation; T32N R100W S30 SW1/4 of SW/14; 28 Jun 1995; *Mills 6* (RM).  E of Lander. 6,000' elevation; T32N R99W S11 SW1/4; 18 May 1981; *Lichvar 4215* (RM).  SE edge of Wind River Range; E of Dry Lake \& NE of US 287/WY 28 jct., ca. 7 air mi. SE of Lander; Lee Ranch. 5,600' elevation; T32N R99W S23; 16 Jun 1989; *Marriott 11013* (RM).  13 mi. SE of Lander. 5,800' elevation; T32N R98W S33; 1 Jul 1991; *Dorn 5242* (RM, NY).  SE Wind River Range: Red Canyon Rim on E side of Red Canyon, ca. 11 air mi. SSE of Lander.  5,700-5,800' elevation; T31N R99W S10 SW1/4NW1/4 \& S9 NE1/4NE1/4; 7 Jun 1994; *Fertig 14811* (RM, NY).  E slope Wind River Range: Red Canyon Rim, on east side of county road, ca. 1.25 mi. W of WY Hwy 28. 5,800-5,900' elevation; T31N R99W S10 NW1/4; 14 May 1994; *Fertig 14672* (RM).  E Slope Wind River Range, Red Canyon,  slopes W of Red Canyon Creek ca. 13 air mi. SSE of Lander. 6,000-6,200' elevation; T31N R99W S15 SW1/4; 24 May 1993; *Fertig \& Studenmund 13573* (RM).  Red Canyon Rim, ca. 14 air mi. SSE of Lander. 6,100' elevation; T31N R99W S23, S25, \& S26; 20 Jun 1986; *Haines 6712* (RM).  E slope Wind River Range, Red Canyon Rim above Foster Draw ca. 15 air mi. SSE of Lander.  6,350' elevation; T31N R99W S26 NE1/4 SE1/4; 8 Jun 1994; *Fertig 14829* (RM).  17.9 mi. S of Lander on Hwy 28; E side of rd. 2,059m elevation; 42° 36' 47" N, 108° 36' 20" W; 26 Jun 1996; *Salywon \& Dierig 3153* (ISTC).  Near State Route 28, 18.4 mi. SW of Lander. 22 Jul 1983; *R. C. \& K. W. Rollins 83331* (NY, GH, US).  21 mi. S of Lander. 20 Jun 1969; *Barneby 15100* (NY, GH).  17.65 mi. S of Lander on Atlantic City Road. 8,100' elevation; T30N R99W S17; 19 May 1946; *Wiegand 207* (RM).  18 mi. SSE of Lander. 6,800' elevation; T30N R98W S8 NE1/4 SW1/4; 12 Jun 1991; *Dorn 5177* (RM, NY, MO).  Great Divide Basin Area, Popo Agie River Drainage, Box Spring, ca. 20 air mi. SE of Lander; ca. 17 air mi. W of Sweetwater Station. 6,220-6,600' elevation; T30N R98W S12; 5 Aug 1995; Welp 7397 (RM).  Hills along Twin Creek. 6,100' elevation; T31N R98W S36 NW1/4; 30 May 1990; *Dorn 5054* (RM, NY).  Sheep Mountain, ca. 11 air mi. SE of jct US Hwy 287 and Wyo. Hwy 28. 6,200' elevation; T31N R98W S36, T31N R97W S31; 31 May 1985; *Hartman 20103* (NY).  Great Divide Basin Area, Popo Agie River Drainage, above Red Bluff Canyon, ca. 6 air mi. N of Schoettlin Mountain; ca. 5.5 air mi. SE of Weiser Pass. 6,600-6,700' elevation; T30N R97W S4; 4 Aug 1995; *Welp 7180* (RM).  Sweetwater River Plateau, south end of Beaver Rim, ridge on N side of Red Canyon on east bank of Beaver Creek, 2-2.5 mi. W of US Hwy 287. 5,800-6,100' elevation; T30N R97W S1 NE1/4 of SW1/4 \& N2 of SE1/4; 30 Jun 1995; *Fertig \& Studenmund 15808* (RM).  Beaver Rim Divide, ca. 5 mi NW from Sweetwater Junction on Hwy 287, and over 1 mi. NW on abandoned highway. 6,720' elevation; T30N R96W S3 NE1/4 of SE1/4; 9 Jun 2003; *Heidel 2300* (RM, NY).  Beaver Rim, ca. 6 air mi. NW of Sweetwater Station. 6,680' elevation; T30N R96W S2; 20 Jul 1986; *Haines 6926* (RM).  Beaver Hill, about 35 mi. SE of Lander along Route 287. 6,400' elevation; T30N R96W S11; 8 Jun 1960; *Wetherell 256* (RM).  Top of Beaver Rim, Devil's Gap, ca. 7 air mi. NNW of Sweetwater Station. 6,800' elevation; T31N R96W S25; 14 Jun 1986; *Haines 6429 \& 6387* (RM).  W Wind River Basin, Beaver Rim; ridges in vicinity of Devil's Gap, ca. 1.5 air mi. W of Dishpan Butte. 6,900-7,000' elevation; T31N R95W S30 NW1/4; 13 Jun 1994; *Fertig 14848* (RM).  Sweetwater River Plateau, Beaver Divide, SE end of Dishpan Butte, 1.25 air mi. W of junction of Dishpan Butte Road \& WY Hwy 135. 6,820-6,880' elevation; T31N R95W S29 NE1/4 of SE1/4 of NW1/4; 9 Jul 1997; *Fertig \& Welp 17668* (RM).  Sw edge of Cedar Rim, ca. 0.7 mi. E of Rte 135, 28 air mi. SSE of Riverton.  T31N R95W S27 SW1/4 \& S34 NW1/4; 11 Jun 1993; *Anderson 14371* (NY).  6 mi. N of Sweetwater Station. 6,700' elevation; T31N R95W S27 SW1/4 \& S34 NW1/4; 1 Jul 1991; *Dorn 5246* (NY, MO).  W Wind River Basin, Beaver Rim Divide, south end of Cedar Rim. 6,800' elevation; T31N R95W S27 SE1/4 SW1/4; 13 Jun 1994; *Fertig 14841* (RM).  Sweetwater River Plateau, ridge system on N side of Government Meadow Draw, ca. 1-1.5 mi. E of WY Hwy 135, ca. 4.75-5 air mi. NNE of Sweetwater Station. 6,700-6,800' elevation; T31N R95W S34 NE1/4 of SE1/4, SE1/4 of NE1/4, \& E1/4 of SE1/4 of SE1/4, S35 SW1/4 of SW1/4 \& S1/2 of SE1/4 of SW1/4; 13 Jul 1997; *Fertig 17696* (RM).  Beaver Rim, ca. 9 mi. N of Sweetwater Station. 6,800' elevation; T31N R95W S9; 27 Jun 1981; *Dueholm 11670* (RM).  Wind River Basin, Beaver Divide, adj to WY Hwy 135, ca. 28 air mi. S of Riverton. 6,760-6,800' elevation; T31N R95W S3 NE1/4 of NW1/4; 13 Jul 1992; *Fertig 13016* (RM).  Ca. 3 air mi. SSW of Big Sand Draw oil and gas field. 6,000' elevation; T32N R95W S28; 3 Jul 1981; *Hartman 13510* (RM).  Beaver Rim. 7,100' elevation; T32N R95W S26 S1/2 of S1/2 of NE1/4; 2 Jul 1991; *Dorn 5252* (RM).



***Physaria didymocarpa***

***Physaria integrifolia***

\textbf{Wyoming} % WYOMING STATE SPECIMENS

\textit{Teton}

Gros Ventre Fork. 10 Jun 1860, *Hayden s.n.* (MO-3833631).  Teton Peaks. *Davis s.n.* (NY, IDS-0030427).  Jackson Hole. 6 Jul 1860; *Hayden s.n.* (MO-3833625).  Teton National Forest, 2 mi. above the mouth of Game Creek. 6,400' elevation; 11 Jul 1932; *Buchenroth 159* (NY).  Infrequent in rocks on Switchbacks up to "The Wall" in South Cascade Canyon. 5 Sep 1955; *Shaw 1011* (UTC).  Plants of Grand Teton National Park, possibly on Mt. Woodring. *Woodring s.n.* (MO-1066430).  On ridge west of The Wall, N of Alaska Basin, Targhee Nat'l Forest. 10,500' elevation; 3 Aug 1955; *Anderson 230* (NY, UTC).  Grand Teton National Park, 1/2 mi. E of Mount Hunt Divide Trail. 9,500' elevation; 11 Jul 1965; *Shaw 1457* (UTC).  Teton Range East Slope; Rendevzvous Mountain, just below ridge between Granite Canyon and Jackson Hole, ca. 3/4 mi. W of Apres Vous Peak, ca. 1.5 air mi. NW of Teton Village; ca. 8 air mi. NW of Jackson. 9,200-9,500' elevation; T42N R117W S14; 31 Jul 1996; *Markow 11323* (RM).  Targhee National Forest, West Slope Teton Range; ca. 6-7 air mi E of Victor, Idaho.  8,400-9,400' elevation; T42N R118W S10, S11, S12, S14, \& S15; 19 Aug 1991; *Markow 6550* (RM).  Targhee NF, Teton Range; E-facing slope of Taylor Mtn, ca. 100 yds. below summit ridge, ca. 12 air mi. NW of Jackson.  9,800-10,200' elevation; T41N R118W S11; 25 Sep 1995; *Markow 11199* (RM).  Targhee National Forest, Northwest Slope Snake River Range; Mail Cabin Creek, ca. 11 air mi. W of Jackson.  7,600-8,000’ elevation; T41N R118W S21 & S22; 22 Jul 1991; Markow 3781 (RM).  In Pumice formation on hills near Adam's Ranch, Jackson. 6,200' elevation; T40N R117W S4; 10 Jul 1901; *Merrill \& Wilcox 960* (RM).  Teton National Forest, big point ending in Snake R. below Fall Creek. 6,000' elevation; T39N R116W S27; 12 Jul 1928; *McDonald 816* (RM).  Gros Ventre Area, 1.5 air mi. S to 1 air mi. SE of Pinnacle Peak.  9,200-10,000' elevation; T39N R114W S3 NW1/4 \& S4 SE1/4; 7 Jul 1994; *Hartman 47330* (RM, RSA-POM).  Hillside, ca. 3.5 mi. E of Kelly Warm Springs. 2,053m elevation; 43º38'02"N 110º33'57"W; 25 Jun 1996; *Salywon & Dierig 3146* (ISTC).  Slide at Gros Ventre Lake. 7,000' elevation; 10 Sep 1951; *Munz 16997* (NY, RSA-POM).  Gros Ventre River. 16 Aug 1894; *Nelson 927* (RM, MO).  Gros Ventre slide area, Teton National Forest. 7,000' elevation; 10 Jul 1959; *C.L. \& M.W. Porter 7891* (UC, CAS).  Gros Ventre Slide Road, 1/4 mi. W of Forest Service Exhibit on Gros Ventre Slide. 6,800' elevation; 8 Jul 1971; *Shaw 1812* (UTC).  Lower Slide Lake. 7,100' elevation; T42N R114W S5; 24 May 1977; *Lichvar 99* (RM).  Mount Leidy Highland Area, Gros Ventre River Road, ca. 2 air mi. E of Forest boundary; just W to overlooking Lower Slide Lake, on the north side.  6,900-7,400' elevation; T42N R114W S5; 26 Jun 1995; *Hartman 51256* (RM).  Bridger-Teton Natl. For.; 1/3 mi. E of Atherton Creek Campground, Gros Ventre Canyon Road. 7,000' elevation; 18 Jun 1979; *Shaw 2482* (UTC).  Crystal Creek. 6,950' elevation; R113W T42N S18; 7 Jul 1977; *Lichvar 691a* (RM).  Teton Forest - Miner Creek. 8,000' elevation; T42N R113W S17; 20 May 1913; *Maris 58* (RM).  Mount Leidy Highland Area, Gray Hils; N of Gros Ventre River, ca. 2 air mi. NW of west end of Uppper Slide Lake, ca. 21 air mi. ENE of Jackson. 7,600' elevation; T42N R113W S14; 6 Jul 1990; *Nelson 19288* (RM).  Gros Ventre Area, Gros Ventre Wilderness Area; ridge E of Crystal Creek, ca. 12 air mi SE of Kelly.  7,200-8,000' elevation; T42N R113W S34; 24 Jun 1994; *Hartman 46465* (RM).  Mount Leidy Highland Area, Gray Hills; ridge and adjacent area overlooking Slate Creek, ca. 4.5 air mi. NW of west end of Upper Slide Lake. 7,900-8,100' elevation; T42N R113W S35; 6 Jul 1990; *Hartman 26343* (RM).  Gros Ventre Area, Gros Ventre Wilderness Area; ridge E of Crystal Creek, ca. 14 air mi. SE of Kelly. 8,600-9,450' elevation; T41N R113W S1, S2, \& S13; 24 Jun 1994; *Hartman 46566* (RM).  Mount Leidy Highland Area, Burnt Creek and ridge to east. 7,200-8,200' elevation; T42N R112W S30 \& S31; 24 Jul 1995; *Hartman 52772* (RM, RSA-POM).  Gros Ventre Area; upper Gros Ventre River Road, ca. 1.5 air mi. SE of Goosewing Guard Station. 7,100' elevation; T41N R112W S3 NE1/4; 26 Jun 1994; *Hartman 46738* (RM).  West Slope Wind River Range; hills on eastern bank of Cottonwood Creek, ca. 28 air mi. E of Jackson. 7,600-8,280' elevation; T42N R111W S29, T42N R111W S30 \& S31, \& T42N R112W S36; 6 Jul 1990; *Fertig 3025* (RM).  West Slope Wind River Range; Fish Creek/Moccasin Basin area, along Fish Creek just below confluence of North and South Forks, ca. 28.5-29.5 air mi. ENE of Jackson. 7,600-7,700' elevation; T42N R111W S28 \& S29; 25 Aug 1990; *Nelson 20322* (RM).  West Slope Wind River Range; South Fork Fish Creek between Hackamore and Bell Creeks. 7,800-7,850' elevation; T42N R111W S36; 25 Aug 1990; *Hartman 28350* (RM).  N Wind River Range; N bank of South Fork Fish Creek just W of confluence of Devils Basin Creek, ca. 4.25 air mi. NNW of Union Pass Rd. 8,000' elevation; T41N R110W S8 NW1/4 of SE1/4; 11 Aug 1995; *Fertig 16262* (RM).  West Slope Wind River Range; Fish Creek / Moccasin Basin Area, North Fork Fish Creek between Packsaddle Creek and Harness Gulch. 7,800-7,900' elevation; T42N R111W S11 \& S15; 12 Jul 1990; *Hartman 27148* (RM).  West Slope Wind River Range; North Fork of Fish Creek, ca. 32 air mi. NE of Jackson. 8,000-8,530' elevation; T42N R111W S2; 11 Jul 1990; *Fertig 3535* (RM).  West Slope Wind River Range; Cottonwood and Moosehorn Creeks. 8,100-8,300' elevation; T42N R111W S5 \& T43N R111W S32; 23 Aug 1990; *Hartman 28213* (RM).  Mount Leidy Highland Area; Grouse Mountain, southern ridge. 8,800-9,800' elevation; T43N R112W S2 \& NE1/4 S11; 15 Aug 1995; *Hartman 53644* (RM).  Mount Leidy Highland Area; summit and upper slopes of Mount Leidy proper. 9,800-10,200' elevation; T43N R113W S3 \& NW1/4 S3; 13 Aug 1995; *Hartman 53435* (RM).  Mt. Leidy. 10,000' elevation; *Tweedy 391* (NY).  Mount Leidy Highland Area; Spread Creek, ca. 5 air mi. SW of Black Rock Ranger Station. 7,300-7,500' elevation; T44N R113W S16; 14 Jul 1995; *Hartman 51830* (RM).  Mount Leidy Highland Area; 0.5-1 air mi. NW of Gunsight Pass on flank of plateau. 9,100-9,200' elevation; T42N R112W S10 W1/2; 12 Aug 1995; *Hartman 53278* (RM).  Steep shale cliffs above Spread Creek. 7,500' elevation; T44N R113W S19; 13 Jun 1948; *J.F. \& M.S. Reed 2302* (RM).  1.5 mi. E of Elk Ranch Reservoir. 7,300' elevation; T44N R114W S3; 22 Jun 1971; *Dorn 1281* (RM).  Grand Teton National Park and Vicinity, Jackson Hole; "Wolff Ridge" on the N side of Spread Creek Valley, ca. 3.5 air mi. SW of Moran; ca. 25 air mi. NE of Jackson. 6,900-7,160' elevation; T44N R114W S9 E1/2; 15 Jun 2006; *Nelson 68918* (NY).  



***Physaria acutifolia***

\textbf{Wyoming} % WYOMING STATE SPECIMENS

\textit{Carbon}




