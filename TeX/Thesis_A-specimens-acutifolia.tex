\begin{center}
\textit{Physaria acutifolia}
\end{center}

%%% WYOMING STATE SPECIMENS
\textbf{Wyoming: Park County:} Bighorn Basin; on road between Silvertip and Elk Basin Oil Fields, SSE of Elk Basin Oil Field; ca. 14.5 air mi. NW of Powell.  	4,600' elevation;	T57N R100W S1; 21 Jun 1987; \textit{Nelson	13688} (NY).  Bighorn Basin; near the head of Spring Creek, ca. 6 air mi. S of Cody. 5,500' elevation; T52N R101W S31; 26 Jun 1983; \textit{Nelson 9944} (ISTC).  Bighorn Basin; north end of Oregon Basin, ca. 8 air mi. SE of Cody. 5,300-5,600' elevation; T52N R100W S8, S16, and S17; 25 May 1983; \textit{Hartman and Hamann 14449} (ISTC).  Along W side of Oregon Basin Rd., ca. 1.5 mi. S of Hwy. 14, 16, 20.	5250' elevation; T52N R100W S15 and S16;	13 Jun 1990; \textit{Evert 18832} (RM).  Bighorn Basin; Burlington Meeteetse Road, 16.5 mi. NE of intersection with Wyo Hwy 120. T51N R98W S14, S15; 24 May 1980; \textit{Hartman with Dueholm 11146} (ISTC, UTC).  Bighorn Basin; on the divide between Gooseberry Creek and Renner Draw, ca. 1.5 air mi. S of Gooseberry Creek; ca. 11 air mi. S of Meeteetse.	6,400' elecation; T47N R100W S33; 27 Jun 1983; \textit{Nelson 9999}	(ISTC).  \textbf{Hot Springs County:} Bighorn Basin; west side of Hillberry Rim, ca. 0.8 mi. E of WY Highway 120. 5,600' elecation;; T47N R99W S13 SW4 of NE4; 1 Jul 1998; \textit{Welp 7860} (RM).  Bighorn Basin; ridge along north side of Grass Creek Road, ca. 6 mi. W of Wyo Hwy 120. 5,800' elevation; T46N R99W S14; 27 Jun 1993; \textit{Evert 25349} (RM).  Bighorn Basin; along east side of Murphy Draw Road, ca. 3/4 mi. N of Wyo Hwy 431. 5,200' elevation; T47N R97W S28; 11 Jun 1995; \textit{Evert 28855} (RM).  Bighorn Basin; off Wyo Hwy 120 on Red Ridge, ca. 5.5 air mi. E of Grass Creek Post Office; ca. 26.5 mi. NW of Thermopolis. 5,500' elevation; T46N R97W S19; 29 Jun 1983; \textit{Nelson 10164} (NY).  Bighorn Basin; ca. 5 air mi. SE of Grass Creek; ca. 25 air mi. NW of Thermopolis. 5,500' elevation; T45N R98W S1; 3 Jun 1981; \textit{Nelson 7463} (ISTC).  Bighorn Basin; north slope of Ilo Ridge, ca. 1 mi. S of Wyo Hwy 171 and 1.2 mi. S of Grass Creek.	5,500' elevation; T45N R98W S1 NW4 of SW4 of NW4; 29 Jun 1998; \textit{Welp 7851} (RM).  Bighorn Basin; north slope of Ilo Ridge, ca. 1 mi. S of Wyo Hwy 171 and 1.2 mi. S of Grass Creek.	5,500' elevation; T45N R98W S2 SW4 of NE4; 29 Jun 1998; \textit{Welp 7855} (RM).  Bighorn Basin; ca. 11.5 air mi. SE of Grass Creek; ca. 19 air mi. NW of Thermopolis on the Cottonwood Creek Road. 5,000' elevation; T45N R97W S14; 3 Jun 1981; \textit{Nelson 7505} (NY, UTC).  E Foothills Absaroka Mountains; Mount 7049, ca. 1.25 air mi. NW of summit of Adam Weiss Peak, ca. 33.5 air mi. NW of Thermopolis. 6,600-6,800' elevation; T45N R99W S6	SE4NW4; 9 Jul 1992; \textit{Fertig 12949a} (RM).  S Bighorn Basin; ridge on western rim of Wagonhound Bench, ca. 27 air mi. NW of Themopolis. 5,740-6,060' elevation; T44N R99W S1, S2, S3; 10 Jul 1992; \textit{Fertig 12980} (RM).  Foothills of the Absaroka and Owl Creek Mountains; west end of Padlock Rim, ca. 3 air mi. SE of Hamilton Dome, ca. 17 air mi. WNW of Thermopolis. 5,300' elevation; T44N R97W S29, S31 NE4, and S32 NW4; 6 Jul 1992; \textit{Fertig 12896} (RM).  S. Big Horn Basin; Sand Draw Ridge, ca. 13.5 air mi. NNW of Thermopolis. 4,760-4,900' elevaiton; T44N R96W S3 SE4 and S10 NE4; 3 Jul 1992; \textit{Fertig 12855} (RM).  Bighorn Basin; ca. 2 mi. W of Gebo (abandoned townsite), ca. 12 mi. N of Thermopolis. 4,500' elevation; T44N R95W S9; 7 Jun 1995; \textit{Evert 28732} (RM).  Bighorn Basin; ca. 1 mi. W of Gebo; ca. 10.5 air mi. NNW of Thermopolis.	4,500' elevation; T44N R95W S10; 18 Jun 1987; \textit{Nelson 13540} (UTC).  Bighorn Basin; ca. 10 air mi. E of Thermopolis, chugwater cliffs and adjacent plains; T42N R93W S3 and S10; 28 May 1981; \textit{Hartman and Dueholm 12821} (UTC).  Bridger Creek Road, 0.25 mi. N of Fremont County line. 5,740' elevation; T41N R90W S34 SW, SE; 	28 Jun 1981; \textit{Martin 1575} (RM).  20 miles SQ of Grass Creek. 14 Jul 1964; \textit{Despain 28} (RM).  Red Canyon. 5,200' elevation; T42N R92W; 21 May 1979; \textit{Lichvar 1671} (NY).  Owl Creek Mountains; South Fork Owl Creek, north rim of canyon below Anchor Dam, ca. 30 air mi. W of Thermopolis. 6,800' elevation; T43N R100W S26	SE1/4 NE1/4; 21 Jun 1991; \textit{Marriott 11356} (RM).  Wind River Indian Reservation, south flank of the Owl Creek Mountains. 6,400' elevation; T7N R1W S23; 24 Jun 1982; \textit{Lichvar 5177} (RM).  Boysen Dam, on the Wind River. 4,800' elevation; 26 Jun 1960; \textit{Porter and Porter 8205} (RM).  Wind River Basin; slopes on west side of Boysen Reservoir, ca. 0.25 mi. W of Power House at Boysen Dam. 5,200' elevation; T5N R6E S8	SE1/4 SW1/4 SE1/4; 12 Jun 1993; \textit{Fertig 13849} (RM).  Cedar Ridge and Barren Hills to the N. 5,600-6,000' elevation; T39N R92W S9, S16, and S21; 28 Jun 1981; \textit{Hartman and Dueholm 13232} (RM).  Bridger Mountains; E end of Copper Mountain N of Point of Mountain Road, ca. 16 air mi. N of US Hwy 20-26. 6,200-6,600' elevation; T40N R91W S32	S1/2 NW1/4; 6 Jun 1996; \textit{Fertig 16544} (RM).  Cottonwood creek, 7 air mi. NE of Lysite. 5,400-5,600' elevation; T39N R90W S14 and S15; 28 May 1985; \textit{Hartman 19963} (RM).  Lysite Badlands, Badlands draining north into Alkali Creek; ca. 2.5 air mi. W of Lysite. 5,350' elevation; T38N R91W S15	N1/2; 17 Jun 1986; \textit{Marriott 10133} (RM).  Lysite Badlands, Pony CR \#1 gas line site just W of Moneta-Lysite rd, 3.3 rd. mi. N of Moneta. 5,800' elevation; T37N R91W S2	E1/2; 16 Jun 1986; \textit{Marriott 10089} (RM).  Edge of (badlands) Moneta Hills; ca. 3.75 air mi. NNE of Moneta. 5,630-5,730' elevation; T37N R91W S1; 20 May 1986; \textit{Haines 5939} (RM).  Lysite Badlands, E of Moneta-Lysite rd. ca. 5 air mi. ENE of Moneta, badlands and ridge to south. 5,650' elevation; T37N R90W S9	E1/2; 18 Jun 1986; \textit{Marriott 10170} (RM).  Burma Road, ca. 5.8 air mi. N of jct. US Hwy 26 and Wyo Hwy 789 in Riverton. 5,000-5,200' elevation; T2N R4E S34; \textit{Hartman and Haines 20054} (RM).  Flats, across US Hwy 26 from Paradise Valley Road, ca. 6 air mi. WNW of Riverton. 5,300' elevation; T1N R3E S11 and S14; 15 Jun 1986; \textit{J. Haines with G. Haines 6480} (RM, GH).  Steep slopes below sandstone ledges along the Oil Springs Road, 7 mi. N of Hwy 130. 5,700' elevation; T34N R95W S30; 30 May 1985; \textit{Scott 4159} (RM).  Gas hills mining district; ca. 2.5 air mi. E of the intersection of Wyo 136 and Ore Road (Main Road through gas hills mining district). 6,480' elevation; T33N R90W S24; 21 Jun 1985; \textit{Haines 4281} (RM).  On the eroded badlands hills near Lysite. 5,500' elevation; 10 Jul 1951; \textit{Porter 5741} (RM).  \textbf{Natrona County:} SE Bighorn Mountains, north slope of Cedar Ridge, S of Badwater Creek, ca, 1.5-2 air mi. NW of Badwater. 6,200-6,700’ elevation; T39N R89W S24; 13 Jun 1993; \textit{Fertig 13877} (RM).  South Fork of Sand Creek, ca. 7.25 air mi. E of Lost Cabin. 5,920-6,000’ elevation; T38N R89W S11 and S14; 1 Jun 1986; \textit{J. Haines with G. Haines 6063} (RM, GH).  Cedar Ridge, ca. 13 air mi. NNW of Hiland. 6,100-6,825’ elevation; T38N R88W S5 and S6, T39N R88W S31 and S32; 10 Jul 1985; \textit{Haines 4792} (RM). SE Bighorn Mountains, southeast slope of Cedar Ridge, ca. 2.5 air mi. S of Badwater. 6,500-6,740’ elevation; T38N R88W S8 and S9; 13 Jun 1993; \textit{Fertig 13861} (RM).  Southern Powder River Basin, ca. 6 air mi. NW of Arminto. 6,230’ elevation; T38N R87W S19 SE4 and S30 NE4; 6 Jul 1993; \textit{Hartman and Nelson 40363} (RM, MONTU).  Bad Water. 6 Jun 1910; \textit{Nelson 9403} (RM).  5 air mi. NNE of Arminto. 6,300’ elevation; T38N R87W S24; 12 Jul 1979; \textit{Hartman 10048} (RM).  SE foothills Big Horn Mountains, draw due W of E-K Creek and ridge on east side of creek, ca. 5.5 air mi. NNE of Arminto. 6,500’ elevation; T38N R87W S13 SE1/4; 8 Jun 1992; \textit{Fertig 12639} (RM).  E-K Creek, ca. 7.5 air mi. NNE of Arminto. 6,520’ elevation; T38N R86W S7; 1 Jun 1986; \textit{J. Haines with G. Haines 6012} (RM, GH).  Southern Powder River Basin, along Baker Cabin or County Road 108 between North Fork Cottonwood Creek and Gray Wall, ca. 12 air mi. NE of Arminto, ca. 51 air mi. NW of Casper. 6,340-6,490’ elevation; T39N R86W S25 W2; 23 May 1994; \textit{Nelson 30628} (RM).  Ca. 24 air mi. N Powder River. 6,400’ elevation; T39N R85W S4 and S9; 3 Jul 1979; \textit{Dueholm 7727} (RM).  6,500' elevation; T40N R85W S11 S1/2;	43º27'N 107º01' W; 29 May 1999; \textit{Dorn 7931} (RM).  NW Rattlesnake Hills, slopes adj. to Poison Spider Road, ca. 1 mi. SE of junction with Gas Hills Road. 6,800-7,000’ elevation; T34N R88W S32 E1/2; 15 Jun 1993; \textit{Fertig 13926} (RM).  Wallace Creek; 29 Jul 1894; \textit{Nelson 674} (NY).  Southern Powder River Basin; Blue Canyon at south end of Pine Mountain, ca. 28 air mi. WNW of Casper.	5,750-5,900' elevation; T34N R84W S13	N1/2 and S24 NE1/4; 24 Jun 1993; \textit{Hartman 38283} (RM, UC).  Ca. 14.5 mi. SE Powder River. 5,700’ elevation; T33N R83W S33; 5 Jul 1979; \textit{Dueholm 7850} (RM). 6,400; T30N R83W S19 N1/2; 20 Jun 1979; \textit{Dorn 3227} (RM).  33 mi. SW of Casper. 5,900’ elevation; 2 Jun 1963; \textit{Mulligan and Mosquin 2791} (RSA-POM).  Satanka Formation, 4 mi. W of Alcova. 9 May 1948; \textit{Porter 4427} (RM, RSA-POM).  Near Alcova, WY.  5,300’ elevation; 10 May 1980; \textit{Tresler 484} (RM).  North Platte River Basin, slopes on N side of Platte River, 0.5 mi. S of WY Hwy 220, ca. 1.25 air mi. NE of Gray Reef Dam. 5,500’ elevation; T30N R82W S8 NE1/4 SW1/4 and SE1/4 NW1/4; 5 Jun 1993; \textit{Fertig 13791} (RM).  North Platte River Basin, slopes on west side of Platte River, ca. 21 air mi. SW of Casper. 5,400’ elevation; T31N R82W S23 W1/2; 5 Jun 1993; \textit{Fertig 13790} (RM).  Satanka formation, about 10 mi. SE of Alcova. 28 Jun 1950; \textit{Porter 5400} (RM, NY).  Near top of pass, between Alcova and Independence Rock, Wyo Hwy 220. 21 May 1979; \textit{Rollins and Rollins 7920} (NY).  One mi. N of Alcova. 21 May 1979; \textit{Rollins and Rollins 7919} (NY).   North of Casper Mt., 20.5 mi. NE of Alcova on Wyo. Hwy 220. 21 May 1979; \textit{Rollins 7914} (RM, NY, UC, US, MO).  13 mi. SW of Casper. 5,200’ elevation; 4 Jun 1963; \textit{Mulligan and Mosquin 2792} (UC, UTC).  N. Laramie Range, south slope of Casper Mountain along ridge adj. to tributary of Little Red Creek and Casper Mountain Loop Rd. 7,000-7,200’ elevation; T32N R79W S32; 9 Jul 1993; \textit{Fertig 14050} (RM).  Along road cut on road to Muddy Mountain. 7,000’ elevation; T31N R79W S4; 20 Jul 1987; \textit{Nelson and Nelson 8681} (RM).  29 mi. S of Casper. 8 Jul 1965; \textit{Mulligan and Crompton 3083} (RM).  Twin Buttes. 6,500’ elevation; T30N R80W S34 NW1/4; 9 Jun 1981; \textit{Lichvar 4334} (RM).  Bates Hole. 20 Jun 1920; \textit{Payson and Payson 4781} (RM, MO).  North Platte River Basin; low ridges ca. 4.5 air mi. E of Alcova-Seminoe Scenic highway, ca. 1.25 air mi. N of Carbon County line. 7,000’ elevation; T29N R82W S34 NE1/4; 11 Jul 1993; \textit{Fertig 14070} (RM).  East rim of Bates Hole, "Castle Rock" cliffs by Rimo Point ca. 2 mi. E of Wyo Hwy 487, ca. 3 mi. N of jct. with Hwy 77. 7,350’ elevation; T29N R79W S3 NW1/4 SW1/4; 21 Jun 1997; \textit{Fertig 17550} (RM).  West slope Laramie Range, divide between Mud Springs Draw and Chalk Creek, ca. 1.2 mi. S of the confluence of Chalk and Bates creeks, ca. 8.5 air mi. E of Wyo Hwy 487. 6,940-7,000' elevation; T29N R79W S3 NW1/4 SW1/4; 20 Jun 1997; \textit{Fertig 17547} (RM).  Bates Creek. 5 Jul 1901; \textit{Goodding} (RM).  \textbf{Sublette County:} Green River Basin; Red Canyon, ca. 18.5 air mi. WNW of Big Piney. 8,150-8,200’ elevation; T31N R114W S15 W1/4; 22 Jul 1995; \textit{Cramer and Kellett 9011b} (RM).  Ca. 11.5 air mi. SW of Boulder. 7,200-7,300’ elevation; T31N R109W S19 SW1/4; 7 Jul 1995; \textit{Cramer 7446} (RM).  North side of Ross Butte overlooking New Fork River, ca. 10 air mi. E of Big Piney. 6,800-7,480’ elevation; T30N R110W S13 SE1/4 and S14 W1/4; 11 Jun 1994; \textit{Cramer 310} (RM). North end of Ross Butte, ca. 0.5 mi. S of the New Fork River and ca. 20 air mi. S of Pinedale. 7,100-7,460’ elevation; T30N R110W S13 N1/2 of SW4 and S14 N1/2 of SE1/4; 9 Jul 1995; \textit{Fertig 15910} (RM).  Badlands at base of east slope of southern lobe of Ross Butte, ca. 1 mi. E of the New Fork River, ca. 3.2 mi. ENE of confluence of the New Fork and Green rivers. 7,300’ elevation; T30N R110W S24 NW1/4 of SW1/4 of NW1/4; 30 May 1997; \textit{Fertig 17415} (RM).  North Alkali Draw, ca. 15 air mi. E of Big Piney. 7,200’ elevation; T30N R109W S33 SE1/4; 25 Jun 1995; \textit{Cramer and Hartman 6979} (RM).  Ca. 1.5 air mi. E of Marbleton. 6,900-7,000’ elevation; T30N R111W S28 NW1/4; 21 Jun 1995; \textit{Cramer and Kellett 6545} (RM).  Ca. 5 air mi. ESE of Big Piney. 6,900-7,000’ elevation; T29N R111W S11 E1/4; 8 Jul 1995; \textit{Cramer 7472} (RM).  Ca. 9 air mi. SW of Ross Butte. 7,100-7,200’ elevation; T29N R110W S30; 25 Jun 1995; \textit{Hartman 51227} (RM).  Lower part of Chapel Canyon, ca. 11 air mi. NE of La Barge. 6,850-7,130’ elevation; T28N R111W S27 W1/4; 12 Jun 1994; \textit{Cramer 403} (RM).  Ca. 4 mi. N of LaBarge on crest of ridge, Bird Canyon. 2,040m elevation; T27N R112W S15; 6 Jun 1993; \textit{R. Kass and Kass 3706} (RM).  Bess Canyon. 6,800’ elevation; T27N R112W S22 SE1/4; 15 Jun 1993; \textit{Kass 3757} (RM).  North rim of Dry Basin just off Calpet Road, ca. 10 air mi. SW of Big Piney. 7,480-7,600’ elevation; T29N R113W S27 E1/2; 28 Jun 1993; \textit{Nelson and Nelson 26591 and 26618} (RM).  Cretaceous Mountain; A branch of Deloney Canyon above Dry Basin, ca. 13.5 air mi. SW of Big Piney. 7,800-8,160’ elevation; T28N R114W S12 NE1/4; 23 Jun 1993; \textit{Nelson 26208} (RM). Fogarty Canyon at the south end of Cretaceous Mountain, ca. 14 air mi. SW of Big Piney. 7,480-7,680’ elevation; T28N R113W S18 SW1/4; 28 Jun 1993; \textit{Nelson and Nelson 26681} (RM).  Cretaceous Mountain/Hogsback Ridge area; north end of Big Mesa above Dry Piney Creek, ca. 13.5 air mi SW of Big Piney.	7,680-7,930' elevation; T28N R113W S28	SE1/4; 28 Jun 1993; \textit{Nelson and Nelson 26706} (RM).  Wyoming Range; Ridge on E side of LaBarge Creek just N of confluence of Packsaddle Creek, ca. 1 mi. E of Scaler Cabin (Guard Station). 8,000’ elevation; T28N R115W S29 E1/2 of NW/4; 10 Aug 1995; \textit{Fertig 16258} (RM).  Formation just NW of Squaw Teat, ca. 4 air mi. S of Elkhorn Junction. 7,200-7,537' elevation; T28N R104W S27 NW1/4; 20 Jun 1994; \textit{Cramer 960} (RM).  Lower end of Wildcat Canyon, ca. 3 air mi. WSW of Clara Birds Nipple (Bird Nipple); ca. 15.5 air mi. SSW of Big Piney. 7,150’ elevation; T27N R113W S10 SE1/4; 28 Jun 1993; \textit{Nelson and Nelson 26777} (RM).  Saddle ridge. 7,400’ elevation; T27N R113W S8 NE1/4; 7 Jun 1993; \textit{Kass and Kass 3716} (RM).  Cretaceous Mountain / Hogsback Ridge Area; northeast side of the Hogsback, ca. 18 air mi. SW of Big Piney. 7,900-8,600’ elevation; T27N R113W S7 SC; 25 Aug 1993; \textit{Hartman 44960} (RM).  East side of Hogsback Ridge above Calpet, ca. 7.5 air mi. W of La Barge. 7,600-8,300’ elevation; T27N R113W S19 S1/4; 12 Jun 1994; \textit{Cramer 492} (RM).  \textbf{Lincoln County:} Ca 1.8 mi. due E of Id-Wyo state line, as confluence of Shale Hollow w/ Salt Cany. 6,400' elevation; T28N R119W S19 SE and S20 SW; 1 Jul 1986; \textit{Franklin 3667} (RM, NY).  Salt River Range; along Smiths Fork at and above the mouth of Hobble Creek, ca, 20 air mi. N of Cokeville; ca. 44 air mi. NNW of Kemmerer. 6,950-7,400' elevation; T28N R118W S27	SW1/4 and S28 E1/4; 30 Jun 1995; \textit{Nelson 35958} (RM).  Green River Basin; ca. 15.5 air mi. W of La Barge at the south end of Fontenelle Hogbacks. 8,160-8,280' elevation; T26N R115W S28	SE1/4; 3 Aug 1995; \textit{Cramer, Kellett and Laster 10269} (RM).  La Barge Creek. 7,200' elevation; T27N R114W S32	NW1/4; 27 May 1981; \textit{Lichvar 4207} (RM).  LaBarge Creek, on slope on northeast side of drainage, ca. 7.5 air mi. WSW of LaBarge. 7,000’ elevation; T26N R113W S18 NW1/4 SW1/4; 18 Jun 1988; \textit{Marriott and Horning 10827} (RM); 7,300’ elevation; T26N R113W S19 NE1/4 NW1/4; 18 Jun 1988; \textit{Marriott and Horning 10828} (RM).  Cretaceous Mountain / Hogsback Ridge Area; southeastern corner of Hogsback Ridge, ca. 22 air mi. SW of Big Piney. 7,200-7,600'elevation; T26N R113W S8	NW1/4; 13 Aug 1993; \textit{Hartman 43802} (RM).  Ca. 5.9 mi. W of Hwy 189, on LaBarge Cr. Road. 6,200' elevation; T26N R113W S20; 5 Jun 1987; \textit{Atwood 12830} (NY).  Green River Basin; ca. 2 air mi. E of La Barge. 7,200' elevation; T26N R112W S9 NE1/4; 8 Jul 1995; \textit{Cramer 7519} (RM).  Ca. 3.5 mi. SE of LaBarge. 6,800' elevation; T26N R112W S21 SE1/4 NE1/4; 1 Jul 1993; \textit{Kass 3794} (RM).  Green River Basin; ca. 7 air mi. W of La Barge. 7,000-7,100' elevation; T25N R112W S1 NW1/4 and S2 NE1/4, T25N R112W S35 SW1/4 and S36 SE1/4; 22 May 1994; \textit{Hartman, Cramer, and Refsdal 45172} (RM, NY).  Shale cliff near the Green River, 6 mi. S of Labarge. 6,500' elevation; 19 Jul 1939; \textit{Rollins and Munoz 2867} (CAS).  Green River Basin; ca. 15.5 air mi. SSW of La Barge. 6,900' elevation; T24N R114W S14 W1/2; 21 Jun 1995; \textit{Cramer and Kellett 6602} (RM).  Green River Basin; south end of Fontenelle Reservoir, ca. 2 air mi. NW of Fontenelle Dam. 6,550' elevation; T24N R112W S23 SE1/4; 5 Jul 1995; \textit{Cramer 7312} (RM).  Green River Basin; Slate Creek Butte, ca. 5 air mi. SW of Fontenelle Dam. 6,750' elevation; T23N R112W S14 W1/2; 21 Jun 1995; \textit{Cramer and Kellett 6653} (RM).  Green River Basin; ca. 13 air mi. N of Opal. 6,720-6,760' elevation; T23N R113W S20 SE1/4; 11 Jul 1995; \textit{Cramer and Kellett 7754} (RM).  Overthrust Belt; Hams Fork Plateau on south side of Robinson Creek Canyon, ca. 2.5 air mi. W of Kemmerer Reservoir, ca. 13 air mi. NW of Kemmerer, ca. 1 mi. N of Hams Fork Road (near emigrant graves). 7,500' elevation; T23N R117W S29 NW1/4 of SE1/4; 2 Jul 1995; \textit{Fertig 16740} (RM).  Southern Salt River Range and Vicinity; Southern Sublette Range, ca. 7 air mi. N of Cokeville. 6,140-7,600' elevation; T25N R119W S4 S1/4, S8 NE1/4, and S9 N1/4; 28 Jun 1994; \textit{Cramer and Kellett 1227} (RM).  Steep shaley road-cut, US Hwys 30 and 89, 7 mi. S of the Idaho State Line. 6,200-6,300' elevation; T25N R119W S21; 22 Jun 1986; \textit{Rollins and Rollins 8683} (RM, NY, GH, UTC).  Cokeville. 11 Jun 1898; \textit{Nelson 4637} (RM, UC).  Southern Salt River Range and Vicinity; Snow Hollow, ca. 1.5 air mi. E of Idaho, ca. 5.5 air mi. SW of Cokeville; ca. 29 air mi. NW of Kemmerer.  6,800-6,900' elevation; T24N R120W S35 NW1/4; 8 Jul 1995; \textit{Nelson and Refsdal 36568} (RM).  Basins and Mountains of Southwest Wyoming; ca. 4.5 air mi. SW of Opal; 4.7 mi. S on Wagon Wheel Road off of US Hwy 30; 6,680-6,720' elevatin; T20N R115W S14 E1/2; 07/01/1995; \textit{Refsdal and Refsdal 4755} (RM).  Basins and Mountains of Southwest Wyoming; benchland between Zieglers Wash and Dry Muddy Creek ca. 13 air mi. WNW of Granger, ca. 22 air mi. SE of Kemmerer. 6,590-6,680' elevation; T20N R113W S32 SE1/4 and S33 SW1/4; 3 Jul 1995; \textit{Nelson and Refsdal 36196} (RM).  Basins and Mountains of Southwest Wyoming; benchland between Zieglers Wash and Dry Muddy Creek ca. 9 air mi. WNW of Granger, ca. 24.5 air mi. SE of Kemmerer. 6,460-6,500' elevation; T19N R113W S12 SW1/4; 3 Jul 1995; \textit{Nelson and Refsdal 36072} (RM).  Basins and Mountains of Southwest Wyoming; benchland between Zieglers Wash and Dry Muddy Creek ca. 9 air mi. WNW of Granger, ca. 24.5 air mi. SE of Kemmerer. 6,460-6,500' elevation; T19N R113W S12 SW1/4; 3 Jul 1995; \textit{Nelson and Refsdal 36073} (RM).  Basins and Mountains of Southwest Wyoming; ca. 12 air mi. SE of Opal, 18.3 mi. E of jct. US Hwy 30 / Wyo Hwy 240. 6,400-6,500' elevation; T20N R112W S28 E1/2; 30 Jun 1995; \textit{Refsdal 4638} (RM, NY).  Little Muddy Creek, 5 mi. W of US 189, 14 mi. SW of Kemmerer. 7,000' elevation; T19N R117W S29; 17 Jul 1982; \textit{Atkins, Neely, and Carpenter 8238} (UTC).  14 mi. S of Kemmerer. 6,500' elevation; T19N R116W S33 E1/2; 27 May 1994; \textit{Dorn 5592} (RM, MO).  Overthrust Belt; south slopes of Hanks Hill at north end of Woodruff Narrows Reservoir, ca. 1 mi. E of the WY-UT state line, ca. 10 air mi. N of Evanston. 6,500-6,760' elevation; T18N R120W S29 SW1/4 SW1/4, S30 SE1/4, S31 N1/2 of NE14, and S32 NW1/4 of NW1/4; 22 May 1996; \textit{Fertig 16460} (RM).  Basins and Mountains of Southwest Wyoming; north end of Woodruff Narrows Resevoir, ca. 16 air mi. N of Evanston, 8.5 mi. E on County Road 101. 6,460-6,750' elevation; T18N R120W S32 NW1/4, S31 NE1/4, S29 SE1/4, and S28 SW1/4; 22 Jun 1995; \textit{Refsdal 4228} (RM).  US 189, 8.7 road mi. NE of I-80. 7,000' elevation; T17N R117W S34; 6 Jul 1983; \textit{Hartman 15736} (RM).  18.5 mi. SSW of Diamondville. 6,650’ elevation; 11 Jul 1965; \textit{Mulligan and Crompton 3095} (MONT).  Basins and Mountains of Southwest Wyoming; ca. 6 air mi. SE of Cumberland Gap, ca. 14.2 road mi. N of I-80 on east side of Wyo Hwy 412. 6,700-6,860' elevation; T18N R116W S28 W1/2; 19 May 1994; \textit{Refsdal and Atwood 200, 201} (RM).  Ca. 6 air mi. SE of Cumberland Gap, ca. 14.2 road mi. N of I-80 on east side of Wyo Hwy 412. 6,700-6,860' elevation; T18N R116W S28 W1/2; 19 May 1994; \textit{Refsdal and Atwood 202} (RM).  Overthrust Belt; SW end of the "Carter Cedars" along Wyo Hwy 412, ca. 4 air mi. NW of Carter, ca. 6.5 mi. E of US Hwy 189. 6,800' elevation; T17N R116W S2 NE1/4 of SW1/4; 18 Jun 1997; \textit{Fertig 17520, 17521} (RM).  Basins and Mountains of Southwest Wyoming; Wildcat Butte between Church Butte Road and I-80 at Sweetwater County, ca. 14.8 air mi. NE of Lyman, ca. 49 air mi. ENE of Evanston.  6,820-6,980' elevation; T17N R112W S22 NW1/4; 18 Jun 1995; \textit{Nelson and Refsdal 35212} (RM).  Basins and Mountains of Southwest Wyoming; ca. 12 air mi. NE of Lyman. 6,940-7,000' elevation; T16N R113W S1 S1/2; 22 Jun 1995; \textit{Refsdal 4313} (RM).  6 mi. E of Lyman. 6,600’ elevation; 19 Jun 1956; \textit{Porter 7005} (RM).  9 mi. E NE of Fort Bridger. 6,500’ elevation; 7 Jul 1965; \textit{Mulligan and Crompton 3080} (CAS).  6 mi. E of Lyman. 3 Jun 1970; \textit{Rollins 79152} (NY, US).  Sandy ravine near Blacks Fork River, 3 mi. N of Lyman.	6,500' elevation; 10 Jun 1937; \textit{Rollins 1653} (RM, NY, UC, MO).  Lyman. 15 Jun 1932; \textit{Rollins 182} (RM, MO).  S of Carter. 7,000' elevation; T17N R115W S34 NE1/4; 4 Jun 1980; \textit{Lichvar 2777} (RM).  6 mi. N Ft. Bridger. 6,500' elevation; 13 Jun 1938; \textit{Rollins 2316} (RM).  2 mi. W of Fort Bridger. 6,700’ elevation; 21 Jul 1963; \textit{Mulligan and Crompton 2785} (UTC).  2 mi. W of Fort Bridger. 6,700’ elevation; 7 Jul 1965; \textit{Mulligan and Crompton 3079} (MONTU).  Ca. 5 mi. SSW of Carter. 6,600' elevation; T16N R116W S13; 12 Jun 1980; \textit{Lichvar 2866} (RM).  Foothills of Bridger Butte, 3 mi. W Ft. Bridger. 6,500’ elevation; 24 Jun 1938; \textit{Rollins 2387} (NY). 3 mi. W of Ft. Bridger, topotype.	7,000' elevation; T16N R116W S35; 24 May 1979; \textit{Lichvar 1704} (RM).  About 3 mi. W of Ft. Bridger. 7,000' elevation;	7 Jul 1977; \textit{Dorn 2974} (RM).  2 mi. W of Fort Bridger. 2,134m; 41º19'51"N, 110º25'10"W; 3 Jun 1996; \textit{O'Kane 3785} (ISTC).  Foothills of Bridger Butte, 3 mi. W Ft. Bridger. 6,500' elevation; 24 Jun 1938; \textit{Rollins 2387} (NY).  Fort Bridger, Wyoming Territory. July 1873, \textit{Porter 10462} (NY); \textit{Porter s.n.} (NY, NY, F).  Fort Bridger. 9 Jun 1898; \textit{Nelson 4602} (RM, F).  Basins and Mountains of Southwest Wyoming; along Leavitt Creek below the south end of Cottonwood Bench, ca. 7 air mi. ESE of Mountain View, ca. 39.5 air mi. E of Evanston.	6,700-6,860' elevation; T15N R114W S36; 18 Jun 1995; \textit{Nelson 35163} (RM).  8 air mi. SE of Mountainview, Leavitt Cr. 6,800' elevation; T15N R114W S36 SE1/4; 30 Jun 1982; \textit{Goodrich and Atwood 17162} (RM, NY).  Flat above barren cliffs overlooking Laevitt Creek, 1 km (0.6 mi.) S of Wyo Hwy 414, 11.5 km (7mi.) air distance east-southeast of Mountain View. 6,800' elevation; 2,075m; 41º13'45"N, 110º12'56"W; 23 May 1999; \textit{Holmgren and Holmgren 13447} (ISTC, NY, UTC).  Grizzly Buttes, Canyonlands and erosional badlands near Mountainview. 6,800' elevation;	T14N and 15N R114W S2 and 36; 13 Jul 1973; \textit{Hill 881} (RM).  Sage Creek Mountain, ca. 12 air mi. SE of Mountain View. 7,200' elevation; T14N R113W S20 SE and S21 SW; 12 Jun 1981; \textit{Dueholm 11434} (RM, NY).  Basins and Mountains of Southwest Wyoming; east end of Sage Creek Mountain, ca. 5.3 air mi. N of Lonetree. 8,200-8,420' elevation; T13N R113W S2 NW1/4, T14N R113W S35 S1/2; 23 Jul 1995; \textit{Refsdal 5887} (RM).  North Slope Uinta Mountains; Hickey Mountain, ca. 5.5 air mi. NW of Lonetree. 7,480-8,000' elevation; T13N R114W S12; 22 Jun 1994; \textit{Refsdal and Fertig 1047} (RM).  5 mi. N 25 dg W of Lonetree, E side Hickey Mtn. 7,800' elevation; T13N R113W S18 SE1/4; 30 Jun 1982; \textit{Goodrich and Atwood 17171} (RM, NY).  Hickey Mountain, one mi. N of State Hwy 414; 20 Jun 1986; \textit{Rollins and Rollins 8670} (RM, NY, GH, UTC, MONTU).  Uinta County Road 290, 3.7 air mi. W of Lonetree. 7,800' elevation; T12N R114W S1; 7 Jul 1983; textit{Hartman 15766} (RM).  Clay knolls and hillsides, County Road 290, 4 mi. W of Lonetree. 19 Jun 1986; \textit{Rollins and Rollins 8666} (RM).  Basins and Mountains of Southwest Wyoming; Cedar Mountain, ca. 3 air mi. NE of Lonetree; ca. 3.3 road mi. E of Cedar Mountain Road from Wyo Hwy 414, west flank of the mountain. 7,700-7,800' elevation; T13N R113W S22 W1/2 and S15 S1/2; 11 Jun 1994; \textit{Refsdal and Lathrop 725} (RM).  SW side of Cedar Mtn. 7,700' elevation; T13N R113W S24 S1/2; 28 Jun 1999; \textit{Dorn 7997} (RM).  Basins and Mountains of Southwest Wyoming; ca. 6.5 air mi. NNW of Lonetree, ca. 1.2 road mi. SW of Wyo Hwy 414. 7,260-7,410' elevation; T13N R113W S26 W1/2; 22 Jun 1994; \textit{Refsdal and Fertig 1023} (RM).  North Slope Uinta Mountains; ca. 2 air mi. E of Lonetree, ca. 8.0 road mi. E of the junction of County Road 1 with Wyo Hwy 414 on south side of Wyo Hwy 414. 7,400-7,600' elevation; T12N R113W S1 SW1/4 and S2 SE1/4; 7 Jun 1994; \textit{Refsdal 517} (RM).  Hoop Lake Road (Uinta County Road 295), 4 air mi. S of Lonetree. 7,800' elevation; T12N R113W S21; 7 Jul 1983; \textit{Hartman 15761} (RM).  North Slope Uinta Mountains; ca. 4 air mi. S of Lonetree, just N of Utah on Hoop Lake Road. 7,900-8,000' elevation; T12N R113W S21 S1/2; 7 Jun 1994; \textit{Refsdal 549} (RM).  28 mi. W SW of Green River. 6,625’ elevation; 11 Jul 1965; \textit{Mulligan and Crompton 3096} (NY).  \textbf{Sweetwater county:} Top of Cedar Mtn. T13N R112W S28 NE1/4; 1 Jul 1982; \textit{Atwood and Goodrich 9053} (NY).  Basins and Mountains of Southwest Wyoming; Cedar Mountain, ca. 5 air mi NW of McKinnon, ca. 0.5 road mi. N of Turtle Bluff Rim Road. 8,380-8,560' elevation; T13N R112W S22 NW1/4 and S21 NE1/4; 22 Jun 1994; \textit{Refsdal 1010} (RM).  Cedar Mountain, ca. 5 air mi. NW of McKinnon; ca. 0.5 road mi N of Turtle Bluff Rim Road. 8,340-8,500' elevation; T13N R112W S22 NW1/4 and NE1/4 S21; 22 Jun 1994; \textit{Refsdal and Fertig 986} (RM).  Basins and Mountains of Southwest Wyoming; ca. 0.5 to 1 air mi. N of Henry's Fork W of County Road 1 on lower southeast slopes of Cedar Mountain, ca. 40 air mi. SW of Green River, ca. 11 air mi. NW of Manila, Utah. 7,000-7,100' elevation; T13N R111W S32; 18 Jun 1995; \textit{Nelson and Refsdal 35135} (RM).  Ca. 6.2 air mi. N of McKinnon, ca. 6.1 mi. N on County Road 1 from Wyo Hwy 414. 7,000-7,100' elevation; T13N R111W S14 SW1/4 and S15 SE1/4; 13 Jul 1995; \textit{Refsdal 5289} (RM).  Southeast slope of Cedar Mountain. 8,000' elevation; 28 Jun 1951; \textit{Rollins and Porter 5136} (RM, NY, US).  Flaming Gorge; N of Linwood Canyon, ca. 6 air mi. NE of Manila, Utah. 6,660' elevation; T12N R108W S18 SE1/4; 9 Jun 2011; \textit{Heidel 3520} (RM).  Flaming Gorge National Recreation Area; ca. 8 air mi. NW of Dutch John, 2.1 mi. W of the Forest Boundary. 6,400' elevation; T12N R108W S14 NE1/4; 7 Jun 1995; \textit{Refsdal and Goodrich 3782} (RM).  South-central Wyoming; ridge E of Muddy Creek and Wyo Hwy 789, ca. 36.5 air mi. SW of Rawlins, ca. 23 air mi. N of Baggs. 6,680-7,015' elevation; T16N R92W S10 S1/4; 6 Jun 1996; \textit{Nelson and Ward 37782} (RM).  Ca. 3 airline mi. NNW of The Gap, NW of Dutch John.	3 Jun 1980; \textit{Atwood 7542} (UTC).  Just across state line from Utah on Highway 191. 2,179m; 41º01'21"N, 109º25'09"W; 2 Jun 1996; \textit{O'Kane 3778} (ISTC, MO).  Lower Henry's Fork, 10 mi. N of Manila, Utah. 16 May 1966; \textit{Tresler 268} (RM).  Green River Basin; ridge system between south bank of North Fork Anvil Wash and summit of NE end of Black Mountain, ca. 2 mi. NE of Twin Buttes, ca. 5.25 air mi. W of WY Hwy 530.  6,840-7,300' elevation; T14N R109W S33 NE1/4 and S28 E1/2; 2 Jun 1995; \textit{Fertig 15718} (RM).  Black Mountain. 7,500' elevation; T14N R109W S21 SE1/4; 13 Jun 1988; \textit{Atwood 13322} (RM).  North end of Black Mountain and Pine Spring area, ca. 26 air mi. S of Green River. 6,800-7,900' elevation; T14N R110W S13 S1/2, S24, and S25 NE1/4; T14N R109W S19 W1/2 and S30; 24 May 1994; \textit{Hartman, Cramer, Refsdal 45215} (RM).  Basins and Mountains of Southwest Wyoming; ca. 19 air mi. SW of Green River, 1.1 mi. W on BLM Road 4315 (Burnt Fork Road).  6,460-6,680' elevation; T15N R110W S26 S1/2 and S35 N1/2; 14 Jun 1995; \textit{Refsdal 3975} (RM). 25 mi. SW of Green River on road to Manila, Utah. 25 Jun 1950; \textit{Ownbey 3250} (UC).  Ca. 13.5 air mi. SW of Green River, 3.9 mi. W of Wyo Hwy 530 on 2 track. 6,240-6,360' elevation; T16N R109W S27 NW1/4; 10 Jun 1995; \textit{Refsdal 3818} (RM).  Ca. 11 air mi. W of Green River; ca. 2.4 road mi. S of I-80; 6,120-6,200' elevation; T18N R109W S26; 10 Jun 1994; \textit{Refsdal 647} (RM).  Granger. 13 Jun 1898; \textit{Nelson 4688} (RM).  South-central Wyoming; northeastern slopes and draws of Richards Mountain, ca. 2 air mi. NW of Richards Gap, ca. 37 air mi. SSW of Rock Springs. 6,850-7,500' elevation; T12N R105W S8 S1/2, S9 SW1/4, and S17 NE1/4; 5 Jul 1996; \textit{Ward 1960} (RM).  Red Creek Basin; adj. to Richard's Gap road at junction with pipeline road ca. 0.25 mi. N of Daniels Creek, ca. 2.5 mi. N of Utah State Line. 6,760' elevaiton; T12N R105W S3 SE1/4 SW1/4 and S10 NE1/4 NW1/4;	27 May 1993; \textit{Fertig 13671, 13648} (RM).  Red Creek Basin. 6,700' elevation; T13N R105W ca. S36 NW1/4; 3 Jul 1999; \textit{Dorn 8013} (RM).  E of Little Mountain, ca. 35 air mi. S of Rock Springs. 6,960' elevation; T13N R105W S26 NE1/4 of NE1/4; 12 Jun 1979; \textit{Aldrich 103} (RM).  Red Creek Badlands. 6,900' elevation; T13N R105W S24; 10 Jul 1981; \textit{Dueholm 11728} (RM).  Washakie Basin; ridges and flats between Richard's Gap and Tepee Mountain. 6,900-7,300' elevation; T12N R104W S18; T12N R105W S13, S14, and S15; 23 May 1981; \textit{Hartman and Dueholm 12592} (RM).  Red Creek Badlands along rim above Red Creek Ranch.	7,500-7,750' elevation; T13N R104W S23 and S26; 19 Jun 1997; \textit{Atwood 22750, 22761} (RM).  Red Creek badlands on north and east slopes of Telephone Canyon, ca. 6 air mi. N of Utah/Wyoming/Colorado border. 7,500' elevation; T13N R104W S24 SE1/4; 27 May 1993; \textit{Fertig 13627} (RM).  Potter Mountain. 7,400-8,000' elevation; T14N R103W S34 and and 28, T13N R103W S2; 7 Jul 1980; \textit{Dueholm 10415, 10416} (RM).  Green River Basin; slope from Currant Creek Ridge draining northward to Currant Creek, ca. 5 air mi. E of Flaming Gorge Reservoir, ca. 26.5 air mi. SSW of Rock Springs. 6,770-7,050' elevation; T14N R107W S2 SW1/4 and S11 NW1/4; 4 Jul 1996; \textit{Ward 1932} (RM).  Sage Creek Road (Sweetwater CO 36) 3.7 road mi. W of Wyo Hwy 430. 6,600-7,300' elevation; T15N R105W S7 and 18, R 106W S12 and S13; 14 Jun 1981; \textit{Dueholm 11511} (RM).  S of Rock Springs on east side of U.S. Hwy 191. 6,400' elevation; T16N R105W S18 NE1/4; 16 May 1994; \textit{Refsdal 153} (RM).  Flaming Gorge National Recreation Area; south peninsula at the confluence of Blacks Fork and Green River, ca. 17 air mi. S of Green River. 6,040-6,080' elevation; T15N R108W S13 NW1/4; 20 Jun 1995; \textit{Nelson, Refsdal, and Welp 35506} (RM).  Flaming Gorge National Recreation Area; small cove on east side ca. 1.5 mi. below Blacks Fork, ca. 4 air mi. NE of Buckboard Crossing. ca. 17.5 air mi. S of Green River. 6,040-6,290' elevaiton; T15N R108W S13 SE1/4 and T15N R107W S18 SW1/4; 20 Jun 1995; \textit{Nelson, Refsdal, and Welp 35277} (RM).  Flaming Gorge National Recreation Area; south peninsula at the confluence of Blacks Fork and Green River, ca. 17 air mi. S of Green River. 6,120-6,264' elevaiton; T15N R108W S14 NE1/4; 20 Jun 1995; \textit{Nelson, Refsdal, and Welp 35449} (RM).  Flaming Gorge National Recreation Area; white semibarren knoll of exposed Green River Shale. 6,200' elevation; T16N R107W S33 SE1/4 of SW1/4; 26 May 1999; \textit{Goodrich 26006} (RM, NY).  West side of Flaming Gorge Reservoir between Firehole Canyon and Sage Creek Basin, ca. 13 air mi. S of Green River. 6,040-6,240' elevation; T16N R107W S27 W1/4 and S28 E1/4; 20 Jun 1995; \textit{Nelson, Refsdal, and Welp 35418} (RM).  18 mi. 167 deg. From Green River, Flaming Gorge National Recreation Area, South Chimney Rock. 6,400-6,600' elevation; T16N R107W S23 and S24; 15 Jun 1988; \textit{Goodrich and Atwood 22529} (RM).  Flaming Gorge National Recreational Area; Firehole Canyon, ca. 18 air mi. SW of Rock Springs, southwest side of South Chimney Rock. 6,000-6,600' elevation; T16N R107W S23 SE1/4; 16 Jun 1994; \textit{Refsdal 881} (RM).  Flaming Gorge National Recreation Area; canyon N of North Chimney Rock and E of the Green River, ca. 11 air mi. S of Green River. 6,040-6,890' elevation; T16N R107W S12 S1/2 and S13 N1/2; 20 Jun 1995; \textit{Nelson, Refsdal, and Welp 35337} (RM).  Green River Basin; Slippery Jim Bottom on Flaming Gorge Reservoir, ca. 1 air mi. N of Little Firehole Canyon; ca. 14 air mi. SSW of Rock Springs. 6,040-6,140' elevation; T17N R106W S30 N1/2 and S19 S1/2; 4 Jul 1996; \textit{Ward 1852} (RM).  Rock Springs Uplift; Point 6465 on the west side of the Green River, ca. 0.75 mi. S of Cordwood Bottom, ca. 2.25 mi. ESE of Whalen Butte, ca. 3 air mi. SE of the city of Green River. 6,400' elevation; T17N R106W S8 NE1/4 of SW1/4 of SW1/4; 6 Jun 1997; \textit{Fertig 17450} (RM).  Ca. 4 air mi. SE of Green River; E of FMC picnic area on old O \& G dirt road. 6,300-6,645' elevation; T18N R106W S32 SW1/4, T17N R106W S5 NW1/4; 3 Jun 1994; \textit{Refsdal 473} (RM).  Wyo Hwy 530, 1.6 mi. S of Green River (bridge south of town). 6,600' elevation; T18N R107W S28; 5 Jun 1971; \textit{Holmgren and Holmgren 5034} (NY, MONTU).  Just west of Green River, near the edge of the city. 3 Jun 1979; \textit{Rollins 79151} (NY, MO, US).  Limey and rocky slope, 1 mi. S of Green River. 1 Jun 1938; \textit{Rollins 2241} (NY).  Green River. 14 Jun 1898; \textit{Nelson 4714} (MONT).  Green River. 30 May 1897; \textit{Nelson 3032} (RM).  Green River. 6,000' elevation; 23 Jun 1896; \textit{Jones s.n.} (RSA-POM).  Green River. 9 Jul 1897; \textit{Williams s.n.} (RM).  Green River, Wyo. 24 Jun 1895; \textit{Shear 4364} (RM).  Green River. 25 Jun 1895; \textit{Rydberg s.n.} (NY, NY).  The Towers; ca. 0.5 air mi. N of Green River, northwest side of White Mountain Road. 6,460-6,500' elevation; T18N R107W S12 N1/2; 29 Jun 1994; \textit{Refsdal 1295} (RM).  4 mi. E of Green River. 6,225' elevation; 11 Jul 1965; \textit{Mulligan and Crompton 3097} (RM).  Green River Basin; drainage into Scott Canyon, ca. 6 air mi. WNW of Pilot Butte, ca. 12.5 air mi. WNW of Rock Springs. 6,660-6,950' elevation; T19N R107W S3 NE1/4 and S2 NW1/4, T20N R107W S34 SE1/4; 10 Jul 1996; \textit{Ward 2194} (RM).  Green River Basin; vicinity of Alkali Creek including Alkali Spring, ca. 3.5 air mi. E of southern extent of Blue Rim, ca. 18 air mi. NW of Rock Springs. 6,470-6,550' elevation; T21N R107W S35 N1/2 and S26 S1/4; 10 Jul 1996; \textit{Ward 2217} (RM).  6 mi. NW of Green River, Wyo. 13 Jun 1971; \textit{Hatch 1257} (NY, UTC).  Green River Basin; ca. 3 air mi. E of Fontenelle.  6,500-6,550' elevation; T23N R111W S9 N1/2; 22 May 1994; \textit{Hartman, Cramer, and Refsdal 45145} (RM).  Green River Basin; east end of Fontenelle Reservoir Dam, ca. 3 air mi. N of Fontenelle. 6,550' elevation; T24N R111W S30 NE1/4 and S19 SE1/4; 4 Aug 1994; \textit{Cramer 2717} (RM).  Green River Basin; lower Eighteen Mile Canyon, ca. 21 air mi. W of Farson. 6,500-6,750' elevation; T25N R109W S31 N1/2; 14 Jun 1994; \textit{Cramer 578} (RM, RSA-POM).  Green River Basin; ca. 26.5 air mi. WNW of Farson. 7,020-7,120' elevation; T26N R110W S3 W1/4; 2 Aug 1995; \textit{Cramer and Kellett 10241} (RM).  Green River Basin; sand knolls ca. 1.5 air mi. N of The Wells; ca. 5 air mi. W of White Mountain; ca. 23.5 air mi. NNW of Rock Springs. 6,960-7,065' elevation; T23N R105W S31 SW1/4, T23N 106W S36 SE1/4; 10 Jul 1996; \textit{Ward 2255} (RM).  Rock Springs Uplift; drainages from White Mountain to Killpecker Creek, ca. 4 air mi. E of The Wells, ca. 20.5 air mi. NNW of Rock Springs. 7,010-7,300' elevation; T22N R105W S11 S1/2 and S14; 23 Jun 1996; \textit{Ward 1352} (RM).  East Flaming Gorge Rd. 5 mi. S of I-80. 29 May 1982; \textit{Atkins s.n.} (UTC-36377).  Common near Rock Springs. 6,600' elevation; 30 May 1947; \textit{Larsen 21} (RM).  Southerly drainages on highlands between Crooked Canyon and North Baxter Basin, ca. 2 air mi. ESE from Twin Rocks, ca. 15.5 air mi. NNE of Rock Springs. 6,930-7,120' elevation; T21N R103W S16 NW1/4; 23 Jun 1996; \textit{Ward 1377} (RM).  7,200' elevation; T25N R102W S23; 4 Jul 1977; \textit{Dorn 2969} (RM).  Great Divide Basin; ridge on N side of Alkali Draw, S of County Rd 21, ca. 4.5 air mi. E of Bush Rim. 7,100-7,200' elevation; T24N R100W S6 SE1/4 of SE1/4, S5 SW1/4, and S8 NW1/4 of NW1/4; 27 Jul 1995; \textit{Fertig 16125} (RM).  Oregon Buttes, ca. 34 mi. NE of Farson. 8,500' elevation; T26N R101W S3 and S10; 27 Jun 1981; \textit{Dueholm 11642} (RM).  Slopes of the North Oregon Butte on the edge of the Red Desert. 8,000-8,600' elevation; T26N R101W S2; 24 May 1985; \textit{Scott 4105} (RM).  47 mi. E of Rock Springs along Hwy 430.	7,300' elevation; T13N R101W S17 SE; 18 Jun 1997; \textit{Atwood 22699} (RM).  Washakie Basin; from Cooper Ridge to Salt Wells Creek E of the confluence of Pretty Water and Salt Wells creeks, ca. 21 air mi. SSE of Rock Springs. 6,705-6,880' elevation; T16N R102W S20 NW1/4 and S19 NE1/4; 5 Jul 1996; \textit{Ward 2142} (RM).  Mesa W of north end of Cooper Ridge, S of Cutthroat Draw, ca. 1-2 air mi. E of Wyo Hwy 430, ca. 18 air mi. SE of Rock Springs. 6,620-7,230' elevation; T17N R102W S16 N1/4 and S9 S1/4; 7 Jun 1996; \textit{Ward 1} (RM).  Black Buttes, ca. 9 air mi. S of Point of Rocks. 7,500' elevation; T18N R101W S9; 22 Jun 1980; \textit{Dueholm 10246} (RM).  Rock Springs Uplift; north facing drainages into Scheggs Draw, ca. 1 air mi. W of Rifes Rim, ca. 33 air mi. SE of Rock Springs. 7,160-7,380' elevation; T15N R101W S35 SW1/4 and S34 SE1/4 S34, T14N R101W S3 NE1/4 and S2 NW1/4; 27 Jul 1996; \textit{Ward 3281} (RM).  Ca. 2 air mi. SW of Pine Butte. 7,400' elevation; T15N R100W S7 and S18; 16 Jul 1980; \textit{Dueholm 10582} (RM).  Washakie Basin; small butte above Pine Butte Basin between Pine and Sand butes, ca. 33 air mi. SE of Rock Springs, ca. 24.5 air mi. SSE of Point of Rocks. 7,760-8,170' elevation; T16N R100W S33 N1/4 and S28 S1/4; 7 Jun 1996; \textit{Ward 72} (RM).  Washakie Basin; east side of Kinney Rim, ca. 1 air mi. NW of County Road 19, ca. 43.5 air mi. SW of Wamsutter. 7,280-7,740' elevation; T14N R99W S8, S7 NE1/4, S9 SW1/4; 13 Jun 1996; \textit{Ward 489} (RM).  Washakie Basin; ca. 6.5 air mi. E of Sand Butte, ca. 2 air mi. SW of the confluence of Pine Creek Wash and Antelope Creek, ca. 36 air mi. SW of Mansutter. 7,190-7,330' elevation; T16N R99W S22 SW1/4 and S21 E1/4; 13 Jun 1996; \textit{Ward 516} (RM).  Great Divide Basin; south of Bitter Creek on County Hwy 19 on E-side of Antelope Creek. 7,000-7,200' elevation; T17N R99W S36; 14 Jun 1994; \textit{Fertig 14865} (RM).  Washakie Basin; butte N of Red Wash, ca. 1 mi. N of confluence with Bitter Creek, ca. 1.5 mi. E of County Road 19, ca. 37 air mi. ESE of Rock Springs, ca. 28.5 air mi. WSW of Wamsutter. 6,890-7,270' elevation; T18N R98W S31 NE1/4, S32 NW1/4; S29 SW1/4, and S30 SE1/4; 12 Jun 1996; \textit{Ward 248} (RM).  Ca. 13 air mi SSW of Red Desert. 7,000' elevation; T17N R95W S7 and T17N R96W S12; 11 Jun 1980; \textit{Dueholm 9958} (RM). South end of Delaney Rim just N of North Barrel Springs Draw, ca. 6 air mi. E of Man and Boy Buttes, ca. 13 air mi. SSW of Wamsutter. 6,760-6,880' elevation; T17N R94W S6 NW1/4; 7 Jun 1996; \textit{Nelson 37905} (RM).  Ridge between Barrel Springs Draw and Mulligan Draw, ca. 5 air mi. E of The Haystacks, ca. 17.5 air mi. SSW of Wamsutter. 6,880-6,920' elevation; T17N R95W S25 SW1/4; 7 Jun 1996; \textit{Nelson 37954} (RM).  Bluffs above Windmill Draw, ca. 4 air mi. N of Courthouse Butte, ca. 24.5 air mi, S of Wamsutter, ca. 24 air mi. NW of Baggs. 6,780-6,860' elevation; T15N R94W S4 NE1/4; 7 Jun 1996; \textit{Nelson 37980} (RM).  West Flat Top Mountain, ca. 30 air mi. S of Wamsutter, ca. 18 air mi. NW of Baggs. 7,400-7,420' elevation; T14N R94W S1 NE1/4; 10 Jul 1996; \textit{Nelson 38423} (RM).  Southeast side of butte overlooking Sand Creek, ca. 2 air mi. W of McPherson Springs, ca. 17.5 air mi. WNW of Baggs. 6,520-6,828' elevation; T13N R94W S21 SW1/4 and S20 SE1/4; 28 Jun 1996; \textit{Ward	1728} (RM).  Between Cherokee Basin and Cherokee Rim, ca. 14.5 air mi. W of Baggs. 6,350-6,520' elevation; T12N R94W S11; 27 Jun 1996; \textit{Ward 1582} (RM).  From Powder Rim N to Sand Creek, ca. 7.5 air mi. ENE of Powder Mountain, ca. 22.5 air mi. WNW of Baggs. 7,010-7,170' elevation; T13N R95W S28; 27 Jun 1996; \textit{Ward 1616} (RM).  White clay-like soil, 48 mi. S of Rock Springs. 19 Jun 1981; \textit{Rollins and Rollins 81358} (RM, NY, UC, US).  \textbf{Carbon County:} Dyer’s Ranch. 21 Jun 1901; \textit{Goodding 80} (RM, MO-3833614, MO-5120378).  Saratoga, Wyoming. 23 May 1924; \textit{Nelson 10092} (RM, NY, UC).  Seminoe Reservoir, ca. 10 air mi. SSE of Seminoe Dam. 6360’ elevation; T24N R84W S14 and S24; 11 July 1979; \textit{Hartman 9925} (NY).  E end Shirley Mountains, canyon at head of First Ranch Creek, ca. 6.5 air mi. SW of Wyo Hwy 77. 8,200-8,400’ elevation; T25N R81W S12 SW1/4; 11 Jun 1996; \textit{Fertig 16588} (RM). Shirley Basin, foothill ridges at NE end of Shirley Mountains, ca. 0.5 air mi. SE of Sullivan Creek, ca. 4 air mi. W of Pine Hill, ca. 5.25 mi. S of County Rd 102 (Leo Road). 7,700-7,800’ elevation; T25N R81W S1 SW1/4; 11 Jun 1996; \textit{Fertig 16573} (RM).  Shirley Basin, W side of BLM Rd 3115 ca. 3.5 air mi. S of jct with County Rd 102, between Sullivan Creek and First Ranch Creek, ca. 2 air mi. NE of base of Shirley Mountains, ca. 4 air mi. SE of WY state highway 77. 7,160-7,200’ elevation; T26N R80W S31 NE1/4 of NW1/4; 11 Jun 1996; \textit{Fertig 16560} (RM).  13.6 mi. N of Rawlins on Hwy 287. 2,277m elevation; 22 Jun 1996; \textit{Salywon and Dierig 3119} (MO).  22 mi. N of Rawlins. 6,500’ elevation; T23N R88W S15 NW1/4; 30 May 1981; \textit{Lichvar 4302} (RM).  Fort Steele. 18 Jun 1898; \textit{Nelson 4834} (RM).  Fort Steele. 6,500’ elevation; 25 May 1901; \textit{Tweedy 4488} (NY).  Gumbo flats 2 mi. S of Sinclair. 6,500’ elevation; 25 May 1947; \textit{Porter 4145} (RM).  Ca. 1/2 mi. E of Little Sage Creek Reservoir, ca. 13 air mi. S of Rawlins. 7,220’ elevation; T19N R88W S22 S1/2 of NE1/4; 23 Jun 1983; \textit{Warren 573} (RM).  Ca. 0.5 mi. SW of Bridger Pass on the southeast edge of Atlantic Rim, ca. 20 air mi. SW of Rawlins. 7,600-7,840’ elevation; T18N R89W S8; 4 Jun 1996; \textit{Nelson 37698} (RM).  Loco Creek Upland, Stratton Hydro. Study Area. 7,000-8,000’ elevation; T17N R87W S24; \textit{Schroeder and Ranch 4200-1} (RM).  Sage Creek Basin, ca. 27 air mi. S of Rawlins. 7,800’ elevation; T17N R88W S32; 22 Jun 1980; \textit{Mastrella 22} (RM).  Open juniper forest with Arabis, Cercocarpus. 7,100’ elevation; T16N R92W S10; 1 Jul 1979; \textit{Dorn 3301} (RM).  Ridge E of Muddy Creek and Wyo Hwy 789, ca. 36.5 air mi. SW of Rawlins, ca. 23 air mi. N of Baggs. 6,680-7,015’ elevation; T16N R92W S10 S1/4; 5 Jun 1996; \textit{Nelson and Ward 37782} (RM).  On ridge between Muddy, Cow, and Wild Cow creeks and Wild Cow or County Road 608, ca. 40 air mi. SW of Rawlins, ca. 19.5 air mi. N of Baggs. 6,506-6,730’ elevation; T15N R91W S8, S17, and S18; 5 Jun 1996; \textit{Nelson and Ward 37848} (RM).  Southwest facing slope of Coal Gulch draining from Browns Hill to Savery Creek, ca. 2.5 air mi. from the confluence of Coal Gulch with Savery Creek, ca. 16.5 air mi. NE of Baggs.	7,280-7,690' elevation; T14N R89W S9 NE1/4 and S10 NW1/4; 20 Jun 1996; \textit{Ward 961} (RM).  13 mi. N of Baggs, 2 mi. off Wyo Hwy 789. 22 May 1979; \textit{Rollins 7932} (NY, MO).  East end of Little Robber Gulch, ca. 12 air mi. NNW of Baggs. 6,500-6,800’ elevation; T14N R92W S11; 1 Jun 1996; \textit{Hartman and Ward 54230} (RM).  Deep Creek Rim and draws on west-facing slope, ca. 3 air mi. E of Wyo Hwy 789, ca. 8.5 air mi. NNE of Baggs. 6,680-7,210’ elevation; T14N R91W S26; 17 Jun 1996; \textit{Ward 860} (RM).  Washakie Basin, draw into Cottonwood Creek from the south between Streckfus Draw and Wyo Hwy 789, ca. 4 air mi. NNW of Baggs. 6,480-6,620’ elevation; T13N R91W S7; 17 Jun 1996; \textit{Ward 824} (RM).  Washakie Basin, southeast slope of headland between Middle Prong Red and North Prong Red creeks, ca. 3 air mi. S of Flat Top Mountain, ca. 10 mi. NW of Baggs. 6,510-6,885’ elevation; T13N R93W S1 N1/2 and T14N R93W S36 SW1/4; 17 Jun 1996; \textit{Ward 785} (RM).  Washakie Basin, Tree Draw leading from Hangout Ridge W to Hangout Wash, ca. 2-2.5 air mi. from the confluence of Hangout Wash and Sand Creek, ca. 12 air mi. WNW of Baggs. 6,260-6,620’ elevation; T13N R93W S16 W1/2, S17 N1/2, and S18 SE1/4; 16 Jun 1996; \textit{Ward 673} (RM).  Washakie Basin, rim N and W of Red Creek, ca. 2.5-3 air mi. above the confluence of Red and Sand creeks, ca. 9.5 air mi. WNW of Bagggs. 6,310-6,530’ elevation; T13N R93W S26 NW1/4 and S23 SW1/4; 17 Jun 1996; \textit{Ward 722} (RM).  Poison Buttes. 6,800’ elevation; T12N R93W S2 SW1/4; 14 Jun 1979; \textit{Lichvar 1756} (NY).  Southern extension of the bluffs NW of Baggs between the Little Snake River plain and Devils Canyon, ca. 3 air mi. W of Baggs. 6,420-6,880’ elevation; T12N R92W S2; 16 Jun 1996; \textit{Ward 634} (RM).  Ca. 3 air mi. SE of Dixon. 6,600’ elevation; T12N R90W S15 SE1/4; 14 Jun 1979; \textit{Hartman and Coffey 8970} (RM, NY).  \textbf{Albany County:} Near Centennial, Wyoming. July 1938; \textit{Mauzy s.n.} (NY).  State Hwy. 230, 4 mi. E of Jelm. 15 Jun 1986; \textit{Rollins and Rollins 8621} (RM, NY, UTC).  Stony canyon, Camel Rock. 21 Jun s.n., \textit{Schwartz 51} (UC).  Canyon of the North Fork of Sybelle Creek, 14 mi. E of Bosler Junction (U.S. 30 and Wyo State Route 34). 7,000’ elevation; 24 Jun 1951; \textit{Rollins and Porter 5114} (RM, NY, UC, US).  Morton’s Pass, NE of Bosler. 6,500’ elevation; 18 Jul 1950; \textit{Ripley and Barneby 10543} (NY, CAS).  




  







